%% Generated by Sphinx.
\def\sphinxdocclass{jupyterBook}
\documentclass[letterpaper,10pt,english]{jupyterBook}
\ifdefined\pdfpxdimen
   \let\sphinxpxdimen\pdfpxdimen\else\newdimen\sphinxpxdimen
\fi \sphinxpxdimen=.75bp\relax
\ifdefined\pdfimageresolution
    \pdfimageresolution= \numexpr \dimexpr1in\relax/\sphinxpxdimen\relax
\fi
%% let collapsible pdf bookmarks panel have high depth per default
\PassOptionsToPackage{bookmarksdepth=5}{hyperref}
%% turn off hyperref patch of \index as sphinx.xdy xindy module takes care of
%% suitable \hyperpage mark-up, working around hyperref-xindy incompatibility
\PassOptionsToPackage{hyperindex=false}{hyperref}
%% memoir class requires extra handling
\makeatletter\@ifclassloaded{memoir}
{\ifdefined\memhyperindexfalse\memhyperindexfalse\fi}{}\makeatother

\PassOptionsToPackage{warn}{textcomp}

\catcode`^^^^00a0\active\protected\def^^^^00a0{\leavevmode\nobreak\ }
\usepackage{cmap}
\usepackage{fontspec}
\defaultfontfeatures[\rmfamily,\sffamily,\ttfamily]{}
\usepackage{amsmath,amssymb,amstext}
\usepackage{polyglossia}
\setmainlanguage{english}



\setmainfont{FreeSerif}[
  Extension      = .otf,
  UprightFont    = *,
  ItalicFont     = *Italic,
  BoldFont       = *Bold,
  BoldItalicFont = *BoldItalic
]
\setsansfont{FreeSans}[
  Extension      = .otf,
  UprightFont    = *,
  ItalicFont     = *Oblique,
  BoldFont       = *Bold,
  BoldItalicFont = *BoldOblique,
]
\setmonofont{FreeMono}[
  Extension      = .otf,
  UprightFont    = *,
  ItalicFont     = *Oblique,
  BoldFont       = *Bold,
  BoldItalicFont = *BoldOblique,
]



\usepackage[Bjarne]{fncychap}
\usepackage[,numfigreset=1,mathnumfig]{sphinx}

\fvset{fontsize=\small}
\usepackage{geometry}


% Include hyperref last.
\usepackage{hyperref}
% Fix anchor placement for figures with captions.
\usepackage{hypcap}% it must be loaded after hyperref.
% Set up styles of URL: it should be placed after hyperref.
\urlstyle{same}

\addto\captionsenglish{\renewcommand{\contentsname}{0 - Intro and Syllabus}}

\usepackage{sphinxmessages}



        % Start of preamble defined in sphinx-jupyterbook-latex %
         \usepackage[Latin,Greek]{ucharclasses}
        \usepackage{unicode-math}
        % fixing title of the toc
        \addto\captionsenglish{\renewcommand{\contentsname}{Contents}}
        \hypersetup{
            pdfencoding=auto,
            psdextra
        }
        % End of preamble defined in sphinx-jupyterbook-latex %
        

\title{PHY 415 Fall 2023}
\date{Aug 23, 2023}
\release{}
\author{Danny Caballero, Alia Valentine}
\newcommand{\sphinxlogo}{\vbox{}}
\renewcommand{\releasename}{}
\makeindex
\begin{document}

\pagestyle{empty}
\sphinxmaketitle
\pagestyle{plain}
\sphinxtableofcontents
\pagestyle{normal}
\phantomsection\label{\detokenize{content/intro::doc}}


\sphinxAtStartPar
PHY 415, called, “Mathematical Methods for Physicists” is a course the brings together many of the mathematical approaches that we commonly use in physics and apply them to variety of problems. In this course, we will take a modeling\sphinxhyphen{}based approach where we focus on the mathematical descriptions of physical phenomenon and determine what mathematical and analytical approaches are useful in exploring those models.

\sphinxAtStartPar
To get a sense of the course, please read all the pages associated with our syllabus.

\begin{DUlineblock}{0em}
\item[] \sphinxstylestrong{\Large Learning Objectives}
\end{DUlineblock}

\sphinxAtStartPar
In this course, you will learn to:
\begin{itemize}
\item {} 
\sphinxAtStartPar
investigate physical systems using a variety of tools and approaches,

\item {} 
\sphinxAtStartPar
construct and document a reproducible process for those investigations,

\item {} 
\sphinxAtStartPar
use analytical, computational, and graphical approaches to answer specific questions in those investigations,

\item {} 
\sphinxAtStartPar
provide evidence of the quality of work using a variety of sources, and

\item {} 
\sphinxAtStartPar
collaborate effectively and contribute to a inclusive learning environment

\end{itemize}

\sphinxAtStartPar
Table of contents:
\begin{itemize}
\item {} 
\sphinxAtStartPar
0 \sphinxhyphen{} Intro and Syllabus

\begin{itemize}
\item {} 
\sphinxAtStartPar
{\hyperref[\detokenize{content/0_course/0_syllabus::doc}]{\sphinxcrossref{Syllabus and Overview of PHY 415}}}

\end{itemize}
\end{itemize}
\begin{itemize}
\item {} 
\sphinxAtStartPar
1 \sphinxhyphen{} Mechanics and ODEs

\begin{itemize}
\item {} 
\sphinxAtStartPar
{\hyperref[\detokenize{content/1_mechanics/modeling/what_is_modeling::doc}]{\sphinxcrossref{What is Mathematical Modeling?}}}

\item {} 
\sphinxAtStartPar
{\hyperref[\detokenize{content/1_mechanics/modeling/activity-what_is_a_model::doc}]{\sphinxcrossref{01 Sept 22 \sphinxhyphen{} Activity: What is a Model?}}}

\item {} 
\sphinxAtStartPar
{\hyperref[\detokenize{content/1_mechanics/mechanics_intro::doc}]{\sphinxcrossref{Intro to Classical Mechanics}}}

\end{itemize}
\end{itemize}

\sphinxstepscope


\part{0 \sphinxhyphen{} Intro and Syllabus}

\sphinxstepscope


\chapter{Syllabus and Overview of PHY 415}
\label{\detokenize{content/0_course/0_syllabus:syllabus-and-overview-of-phy-415}}\label{\detokenize{content/0_course/0_syllabus::doc}}
\sphinxAtStartPar
In designing this course, we plan to emphasize more independent learning on your part and greater agency for you in determining what you learn and how you demonstrate you have learned. So you should expect:
\begin{itemize}
\item {} 
\sphinxAtStartPar
to read a variety of pieces of information to coordinate information

\item {} 
\sphinxAtStartPar
to present your ideas publicly and to discuss them

\item {} 
\sphinxAtStartPar
to learn new approaches and novel techniques on your own

\item {} 
\sphinxAtStartPar
to become more expert than me in the areas of your interest

\item {} 
\sphinxAtStartPar
to learn more about scientists that you have not learned about

\end{itemize}

\sphinxAtStartPar
This is not to say that you are on your own. Here’s what you can expect from us:
\begin{itemize}
\item {} 
\sphinxAtStartPar
resources, information, and tools to help you learn

\item {} 
\sphinxAtStartPar
support and scaffolding to move you towards more independence in your learning

\item {} 
\sphinxAtStartPar
timely and detailed feedback to help you along

\item {} 
\sphinxAtStartPar
a commitment to an inclusive classroom

\end{itemize}

\sphinxAtStartPar
\sphinxhref{./6\_environment.html\#use-of-generative-ai-tools}{Statement on the use of Generative AI tools}


\section{Contact Information}
\label{\detokenize{content/0_course/0_syllabus:contact-information}}

\subsection{Web page}
\label{\detokenize{content/0_course/0_syllabus:web-page}}\begin{itemize}
\item {} 
\sphinxAtStartPar
Web page for this class:
\sphinxurl{https://dannycab.github.io/phy415fall23/content/intro}

\end{itemize}


\subsection{Instructor}
\label{\detokenize{content/0_course/0_syllabus:instructor}}\begin{itemize}
\item {} 
\sphinxAtStartPar
\sphinxhref{http://dannycab.github.io}{Prof. Danny Caballero} (he/him/his)

\item {} 
\sphinxAtStartPar
Class Meetings: Tuesdays and Thursdays 10:20am\sphinxhyphen{}12:10pm (Location: 1300 BPS)

\item {} 
\sphinxAtStartPar
Email: \sphinxhref{mailto:caball14@msu.edu}{caball14@msu.edu}, office: 1310\sphinxhyphen{}A BPS

\item {} 
\sphinxAtStartPar
Office hrs: To be scheduled, but I also have an open door policy. I enjoy visiting and talking with you about physics.

\end{itemize}


\section{Teaching Assistant/Grader}
\label{\detokenize{content/0_course/0_syllabus:teaching-assistant-grader}}\begin{itemize}
\item {} 
\sphinxAtStartPar
\sphinxhref{https://directory.natsci.msu.edu/directory/Profiles/Person/101905}{Ian Neuhart} (he/him/his)

\item {} 
\sphinxAtStartPar
Email: \sphinxhref{mailto:neuharti@msu.edu}{neuharti@msu.edu}

\item {} 
\sphinxAtStartPar
Office hrs: TBD

\end{itemize}


\subsection{Learning Assistant}
\label{\detokenize{content/0_course/0_syllabus:learning-assistant}}\begin{itemize}
\item {} 
\sphinxAtStartPar
\sphinxhref{https://valentine-alia.github.io/}{Alia Valentine} (she/her/hers)

\item {} 
\sphinxAtStartPar
Email: \sphinxhref{mailto:valen176@msu.edu}{valen176@msu.edu}

\item {} 
\sphinxAtStartPar
Office hrs: TBD, feel free to email me if you want to set up a time. I too enjoy visiting and talking with you about math and physics.

\end{itemize}


\section{Grading}
\label{\detokenize{content/0_course/0_syllabus:grading}}
\sphinxAtStartPar
Details about \DUrole{xref,myst}{course activities are here} and \DUrole{xref,myst}{information regarding assessment is here}. Your grade will be comprised of completing weekly discussion questions, seven of eight worked problems, and three of four projects that you will complete in the form of a Jupyter notebook (a “computational essay”, which we will discuss later).

\sphinxAtStartPar
Your grade is comprised of the following:


\begin{savenotes}\sphinxattablestart
\centering
\begin{tabulary}{\linewidth}[t]{|T|T|}
\hline
\sphinxstyletheadfamily 
\sphinxAtStartPar
Activity
&\sphinxstyletheadfamily 
\sphinxAtStartPar
Percent of Grade
\\
\hline
\sphinxAtStartPar
Weekly Reading Questions (completion)
&
\sphinxAtStartPar
5\%
\\
\hline
\sphinxAtStartPar
Seven of Eight Worked Problems (5\% each)
&
\sphinxAtStartPar
35\%
\\
\hline
\sphinxAtStartPar
Three of Four Projects (see below)
&
\sphinxAtStartPar
60\%
\\
\hline
\sphinxAtStartPar
\sphinxstylestrong{Total}
&
\sphinxAtStartPar
\sphinxstylestrong{100\%}
\\
\hline
\end{tabulary}
\par
\sphinxattableend\end{savenotes}

\sphinxAtStartPar
Your grade on each project is split between completion (50\%) and quality (50\%). We will collectively define “quality” in class, but we have provided \DUrole{xref,myst}{a preliminary rubric} for us to work from for the first project. Your final grade will be scaled based on your best performances; there will be slightly more projects than what comprises your grade.  \sphinxstyleemphasis{The intent here is to to allow you space to explore a model or project that you really enjoy, and to reward you for doing that.} How your project grade is calculated appears below.


\begin{savenotes}\sphinxattablestart
\centering
\begin{tabulary}{\linewidth}[t]{|T|T|}
\hline
\sphinxstyletheadfamily 
\sphinxAtStartPar
Activity
&\sphinxstyletheadfamily 
\sphinxAtStartPar
Percent of Grade
\\
\hline
\sphinxAtStartPar
Best Project Grade
&
\sphinxAtStartPar
25\%
\\
\hline
\sphinxAtStartPar
2nd Best Project Grade
&
\sphinxAtStartPar
20\%
\\
\hline
\sphinxAtStartPar
3rd Best Project Grade
&
\sphinxAtStartPar
15\%
\\
\hline
\end{tabulary}
\par
\sphinxattableend\end{savenotes}

\sphinxAtStartPar
\sphinxstylestrong{While attendance is not required, you are unlikely to succeed with your projects without regular attendance and engagement.}


\section{“Extra” Credit}
\label{\detokenize{content/0_course/0_syllabus:extra-credit}}
\sphinxAtStartPar
We get that you might want to do more of the projects or maybe you fell a little behind during the semester. If you complete 8 of the 8 worked problems, you can earn up to another 5\%. If you complete four projects, then your lowest scoring project will be included up to an additional 10\%. These have to be completed within the usual due dates for the worked problems and projects; not at the end of the semester.

\sphinxstepscope


\section{Course Objectives}
\label{\detokenize{content/0_course/1_goals:course-objectives}}\label{\detokenize{content/0_course/1_goals::doc}}
\sphinxAtStartPar
This course emphasizes making models of physical phenomenon and how we use various tools at our disposal to investigate those models. Hence, we have learning objectives for making models of these systems and for learning specific tools.


\subsection{Investigate physical systems}
\label{\detokenize{content/0_course/1_goals:investigate-physical-systems}}
\sphinxAtStartPar
Clearly, one of our central goals is learning how to make models of physical systems. This means learning about and developing fluency with a wide variety of mathematical and computational tools. In this courses, we will make extensive use of \sphinxhref{http://anaconda.org}{Jupyter notebooks} for homework and projects. In fact, what you are reading is a set of Jupyter notebooks! Below, you will see the list of objectives for this principal objective.

\begin{sphinxadmonition}{note}{Investigating Physical Systems Learning Objectives}

\sphinxAtStartPar
Students will demonstrate they can:
\begin{itemize}
\item {} 
\sphinxAtStartPar
use mathematical techniques to predict or explain some physical phenomenon

\item {} 
\sphinxAtStartPar
employ computational models and algorithms to investigate physical systems

\item {} 
\sphinxAtStartPar
compare analytical and computational approaches to these investigations

\item {} 
\sphinxAtStartPar
provide coherent explanations for their investigations buttressed by physical, mathematical, and/or computational knowledge and principles

\end{itemize}
\end{sphinxadmonition}

\begin{sphinxadmonition}{note}{Principle Learning Objectives}

\sphinxAtStartPar
Students will demonstrate they can:
\begin{itemize}
\item {} 
\sphinxAtStartPar
investigate physical systems of their choosing using a variety of tools and approaches

\item {} 
\sphinxAtStartPar
construct and document a reproducible process for those investigations

\item {} 
\sphinxAtStartPar
use analytical, computational, and graphical approaches to answer specific questions in those investigations

\item {} 
\sphinxAtStartPar
provide evidence of the quality of their work using a variety of sources

\item {} 
\sphinxAtStartPar
collaborate effectively and contribute to a inclusive learning environment

\end{itemize}

\sphinxAtStartPar
Each of these learning objectives contributes to your development as a physicist. I recognize that these are \sphinxstylestrong{big} ideas to think about. What I mean is that the objectives above are quite broad and you might be able to see a little about what or why they are included. But, below, I added more detail about each one along with a smaller scale list of objectives that you will engage with. Throughout our course, you will have opportunities to demonstrate these objectives in your work. \sphinxstyleemphasis{My aim is to make what you are assessed on in this course something you are interested in, so these objectives reflect that.}
\end{sphinxadmonition}


\subsection{Construct and document a reproducible process}
\label{\detokenize{content/0_course/1_goals:construct-and-document-a-reproducible-process}}
\sphinxAtStartPar
A critical element of physics work is making sure that with the same setup and approach, others can reproduce the work you have done. This provides validity to your work and evidences how we develop collective understanding of physics. Physics is a social enterprise and the ensuring the reproducibility of work supports that enterprise. Below are the learning objectives for this principal objective.

\begin{sphinxadmonition}{note}{Reproducibility Learning Objectives}

\sphinxAtStartPar
Students will demonstrate they can:
\begin{itemize}
\item {} 
\sphinxAtStartPar
document their work and analysis such that others can reproduce their work

\item {} 
\sphinxAtStartPar
consistently reproduce their work and results in a variety of contexts

\item {} 
\sphinxAtStartPar
provide an explanation for why certain work or results are not (or should not be) reproducible

\end{itemize}
\end{sphinxadmonition}


\subsection{Use analytical, computational, and graphical approaches}
\label{\detokenize{content/0_course/1_goals:use-analytical-computational-and-graphical-approaches}}
\sphinxAtStartPar
The main approaches that we use to make models are mathematical, computational, and graphical. In this class, we will aim to leverage the benefits of each to learn more about the physical systems that we are investigating. Indeed, much of the “knowledge” that you are going to develop will be about specific analytical, computational, or graphical approaches to investigate physical systems. Below are the learning objectives for this principal objective.

\begin{sphinxadmonition}{note}{Modeling Approaches Learning Objectives}

\sphinxAtStartPar
Students will demonstrate they can:
\begin{itemize}
\item {} 
\sphinxAtStartPar
Use a wide variety of modeling techniques to investigate different physical systems

\item {} 
\sphinxAtStartPar
Choose and employ appropriate approaches to modeling physical systems of their choosing

\item {} 
\sphinxAtStartPar
Explain how those approaches lead to different results or conclusions

\end{itemize}
\end{sphinxadmonition}


\subsection{Provide evidence of the quality of their work}
\label{\detokenize{content/0_course/1_goals:provide-evidence-of-the-quality-of-their-work}}
\sphinxAtStartPar
The definition of the quality of a piece of science is a collective decision by the scientific community. In established communities, like physics, there are commonly\sphinxhyphen{}accepted ways of defining the quality of work (norms, customs, and rules all play a role). But that is not to mean those ways can’t change; papers describing quantum physics and relativity brushed up hard against this issue of quality and were both dismissed and celebrated. Newer disciplines are still establishing those norms and rules. And in some cases, disciplines are pushing back against Western norms of quality. In our class, we will collectively decide what we mean by ‘’high quality’’ work. Below are the learning objectives for this principal objective.

\begin{sphinxadmonition}{note}{Quality Control Learning Objectives}

\sphinxAtStartPar
Students will demonstrate they can:
\begin{itemize}
\item {} 
\sphinxAtStartPar
describe what it means to have high quality work in our class

\item {} 
\sphinxAtStartPar
look for and evaluate when work meets those standards

\item {} 
\sphinxAtStartPar
provide suggestions (or act on suggestions) to improve the quality of their work

\end{itemize}
\end{sphinxadmonition}


\subsection{Collaborate effectively}
\label{\detokenize{content/0_course/1_goals:collaborate-effectively}}
\sphinxAtStartPar
Physics is a social enterprise that relies on effective and productive collaborations. Very little (if any) science is done alone; the scale of science is too grand for individuals to effectively work – everyone needs a team. In this spirit, in this classroom, we deeply encourage collaboration. We will try to develop effective collaboration through your work on projects and our in\sphinxhyphen{}class activities. Below are the learning objectives for this principal objective.

\begin{sphinxadmonition}{note}{Collaboration Learning Objectives}

\sphinxAtStartPar
Students will demonstrate they can:
\begin{itemize}
\item {} 
\sphinxAtStartPar
Collaborate on a variety of activities in and out of class

\item {} 
\sphinxAtStartPar
Document the contributions in these collaborations and make changes if contributions are unbalanced

\item {} 
\sphinxAtStartPar
Develop personally effective strategies for collaboration

\end{itemize}
\end{sphinxadmonition}

\sphinxstepscope


\section{Course Design}
\label{\detokenize{content/0_course/2_design:course-design}}\label{\detokenize{content/0_course/2_design::doc}}
\sphinxAtStartPar
For most of you, 415 is an elective course that you are taking to learn more about how we use mathematical techniques in physics. As such, this course is designed under several different principles than a standard course. Below, I provide those principles and their rationale.
\begin{itemize}
\item {} 
\sphinxAtStartPar
415 should help you learn the central tenets of modeling physical systems
\begin{itemize}
\item {} 
\sphinxAtStartPar
The sheer volume of mathematical and computational physics knowledge out there is immense and impossible for any one person to learn. However, the central elements of making models, how to learn about specific techniques, and how to debug your approaches are things we can learn and employ broadly as well as to specific problems.

\end{itemize}

\item {} 
\sphinxAtStartPar
415 should be a celebration of your knowledge
\begin{itemize}
\item {} 
\sphinxAtStartPar
For most of you, this course is part of your senior level coursework. What you have achieved in the last three to four years should be celebrated and enjoyed. This course will provide ample opportunities for you to share what things you know and what things you are learning with me and with each other.

\end{itemize}

\item {} 
\sphinxAtStartPar
415 should give you opportunities to engage in professional practice
\begin{itemize}
\item {} 
\sphinxAtStartPar
As you start towards your professional career, it’s important to learn what professional scientists do. You have probably already begun this work in advanced lab and research projects that you have worked on. We will continue developing your professional skills in this course through the use of course projects.

\end{itemize}

\item {} 
\sphinxAtStartPar
415 will illustrate that we can learn from each other
\begin{itemize}
\item {} 
\sphinxAtStartPar
Even though I’ve been learning physics for almost 20 years, I don’t know everything. I am excited to learn from you and I hope that you are excited to learn from me and each other.

\end{itemize}

\end{itemize}


\subsection{Optional purchases:}
\label{\detokenize{content/0_course/2_design:optional-purchases}}
\sphinxAtStartPar
The core readings and work for this course will be this jupyterbook. I will find resources online, make my own, and provide as much organized free material as possible. If you want to have a textbook that helps you organize your readings, please obtain copies of:
\begin{enumerate}
\sphinxsetlistlabels{\arabic}{enumi}{enumii}{}{.}%
\item {} 
\sphinxAtStartPar
Mary Boas, \sphinxhref{https://bookshop.org/p/books/mathematical-methods-in-the-physical-sciences-mary-l-boas/7483888}{\sphinxstyleemphasis{Mathematical Methods in the Physical Sciences}} (Wiley; 2005). This book is the definitive text on mathematical approaches, written by Dr. Boas originally in 1966! Any 3rd edition will be useful and I will put the section numbers from Boas in the online readings.

\item {} 
\sphinxAtStartPar
Mark Newman, \sphinxhref{https://bookshop.org/p/books/computational-physics-mark-newman/9385815}{\sphinxstyleemphasis{Computational Physics}} (CreateSpace Independent Publishing Platform; 2012). This book is a great introduction to a variety of computational physics techniques, written by UMich professor Mark Newman for a computational physics course. I will put section numbers from Newman in the online readings.

\end{enumerate}


\subsubsection{Additional sources}
\label{\detokenize{content/0_course/2_design:additional-sources}}
\sphinxAtStartPar
In addition, I will draw from the following books. I have copies of them if you want or need scans of sections. But they can found online in Google Books and other places as well. No need to purchase unless you want a copy for your personal library.


\paragraph{Mechanics}
\label{\detokenize{content/0_course/2_design:mechanics}}\begin{itemize}
\item {} 
\sphinxAtStartPar
Edwin Taylor, Mechanics

\item {} 
\sphinxAtStartPar
Jerry Marion and Stephen Thornton, Classical Dynamics of Particles and Systems

\item {} 
\sphinxAtStartPar
Charles Kittel, Walter D. Knight, Malvin A. Ruderman, A. Carl Helholtz, and Burton J. Moyer, Mechanics

\end{itemize}


\paragraph{Electromagnetism}
\label{\detokenize{content/0_course/2_design:electromagnetism}}\begin{itemize}
\item {} 
\sphinxAtStartPar
Edward Purcell, Electricity and Magnetism

\item {} 
\sphinxAtStartPar
David J. Grriffths, Introduction to Electromagnetism

\end{itemize}


\paragraph{Quantum Mechanics}
\label{\detokenize{content/0_course/2_design:quantum-mechanics}}\begin{itemize}
\item {} 
\sphinxAtStartPar
David McIntyre, Quantum Mechanics

\item {} 
\sphinxAtStartPar
David J. Griffiths, Introduction to Quantum Mechanics

\end{itemize}


\paragraph{Waves and Thermal Physics}
\label{\detokenize{content/0_course/2_design:waves-and-thermal-physics}}\begin{itemize}
\item {} 
\sphinxAtStartPar
Frank S. Crawford, Waves

\item {} 
\sphinxAtStartPar
Charles Kittel, Thermal Physics

\item {} 
\sphinxAtStartPar
Ashley Carter, Classical and Statistical Thermodynamics

\item {} 
\sphinxAtStartPar
Daniel Schroeder, Thermal Physics

\end{itemize}


\paragraph{Additional Physics Topics}
\label{\detokenize{content/0_course/2_design:additional-physics-topics}}\begin{itemize}
\item {} 
\sphinxAtStartPar
Steven H. Strogatz, Nonlinear Dynamics and Chaos

\item {} 
\sphinxAtStartPar
B Lautrup, Physics of Continuous Matter

\item {} 
\sphinxAtStartPar
Frank L. Pedrotti and Leno S. Pedrotti, Introduction to Optics

\end{itemize}


\paragraph{Mathematics}
\label{\detokenize{content/0_course/2_design:mathematics}}\begin{itemize}
\item {} 
\sphinxAtStartPar
Susan M. Lea, Mathematics for Physicists

\item {} 
\sphinxAtStartPar
William E. Boyce and Richard C. DiPrima, Elementary Differential Equations

\item {} 
\sphinxAtStartPar
James Brown and Ruel Churchill, Complex Variables and Applications

\item {} 
\sphinxAtStartPar
Jerrold Marsden and Anthony Tromba, Vector Calculus

\item {} 
\sphinxAtStartPar
Sheldon Ross, A First Course in Probability

\end{itemize}


\paragraph{Presenting (Visual) Information}
\label{\detokenize{content/0_course/2_design:presenting-visual-information}}\begin{itemize}
\item {} 
\sphinxAtStartPar
Edward Tufte, The Visual Display of Quantitative information

\item {} 
\sphinxAtStartPar
Albert Cairo, The Truthful Art

\item {} 
\sphinxAtStartPar
Stephen E. Toulmin, The Uses of Argument

\end{itemize}


\section{Course Activities}
\label{\detokenize{content/0_course/2_design:course-activities}}

\subsection{“Readings”}
\label{\detokenize{content/0_course/2_design:readings}}
\sphinxAtStartPar
\sphinxstylestrong{“Reading”} is an essential part of 415! Reading the notes before class is very important. I use “reading” in quotes, because in our class this idea goes beyond just reading text and includes understanding figures and watching videos. These should help inform the basis of your understating that we will draw on in class to clarify your understanding and to help you make sense of the material. I will assume you have done the required readings in advance! It will make a huge difference if you spend the time and effort to carefully read and follow the resources posted. The calendar has the details on videos and readings that will be updated.

\sphinxAtStartPar
\sphinxstylestrong{Weekly Questions}: To encourage and reward you for keeping up with the “readings”, there will be weekly questions about the readings posted for you to respond to. These are not meant to test your knowledge, but rather to focus your “reading” towards what you understand, and what you don’t yet understand. I will ask you about those things weekly and use that information to tailor in\sphinxhyphen{}class activities based on what I am hearing is confusing, unclear, or challenging. These questions are only graded for completion, but I do want your honest attempt.


\subsection{Class Meetings}
\label{\detokenize{content/0_course/2_design:class-meetings}}
\sphinxAtStartPar
\sphinxstylestrong{Classroom Etiquette:} Please silence your electronic devices when entering the classroom. I don’t mind you using them (in fact, see below, we will use them). But, sometimes, they can very distracting to your neighbors, so use your judgement. I appreciate that you might have questions or comments about things in class. We are going to be having short lectures combined with longer project work in class. So you will have plenty of time to catch up with social media and the news.

\sphinxAtStartPar
If you and/or your group mates are confused, just raise your hand and ask questions. If you are confused, you are likely not the only one and it’s better to chat about it, then move on. Questions are always good, and are strongly encouraged! \sphinxstyleemphasis{The only way we learn is to question what we know and how we know it.}

\sphinxAtStartPar
\sphinxstylestrong{Computing Devices:} Please bring some sort of computing device to class everyday. You might be researching information online, reviewing work you have done, or actively building models of systems together. This device can be a computer, a tablet, or a phone. You can also partner up with folks because we will use them in groups. \sphinxstyleemphasis{If you need a computing device brought to class for you or your group mates to use, let me know. I will organize for some small collection of laptops if we need it.}

\sphinxAtStartPar
\sphinxstylestrong{In\sphinxhyphen{}Class:} We will have some short lectures about topics or concepts; some of those will be in\sphinxhyphen{}the\sphinxhyphen{}moment as needed. The idea is that you are developing a basic understanding through readings and videos, practicing using those new ideas with me and with your classmates in class, and then applying what you are learning to new ideas. So, we will also use a variety of in\sphinxhyphen{}class activities that help you construct an understanding of a particular topic or concept. These will not be collected or graded, but we will discuss the solutions in class. \sphinxstyleemphasis{I will not post solutions for these activities as we have no exams or quizzes.}


\subsection{“Homework”}
\label{\detokenize{content/0_course/2_design:homework}}
\sphinxAtStartPar
\sphinxstylestrong{Worked Problems:} We will spend much of our time learning specific techniques and approaches to use withh many different kinds of problems and models. Your “weekly” homework will be to select a particular example where the method or model applies, and work that problem yourself. You will need to explain your approach and findings on paper or in a notebook. You will turn in seven of eight of these worked problems. \sphinxstyleemphasis{I will not post solutions for these problems as we have no exams or quizzes.}


\subsection{Projects}
\label{\detokenize{content/0_course/2_design:projects}}
\sphinxAtStartPar
\sphinxstylestrong{In\sphinxhyphen{}class Projects:} The class is designed to support your independent research into ideas that you are excited about. So in\sphinxhyphen{}class projects are meant to equip you with the knowledge and practice to learn new things for your projects. These in\sphinxhyphen{}class projects will be short demonstrations of models that you complete in groups. We will circulate around the room and check on you and your group’s progress and understanding. At the end of the class period, we will share the results of the in\sphinxhyphen{}class project and discuss any sticking points. These in\sphinxhyphen{}class activities will not be graded, but they will be essential for your out\sphinxhyphen{}of\sphinxhyphen{}class projects.

\sphinxAtStartPar
\sphinxstylestrong{Out\sphinxhyphen{}of\sphinxhyphen{}class Projects:} For this class, we anticipate 6 projects to be turned in roughly every 2\sphinxhyphen{}3 weeks, with a weeklong turn\sphinxhyphen{}in window (see calendar). Except for the first project, up to 3 of these projects can be completed as partner projects. Partner projects are subject to a different grading rubric that evaluates collaborative efforts and increases the expectation for other areas compared to an individual project. A preliminary rubric appears here, but we will define these collectively after the first project.

\sphinxAtStartPar
These projects will take the form a \sphinxhref{https://uio-ccse.github.io/computational-essay-showroom/}{computational essay}, which provides documentation and rationale for the exploration that you are completing. We will model a computational essay project in our first project and we will reflect on the rubric after it, and make changes collectively as a class to it.

\sphinxAtStartPar
\sphinxstyleemphasis{I strongly encourage collaboration}, an essential skill in science and engineering (and highly valued by employers!) Social interactions are critical to scientists’ success – most good ideas grow out of discussions with colleagues, and essentially all physicists work as part of a group. Find partners and work together. However, it is also important that you OWN the material. I strongly suggest you start working by yourself (and that means really making an extended effort on every activity). Then work with a group, and finally, finish up on your own – write up your own work, in your own way. There will also be time for peer discussion during classes – as you work together, try to help your partners get over confusions, listen to them, ask each other questions, critique, teach each other. You will learn a lot this way! For all assignments, the work you turn in must in the end be your own: in your own words, reflecting your own understanding. (If, at any time, for any reason, you feel disadvantaged or isolated, contact me and I can discretely try to help arrange study groups.)


\subsubsection{Help Session}
\label{\detokenize{content/0_course/2_design:help-session}}
\sphinxAtStartPar
Help sessions/office hours are to facilitate your learning. We encourage attendance \sphinxhyphen{} plan on working in small groups, our role will be as learning coaches. The sessions are concept and project\sphinxhyphen{}centric, but we will not be explicitly telling anyone how to do your project (how would that help you learn?) I strongly encourage you to start all projects on your own. If you come to help sessions “cold”, the value of the project to you will be greatly reduced.

\sphinxstepscope


\section{Assessments}
\label{\detokenize{content/0_course/3_assessments:assessments}}\label{\detokenize{content/0_course/3_assessments::doc}}

\subsection{Formative Assessment}
\label{\detokenize{content/0_course/3_assessments:formative-assessment}}
\sphinxAtStartPar
Formative assessment is often ungraded and reflective assessment. It is meant to help you make changes to your thinking, approaches, or practice. It is not evaluative, it’s corrective; to help you make changes. We will make heavy use of ungraded formative feedback throughout the course. Our weekly readings and your worked problems are meant to be formative; meaning most folks will receive full credit, but the important part to review the feedback and reflect on it.


\subsection{Summative Assessment}
\label{\detokenize{content/0_course/3_assessments:summative-assessment}}
\sphinxAtStartPar
Summative assessment is typically evaluative and will take the form of course projects completed out of class. These projects will take the form of a computational essay in which you write mathematics and code to investigate and explain a given phenomenon of interest. We will explore those essays in class and talk about what makes a useful one as we define a rubric for evaluation.


\subsubsection{Preliminary Rubric}
\label{\detokenize{content/0_course/3_assessments:preliminary-rubric}}
\sphinxAtStartPar
A preliminary rubric has been posted. We will use this rubric for the first out\sphinxhyphen{}of\sphinxhyphen{}class project evaluation. We will then reflect on it and make changes to collectively as a class.


\subsubsection{Resources for Computational Essays}
\label{\detokenize{content/0_course/3_assessments:resources-for-computational-essays}}
\sphinxAtStartPar
If you want to read more about computational essays, here’s a few links in the order utility/readability:
\begin{itemize}
\item {} 
\sphinxAtStartPar
Steven Wolfram \sphinxhyphen{} \sphinxhref{https://writings.stephenwolfram.com/2017/11/what-is-a-computational-essay/}{What is a Computational Essay?}

\item {} 
\sphinxAtStartPar
University of Oslo Physics \sphinxhyphen{} \sphinxhref{https://uio-ccse.github.io/computational-essay-showroom/}{Examples of Computational Essays}

\item {} 
\sphinxAtStartPar
Odden and Burk, The Physics Teacher \sphinxhyphen{} \sphinxhref{https://aapt.scitation.org/doi/abs/10.1119/1.5145471}{Computational Essays in the Physics Classroom}

\item {} 
\sphinxAtStartPar
Odden, Lockwood, and Caballero, Physical Review PER \sphinxhyphen{} \sphinxhref{https://journals.aps.org/prper/abstract/10.1103/PhysRevPhysEducRes.15.020152}{Physics computational literacy: An exploratory case study using computational essays}

\end{itemize}

\sphinxstepscope


\section{Project Rubrics}
\label{\detokenize{content/0_course/4_rubric:project-rubrics}}\label{\detokenize{content/0_course/4_rubric::doc}}

\subsection{Preliminary (For first out of class project)}
\label{\detokenize{content/0_course/4_rubric:preliminary-for-first-out-of-class-project}}
\sphinxAtStartPar
We have worked together to define elements of a rubric that matter for making physical models. These elements appear as part of major learning goals below.


\begin{savenotes}\sphinxattablestart
\centering
\begin{tabulary}{\linewidth}[t]{|T|T|}
\hline
\sphinxstyletheadfamily 
\sphinxAtStartPar
Goal
&\sphinxstyletheadfamily 
\sphinxAtStartPar
Fractional Importance
\\
\hline
\sphinxAtStartPar
Investigate physical systems
&
\sphinxAtStartPar
0.30
\\
\hline
\sphinxAtStartPar
Construct and document a reproducible process
&
\sphinxAtStartPar
0.10
\\
\hline
\sphinxAtStartPar
Use analytical, computational, and graphical approaches
&
\sphinxAtStartPar
0.30
\\
\hline
\sphinxAtStartPar
Provide evidence of the quality of their work
&
\sphinxAtStartPar
0.10
\\
\hline
\sphinxAtStartPar
Collaborate effectively
&
\sphinxAtStartPar
0.20
\\
\hline
\end{tabulary}
\par
\sphinxattableend\end{savenotes}


\subsubsection{Goal: Investigate physical systems (0.30)}
\label{\detokenize{content/0_course/4_rubric:goal-investigate-physical-systems-0-30}}\begin{itemize}
\item {} 
\sphinxAtStartPar
How well does your computational essay predict or explain the system of interest?

\item {} 
\sphinxAtStartPar
How well does your computational essay allow the user to explore and investigate the system?

\end{itemize}


\subsubsection{Goal: Construct and document a reproducible process (0.10)}
\label{\detokenize{content/0_course/4_rubric:goal-construct-and-document-a-reproducible-process-0-10}}\begin{itemize}
\item {} 
\sphinxAtStartPar
How well does your computational essay reproduce your results and claims?

\item {} 
\sphinxAtStartPar
How  well documented is your computational essay?

\end{itemize}


\subsubsection{Goal: Use analytical, computational, and graphical approaches (0.30)}
\label{\detokenize{content/0_course/4_rubric:goal-use-analytical-computational-and-graphical-approaches-0-30}}\begin{itemize}
\item {} 
\sphinxAtStartPar
How well does your computational essay document your assumptions?

\item {} 
\sphinxAtStartPar
How well does your computational essay produce an understandable and parsimonious model?

\item {} 
\sphinxAtStartPar
How well does your computational essay explain the limitations of your analysis?

\end{itemize}


\subsubsection{Goal: Provide evidence of the quality of their work}
\label{\detokenize{content/0_course/4_rubric:goal-provide-evidence-of-the-quality-of-their-work}}\begin{itemize}
\item {} 
\sphinxAtStartPar
How well does your computational essay present  the case for its claims?

\item {} 
\sphinxAtStartPar
How well validated  is your model?

\end{itemize}


\subsubsection{Goal: Collaborate effectively}
\label{\detokenize{content/0_course/4_rubric:goal-collaborate-effectively}}\begin{itemize}
\item {} 
\sphinxAtStartPar
How well did you share  in the class’s knowledge?
\begin{itemize}
\item {} 
\sphinxAtStartPar
How well is that documented in your computational essay?

\end{itemize}

\item {} 
\sphinxAtStartPar
How well did you work with your partner ? \sphinxstyleemphasis{For those choosing to do so}

\end{itemize}

\sphinxstepscope


\section{Calendar}
\label{\detokenize{content/0_course/5_calendar:calendar}}\label{\detokenize{content/0_course/5_calendar::doc}}
\sphinxAtStartPar
In this course, we will cover four principal topics in physics (in this order):
\begin{enumerate}
\sphinxsetlistlabels{\arabic}{enumi}{enumii}{}{.}%
\item {} 
\sphinxAtStartPar
Classical Mechanics and Ordinary Differential Equations

\item {} 
\sphinxAtStartPar
Electromagnetism and Partial Differential Equations

\item {} 
\sphinxAtStartPar
Waves and Fourier Analysis

\item {} 
\sphinxAtStartPar
Statistical Mechanics and Monte Carlo models

\end{enumerate}

\sphinxAtStartPar
The class is roughly broken into equal parts for each section. Below we indicate how each week will be spent. This is subject to change based on your input and feedback. But roughly you can expect the following:


\subsection{Classical Mechanics and Ordinary Differential Equations}
\label{\detokenize{content/0_course/5_calendar:classical-mechanics-and-ordinary-differential-equations}}\begin{itemize}
\item {} 
\sphinxAtStartPar
Week 1 \sphinxhyphen{} Introduction to the Course, Python, and Jupyter notebooks; Modeling; Coordinate Systems and Frames

\item {} 
\sphinxAtStartPar
Week 2 \sphinxhyphen{} Calculus of Variations; Lagrangian Mechanics; Equations of Motion; Numerical Integration and Trajectories
\begin{itemize}
\item {} 
\sphinxAtStartPar
\sphinxstylestrong{Worked Problem 1 Assigned; due end of Week 3; 9/15}

\end{itemize}

\item {} 
\sphinxAtStartPar
Week 3 \sphinxhyphen{} Dynamical Systems; Phase Space; Stability
\begin{itemize}
\item {} 
\sphinxAtStartPar
\sphinxstylestrong{Worked Problem 2 Assigned; due end of Week 4; 9/22}

\end{itemize}

\item {} 
\sphinxAtStartPar
Week 4 \sphinxhyphen{} Chaos; Dynamical Analysis of ODEs

\item {} 
\sphinxAtStartPar
\sphinxstylestrong{Project 1 (CM and ODEs) Due 9/29}

\end{itemize}


\subsection{Electromagnetism and Partial Differential Equations}
\label{\detokenize{content/0_course/5_calendar:electromagnetism-and-partial-differential-equations}}\begin{itemize}
\item {} 
\sphinxAtStartPar
Week 5 \sphinxhyphen{} Introduction to Electrostatics; Vector fields; Numerical superposition; Laplace’s Equation; Separation of Variables
\begin{itemize}
\item {} 
\sphinxAtStartPar
\sphinxstylestrong{Worked Problem 3 Assigned; due end of Week 6; 10/6}

\end{itemize}

\item {} 
\sphinxAtStartPar
Week 6 \sphinxhyphen{} 2D Partial Differential Equations; Boundary Conditions; Numerical Solutions; Method of Relaxation
\begin{itemize}
\item {} 
\sphinxAtStartPar
\sphinxstylestrong{Worked Problem 4 Assigned; due end of Week 7; 10/13}

\end{itemize}

\item {} 
\sphinxAtStartPar
Week  7 \sphinxhyphen{} Magnetic Fields; Maxwell’s Equations; Vector Calculus; EM Wave Equation

\item {} 
\sphinxAtStartPar
\sphinxstylestrong{Project 2 (EM and PDEs) Due 10/20}

\end{itemize}


\subsection{Waves and Fourier Analysis}
\label{\detokenize{content/0_course/5_calendar:waves-and-fourier-analysis}}\begin{itemize}
\item {} 
\sphinxAtStartPar
Week 8 \sphinxhyphen{} Introduction to Waves; Normal Modes; Matrix Methods; Beats and Superposition
\begin{itemize}
\item {} 
\sphinxAtStartPar
\sphinxstylestrong{Worked Problem 5 Assigned; due end of Week 9; 10/27}

\end{itemize}

\item {} 
\sphinxAtStartPar
Week 9 \sphinxhyphen{} Wave formulations; Complex analysis; the Fourier Transform
\begin{itemize}
\item {} 
\sphinxAtStartPar
\sphinxstylestrong{Worked Problem 6 Assigned; due end of Week 10; 11/3}

\end{itemize}

\item {} 
\sphinxAtStartPar
Week 10 \sphinxhyphen{} Discrete Fourier Transform; Fast Fourier Transform; Applications of Fourier Analysis

\item {} 
\sphinxAtStartPar
\sphinxstylestrong{Project 3 (Waves and Fourier) Due 11/10}

\end{itemize}


\subsection{Statistical Mechanics and Monte Carlo models}
\label{\detokenize{content/0_course/5_calendar:statistical-mechanics-and-monte-carlo-models}}\begin{itemize}
\item {} 
\sphinxAtStartPar
Week 11 \sphinxhyphen{} Distributions; Counting and Probability; Boltzmann Distribution
\begin{itemize}
\item {} 
\sphinxAtStartPar
\sphinxstylestrong{Worked Problem 7 Assigned; due end of Week 12; 11/17}

\end{itemize}

\item {} 
\sphinxAtStartPar
Week 12 \sphinxhyphen{} Statistical Mechanics; The Ideal Gas; Monte Carlo methods
\begin{itemize}
\item {} 
\sphinxAtStartPar
\sphinxstylestrong{Worked Problem 8 Assigned; due end of Week 14; 12/1}

\end{itemize}

\item {} 
\sphinxAtStartPar
Week 13 \sphinxhyphen{} Thanksgiving break

\item {} 
\sphinxAtStartPar
Week 14 \sphinxhyphen{} Markov Chains; Metropolis Algorithm; Applications

\item {} 
\sphinxAtStartPar
\sphinxstylestrong{Project 4 (Distributions and Monte Carlo) Due 12/8}

\end{itemize}

\sphinxstepscope


\section{Classroom Environment}
\label{\detokenize{content/0_course/6_environment:classroom-environment}}\label{\detokenize{content/0_course/6_environment::doc}}

\subsection{Commitment to an Inclusive Classroom}
\label{\detokenize{content/0_course/6_environment:commitment-to-an-inclusive-classroom}}
\sphinxAtStartPar
I am deeply committed to creating an inclusive classroom \sphinxhyphen{} one where you and your classmates
feel comfortable, intellectually challenged, and able to speak up about your ideas
and experiences. This means that our classroom, our virtual environments, and our interactions
need to be as inclusive as possible. Mutual respect, civility, and the ability to listen
and observe others are central to creating a classroom that is inclusive. I will strive to
do this and I ask that you do the same. If I can do anything to make the classroom a better
learning environment for you, please let me know.

\sphinxAtStartPar
\sphinxstylestrong{If you observe or experience behaviors that violate our commitment to inclusivity,
please let me know as soon as possible.}

\sphinxAtStartPar
If I violate this principle, please let me know or please tell the undergraduate department chair, Stuart Tessmer (\sphinxhref{mailto:tessmer@pa.msu.edu}{tessmer@pa.msu.edu}), who I have informed to tell me about any such incidents without conveying student information to me.


\subsection{Comments on preparation:}
\label{\detokenize{content/0_course/6_environment:comments-on-preparation}}
\sphinxAtStartPar
Physics 415 covers material you might have seen before. Many of the topics
stem from a wide variety of physics courses you might have already taken. But, we might be applying them at a higher level of conceptual and mathematical sophistication.

\sphinxAtStartPar
Therefore you should expect:
\begin{itemize}
\item {} 
\sphinxAtStartPar
a large amount of material to review and digest.

\item {} 
\sphinxAtStartPar
no recitations, and few examples covered in lecture. Most of the learning will be done through projects and questions you and your group mates raise.

\item {} 
\sphinxAtStartPar
long, hard problems that usually cannot be completed by one individual alone.

\item {} 
\sphinxAtStartPar
challenging projects.

\item {} 
\sphinxAtStartPar
to learn more about being a physicist that you have in another class (I hope!).

\end{itemize}

\sphinxAtStartPar
Physics 415 is a challenging, upper‐division physics course. Unlike more introductory courses, you are fully responsible for your own learning. In particular, you control the pace of the course by asking questions in class. I tend to speak quickly, and questions are important to slow down. This means that if you don’t understand something, it is your responsibility to ask questions. Attending class and the help sessions gives you an opportunity to ask questions. I am here to help you as much as possible, but I need your questions to know what you don’t understand.

\sphinxAtStartPar
Physics 415 covers some of the most important physics and mathematical methods in the field. Your reward for the hard work and effort will be learning important and elegant material that you will use over and over as a physics major. Here is what I have experienced, and heard from
other faculty teaching upper division physics in the past:
\begin{itemize}
\item {} 
\sphinxAtStartPar
most students reported spending a minimum of 10 hours per week on the
homework (!!)

\item {} 
\sphinxAtStartPar
students who didn’t attend the help sessions
often did poorly in the class.

\item {} 
\sphinxAtStartPar
students reported learning a tremendous amount in this class.

\end{itemize}

\sphinxAtStartPar
\sphinxstylestrong{The course topics that we will cover in Physics 415 are among the
greatest intellectual achievements of humans. Don’t be surprised if you
have to think hard and work hard to master the material.}


\subsection{Use of Generative AI Tools}
\label{\detokenize{content/0_course/6_environment:use-of-generative-ai-tools}}
\sphinxAtStartPar
You are welcome to use generative AI tools (e.g. ChatGPT, Dall\sphinxhyphen{}e, etc.) in this class as doing so aligns with the course learning goals. These tools can be useful in gathering information, troubleshooting code, and developing potential directions. However, you are responsible for the information you submit based on an AI query (for instance, that it does not violate intellectual property laws, or contain misinformation or unethical content). Your use of AI tools must be properly documented and cited in order to stay within university policies on academic integrity and the \sphinxhref{https://spartanexperiences.msu.edu/about/handbook/spartan-code-of-honor-academic-pledge/index.html}{Spartan Code of Honor Academic Pledge}.

\sphinxAtStartPar
For example, if generative AI is used to develop code or make sense of results, the original query, the resulting text, and your discussion of how that information was synthesized and used is \sphinxstylestrong{required} to be submitted with your work. This can be in an appendix if it distracts from the presentation of your work. Remember, AI is not likely to generate a response that would be seen as quality work and should be modified and improved. AI cannot think critically, so you must do that work. More details or the rationale for this policy can be found here: \sphinxurl{https://msu-cmse-courses.github.io/cmse202-F23-jb/course\_materials/CMSE202\_GenerativeAI\_Policy.html}

\sphinxstepscope


\section{Resources}
\label{\detokenize{content/0_course/7_resources:resources}}\label{\detokenize{content/0_course/7_resources::doc}}

\subsection{Confidentiality and Mandatory Reporting}
\label{\detokenize{content/0_course/7_resources:confidentiality-and-mandatory-reporting}}
\sphinxAtStartPar
College students often experience issues that may interfere with academic success such as academic stress, sleep problems, juggling responsibilities, life events, relationship concerns, or feelings of anxiety, hopelessness, or depression.
As your instructor, one of my responsibilities is to help create a safe learning environment and to support you through these situations and experiences.
I also have a mandatory reporting responsibility related to my role as a University employee.
It is my goal that you feel able to share information related to your life experiences in classroom
discussions, in written work, and in one\sphinxhyphen{}on\sphinxhyphen{}one meetings.
I will seek to keep information you share private to the greatest extent possible.
However, under Title IX, I am required to share information regarding sexual misconduct, relationship violence, or information
about criminal activity on MSU’s campus with the University including the Office of Institutional Equity (OIE).

\sphinxAtStartPar
\sphinxstylestrong{Students may speak to someone confidentially by contacting MSU Counseling and Psychiatric Service (CAPS) (\sphinxhref{http://caps.msu.edu}{caps.msu.edu}, 517\sphinxhyphen{}355\sphinxhyphen{}8270), MSU’s 24\sphinxhyphen{}hour Sexual Assault Crisis Line (\sphinxhref{http://endrape.msu.edu}{endrape.msu.edu}, 517\sphinxhyphen{}372\sphinxhyphen{}6666), or Olin Health Center (\sphinxhref{http://olin.msu.edu}{olin.msu.edu}, 517\sphinxhyphen{}884\sphinxhyphen{}6546).}


\subsection{Spartan Code of Honor Academic Pledge}
\label{\detokenize{content/0_course/7_resources:spartan-code-of-honor-academic-pledge}}
\sphinxAtStartPar
As a Spartan, I will strive to uphold values of the highest ethical standard. I will practice honesty in my work, foster honesty in my peers, and take pride in knowing that honor is worth more than grades. I will carry these values beyond my time as a student at Michigan State University, continuing the endeavor to build personal integrity in all that I do.


\subsection{Handling Emergency Situations}
\label{\detokenize{content/0_course/7_resources:handling-emergency-situations}}
\sphinxAtStartPar
\sphinxstyleemphasis{In the event of an emergency arising within the classroom, Prof. Caballero will notify you of what actions that may be required to ensure your safety. It is the responsibility of each student to understand the evacuation, “shelter\sphinxhyphen{}in\sphinxhyphen{}place,” and “secure\sphinxhyphen{}in\sphinxhyphen{}place” guidelines posted in each facility and to act in a safe manner. You are allowed to maintain cellular devices in a silent mode during this course, in order to receive emergency SMS text, phone or email messages distributed by the university. When anyone receives such a notification or observes an emergency situation, they should immediately bring it to the attention of Prof. Caballero in a way that causes the least disruption. If an evacuation is ordered, please ensure that you do it in a safe manner and facilitate those around you that may not otherwise be able to safely leave. When these orders are given, you do have the right as a member of this community to follow that order. Also, if a shelter\sphinxhyphen{}in\sphinxhyphen{}place or secure\sphinxhyphen{}in\sphinxhyphen{}place is ordered, please seek areas of refuge that are safe depending on the emergency encountered and provide assistance if it is advisable to do so.}

\sphinxstepscope


\part{1 \sphinxhyphen{} Mechanics and ODEs}

\sphinxstepscope


\chapter{What is Mathematical Modeling?}
\label{\detokenize{content/1_mechanics/modeling/what_is_modeling:what-is-mathematical-modeling}}\label{\detokenize{content/1_mechanics/modeling/what_is_modeling::doc}}
\sphinxAtStartPar
Nature reveals itself to us through interactions. We can tell from observations that it is nature’s interactions that lead to its evolution. How nature is changing and predicting how it will change in the future is the work of science. In this work, we observe nature and its interactions to make models of those observations. We aim to predict and explain our observations of nature through this building of models.

\sphinxAtStartPar
In physics, our goals are typically to explain and predict observations of physical phenomenon. Here, we focus ourselves to those canonical things that physicists concern themselves with: motion, fields, waves, atoms, nuclei, and so on.

\sphinxstepscope


\chapter{01 Sept 22 \sphinxhyphen{} Activity: What is a Model?}
\label{\detokenize{content/1_mechanics/modeling/activity-what_is_a_model:sept-22-activity-what-is-a-model}}\label{\detokenize{content/1_mechanics/modeling/activity-what_is_a_model::doc}}

\section{Notes}
\label{\detokenize{content/1_mechanics/modeling/activity-what_is_a_model:notes}}\begin{itemize}
\item {} 
\sphinxAtStartPar
\sphinxstyleemphasis{Class Meeting:} 01 Sept 22

\item {} 
\sphinxAtStartPar
\sphinxstyleemphasis{Prior Reading:} None necessary

\end{itemize}


\section{Models and modeling}
\label{\detokenize{content/1_mechanics/modeling/activity-what_is_a_model:models-and-modeling}}
\sphinxAtStartPar
This course focuses on common physical models and how we use mathematics and computing tools to investigate these models. But, we need to agree on an mutual understanding of a model. In this activity, you will be working in a group to define aspects of a good model for the physical sciences.

\sphinxAtStartPar
There are plenty of ideas on the internet, I can even \sphinxhref{https://bfy.tw/TUdA}{provide you a link} with plenty of information. But, the point of this activity is for us to define a good model now (given all your prior experiences with models and modeling), and to reflect on that definition as you develop expertise with models and modeling (through your learning in this class).


\section{What is a model?}
\label{\detokenize{content/1_mechanics/modeling/activity-what_is_a_model:what-is-a-model}}
\sphinxAtStartPar
We are going to start with this short video made by Geoscientist John Aiken when he was a graduate student at Georgia Tech.

\begin{sphinxuseclass}{cell}\begin{sphinxVerbatimInput}

\begin{sphinxuseclass}{cell_input}
\begin{sphinxVerbatim}[commandchars=\\\{\}]
\PYG{k+kn}{from} \PYG{n+nn}{IPython}\PYG{n+nn}{.}\PYG{n+nn}{display} \PYG{k+kn}{import} \PYG{n}{YouTubeVideo}
\PYG{n}{YouTubeVideo}\PYG{p}{(}\PYG{l+s+s2}{\PYGZdq{}}\PYG{l+s+s2}{dkTncoPqo5Y}\PYG{l+s+s2}{\PYGZdq{}}\PYG{p}{,} \PYG{n}{width} \PYG{o}{=} \PYG{l+m+mi}{800}\PYG{p}{,} \PYG{n}{height} \PYG{o}{=} \PYG{l+m+mi}{600}\PYG{p}{)}
\end{sphinxVerbatim}

\end{sphinxuseclass}\end{sphinxVerbatimInput}
\begin{sphinxVerbatimOutput}

\begin{sphinxuseclass}{cell_output}
\noindent\sphinxincludegraphics{{activity-what_is_a_model_2_0}.jpg}

\end{sphinxuseclass}\end{sphinxVerbatimOutput}

\end{sphinxuseclass}

\subsection{History and Philosophy of Science}
\label{\detokenize{content/1_mechanics/modeling/activity-what_is_a_model:history-and-philosophy-of-science}}
\sphinxAtStartPar
If you would like to dive deeper into models and modeling, there’s excellent work in history and philosophy of science. The field studies how science develops knowledge, practice, culture, and so on. It studies important events and provides critical information on important and, often, overlooked folks who do science. For example, historian and gender studies professor \sphinxhref{https://en.wikipedia.org/wiki/Sharon\_Traweek}{Sharon Traweek} studies the high energy physics field. Her book, \sphinxhref{https://en.wikipedia.org/wiki/Beamtimes\_and\_Lifetimes}{Beamtimes and Lifetimes: The World of High Energy Physicists} {[}\hyperlink{cite.jb_reference/markdown:id5}{Tra09}{]} is excellent.

\begin{sphinxadmonition}{note}{Dame Nancy Cartwright (philosopher of science)}

\sphinxAtStartPar
One of the more interesting scholars in \sphinxhref{https://en.wikipedia.org/wiki/Nancy\_Cartwright\_(philosopher)}{Dame Nancy Cartwright} who wrote a lot about the ‘practice of science.’ Her philosophical work informed many of the innovations in physics and broader science education – including many science courses at MSU.

\sphinxAtStartPar
Her writing is very interesting, but the style of writing can be a challenge to read. This is the nature of academic writing in different disciplines. Her book called “How The Laws of Physics Lie” {[}\hyperlink{cite.jb_reference/markdown:id6}{Car83}{]} is worth a read. Here’s a link to the \sphinxhref{http://www.generativescience.org/papers/nature/Cartwright-\_1983.pdf}{first chapter}.
\end{sphinxadmonition}


\section{What has been your experience with models?}
\label{\detokenize{content/1_mechanics/modeling/activity-what_is_a_model:what-has-been-your-experience-with-models}}
\sphinxAtStartPar
Take 2\sphinxhyphen{}3 minutes to think about your prior physics class.
\begin{itemize}
\item {} 
\sphinxAtStartPar
What models have you used? What makes that a model?

\item {} 
\sphinxAtStartPar
What modeling have you done? What makes that modeling?

\end{itemize}

\begin{sphinxuseclass}{cell}\begin{sphinxVerbatimInput}

\begin{sphinxuseclass}{cell_input}
\begin{sphinxVerbatim}[commandchars=\\\{\}]
\PYG{c+c1}{\PYGZsh{}\PYGZsh{} Take notes for yourself here}

\PYG{c+c1}{\PYGZsh{} Idea 1}
\PYG{c+c1}{\PYGZsh{} Idea 2}
\end{sphinxVerbatim}

\end{sphinxuseclass}\end{sphinxVerbatimInput}

\end{sphinxuseclass}
\sphinxAtStartPar
After we have discussed the above questions and generated a long list of experiences, you can move onto the next set of questions below.

\begin{sphinxuseclass}{toggle}
\sphinxAtStartPar
Take 2\sphinxhyphen{}3 minutes to think about your prior work with models and modeling?
\begin{itemize}
\item {} 
\sphinxAtStartPar
What made a model good or not so good?

\item {} 
\sphinxAtStartPar
What kinds of things could you do to make a better model?

\end{itemize}

\end{sphinxuseclass}
\begin{sphinxuseclass}{cell}\begin{sphinxVerbatimInput}

\begin{sphinxuseclass}{cell_input}
\begin{sphinxVerbatim}[commandchars=\\\{\}]
\PYG{c+c1}{\PYGZsh{}\PYGZsh{} Take notes for yourself here}

\PYG{c+c1}{\PYGZsh{} Idea 1}
\PYG{c+c1}{\PYGZsh{} Idea 2}
\end{sphinxVerbatim}

\end{sphinxuseclass}\end{sphinxVerbatimInput}

\end{sphinxuseclass}
\sphinxAtStartPar
After we have discussed the above questions and generated a consensus list of good model features, you can move onto the next set of questions below.

\begin{sphinxuseclass}{toggle}
\sphinxAtStartPar
Take 2\sphinxhyphen{}3 minutes and write down all the mathematical and/or computing models you have experienced, heard about, seen, or are otherwise familiar.

\end{sphinxuseclass}
\begin{sphinxuseclass}{cell}\begin{sphinxVerbatimInput}

\begin{sphinxuseclass}{cell_input}
\begin{sphinxVerbatim}[commandchars=\\\{\}]
\PYG{c+c1}{\PYGZsh{}\PYGZsh{} Take notes for yourself here (use comments)}

\PYG{c+c1}{\PYGZsh{} Model 1}
\PYG{c+c1}{\PYGZsh{} Model 2}
\end{sphinxVerbatim}

\end{sphinxuseclass}\end{sphinxVerbatimInput}

\end{sphinxuseclass}
\sphinxstepscope


\chapter{Intro to Classical Mechanics}
\label{\detokenize{content/1_mechanics/mechanics_intro:intro-to-classical-mechanics}}\label{\detokenize{content/1_mechanics/mechanics_intro::doc}}
\sphinxAtStartPar
Welcome to Mathematical Modeling in physics! Our first unit will focus on Classical Mechanics and Ordinary Differential Equations (ODEs). Classical Mechanics is all about the motion of macrosopic objects, typically ones that are moving slow enough so that we can ignore special relativity. In particular, we are interested in systems that obey Newton’s second law:
\begin{equation*}
\begin{split}\mathbf{F} = \frac{d\mathbf{p}}{dt} = \dot{\mathbf{p}}\end{split}
\end{equation*}
\sphinxAtStartPar
or equivalently (if m is constant):
\begin{equation*}
\begin{split}\mathbf{F} = m \frac{d\mathbf{r}}{dt^2} = m\ddot{\mathbf{r}}\end{split}
\end{equation*}
\sphinxAtStartPar
Here we’re using dot notation to mean derivatives with repsect to time. We’ll continue to see this notation through this course.

\sphinxAtStartPar
The key thing to note here is that both of these equations are statements of \sphinxstylestrong{ordinary differential equations}. A huge portions of problems in physics boil down to differential equations, its probably more difficult to think of physics problems that aren’t differential equations than ones that are. Broadly speaking, differential equations are equations of some \sphinxstylestrong{unkown} function and its derivatives. Often we are concerned with the form and/or behavior of this unknown function. In terms of newtons laws, if we are able to solve the differential equation \(\mathbf{F} = m\ddot{\mathbf{r}}\), then we get some function \(\mathbf{r}(t)\) that tells us exactly how our system evolves in time.


\section{Interpreting the statement of ODEs}
\label{\detokenize{content/1_mechanics/mechanics_intro:interpreting-the-statement-of-odes}}
\sphinxAtStartPar
A nice way to think of differential equations is as a set of instructions for how a function should change in time. For example, consider the first order ODE:
\begin{equation*}
\begin{split}
\dot{x} = x
\end{split}
\end{equation*}
\sphinxAtStartPar
This equation is saying that the function \(x(t)\) should change according to what its current value is. We can also think of \(\dot{x}\) as the velocity of function \(x(t)\), so this equation is also saying that the velocity of \(x(t)\) needs to be equal to its current position at all times, or that when we take a derivative of this function we get itself back. The function \(e^t\) has this property, so it might be our go\sphinxhyphen{}to guess for the solution of this system. Since this equation is \sphinxhref{https://math.libretexts.org/Courses/Monroe\_Community\_College/MTH\_211\_Calculus\_II/Chapter\_8\%3A\_Introduction\_to\_Differential\_Equations/8.3\%3A\_Separable\_Differential\_Equations}{seperable}, we can solve it exactly by integrating:
\begin{equation*}
\begin{split}
\int \frac{dx}{x} = \int dt \implies \log(x(t)) = t + c
\end{split}
\end{equation*}\begin{equation*}
\begin{split}
\implies x(t) = e^{(x + c)} = Ae^t \text{ if we let } A = e^c
\end{split}
\end{equation*}
\sphinxAtStartPar
Our guess at a solution was close, but the actual solution ended up with an extra \(A\) term. In fact we’ve found \sphinxstylestrong{infinitley many} solutions, or the \sphinxstylestrong{general solution} since we don’t know the value of \(A\) that came from the integration constant.  Think for second about how we might go about finding what \(A\) is.


\section{Initial Conditions}
\label{\detokenize{content/1_mechanics/mechanics_intro:initial-conditions}}
\sphinxAtStartPar
To find \(A\), we would need to know what \(x(t=0)\) is. Let’s say for the particular solution we’re interested in has \(x(t=0) = x_0\). Then its straightforward to solve for \(A\):
\begin{equation*}
\begin{split}
x(t=0) = x_0 = Ae^0 = A \implies A = x_0 \implies x(t) = x_0e^t
\end{split}
\end{equation*}
\sphinxAtStartPar
So we’ve found a \sphinxstylestrong{specific} or \sphinxstylestrong{unique} solution to this ODE. In general, for an \(n\) th\sphinxhyphen{}order (the highest order is of \(n\) th degree) ODE, you need \(n\) iniial conditions to find a specific solution. We’ll see why this is in the coming weeks.


\section{A note on differentiability}
\label{\detokenize{content/1_mechanics/mechanics_intro:a-note-on-differentiability}}
\sphinxAtStartPar
When we are concerned with differential equations that show up in classical mechanics, we often secretly make the assumption that the function that we are looking for is differentiable in the first place, i.e. that the function is sufficiently smooth so that \(\lim_{t\to 0} \frac{\Delta \mathbf{r}}{\Delta t}\) exists. For the systems we’ll concern ourselves with in this class, we will take this for granted. Some differential equation models do run into this being an issue though, such as in fracture mechanics or models of swarming behavior, where one needs to employ what is called nonlocal analysis, which is a really interesting bit of math that we just don’t have the time to cover in this class sadly.

\sphinxAtStartPar
But before we can start modeling more or less whatever we’d like with ODEs, we need to get familiar with some different frames of reference, or coordinate systems that solutions to ODEs often live in. That leads us to our \sphinxhref{https://valentine-alia.github.io/phy415fall23/content/1\_mechanics/frames.html}{first in\sphinxhyphen{}class activity}.







\renewcommand{\indexname}{Index}
\printindex
\end{document}