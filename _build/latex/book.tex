%% Generated by Sphinx.
\def\sphinxdocclass{jupyterBook}
\documentclass[letterpaper,10pt,english]{jupyterBook}
\ifdefined\pdfpxdimen
   \let\sphinxpxdimen\pdfpxdimen\else\newdimen\sphinxpxdimen
\fi \sphinxpxdimen=.75bp\relax
\ifdefined\pdfimageresolution
    \pdfimageresolution= \numexpr \dimexpr1in\relax/\sphinxpxdimen\relax
\fi
%% let collapsible pdf bookmarks panel have high depth per default
\PassOptionsToPackage{bookmarksdepth=5}{hyperref}
%% turn off hyperref patch of \index as sphinx.xdy xindy module takes care of
%% suitable \hyperpage mark-up, working around hyperref-xindy incompatibility
\PassOptionsToPackage{hyperindex=false}{hyperref}
%% memoir class requires extra handling
\makeatletter\@ifclassloaded{memoir}
{\ifdefined\memhyperindexfalse\memhyperindexfalse\fi}{}\makeatother

\PassOptionsToPackage{warn}{textcomp}

\catcode`^^^^00a0\active\protected\def^^^^00a0{\leavevmode\nobreak\ }
\usepackage{cmap}
\usepackage{fontspec}
\defaultfontfeatures[\rmfamily,\sffamily,\ttfamily]{}
\usepackage{amsmath,amssymb,amstext}
\usepackage{polyglossia}
\setmainlanguage{english}



\setmainfont{FreeSerif}[
  Extension      = .otf,
  UprightFont    = *,
  ItalicFont     = *Italic,
  BoldFont       = *Bold,
  BoldItalicFont = *BoldItalic
]
\setsansfont{FreeSans}[
  Extension      = .otf,
  UprightFont    = *,
  ItalicFont     = *Oblique,
  BoldFont       = *Bold,
  BoldItalicFont = *BoldOblique,
]
\setmonofont{FreeMono}[
  Extension      = .otf,
  UprightFont    = *,
  ItalicFont     = *Oblique,
  BoldFont       = *Bold,
  BoldItalicFont = *BoldOblique,
]



\usepackage[Bjarne]{fncychap}
\usepackage[,numfigreset=1,mathnumfig]{sphinx}

\fvset{fontsize=\small}
\usepackage{geometry}


% Include hyperref last.
\usepackage{hyperref}
% Fix anchor placement for figures with captions.
\usepackage{hypcap}% it must be loaded after hyperref.
% Set up styles of URL: it should be placed after hyperref.
\urlstyle{same}

\addto\captionsenglish{\renewcommand{\contentsname}{0 - Intro and Syllabus}}

\usepackage{sphinxmessages}



        % Start of preamble defined in sphinx-jupyterbook-latex %
         \usepackage[Latin,Greek]{ucharclasses}
        \usepackage{unicode-math}
        % fixing title of the toc
        \addto\captionsenglish{\renewcommand{\contentsname}{Contents}}
        \hypersetup{
            pdfencoding=auto,
            psdextra
        }
        % End of preamble defined in sphinx-jupyterbook-latex %
        

\title{PHY 415 Fall 2023}
\date{Aug 21, 2023}
\release{}
\author{Danny Caballero, Alia Valentine}
\newcommand{\sphinxlogo}{\vbox{}}
\renewcommand{\releasename}{}
\makeindex
\begin{document}

\pagestyle{empty}
\sphinxmaketitle
\pagestyle{plain}
\sphinxtableofcontents
\pagestyle{normal}
\phantomsection\label{\detokenize{content/intro::doc}}


\sphinxAtStartPar
PHY 415, called, “Mathematical Methods for Physicists” is a course the brings together many of the mathematical approaches that we commonly use in physics and apply them to variety of problems. In this course, we will take a modeling\sphinxhyphen{}based approach where we focus on the mathematical descriptions of physical phenomenon and determine what mathematical and analytical approaches are useful in exploring those models.

\sphinxAtStartPar
To get a sense of the course, please read all the pages associated with our syllabus.

\begin{DUlineblock}{0em}
\item[] \sphinxstylestrong{\Large Learning Objectives}
\end{DUlineblock}

\sphinxAtStartPar
In this course, you will learn to:
\begin{itemize}
\item {} 
\sphinxAtStartPar
investigate physical systems using a variety of tools and approaches,

\item {} 
\sphinxAtStartPar
construct and document a reproducible process for those investigations,

\item {} 
\sphinxAtStartPar
use analytical, computational, and graphical approaches to answer specific questions in those investigations,

\item {} 
\sphinxAtStartPar
provide evidence of the quality of work using a variety of sources, and

\item {} 
\sphinxAtStartPar
collaborate effectively and contribute to a inclusive learning environment

\end{itemize}

\sphinxAtStartPar
Table of contents:
\begin{itemize}
\item {} 
\sphinxAtStartPar
0 \sphinxhyphen{} Intro and Syllabus

\begin{itemize}
\item {} 
\sphinxAtStartPar
{\hyperref[\detokenize{content/0_course/syllabus::doc}]{\sphinxcrossref{Syllabus and Overview of PHY 415}}}

\end{itemize}
\end{itemize}
\begin{itemize}
\item {} 
\sphinxAtStartPar
1 \sphinxhyphen{} Mechanics and ODEs

\begin{itemize}
\item {} 
\sphinxAtStartPar
{\hyperref[\detokenize{content/1_mechanics/mechanics_intro::doc}]{\sphinxcrossref{Intro to Classical Mechanics}}}

\item {} 
\sphinxAtStartPar
{\hyperref[\detokenize{content/1_mechanics/frames::doc}]{\sphinxcrossref{31 Aug 23 \sphinxhyphen{} Frames and Coordinates}}}

\item {} 
\sphinxAtStartPar
{\hyperref[\detokenize{content/1_mechanics/lagrange_1::doc}]{\sphinxcrossref{5 Sep 23 \sphinxhyphen{} Calculus of Varations and Lagrangian Dynamics}}}

\item {} 
\sphinxAtStartPar
{\hyperref[\detokenize{content/1_mechanics/lagrange_2::doc}]{\sphinxcrossref{7 Sep 23 \sphinxhyphen{} Numerical Integration and More Lagrangians}}}

\item {} 
\sphinxAtStartPar
{\hyperref[\detokenize{content/1_mechanics/dynamical_1::doc}]{\sphinxcrossref{12 Sep 23 \sphinxhyphen{} The Dynamical Systems Approach and Phase Portraits}}}

\item {} 
\sphinxAtStartPar
{\hyperref[\detokenize{content/1_mechanics/dynamical_2::doc}]{\sphinxcrossref{14 Sep 23 \sphinxhyphen{} Dynamical Systems Continued}}}

\item {} 
\sphinxAtStartPar
{\hyperref[\detokenize{content/1_mechanics/CHAOS::doc}]{\sphinxcrossref{19 Sep 23 \sphinxhyphen{} CHAOS}}}

\item {} 
\sphinxAtStartPar
{\hyperref[\detokenize{content/1_mechanics/ODE_games::doc}]{\sphinxcrossref{21 Sept 23 \sphinxhyphen{} Activity: ODE Games}}}

\end{itemize}
\end{itemize}
\begin{itemize}
\item {} 
\sphinxAtStartPar
2 \sphinxhyphen{} E\&M and PDEs

\begin{itemize}
\item {} 
\sphinxAtStartPar
{\hyperref[\detokenize{content/2_EM/EM_intro::doc}]{\sphinxcrossref{Intro to Electricity and Magnetism}}}

\item {} 
\sphinxAtStartPar
{\hyperref[\detokenize{content/2_EM/E_fields_graphing_184::doc}]{\sphinxcrossref{26 Sep 23 \sphinxhyphen{} Graphing Electric Fields}}}

\item {} 
\sphinxAtStartPar
{\hyperref[\detokenize{content/2_EM/laplace_eq::doc}]{\sphinxcrossref{28 Sep 23 \sphinxhyphen{} Laplace’s Equation}}}

\item {} 
\sphinxAtStartPar
{\hyperref[\detokenize{content/2_EM/more_PDEs::doc}]{\sphinxcrossref{3 Oct 23 \sphinxhyphen{} More PDEs}}}

\item {} 
\sphinxAtStartPar
{\hyperref[\detokenize{content/2_EM/relaxation::doc}]{\sphinxcrossref{5 sep 23 \sphinxhyphen{} Method of Relaxation}}}

\item {} 
\sphinxAtStartPar
{\hyperref[\detokenize{content/2_EM/B_fields::doc}]{\sphinxcrossref{10 Oct 23 \sphinxhyphen{} Magnetic Fields}}}

\item {} 
\sphinxAtStartPar
{\hyperref[\detokenize{content/2_EM/EM_waves::doc}]{\sphinxcrossref{Electromagnetic Waves \& the Wave Equation}}}

\end{itemize}
\end{itemize}
\begin{itemize}
\item {} 
\sphinxAtStartPar
3 \sphinxhyphen{} Waves and Complex Analysis

\begin{itemize}
\item {} 
\sphinxAtStartPar
{\hyperref[\detokenize{content/3_waves/waves_intro::doc}]{\sphinxcrossref{Intro to Waves}}}

\item {} 
\sphinxAtStartPar
{\hyperref[\detokenize{content/3_waves/normal_modes::doc}]{\sphinxcrossref{17 Oct 23 \sphinxhyphen{} Normal Modes}}}

\item {} 
\sphinxAtStartPar
{\hyperref[\detokenize{content/3_waves/mech_waves::doc}]{\sphinxcrossref{19 Oct 23 \sphinxhyphen{} Mechanical Waves}}}

\item {} 
\sphinxAtStartPar
{\hyperref[\detokenize{content/3_waves/wave_examples::doc}]{\sphinxcrossref{26 Oct 23 \sphinxhyphen{} Examples of Waves}}}

\item {} 
\sphinxAtStartPar
{\hyperref[\detokenize{content/3_waves/complex::doc}]{\sphinxcrossref{31 Oct 23 \sphinxhyphen{} Complex Analysis \& the Fourier Transform}}}

\item {} 
\sphinxAtStartPar
{\hyperref[\detokenize{content/3_waves/DFT_FFT::doc}]{\sphinxcrossref{2 Nov 23 \sphinxhyphen{} Discrete and Fast Fourier Transforms}}}

\end{itemize}
\end{itemize}
\begin{itemize}
\item {} 
\sphinxAtStartPar
4 \sphinxhyphen{} Random Processes and Distributions

\begin{itemize}
\item {} 
\sphinxAtStartPar
{\hyperref[\detokenize{content/4_distributions/distributions_intro::doc}]{\sphinxcrossref{Intro to Distributions}}}

\end{itemize}
\end{itemize}

\sphinxstepscope


\part{0 \sphinxhyphen{} Intro and Syllabus}

\sphinxstepscope


\chapter{Syllabus and Overview of PHY 415}
\label{\detokenize{content/0_course/syllabus:syllabus-and-overview-of-phy-415}}\label{\detokenize{content/0_course/syllabus::doc}}
\sphinxAtStartPar
In designing this course, I plan to emphasize more independent learning on your part and greater agency for you in determining what you learn and how you demonstrate you have learned. So you should expect:
\begin{itemize}
\item {} 
\sphinxAtStartPar
to read a variety of pieces of information to coordinate information

\item {} 
\sphinxAtStartPar
to present your ideas publicly and to discuss them

\item {} 
\sphinxAtStartPar
to learn new approaches and novel techniques on your own

\item {} 
\sphinxAtStartPar
to become more expert than me in the areas of your interest

\item {} 
\sphinxAtStartPar
to learn more about scientists that you have not learned about

\end{itemize}

\sphinxAtStartPar
This is not to say that you are on your own. Here’s what you can expect from me:
\begin{itemize}
\item {} 
\sphinxAtStartPar
resources, information, and tools to help you learn

\item {} 
\sphinxAtStartPar
support and scaffolding to move you towards more independence in your learning

\item {} 
\sphinxAtStartPar
timely and detailed feedback to help you along

\item {} 
\sphinxAtStartPar
a commitment to an inclusive classroom

\end{itemize}


\section{Contact Information}
\label{\detokenize{content/0_course/syllabus:contact-information}}

\subsection{Web page}
\label{\detokenize{content/0_course/syllabus:web-page}}\begin{itemize}
\item {} 
\sphinxAtStartPar
Web page for this class:
\sphinxurl{https://valentine-alia.github.io/phy415fall23/content/intro}

\end{itemize}


\subsection{Instructor}
\label{\detokenize{content/0_course/syllabus:instructor}}\begin{itemize}
\item {} 
\sphinxAtStartPar
\sphinxhref{http://dannycab.github.io}{Prof. Danny Caballero} (he/him/his)

\item {} 
\sphinxAtStartPar
Class Meetings: Tuesdays and Thursdays 10:20am\sphinxhyphen{}12:10pm (Location: 1300 BPS)

\item {} 
\sphinxAtStartPar
Email: \sphinxhref{mailto:caball14@msu.edu}{caball14@msu.edu}, office: 1310\sphinxhyphen{}A BPS

\item {} 
\sphinxAtStartPar
Office hrs: To be scheduled, but I also have an open door policy. I enjoy visiting and talking with you about physics.

\end{itemize}


\subsection{Learning Assistant}
\label{\detokenize{content/0_course/syllabus:learning-assistant}}\begin{itemize}
\item {} 
\sphinxAtStartPar
\sphinxhref{https://valentine-alia.github.io/}{Alia Valentine} (she/her/hers)

\item {} 
\sphinxAtStartPar
Email: \sphinxhref{mailto:valen176@msu.edu}{valen176@msu.edu}

\item {} 
\sphinxAtStartPar
Office hrs: TBD, feel free to email me if you want to set up a time. I too enjoy visiting and talking with you about math and physics.

\end{itemize}


\section{Grading}
\label{\detokenize{content/0_course/syllabus:grading}}
\sphinxAtStartPar
Details about {\hyperref[\detokenize{content/0_course/design::doc}]{\sphinxcrossref{\DUrole{doc,std,std-doc}{course activities are here}}}} and {\hyperref[\detokenize{content/0_course/assessments::doc}]{\sphinxcrossref{\DUrole{doc,std,std-doc}{information regarding assessment is here}}}}. Your grade will be comprised of weekly discussion questions and several projects that you will complete in the form of a Jupyter notebook (a “computational essay”, which we will discuss later). Your grade on each project is split between completion (50\%) and quality (50\%). We will collectively define “quality” in class, but we have provided {\hyperref[\detokenize{content/0_course/rubric::doc}]{\sphinxcrossref{\DUrole{doc,std,std-doc}{a preliminary rubric}}}} for us to work from for the first project. Your final grade will be scaled based on your best performances; there will be slightly more projects than what comprises your grade.  \sphinxstyleemphasis{The intent here is to to allow you space to explore a model or project that you really enjoy, and to reward you for doing that.} How your grade is calculated appears below.


\begin{savenotes}\sphinxattablestart
\centering
\begin{tabulary}{\linewidth}[t]{|T|T|}
\hline
\sphinxstyletheadfamily 
\sphinxAtStartPar
Activity
&\sphinxstyletheadfamily 
\sphinxAtStartPar
Percent of Grade
\\
\hline
\sphinxAtStartPar
Best Project Grade
&
\sphinxAtStartPar
30\%
\\
\hline
\sphinxAtStartPar
2nd Best Project Grade
&
\sphinxAtStartPar
25\%
\\
\hline
\sphinxAtStartPar
3rd Best Project Grade
&
\sphinxAtStartPar
20\%
\\
\hline
\sphinxAtStartPar
4th Best Project Grade
&
\sphinxAtStartPar
10\%
\\
\hline
\sphinxAtStartPar
Weekly Discussion Questions (completion)
&
\sphinxAtStartPar
15\%
\\
\hline
\end{tabulary}
\par
\sphinxattableend\end{savenotes}

\sphinxAtStartPar
\sphinxstylestrong{While attendance is not required, you are unlikely to succeed with your projects without regular attendance and engagement.}

\sphinxstepscope


\section{Course Objectives}
\label{\detokenize{content/0_course/goals:course-objectives}}\label{\detokenize{content/0_course/goals::doc}}
\sphinxAtStartPar
This course emphasizes making models of physical phenomenon and how we use various tools at our disposal to investigate those models. Hence, we have learning objectives for making models of these systems and for learning specific tools.


\subsection{Investigate physical systems}
\label{\detokenize{content/0_course/goals:investigate-physical-systems}}
\sphinxAtStartPar
Clearly, one of our central goals is learning how to make models of physical systems. This means learning about and developing fluency with a wide variety of mathematical and computational tools. In this courses, we will make extensive use of \sphinxhref{http://anaconda.org}{Jupyter notebooks} for homework and projects. In fact, what you are reading is a set of Jupyter notebooks! Below, you will see the list of objectives for this principal objective.

\begin{sphinxadmonition}{note}{Investigating Physical Systems Learning Objectives}

\sphinxAtStartPar
Students will demonstrate they can:
\begin{itemize}
\item {} 
\sphinxAtStartPar
use mathematical techniques to predict or explain some physical phenomenon

\item {} 
\sphinxAtStartPar
employ computational models and algorithms to investigate physical systems

\item {} 
\sphinxAtStartPar
compare analytical and computational approaches to these investigations

\item {} 
\sphinxAtStartPar
provide coherent explanations for their investigations buttressed by physical, mathematical, and/or computational knowledge and principles

\end{itemize}
\end{sphinxadmonition}

\begin{sphinxadmonition}{note}{Principle Learning Objectives}

\sphinxAtStartPar
Students will demonstrate they can:
\begin{itemize}
\item {} 
\sphinxAtStartPar
investigate physical systems of their choosing using a variety of tools and approaches

\item {} 
\sphinxAtStartPar
construct and document a reproducible process for those investigations

\item {} 
\sphinxAtStartPar
use analytical, computational, and graphical approaches to answer specific questions in those investigations

\item {} 
\sphinxAtStartPar
provide evidence of the quality of their work using a variety of sources

\item {} 
\sphinxAtStartPar
collaborate effectively and contribute to a inclusive learning environment

\end{itemize}

\sphinxAtStartPar
Each of these learning objectives contributes to your development as a physicist. I recognize that these are \sphinxstylestrong{big} ideas to think about. What I mean is that the objectives above are quite broad and you might be able to see a little about what or why they are included. But, below, I added more detail about each one along with a smaller scale list of objectives that you will engage with. Throughout our course, you will have opportunities to demonstrate these objectives in your work. \sphinxstyleemphasis{My aim is to make what you are assessed on in this course something you are interested in, so these objectives reflect that.}
\end{sphinxadmonition}


\subsection{Construct and document a reproducible process}
\label{\detokenize{content/0_course/goals:construct-and-document-a-reproducible-process}}
\sphinxAtStartPar
A critical element of physics work is making sure that with the same setup and approach, others can reproduce the work you have done. This provides validity to your work and evidences how we develop collective understanding of physics. Physics is a social enterprise and the ensuring the reproducibility of work supports that enterprise. Below are the learning objectives for this principal objective.

\begin{sphinxadmonition}{note}{Reproducibility Learning Objectives}

\sphinxAtStartPar
Students will demonstrate they can:
\begin{itemize}
\item {} 
\sphinxAtStartPar
document their work and analysis such that others can reproduce their work

\item {} 
\sphinxAtStartPar
consistently reproduce their work and results in a variety of contexts

\item {} 
\sphinxAtStartPar
provide an explanation for why certain work or results are not (or should not be) reproducible

\end{itemize}
\end{sphinxadmonition}


\subsection{Use analytical, computational, and graphical approaches}
\label{\detokenize{content/0_course/goals:use-analytical-computational-and-graphical-approaches}}
\sphinxAtStartPar
The main approaches that we use to make models are mathematical, computational, and graphical. In this class, we will aim to leverage the benefits of each to learn more about the physical systems that we are investigating. Indeed, much of the “knowledge” that you are going to develop will be about specific analytical, computational, or graphical approaches to investigate physical systems. Below are the learning objectives for this principal objective.

\begin{sphinxadmonition}{note}{Modeling Approaches Learning Objectives}

\sphinxAtStartPar
Students will demonstrate they can:
\begin{itemize}
\item {} 
\sphinxAtStartPar
Use a wide variety of modeling techniques to investigate different physical systems

\item {} 
\sphinxAtStartPar
Choose and employ appropriate approaches to modeling physical systems of their choosing

\item {} 
\sphinxAtStartPar
Explain how those approaches lead to different results or conclusions

\end{itemize}
\end{sphinxadmonition}


\subsection{Provide evidence of the quality of their work}
\label{\detokenize{content/0_course/goals:provide-evidence-of-the-quality-of-their-work}}
\sphinxAtStartPar
The definition of the quality of a piece of science is a collective decision by the scientific community. In established communities, like physics, there are commonly\sphinxhyphen{}accepted ways of defining the quality of work (norms, customs, and rules all play a role). But that is not to mean those ways can’t change; papers describing quantum physics and relativity brushed up hard against this issue of quality and were both dismissed and celebrated. Newer disciplines are still establishing those norms and rules. And in some cases, disciplines are pushing back against Western norms of quality. In our class, we will collectively decide what we mean by ‘’high quality’’ work. Below are the learning objectives for this principal objective.

\begin{sphinxadmonition}{note}{Quality Control Learning Objectives}

\sphinxAtStartPar
Students will demonstrate they can:
\begin{itemize}
\item {} 
\sphinxAtStartPar
describe what it means to have high quality work in our class

\item {} 
\sphinxAtStartPar
look for and evaluate when work meets those standards

\item {} 
\sphinxAtStartPar
provide suggestions (or act on suggestions) to improve the quality of their work

\end{itemize}
\end{sphinxadmonition}


\subsection{Collaborate effectively}
\label{\detokenize{content/0_course/goals:collaborate-effectively}}
\sphinxAtStartPar
Physics is a social enterprise that relies on effective and productive collaborations. Very little (if any) science is done alone; the scale of science is too grand for individuals to effectively work – everyone needs a team. In this spirit, in this classroom, we deeply encourage collaboration. We will try to develop effective collaboration through your work on projects and our in\sphinxhyphen{}class activities. Below are the learning objectives for this principal objective.

\begin{sphinxadmonition}{note}{Collaboration Learning Objectives}

\sphinxAtStartPar
Students will demonstrate they can:
\begin{itemize}
\item {} 
\sphinxAtStartPar
Collaborate on a variety of activities in and out of class

\item {} 
\sphinxAtStartPar
Document the contributions in these collaborations and make changes if contributions are unbalanced

\item {} 
\sphinxAtStartPar
Develop personally effective strategies for collaboration

\end{itemize}
\end{sphinxadmonition}

\sphinxstepscope


\section{Course Design}
\label{\detokenize{content/0_course/design:course-design}}\label{\detokenize{content/0_course/design::doc}}
\sphinxAtStartPar
For most of you, 4415 is an elective course that you are taking to learn more about how we use mathematical techniques in physics. As such, this course is designed under several different principles than a standard course. Below, I provide those principles and their rationale.
\begin{itemize}
\item {} 
\sphinxAtStartPar
415 should help you learn the central tenets of modeling physical systems
\begin{itemize}
\item {} 
\sphinxAtStartPar
The sheer volume of mathematical and computational physics knowledge out there is immense and impossible for any one person to learn. However, the central elements of making models, how to learn about specific techniques, and how to debug your approaches are things we can learn and employ broadly as well as to specific problems.

\end{itemize}

\item {} 
\sphinxAtStartPar
415 should be a celebration of your knowledge
\begin{itemize}
\item {} 
\sphinxAtStartPar
For most of you, this course is part of your senior level coursework. What you have achieved in the last three to four years should be celebrated and enjoyed. This course will provide ample opportunities for you to share what things you know and what things you are learning with me and with each other.

\end{itemize}

\item {} 
\sphinxAtStartPar
415 should give you opportunities to engage in professional practice
\begin{itemize}
\item {} 
\sphinxAtStartPar
As you start towards your professional career, it’s important to learn what professional scientists do. You have probably already begun this work in advanced lab and research projects that you have worked on. We will continue developing your professional skills in this course through the use of course projects.

\end{itemize}

\item {} 
\sphinxAtStartPar
415 will illustrate that we can learn from each other
\begin{itemize}
\item {} 
\sphinxAtStartPar
Even though I’ve been learning physics for almost 20 years, I don’t know everything. I am excited to learn from you and I hope that you are excited to learn from me and each other.

\end{itemize}

\end{itemize}


\subsection{Optional purchases:}
\label{\detokenize{content/0_course/design:optional-purchases}}
\sphinxAtStartPar
The core readings and work for this course will be this jupyterbook. I will find resources online, make my own, and provide as much organized free material as possible. If you want to have a textbook that helps you organize your readings, please obtain copies of:
\begin{enumerate}
\sphinxsetlistlabels{\arabic}{enumi}{enumii}{}{.}%
\item {} 
\sphinxAtStartPar
Mary Boas, \sphinxhref{https://www.amazon.com/Mathematical-Methods-Physical-Sciences-Mary/dp/04711982}{\sphinxstyleemphasis{Mathematical Methods in the Physical Sciences}} (Wiley; 2005). This book is the definitive text on mathematical approaches, written by Dr. Boas originally in 1966! Any 3rd edition will be useful and I will put the section numbers from Boas in the online readings.

\item {} 
\sphinxAtStartPar
Mark Newman, \sphinxhref{https://www.amazon.com/Computational-Physics-Mark-Newman/dp/1480145513}{\sphinxstyleemphasis{Computational Physics}} (CreateSpace Independent Publishing Platform; 2012). This book is a great introduction to a variety of computational physics techniques, written by UMich professor Mark Newman for a computational physics course. I will put section numbers from Newman in the online readings.

\end{enumerate}


\subsubsection{Additional sources}
\label{\detokenize{content/0_course/design:additional-sources}}
\sphinxAtStartPar
In addition, I will draw from the following books. I have copies of them if you want or need scans of sections. But they can found online in Google Books and other places as well. no need to purchase unless you want a copy for your personal library.


\paragraph{Mechanics}
\label{\detokenize{content/0_course/design:mechanics}}\begin{itemize}
\item {} 
\sphinxAtStartPar
Edwin Taylor, Mechanics

\item {} 
\sphinxAtStartPar
Jerry Marion and Stephen Thornton, Classical Dynamics of Particles and Systems

\item {} 
\sphinxAtStartPar
Charles Kittel, Walter D. Knight, Malvin A. Ruderman, A. Carl Helholtz, and Burton J. Moyer, Mechanics

\end{itemize}


\paragraph{Electromagnetism}
\label{\detokenize{content/0_course/design:electromagnetism}}\begin{itemize}
\item {} 
\sphinxAtStartPar
Edward Purcell, Electricity and Magnetism

\item {} 
\sphinxAtStartPar
David J. Grriffths, Introduction to Electromagnetism

\end{itemize}


\paragraph{Quantum Mechanics}
\label{\detokenize{content/0_course/design:quantum-mechanics}}\begin{itemize}
\item {} 
\sphinxAtStartPar
David McIntyre, Quantum Mechanics

\item {} 
\sphinxAtStartPar
David J. Griffiths, Introduction to Quantum Mechanics

\end{itemize}


\paragraph{Waves and Thermal Physics}
\label{\detokenize{content/0_course/design:waves-and-thermal-physics}}\begin{itemize}
\item {} 
\sphinxAtStartPar
Frank S. Crawford, Waves

\item {} 
\sphinxAtStartPar
Charles Kittel, Thermal Physics

\item {} 
\sphinxAtStartPar
Ashley Carter, Classical and Statistical Thermodynamics

\item {} 
\sphinxAtStartPar
Daniel Schroeder, Thermal Physics

\end{itemize}


\paragraph{Additional Physics Topics}
\label{\detokenize{content/0_course/design:additional-physics-topics}}\begin{itemize}
\item {} 
\sphinxAtStartPar
Steven H. Strogatz, Nonlinear Dynamics and Chaos

\item {} 
\sphinxAtStartPar
B Lautrup, Physics of Continuous Matter

\item {} 
\sphinxAtStartPar
Frank L. Pedrotti and Leno S. Pedrotti, Introduction to Optics

\end{itemize}


\paragraph{Mathematics}
\label{\detokenize{content/0_course/design:mathematics}}\begin{itemize}
\item {} 
\sphinxAtStartPar
Susan M. Lea, Mathematics for Physicists

\item {} 
\sphinxAtStartPar
William E. Boyce and Richard C. DiPrima, Elementary Differential Equations

\item {} 
\sphinxAtStartPar
James Brown and Ruel Churchill, Complex Variables and Applications

\item {} 
\sphinxAtStartPar
Jerrold Marsden and Anthony Tromba, Vector Calculus

\item {} 
\sphinxAtStartPar
Sheldon Ross, A First Course in Probability

\end{itemize}


\paragraph{Presenting (Visual) Information}
\label{\detokenize{content/0_course/design:presenting-visual-information}}\begin{itemize}
\item {} 
\sphinxAtStartPar
Edward Tufte, The Visual Display of Quantitative information

\item {} 
\sphinxAtStartPar
Albert Cairo, The Truthful Art

\item {} 
\sphinxAtStartPar
Stephen E. Toulmin, The Uses of Argument

\end{itemize}


\section{Course Activities}
\label{\detokenize{content/0_course/design:course-activities}}

\subsection{“Readings”}
\label{\detokenize{content/0_course/design:readings}}
\sphinxAtStartPar
\sphinxstylestrong{“Reading”} is an essential part of 415! Reading the notes before class is very important. I use “reading” in quotes, because in our class this idea goes beyond just reading text and includes understanding figures and watching videos. These should help inform the basis of your understating that we will draw on in class to clarify your understanding and to help you make sense of the material. I will assume you have done the required readings in advance! It will make a huge difference if you spend the time and effort to carefully read and follow the resources posted. The calendar has the details on videos and readings that will be updated.

\sphinxAtStartPar
\sphinxstylestrong{Weekly Questions}: To encourage and reward you for keeping up with the “readings”, there will be weekly questions about the readings posted for you to respond to. These are not meant to test your knowledge, but rather to focus your “reading” towards what you understand, and what you don’t yet understand. I will ask you about those things weekly and use that information to tailor in\sphinxhyphen{}class activities based on what I am hearing is confusing, unclear, or challenging. These questions are only graded for completion, but I do want your honest attempt.


\subsection{Class Meetings}
\label{\detokenize{content/0_course/design:class-meetings}}
\sphinxAtStartPar
\sphinxstylestrong{Classroom Etiquette:} Please silence your electronic devices when entering the classroom. I don’t mind you using them (in fact, see below, we will use them). But, sometimes, they can very distracting to your neighbors, so use your judgement. I appreciate that you might have questions or comments about things in class. We are going to be having short lectures combined with longer project work in class. So you will have plenty of time to catch up with social media and the news.

\sphinxAtStartPar
If you and/or your group mates are confused, just raise your hand and ask questions. If you are confused, you are likely not the only one and it’s better to chat about it, then move on. Questions are always good, and are strongly encouraged! \sphinxstyleemphasis{The only way we learn is to question what we know and how we know it.}

\sphinxAtStartPar
\sphinxstylestrong{Computing Devices:} Please bring some sort of computing device to class everyday. You might be researching information online, reviewing work you have done, or actively building models of systems together. This device can be a computer, a tablet, or a phone. You can also partner up with folks because we will use them in groups. \sphinxstyleemphasis{If you need a computing device brought to class for you or your group mates to use, let me know. I will organize for some small collection of laptops if we need it.}

\sphinxAtStartPar
\sphinxstylestrong{In\sphinxhyphen{}Class:} We will have some short lectures about topics or concepts; some of those will be in\sphinxhyphen{}the\sphinxhyphen{}moment as needed. The idea is that you are developing a basic understanding through readings and videos, practicing using those new ideas with me and with your classmates in class, and then applying what you are learning to new ideas. So, we will also use a variety of in\sphinxhyphen{}class activities that help you construct an understanding of a particular topic or concept. These will not be collected or graded, but we will discuss the solutions in class. \sphinxstyleemphasis{I will not post solutions for these activties as we have no exams or quizzes.}


\subsection{Projects}
\label{\detokenize{content/0_course/design:projects}}
\sphinxAtStartPar
\sphinxstylestrong{In\sphinxhyphen{}class Projects:} The class is designed to support your independent research into ideas that you are excited about. So in\sphinxhyphen{}class projects are meant to equip you with the knowledge and practice to learn new things for your projects. These in\sphinxhyphen{}class projects will be short demonstrations of models that you complete in groups. We will circulate around the room and check on you and your group’s progress and understanding. At the end of the class period, we will share the results of the in\sphinxhyphen{}class project and discuss any sticking points. These in\sphinxhyphen{}class activites will not be graded, but they will be essential for your out\sphinxhyphen{}of\sphinxhyphen{}class projects.

\sphinxAtStartPar
\sphinxstylestrong{Out\sphinxhyphen{}of\sphinxhyphen{}class Projects:} For this class, we anticipate 6 projects to be turned in roughly every 2\sphinxhyphen{}3 weeks, with a weeklong turn\sphinxhyphen{}in window (see calendar). Except for the first project, up to 3 of these projects can be completed as partner projects. Partner projects are subject to a different grading rubric that evaluates collaborative efforts and increases the expectation for other areas compared to an individual project. A preliminary rubric appears here, but we will define these collectively after the first project.

\sphinxAtStartPar
These projects will take the form a \sphinxhref{https://uio-ccse.github.io/computational-essay-showroom/}{computational essay}, which provides documentation and rationale for the exploration that you are completing. We will model a computational essay project in our first project and we will reflect on the rubric after it, and make changes collectively as a class to it.

\sphinxAtStartPar
\sphinxstyleemphasis{I strongly encourage collaboration}, an essential skill in science and engineering (and highly valued by employers!) Social interactions are critical to scientists’ success – most good ideas grow out of discussions with colleagues, and essentially all physicists work as part of a group. Find partners and work together. However, it is also important that you OWN the material. I strongly suggest you start working by yourself (and that means really making an extended effort on every activity). Then work with a group, and finally, finish up on your own – write up your own work, in your own way. There will also be time for peer discussion during classes – as you work together, try to help your partners get over confusions, listen to them, ask each other questions, critique, teach each other. You will learn a lot this way! For all assignments, the work you turn in must in the end be your own: in your own words, reflecting your own understanding. (If, at any time, for any reason, you feel disadvantaged or isolated, contact me and I can discretely try to help arrange study groups.)


\subsubsection{Help Session}
\label{\detokenize{content/0_course/design:help-session}}
\sphinxAtStartPar
Help sessions/office hours are to facilitate your learning. We encourage attendance \sphinxhyphen{} plan on working in small groups, our role will be as learning coaches. The sessions are concept and project\sphinxhyphen{}centric, but we will not be explicitly telling anyone how to do your project (how would that help you learn?) I strongly encourage you to start all projects on your own. If you come to help sessions “cold”, the value of the project to you will be greatly reduced.

\sphinxstepscope


\section{Assessments}
\label{\detokenize{content/0_course/assessments:assessments}}\label{\detokenize{content/0_course/assessments::doc}}

\subsection{Formative Assessment}
\label{\detokenize{content/0_course/assessments:formative-assessment}}
\sphinxAtStartPar
Formative assessment is often ungraded and reflective assessment. It is meant to help you make changes to your thinking, approaches, or practice. It is not evaluative, it’s corrective; to help you make changes. We will make heavy use of ungraded formative feedback throughout the course.


\subsection{Summative Assessment}
\label{\detokenize{content/0_course/assessments:summative-assessment}}
\sphinxAtStartPar
Summative assessment is typically evaluative and will take the form of course projects completed out of class. These projects will take the form of a computational essay in which you write mathematics and code to investigate and explain a given phenomenon of interest. We will explore those essays in class and talk about what makes a useful one as we define a rubric for evaluation.


\subsubsection{Preliminary Rubric}
\label{\detokenize{content/0_course/assessments:preliminary-rubric}}
\sphinxAtStartPar
A preliminary rubric has been posted. We will use this rubric for the first out\sphinxhyphen{}of\sphinxhyphen{}class project evaluation. We will then reflect on it and make changes to collectively as a class.


\subsubsection{Resources for Computational Essays}
\label{\detokenize{content/0_course/assessments:resources-for-computational-essays}}
\sphinxAtStartPar
If you want to read more about computational essays, here’s a few links in the order utility/readability:
\begin{itemize}
\item {} 
\sphinxAtStartPar
Steven Wolfram \sphinxhyphen{} \sphinxhref{https://writings.stephenwolfram.com/2017/11/what-is-a-computational-essay/}{What is a Computational Essay?}

\item {} 
\sphinxAtStartPar
University of Oslo Physics \sphinxhyphen{} \sphinxhref{https://uio-ccse.github.io/computational-essay-showroom/}{Examples of Computational Essays}

\item {} 
\sphinxAtStartPar
Odden and Burk, The Physics Teacher \sphinxhyphen{} \sphinxhref{https://aapt.scitation.org/doi/abs/10.1119/1.5145471}{Computational Essays in the Physics Classroom}

\item {} 
\sphinxAtStartPar
Odden, Lockwood, and Caballero, Physical Review PER \sphinxhyphen{} \sphinxhref{https://journals.aps.org/prper/abstract/10.1103/PhysRevPhysEducRes.15.020152}{Physics computational literacy: An exploratory case study using computational essays}

\end{itemize}

\sphinxstepscope


\section{Project Rubrics}
\label{\detokenize{content/0_course/rubric:project-rubrics}}\label{\detokenize{content/0_course/rubric::doc}}

\subsection{Preliminary (For first out of class project)}
\label{\detokenize{content/0_course/rubric:preliminary-for-first-out-of-class-project}}
\sphinxAtStartPar
We have worked together to define elements of a rubric that matter for making physical models. These elements appear as part of major learning goals below.


\begin{savenotes}\sphinxattablestart
\centering
\begin{tabulary}{\linewidth}[t]{|T|T|}
\hline
\sphinxstyletheadfamily 
\sphinxAtStartPar
Goal
&\sphinxstyletheadfamily 
\sphinxAtStartPar
Fractional Importance
\\
\hline
\sphinxAtStartPar
Investigate physical systems
&
\sphinxAtStartPar
0.30
\\
\hline
\sphinxAtStartPar
Construct and document a reproducible process
&
\sphinxAtStartPar
0.10
\\
\hline
\sphinxAtStartPar
Use analytical, computational, and graphical approaches
&
\sphinxAtStartPar
0.30
\\
\hline
\sphinxAtStartPar
Provide evidence of the quality of their work
&
\sphinxAtStartPar
0.10
\\
\hline
\sphinxAtStartPar
Collaborate effectively
&
\sphinxAtStartPar
0.20
\\
\hline
\end{tabulary}
\par
\sphinxattableend\end{savenotes}


\subsubsection{Goal: Investigate physical systems (0.30)}
\label{\detokenize{content/0_course/rubric:goal-investigate-physical-systems-0-30}}\begin{itemize}
\item {} 
\sphinxAtStartPar
How well does your computational essay predict or explain the system of interest?

\item {} 
\sphinxAtStartPar
How well does your computational essay allow the user to explore and investigate the system?

\end{itemize}


\subsubsection{Goal: Construct and document a reproducible process (0.10)}
\label{\detokenize{content/0_course/rubric:goal-construct-and-document-a-reproducible-process-0-10}}\begin{itemize}
\item {} 
\sphinxAtStartPar
How well does your computational essay reproduce your results and claims?

\item {} 
\sphinxAtStartPar
How  well documented is your computational essay?

\end{itemize}


\subsubsection{Goal: Use analytical, computational, and graphical approaches (0.30)}
\label{\detokenize{content/0_course/rubric:goal-use-analytical-computational-and-graphical-approaches-0-30}}\begin{itemize}
\item {} 
\sphinxAtStartPar
How well does your computational essay document your assumptions?

\item {} 
\sphinxAtStartPar
How well does your computational essay produce an understandable and parsimonious model?

\item {} 
\sphinxAtStartPar
How well does your computational essay explain the limitations of your analysis?

\end{itemize}


\subsubsection{Goal: Provide evidence of the quality of their work}
\label{\detokenize{content/0_course/rubric:goal-provide-evidence-of-the-quality-of-their-work}}\begin{itemize}
\item {} 
\sphinxAtStartPar
How well does your computational essay present  the case for its claims?

\item {} 
\sphinxAtStartPar
How well validated  is your model?

\end{itemize}


\subsubsection{Goal: Collaborate effectively}
\label{\detokenize{content/0_course/rubric:goal-collaborate-effectively}}\begin{itemize}
\item {} 
\sphinxAtStartPar
How well did you share  in the class’s knowledge?
\begin{itemize}
\item {} 
\sphinxAtStartPar
How well is that documented in your computational essay?

\end{itemize}

\item {} 
\sphinxAtStartPar
How well did you work with your partner ? \sphinxstyleemphasis{For those choosing to do so}

\end{itemize}

\sphinxstepscope


\section{Calendar}
\label{\detokenize{content/0_course/calendar:calendar}}\label{\detokenize{content/0_course/calendar::doc}}
\sphinxAtStartPar
In this course, we will cover four principal topics in physics (in this order):

\sphinxAtStartPar
We will spend roughly 1/3 of the course on each, and the expectaitn is that each topic has 2 completed projects.

\sphinxAtStartPar
A Google Calendar appears below for class. If you review the notes in a given event, you will find the details for a class or the readings to do.

\sphinxstepscope


\section{Classroom Environment}
\label{\detokenize{content/0_course/environment:classroom-environment}}\label{\detokenize{content/0_course/environment::doc}}

\subsection{Commitment to an Inclusive Classroom}
\label{\detokenize{content/0_course/environment:commitment-to-an-inclusive-classroom}}
\sphinxAtStartPar
I am deeply committed to creating an inclusive classroom \sphinxhyphen{} one where you and your classmates
feel comfortable, intellectually challenged, and able to speak up about your ideas
and experiences. This means that our classroom, our virtual environments, and our interactions
need to be as inclusive as possible. Mutual respect, civility, and the ability to listen
and observe others are central to creating a classroom that is inclusive. I will strive to
do this and I ask that you do the same. If I can do anything to make the classroom a better
learning environment for you, please let me know.

\sphinxAtStartPar
\sphinxstylestrong{If you observe or experience behaviors that violate our commitment to inclusivity,
please let me know as soon as possible.}

\sphinxAtStartPar
If I violate this principle, please let me know or please tell the undergraduate department chair, Stuart Tessmer (\sphinxhref{mailto:tessmer@pa.msu.edu}{tessmer@pa.msu.edu}), who I have informed to tell me about any such incidents without conveying student information to me.


\subsection{Comments on preparation:}
\label{\detokenize{content/0_course/environment:comments-on-preparation}}
\sphinxAtStartPar
Physics 415 covers material you might have seen before. Many of the topics
stem from a wide variety of physics courses you might have already taken. But, we might be applying them at a higher level of conceptual and mathematical sophistication.

\sphinxAtStartPar
Therefore you should expect:
\begin{itemize}
\item {} 
\sphinxAtStartPar
a large amount of material to review and digest.

\item {} 
\sphinxAtStartPar
no recitations, and few examples covered in lecture. Most of the learning will be done through projects and questions you and your group mates raise.

\item {} 
\sphinxAtStartPar
long, hard problems that usually cannot be completed by one individual alone.

\item {} 
\sphinxAtStartPar
challenging projects.

\item {} 
\sphinxAtStartPar
to learn more about being a physicist that you have in another class (I hope!).

\end{itemize}

\sphinxAtStartPar
Physics 415 is a challenging, upper‐division physics course. Unlike more introductory courses, you are fully responsible for your own learning. In particular, you control the pace of the course by asking questions in class. I tend to speak quickly, and questions are important to slow down. This means that if you don’t understand something, it is your responsibility to ask questions. Attending class and the help sessions gives you an opportunity to ask questions. I am here to help you as much as possible, but I need your questions to know what you don’t understand.

\sphinxAtStartPar
Physics 415 covers some of the most important physics and mathematical methods in the field. Your reward for the hard work and effort will be learning important and elegant material that you will use over and over as a physics major. Here is what I have experienced, and heard from
other faculty teaching upper division physics in the past:
\begin{itemize}
\item {} 
\sphinxAtStartPar
most students reported spending a minimum of 10 hours per week on the
homework (!!)

\item {} 
\sphinxAtStartPar
students who didn’t attend the help sessions
often did poorly in the class.

\item {} 
\sphinxAtStartPar
students reported learning a tremendous amount in this class.

\end{itemize}

\sphinxAtStartPar
\sphinxstylestrong{The course topics that we will cover in Physics 415 are among the
greatest intellectual achievements of humans. Don’t be surprised if you
have to think hard and work hard to master the material.}

\sphinxstepscope


\section{Resources}
\label{\detokenize{content/0_course/resources:resources}}\label{\detokenize{content/0_course/resources::doc}}

\subsection{Confidentiality and Mandatory Reporting}
\label{\detokenize{content/0_course/resources:confidentiality-and-mandatory-reporting}}
\sphinxAtStartPar
College students often experience issues that may interfere with academic success such as academic stress, sleep problems, juggling responsibiities, life events, relationship concerns, or feelings of anxiety, hopelessness, or depression.
As your instructor, one of my responsibilities is to help create a safe learning environment and to support you through these situations and experiences.
I also have a mandatory reporting responsibility related to my role as a University employee.
It is my goal that you feel able to share information related to your life experiences in classroom
discussions, in written work, and in one\sphinxhyphen{}on\sphinxhyphen{}one meetings.
I will seek to keep information you share private to the greatest extent possible.
However, under Title IX, I am required to share information regarding sexual misconduct, relationship violence, or information
about criminal activity on MSU’s campus with the University including the Office of Institutional Equity (OIE).

\sphinxAtStartPar
\sphinxstylestrong{Students may speak to someone confidentially by contacting MSU Counseling and Psychiatric Service (CAPS) (\sphinxhref{http://caps.msu.edu}{caps.msu.edu}, 517\sphinxhyphen{}355\sphinxhyphen{}8270), MSU’s 24\sphinxhyphen{}hour Sexual Assault Crisis Line (\sphinxhref{http://endrape.msu.edu}{endrape.msu.edu}, 517\sphinxhyphen{}372\sphinxhyphen{}6666), or Olin Health Center (\sphinxhref{http://olin.msu.edu}{olin.msu.edu}, 517\sphinxhyphen{}884\sphinxhyphen{}6546).}


\subsection{Spartan Code of Honor Academic Pledge}
\label{\detokenize{content/0_course/resources:spartan-code-of-honor-academic-pledge}}
\sphinxAtStartPar
As a Spartan, I will strive to uphold values of the highest ethical standard. I will practice honesty in my work, foster honesty in my peers, and take pride in knowing that honor is worth more than grades. I will carry these values beyond my time as a student at Michigan State University, continuing the endeavor to build personal integrity in all that I do.


\subsection{Handling Emergency Situations}
\label{\detokenize{content/0_course/resources:handling-emergency-situations}}
\sphinxAtStartPar
\sphinxstyleemphasis{In the event of an emergency arising within the classroom, Prof. Caballero will notify you of what actions that may be required to ensure your safety. It is the responsibility of each student to understand the evacuation, “shelter\sphinxhyphen{}in\sphinxhyphen{}place,” and “secure\sphinxhyphen{}in\sphinxhyphen{}place” guidelines posted in each facility and to act in a safe manner. You are allowed to maintain cellular devices in a silent mode during this course, in order to receive emergency SMS text, phone or email messages distributed by the university. When anyone receives such a notification or observes an emergency situation, they should immediately bring it to the attention of Prof. Caballero in a way that causes the least disruption. If an evacuation is ordered, please ensure that you do it in a safe manner and facilitate those around you that may not otherwise be able to safely leave. When these orders are given, you do have the right as a member of this community to follow that order. Also, if a shelter\sphinxhyphen{}in\sphinxhyphen{}place or secure\sphinxhyphen{}in\sphinxhyphen{}place is ordered, please seek areas of refuge that are safe depending on the emergency encountered and provide assistance if it is advisable to do so.}

\sphinxstepscope


\part{1 \sphinxhyphen{} Mechanics and ODEs}

\sphinxstepscope


\chapter{Intro to Classical Mechanics}
\label{\detokenize{content/1_mechanics/mechanics_intro:intro-to-classical-mechanics}}\label{\detokenize{content/1_mechanics/mechanics_intro::doc}}
\sphinxAtStartPar
Welcome to Mathematical Modeling in physics! Our first unit will focus on Classical Mechanics and Ordinary Differential Equations (ODEs). Classical Mechanics is all about the motion of macrosopic objects, typically ones that are moving slow enough so that we can ignore special relativity. In particular, we are interested in systems that obey Newton’s second law:
\begin{equation*}
\begin{split}\mathbf{F} = \frac{d\mathbf{p}}{dt} = \dot{\mathbf{p}}\end{split}
\end{equation*}
\sphinxAtStartPar
or equivalently (if m is constant):
\begin{equation*}
\begin{split}\mathbf{F} = m \frac{d\mathbf{r}}{dt^2} = m\ddot{\mathbf{r}}\end{split}
\end{equation*}
\sphinxAtStartPar
Here we’re using dot notation to mean derivatives with repsect to time. We’ll continue to see this notation through this course.

\sphinxAtStartPar
The key thing to note here is that both of these equations are statements of \sphinxstylestrong{ordinary differential equations}. A huge portions of problems in physics boil down to differential equations, its probably more difficult to think of physics problems that aren’t differential equations than ones that are. Broadly speaking, differential equations are equations of some \sphinxstylestrong{unkown} function and its derivatives. Often we are concerned with the form and/or behavior of this unknown function. In terms of newtons laws, if we are able to solve the differential equation \(\mathbf{F} = m\ddot{\mathbf{r}}\), then we get some function \(\mathbf{r}(t)\) that tells us exactly how our system evolves in time.


\section{Interpreting the statement of ODEs}
\label{\detokenize{content/1_mechanics/mechanics_intro:interpreting-the-statement-of-odes}}
\sphinxAtStartPar
A nice way to think of differential equations is as a set of instructions for how a function should change in time. For example, consider the first order ODE:
\begin{equation*}
\begin{split}
\dot{x} = x
\end{split}
\end{equation*}
\sphinxAtStartPar
This equation is saying that the function \(x(t)\) should change according to what its current value is. We can also think of \(\dot{x}\) as the velocity of function \(x(t)\), so this equation is also saying that the velocity of \(x(t)\) needs to be equal to its current position at all times, or that when we take a derivative of this function we get itself back. The function \(e^t\) has this property, so it might be our go\sphinxhyphen{}to guess for the solution of this system. Since this equation is \sphinxhref{https://math.libretexts.org/Courses/Monroe\_Community\_College/MTH\_211\_Calculus\_II/Chapter\_8\%3A\_Introduction\_to\_Differential\_Equations/8.3\%3A\_Separable\_Differential\_Equations}{seperable}, we can solve it exactly by integrating:
\begin{equation*}
\begin{split}
\int \frac{dx}{x} = \int dt \implies \log(x(t)) = t + c
\end{split}
\end{equation*}\begin{equation*}
\begin{split}
\implies x(t) = e^{(x + c)} = Ae^t \text{ if we let } A = e^c
\end{split}
\end{equation*}
\sphinxAtStartPar
Our guess at a solution was close, but the actual solution ended up with an extra \(A\) term. In fact we’ve found \sphinxstylestrong{infinitley many} solutions, or the \sphinxstylestrong{general solution} since we don’t know the value of \(A\) that came from the integration constant.  Think for second about how we might go about finding what \(A\) is.


\section{Initial Conditions}
\label{\detokenize{content/1_mechanics/mechanics_intro:initial-conditions}}
\sphinxAtStartPar
To find \(A\), we would need to know what \(x(t=0)\) is. Let’s say for the particular solution we’re interested in has \(x(t=0) = x_0\). Then its straightforward to solve for \(A\):
\begin{equation*}
\begin{split}
x(t=0) = x_0 = Ae^0 = A \implies A = x_0 \implies x(t) = x_0e^t
\end{split}
\end{equation*}
\sphinxAtStartPar
So we’ve found a \sphinxstylestrong{specific} or \sphinxstylestrong{unique} solution to this ODE. In general, for an \(n\) th\sphinxhyphen{}order (the highest order is of \(n\) th degree) ODE, you need \(n\) iniial conditions to find a specific solution. We’ll see why this is in the coming weeks.


\section{A note on differentiability}
\label{\detokenize{content/1_mechanics/mechanics_intro:a-note-on-differentiability}}
\sphinxAtStartPar
When we are concerned with differential equations that show up in classical mechanics, we often secretly make the assumption that the function that we are looking for is differentiable in the first place, i.e. that the function is sufficiently smooth so that \(\lim_{t\to 0} \frac{\Delta \mathbf{r}}{\Delta t}\) exists. For the systems we’ll concern ourselves with in this class, we will take this for granted. Some differential equation models do run into this being an issue though, such as in fracture mechanics or models of swarming behavior, where one needs to employ what is called nonlocal analysis, which is a really interesting bit of math that we just don’t have the time to cover in this class sadly.

\sphinxAtStartPar
But before we can start modeling more or less whatever we’d like with ODEs, we need to get familiar with some different frames of reference, or coordinate systems that solutions to ODEs often live in. That leads us to our \sphinxhref{https://valentine-alia.github.io/phy415fall23/content/1\_mechanics/frames.html}{first in\sphinxhyphen{}class activity}.

\sphinxstepscope


\chapter{31 Aug 23 \sphinxhyphen{} Frames and Coordinates}
\label{\detokenize{content/1_mechanics/frames:aug-23-frames-and-coordinates}}\label{\detokenize{content/1_mechanics/frames::doc}}

\section{What is a Frame?}
\label{\detokenize{content/1_mechanics/frames:what-is-a-frame}}
\sphinxAtStartPar
A \sphinxstylestrong{Frame of Reference}, \sphinxstylestrong{Reference Frame}, or simply \sphinxstylestrong{Frame} is a set of coordinates. These coordinates describe where things are.

\sphinxAtStartPar
An \sphinxstylestrong{inertial frame} is a Frame where newtons first law holds, which means that in this frame a net force of zero means an acceleration of zero. In practice, this means that if you have two frames that are moving at a constant velocity relative to each other (without rotation), these are inertial frames.


\subsection{Relative Velocities in cartesian coordinates}
\label{\detokenize{content/1_mechanics/frames:relative-velocities-in-cartesian-coordinates}}
\sphinxAtStartPar
Suppose you’re walking down shaw lane eager to get to BPS for your favorite physics class at constant velocity \(v_A\) when you notice a \sphinxhref{https://www.instagram.com/qualitysquirrelsofmsu/}{squirrel} sitting stationary on the sidewalk \(d\) meters in front of you. You’re a physics major so you naturally feel the urge to unnececarily mathematically analyze this situation using reference frames.

\sphinxAtStartPar
Let’s call your reference frame \(A\) and the squirrel’s \(B\). We’ll ignore height differences between you and the squirrel.


\subsection{Questions}
\label{\detokenize{content/1_mechanics/frames:questions}}
\sphinxAtStartPar
A note on notation: \(r_{A/B}\) means “the position of \sphinxstylestrong{object} \(A\) in \sphinxstylestrong{frame} \(B\),” or simply “the position of \(A\) relative to \(B\).”



\sphinxAtStartPar
\sphinxstylestrong{✅ Do this}

\sphinxAtStartPar
Answer the following Questions:
\begin{enumerate}
\sphinxsetlistlabels{\arabic}{enumi}{enumii}{}{.}%
\item {} 
\sphinxAtStartPar
What is the squirrels position in your frame, \(r_{B/A}\)?

\item {} 
\sphinxAtStartPar
What is your position in the squirrel’s frame, \(r_{A/B}\)?

\item {} 
\sphinxAtStartPar
What is your velocity in your frame, \(\dot{r}_{A/A} = v_{A/A}\)?

\item {} 
\sphinxAtStartPar
What is the sqirrel’s velocity in your frame, \(v_{B/A}\)?

\item {} 
\sphinxAtStartPar
What is your velocity in the squirrels frame, \(v_{B/A}\)?

\item {} 
\sphinxAtStartPar
Suppose that the squirrel starts walking to the right at the same velocity that you are walking at. What is \(v_{B/A}\) now?

\end{enumerate}

\sphinxAtStartPar
Hopefully these are intuitive from what you’ve learned back in PHY 183, but there are subtleties to the math you just did. Is there a relationship between \(v_{B/A}\) and \(v_{A/B}\)?

\sphinxAtStartPar
In general, for inertial frames \(A,B,\) and object \(C\) one has the following:
\begin{equation*}
\begin{split}\mathbf{r}_{A/C} = \mathbf{r}_{A/B} + \mathbf{r}_{B/C} \end{split}
\end{equation*}
\sphinxAtStartPar
taking a derivative gives:
\begin{equation*}
\begin{split}\dot{\mathbf{r}}_{A/C} = \dot{\mathbf{r}}_{A/B} + \dot{\mathbf{r}}_{B/C} \end{split}
\end{equation*}
\sphinxAtStartPar
This derivative works out so simply only because the \sphinxstylestrong{unit vectors are fixed in cartesian coordinates}. That is, if we write a generic vector in one of these frames as \(\mathbf{r} = x\mathbf{\hat x} + y\mathbf{\hat y} + z \mathbf{\hat z}\), and then take a time derivative, we simply get \(\dot{\mathbf{r}} = \frac{d}{dt} \mathbf{r} = \frac{dx}{dt}\mathbf{\hat x} + \frac{d\mathbf{\hat x}}{dt} x + \frac{dy}{dt}\mathbf{\hat y} + \frac{d\mathbf{\hat y}}{dt} y + \frac{dz}{dt}\mathbf{\hat z} + \frac{d\mathbf{\hat z}}{dt} z = \dot{x}\mathbf{\hat x} + \dot{y}\mathbf{\hat y} + \dot{z} \mathbf{\hat z}\) because the derivatives of the unit vectors vanish. Later, you’ll figure out what happens when you have non\sphinxhyphen{}constant unit vectors.

\sphinxAtStartPar
Let’s use these transformations in a problem.

\sphinxAtStartPar
A second squirrel appears, which we’ll call squirrel \(C\). This scares the first squirrel into running toward you, while the new squirrel runs away.




\subsection{Questions}
\label{\detokenize{content/1_mechanics/frames:id1}}
\sphinxAtStartPar
\sphinxstylestrong{✅ Do this}

\sphinxAtStartPar
Answer the following question:

\sphinxAtStartPar
What is the velocity of \(B\) relative to \(C\)?


\subsection{The Galilean Transformation}
\label{\detokenize{content/1_mechanics/frames:the-galilean-transformation}}
\sphinxAtStartPar
We just did an example of a \sphinxstylestrong{Galilean Transformation}. Traditionally, if we have a frame \(B\) that with moves with constant velocity \(v_{B/A}\) with respect to frame \(A\), we re\sphinxhyphen{}write the above equations as the following:
\begin{equation*}
\begin{split}\mathbf{v}_B = \mathbf{v}_A - \textbf{v}_{B/A}\end{split}
\end{equation*}\begin{equation*}
\begin{split}\mathbf{r}_B = \mathbf{r}_A - \mathbf{v}_{B/A} t\end{split}
\end{equation*}
\sphinxAtStartPar
Where we have taken \(_A\) to mean \(_{C/A}\) and \(_B\) to mean \(_{C/B}\). Also note that since we are ignoring relativistic effects, we also have that:
\begin{equation*}
\begin{split}t_B = t_A\end{split}
\end{equation*}

\subsection{Checking the physics}
\label{\detokenize{content/1_mechanics/frames:checking-the-physics}}
\sphinxAtStartPar
This is all well and good, but we should make sure to check that newtonian mechanics works the same in both of these frames, that they are really inertial frames.

\sphinxAtStartPar
\sphinxstylestrong{✅ Do this}

\sphinxAtStartPar
In your group, show that forces experienced in frame \(A\) are the same as forces experienced in frame \(B\). (Hint: \(v_{B/A}\) is constant)


\section{Forces in polar coordinates}
\label{\detokenize{content/1_mechanics/frames:forces-in-polar-coordinates}}
\sphinxAtStartPar
Many problems in physics require the use of non\sphinxhyphen{}cartesian coordinates, such as the Hydrogen atom or the two\sphinxhyphen{}body problem. One such coordinate system is \sphinxstylestrong{polar coordinates}. In this coordinate system, any vector \(\mathbf{r}\in \mathbb{R}^2\) is described by a distance \(r\) and angle \(\phi\) insead of cartesian coordinates \(x\) and \(y\). The following four equations show how points transform in these coordinate systems.
\begin{equation*}
\begin{split}
x = r\cos \phi \hspace{1in} y = r\sin \phi
\end{split}
\end{equation*}\begin{equation*}
\begin{split}
r = \sqrt{x^2 + y^2} \hspace{1in} \phi = \arctan(y / x)
\end{split}
\end{equation*}
\sphinxAtStartPar
This also let’s us write kinetic energy as \(T = \frac{1}{2}m(\dot{r} + r^2\dot{\phi}^2)\).

\sphinxAtStartPar
We’d like to know how to write down Newton’s second law in this coordinate system, but this is not as simple as before because the unit vectors in polar coordinate are NOT constant. We’ll denote the unit vectors for polar coordinates by \(\hat{\mathbf{r}}\) and \(\hat{\boldsymbol{\phi}}\). \(\hat{\mathbf{r}}\) points in the direction of increasing \(r\) with \(\phi\) fixed, and similarly \(\hat{\boldsymbol{\phi}}\) points in the direction of increasing \(\phi\) with \(r\) fixed.

\sphinxAtStartPar
However, this doesen’t stop us from being able to break down a net force into its components along each unit vector:
\begin{equation*}
\begin{split}
\mathbf{F} = F_r \hat{\mathbf{R}} + F_\phi \hat{\boldsymbol{\phi}}
\end{split}
\end{equation*}
\sphinxAtStartPar
And the second law still holds:
\begin{equation*}
\begin{split}
\mathbf{F} = m\ddot{\mathbf{r}}
\end{split}
\end{equation*}
\sphinxAtStartPar
It would be very useful if we had expressions for \(F_r\) and \(F_\phi\) in terms of \(r\) and \(\phi\). Toward finding these, we can start by writing:
\begin{equation*}
\begin{split}
\mathbf{r} = r\hat{\mathbf{r}}
\end{split}
\end{equation*}

\subsection{Task}
\label{\detokenize{content/1_mechanics/frames:task}}
\sphinxAtStartPar
\sphinxstylestrong{✅ Do this}

\sphinxAtStartPar
Find \(\dot{\mathbf{r}}\) by differentiating \(\mathbf{r}\) with respect to time. (see hint below)

\sphinxAtStartPar
\sphinxstylestrong{✅ Do this}

\sphinxAtStartPar
Find \(\ddot{\mathbf{r}}\) by differentiating \(\dot{\mathbf{r}}\) with respect to time. Then find \(F_r\) and \(F_\phi\). (see hint below)





\sphinxAtStartPar
Hint: During these problems, you’ll need to find expressions for \(\frac{d\hat{\mathbf{r}}}{dt}\) and \(\frac{d\hat{\boldsymbol{\phi}}}{dt}\), which \sphinxhref{https://raw.githubusercontent.com/valentine-alia/phy415fall23/main/content/assets/frames\_hint.pdf}{these pictures} might help with.


\subsubsection{Angular momentum}
\label{\detokenize{content/1_mechanics/frames:angular-momentum}}
\sphinxAtStartPar
One of the powerful things you get from using this coordinate system is a handy way to represent angular momentum:
\begin{equation*}
\begin{split}
|\mathbf{L}| = |\mathbf{r} \times \mathbf{p}| = |mr^2\dot{\phi}|
\end{split}
\end{equation*}
\sphinxAtStartPar
In the interest of time we won’t go through the whole calculation to arrive at this, but this is handy to know.


\subsection{(time permitting) Non\sphinxhyphen{}Inertial Frames}
\label{\detokenize{content/1_mechanics/frames:time-permitting-non-inertial-frames}}
\sphinxAtStartPar
Broadly speaking, non\sphinxhyphen{}inertial frames are frames that undergo some sort of acceleration. In general for inertial frame \(A\) and non\sphinxhyphen{}inertial frame \(B\), we can write:
\begin{equation*}
\begin{split}r_B = r_A - r_{B/A}\end{split}
\end{equation*}\begin{equation*}
\begin{split}v_B = v_A - v_{B/A}\end{split}
\end{equation*}
\sphinxAtStartPar
as before, as well as:
\begin{equation*}
\begin{split}\ddot{r}_B = \ddot{r}_{A} - \ddot{r}_{B/A}\end{split}
\end{equation*}
\sphinxAtStartPar
Interestingly, this means a new (ficticious) force has sprung into being, equal to \(-m\ddot{r}_{B/A}\) ! This lets us write newton’s second law for frame B as:
\begin{equation*}
\begin{split}m\ddot{r}_{B} = F -m\ddot{r}_{B/A}\end{split}
\end{equation*}

\subsection{Leaning on a bus problem}
\label{\detokenize{content/1_mechanics/frames:leaning-on-a-bus-problem}}
\sphinxAtStartPar
Suppose you’re standing on a bus that is accelerating forward with constant acceleration \(r_{B/A}\). If we approximate you as a pendulum, what angle should you lean at so you can stay at equillibrium without having to exert any force to stay at that angle?

\sphinxAtStartPar
\sphinxstylestrong{✅ Do this}

\sphinxAtStartPar
Solve this problem using the the as a non\sphinxhyphen{}inertial frame. \sphinxhref{https://raw.githubusercontent.com/valentine-alia/phy415fall23/main/content/assets/bus\_soln.pdf}{Solution}


\subsection{Forces in Rotating Frames}
\label{\detokenize{content/1_mechanics/frames:forces-in-rotating-frames}}
\sphinxAtStartPar
For intertial frame A and non\sphinxhyphen{}inertial, rotating frame B, rotating at an angular velocity of \(\mathbf{\Omega}\) relative to A, we can write the second law as:
\begin{equation*}
\begin{split}
m\ddot{\mathbf{r}} = \mathbf{F} + 2m\dot{\mathbf{r}}\times \mathbf{\Omega} + m (\mathbf{\Omega}\times \mathbf{r}) \times \mathbf{\Omega}
\end{split}
\end{equation*}
\sphinxAtStartPar
Where \(\mathbf{F}\) is any “standard” forces from inertial frames, \(2m\dot{\mathbf{r}}\times \mathbf{\Omega}\) is the Coriolis force, and \(m (\mathbf{\Omega}\times \mathbf{r}) \times \mathbf{\Omega}\) is the centrifugal force.

\sphinxstepscope


\chapter{5 Sep 23 \sphinxhyphen{} Calculus of Varations and Lagrangian Dynamics}
\label{\detokenize{content/1_mechanics/lagrange_1:sep-23-calculus-of-varations-and-lagrangian-dynamics}}\label{\detokenize{content/1_mechanics/lagrange_1::doc}}
\sphinxAtStartPar
The name of the game in calculus of variations is finding minimums,maximums, or stationary points of integrals that have the form:
\begin{equation*}
\begin{split}
S = \int_{x_1}^{x_2} f[y(x),\dot{y}(x),x] dx
\end{split}
\end{equation*}
\sphinxAtStartPar
While you are trying to find the minimize \(S\), what you end up finding is the \sphinxstylestrong{function} \(y(x)\) that satisfies this minimization. It turns out that for \(S\) to have extrema, the Euler\sphinxhyphen{}Lagrange equation (below) must be satisfied.
\begin{equation*}
\begin{split}
\frac{\partial f}{\partial y} - \frac{d}{dx}\left(\frac{\partial f}{\partial \dot{y}} \right) = 0
\end{split}
\end{equation*}
\sphinxAtStartPar
In practice, when approaching a varational problem, the typical worflow if something like this:
\begin{enumerate}
\sphinxsetlistlabels{\arabic}{enumi}{enumii}{}{.}%
\item {} 
\sphinxAtStartPar
Write your problem down in the form of an integral like \(S\).

\item {} 
\sphinxAtStartPar
Use the Euler\sphinxhyphen{}Lagrange equation to get a differential equation for the unknown function \(y\).

\item {} 
\sphinxAtStartPar
Solve the differential equation.

\end{enumerate}

\sphinxAtStartPar
We can extend this framework for use in classical mechanics by defining the lagrangian of a system with independent, generalized coordinates \((q_1,\dot{q}_1... q_n,\dot{q}_n)\) as the kinetic energy minus potential energy of a system:
\begin{equation*}
\begin{split}
\mathcal{L(q_1,\dot{q}_1... q_n,\dot{q}_n)} = T(q_1,\dot{q}_1... q_n,\dot{q}_n) - V(q_1,\dot{q}_1... q_n,\dot{q}_n)
\end{split}
\end{equation*}
\sphinxAtStartPar
Then we write the action of the system as:
\begin{equation*}
\begin{split}
S = \int_{t_1}^{t_2} \mathcal{L(q_1,\dot{q}_1... q_n,\dot{q}_n)} dt
\end{split}
\end{equation*}
\sphinxAtStartPar
It turns out that the path a system takes between points \(1\) and \(2\) in the generalized coordinates is the path such that \(S\) is stationary. This is called the principle of least action. This lets us leverage the  Euler\sphinxhyphen{}Lagrange equation for the generalized coordinates of our system \(q_n\).
\begin{equation*}
\begin{split}
\frac{\partial \mathcal{L}}{\partial q_i} - \frac{d}{dx}\left(\frac{\partial \mathcal{L}}{\partial \dot{q}_i} \right) = 0
\end{split}
\end{equation*}
\sphinxAtStartPar
This gives us \(n\) equations of motion (EOM) for our system. Note how we didn’t have to know anything about the forces acting on our system to arrive at equations of motion.


\section{Activity}
\label{\detokenize{content/1_mechanics/lagrange_1:activity}}

\subsection{Simple Harmonic Oscillator (SHO)}
\label{\detokenize{content/1_mechanics/lagrange_1:simple-harmonic-oscillator-sho}}
\sphinxAtStartPar
\sphinxstylestrong{✅ Do this}
\begin{enumerate}
\sphinxsetlistlabels{\arabic}{enumi}{enumii}{}{.}%
\item {} 
\sphinxAtStartPar
Starting with the 1d energy equations (\(T\) and \(V\)) for a SHO; derive the equations of motion. Did you get the sign right?

\end{enumerate}


\subsection{Canonical Coupled Oscillators}
\label{\detokenize{content/1_mechanics/lagrange_1:canonical-coupled-oscillators}}
\sphinxAtStartPar
Let’s assume you have a chain of two mass connected by springs (all with the same \(k\)) as below.



\sphinxAtStartPar
\sphinxstylestrong{✅ Do this}
\begin{enumerate}
\sphinxsetlistlabels{\arabic}{enumi}{enumii}{}{.}%
\item {} 
\sphinxAtStartPar
Write down the energy equations for this system (using \(x_1\) and \(x_2\) for coordinates)

\item {} 
\sphinxAtStartPar
Write the Lagrangian and derive the two equations of motion.

\item {} 
\sphinxAtStartPar
Do all the signs makes sense to you?

\item {} 
\sphinxAtStartPar
Could you have arrived at these equations in the newtonian framework?

\end{enumerate}


\subsection{2\sphinxhyphen{}Body Problem}
\label{\detokenize{content/1_mechanics/lagrange_1:body-problem}}
\sphinxAtStartPar
Consider the 2 body problem of a star and an orbiting planet under the force of gravity. Assume the star is stationary.

\sphinxAtStartPar
\sphinxstylestrong{✅ Do this}
\begin{enumerate}
\sphinxsetlistlabels{\arabic}{enumi}{enumii}{}{.}%
\item {} 
\sphinxAtStartPar
Write down the energy equations for this system using polar coordinates.

\item {} 
\sphinxAtStartPar
Write the Lagrangian and derive 2 equations of motion for \(r\) and \(\phi\)

\end{enumerate}


\section{Adding Constraint Forces}
\label{\detokenize{content/1_mechanics/lagrange_1:adding-constraint-forces}}
\sphinxAtStartPar
The Lagrangian framework also excells at dealing with constrained motion, where it is usually not obvious what the constraint forces are. This is because you can write your generalized coordinates for your system in such a way that it contains the information

\sphinxAtStartPar
Consider a particle of mass \(m\) constrained to move on the surface of a paraboloid \(z =  r^2\) subject to a gravitational force downward, so that the paraboloid and gravity are aligned.

\sphinxAtStartPar
\sphinxstylestrong{✅ Do this}
\begin{enumerate}
\sphinxsetlistlabels{\arabic}{enumi}{enumii}{}{.}%
\item {} 
\sphinxAtStartPar
Using cylindrical coordinates (why?), write down the equation of constraint. Think about where the mass must be if it’s stuck on a paraboloid.

\item {} 
\sphinxAtStartPar
Write the energy contributions in cylindrical coordinates. (This is where you put in the constraint!)

\item {} 
\sphinxAtStartPar
Form the Lagrangian and find the equations of motion (there are two!)

\end{enumerate}

\begin{sphinxuseclass}{cell}\begin{sphinxVerbatimInput}

\begin{sphinxuseclass}{cell_input}
\begin{sphinxVerbatim}[commandchars=\\\{\}]
\PYG{k+kn}{import} \PYG{n+nn}{numpy} \PYG{k}{as} \PYG{n+nn}{np}
\PYG{k+kn}{import} \PYG{n+nn}{matplotlib}\PYG{n+nn}{.}\PYG{n+nn}{pyplot} \PYG{k}{as} \PYG{n+nn}{plt}

\PYG{k}{def} \PYG{n+nf}{parabaloid}\PYG{p}{(}\PYG{n}{x}\PYG{p}{,}\PYG{n}{y}\PYG{p}{,}\PYG{n}{alpha}\PYG{p}{)}\PYG{p}{:}
    \PYG{c+c1}{\PYGZsh{} function of a paraboloid in Cartesian coordinates}
    \PYG{k}{return} \PYG{n}{alpha} \PYG{o}{*} \PYG{p}{(}\PYG{n}{x}\PYG{o}{*}\PYG{o}{*}\PYG{l+m+mi}{2} \PYG{o}{+} \PYG{n}{y}\PYG{o}{*}\PYG{o}{*}\PYG{l+m+mi}{2}\PYG{p}{)}

\PYG{c+c1}{\PYGZsh{} points of the surface to plot}
\PYG{n}{x} \PYG{o}{=} \PYG{n}{np}\PYG{o}{.}\PYG{n}{linspace}\PYG{p}{(}\PYG{o}{\PYGZhy{}}\PYG{l+m+mf}{2.8}\PYG{p}{,} \PYG{l+m+mf}{2.8}\PYG{p}{,} \PYG{l+m+mi}{50}\PYG{p}{)}
\PYG{n}{y} \PYG{o}{=} \PYG{n}{np}\PYG{o}{.}\PYG{n}{linspace}\PYG{p}{(}\PYG{o}{\PYGZhy{}}\PYG{l+m+mf}{2.8}\PYG{p}{,} \PYG{l+m+mf}{2.8}\PYG{p}{,} \PYG{l+m+mi}{50}\PYG{p}{)}
\PYG{n}{alpha} \PYG{o}{=} \PYG{l+m+mi}{1}
\PYG{c+c1}{\PYGZsh{} construct meshgrid for plotting}
\PYG{n}{X}\PYG{p}{,} \PYG{n}{Y} \PYG{o}{=} \PYG{n}{np}\PYG{o}{.}\PYG{n}{meshgrid}\PYG{p}{(}\PYG{n}{x}\PYG{p}{,} \PYG{n}{y}\PYG{p}{)}
\PYG{n}{Z} \PYG{o}{=} \PYG{n}{parabaloid}\PYG{p}{(}\PYG{n}{X}\PYG{p}{,} \PYG{n}{Y}\PYG{p}{,}\PYG{n}{alpha}\PYG{p}{)}

\PYG{c+c1}{\PYGZsh{} do plotting}
\PYG{n}{fig} \PYG{o}{=} \PYG{n}{plt}\PYG{o}{.}\PYG{n}{figure}\PYG{p}{(}\PYG{n}{figsize} \PYG{o}{=} \PYG{p}{(}\PYG{l+m+mi}{10}\PYG{p}{,}\PYG{l+m+mi}{10}\PYG{p}{)}\PYG{p}{)}
\PYG{n}{ax} \PYG{o}{=} \PYG{n}{plt}\PYG{o}{.}\PYG{n}{axes}\PYG{p}{(}\PYG{n}{projection}\PYG{o}{=}\PYG{l+s+s1}{\PYGZsq{}}\PYG{l+s+s1}{3d}\PYG{l+s+s1}{\PYGZsq{}}\PYG{p}{)}
\PYG{n}{plt}\PYG{o}{.}\PYG{n}{title}\PYG{p}{(}\PYG{l+s+sa}{r}\PYG{l+s+s2}{\PYGZdq{}}\PYG{l+s+s2}{Paraboloid (\PYGZdl{}}\PYG{l+s+s2}{\PYGZbs{}}\PYG{l+s+s2}{alpha = \PYGZdl{}}\PYG{l+s+s2}{\PYGZdq{}} \PYG{o}{+} \PYG{n+nb}{str}\PYG{p}{(}\PYG{n}{alpha}\PYG{p}{)}\PYG{o}{+} \PYG{l+s+s2}{\PYGZdq{}}\PYG{l+s+s2}{)}\PYG{l+s+s2}{\PYGZdq{}}\PYG{p}{)}
\PYG{n}{ax} \PYG{o}{=} \PYG{n}{plt}\PYG{o}{.}\PYG{n}{axes}\PYG{p}{(}\PYG{n}{projection}\PYG{o}{=}\PYG{l+s+s1}{\PYGZsq{}}\PYG{l+s+s1}{3d}\PYG{l+s+s1}{\PYGZsq{}}\PYG{p}{)}
\PYG{n}{ax}\PYG{o}{.}\PYG{n}{plot\PYGZus{}surface}\PYG{p}{(}\PYG{n}{X}\PYG{p}{,} \PYG{n}{Y}\PYG{p}{,} \PYG{n}{Z}\PYG{p}{,} \PYG{n}{cmap}\PYG{o}{=}\PYG{l+s+s1}{\PYGZsq{}}\PYG{l+s+s1}{binary}\PYG{l+s+s1}{\PYGZsq{}}\PYG{p}{,} \PYG{n}{alpha}\PYG{o}{=}\PYG{l+m+mf}{0.8}\PYG{p}{)} 
\PYG{n}{ax}\PYG{o}{.}\PYG{n}{set\PYGZus{}xlim}\PYG{p}{(}\PYG{o}{\PYGZhy{}}\PYG{l+m+mi}{3}\PYG{p}{,} \PYG{l+m+mi}{3}\PYG{p}{)}\PYG{p}{;} \PYG{n}{ax}\PYG{o}{.}\PYG{n}{set\PYGZus{}ylim}\PYG{p}{(}\PYG{o}{\PYGZhy{}}\PYG{l+m+mi}{3}\PYG{p}{,} \PYG{l+m+mi}{3}\PYG{p}{)}\PYG{p}{;} \PYG{n}{ax}\PYG{o}{.}\PYG{n}{set\PYGZus{}zlim}\PYG{p}{(}\PYG{o}{\PYGZhy{}}\PYG{l+m+mi}{1} \PYG{p}{,}\PYG{l+m+mi}{15}\PYG{p}{)}
\PYG{n}{ax}\PYG{o}{.}\PYG{n}{set\PYGZus{}xlabel}\PYG{p}{(}\PYG{l+s+s1}{\PYGZsq{}}\PYG{l+s+s1}{x}\PYG{l+s+s1}{\PYGZsq{}}\PYG{p}{)}
\PYG{n}{ax}\PYG{o}{.}\PYG{n}{set\PYGZus{}ylabel}\PYG{p}{(}\PYG{l+s+s1}{\PYGZsq{}}\PYG{l+s+s1}{y}\PYG{l+s+s1}{\PYGZsq{}}\PYG{p}{)}
\PYG{n}{ax}\PYG{o}{.}\PYG{n}{set\PYGZus{}zlabel}\PYG{p}{(}\PYG{l+s+s1}{\PYGZsq{}}\PYG{l+s+s1}{z}\PYG{l+s+s1}{\PYGZsq{}}\PYG{p}{)}
\PYG{n}{plt}\PYG{o}{.}\PYG{n}{show}\PYG{p}{(}\PYG{p}{)}
\end{sphinxVerbatim}

\end{sphinxuseclass}\end{sphinxVerbatimInput}
\begin{sphinxVerbatimOutput}

\begin{sphinxuseclass}{cell_output}
\noindent\sphinxincludegraphics{{lagrange_1_5_0}.png}

\end{sphinxuseclass}\end{sphinxVerbatimOutput}

\end{sphinxuseclass}

\subsection{Roller Coaster}
\label{\detokenize{content/1_mechanics/lagrange_1:roller-coaster}}
\sphinxAtStartPar
Consider 3 roller coaster cars of equal mass \(m\) and positions \(x_1,x_2,x_3\), constrained to move on a one dimensional “track” defined by \(f(x) = x^4 -2x^2 + 1\). These cars are also constrained to stay a distance \(d\) apart, since they are linked. We’ll only worry about that distance \(d\) in the direction for now (though a fun problem would be to try this problem with a true fixed distance!)

\begin{sphinxuseclass}{cell}\begin{sphinxVerbatimInput}

\begin{sphinxuseclass}{cell_input}
\begin{sphinxVerbatim}[commandchars=\\\{\}]
\PYG{n}{x} \PYG{o}{=} \PYG{n}{np}\PYG{o}{.}\PYG{n}{arange}\PYG{p}{(}\PYG{o}{\PYGZhy{}}\PYG{l+m+mf}{1.8}\PYG{p}{,}\PYG{l+m+mf}{1.8}\PYG{p}{,}\PYG{l+m+mf}{0.01}\PYG{p}{)}
\PYG{n}{track} \PYG{o}{=} \PYG{k}{lambda} \PYG{n}{x} \PYG{p}{:} \PYG{n}{x}\PYG{o}{*}\PYG{o}{*}\PYG{l+m+mi}{4} \PYG{o}{\PYGZhy{}} \PYG{l+m+mi}{2}\PYG{o}{*}\PYG{n}{x}\PYG{o}{*}\PYG{o}{*}\PYG{l+m+mi}{2} \PYG{o}{+} \PYG{l+m+mi}{1}
\PYG{n}{y} \PYG{o}{=} \PYG{n}{track}\PYG{p}{(}\PYG{n}{x}\PYG{p}{)}
\PYG{n}{d} \PYG{o}{=} \PYG{l+m+mf}{0.1}
\PYG{n}{x1\PYGZus{}0} \PYG{o}{=} \PYG{o}{\PYGZhy{}}\PYG{l+m+mf}{1.5}
\PYG{n}{x2\PYGZus{}0} \PYG{o}{=} \PYG{n}{x1\PYGZus{}0} \PYG{o}{\PYGZhy{}} \PYG{n}{d}
\PYG{n}{x3\PYGZus{}0} \PYG{o}{=} \PYG{n}{x1\PYGZus{}0} \PYG{o}{\PYGZhy{}} \PYG{l+m+mi}{2}\PYG{o}{*}\PYG{n}{d}
\PYG{n}{plt}\PYG{o}{.}\PYG{n}{plot}\PYG{p}{(}\PYG{n}{x}\PYG{p}{,}\PYG{n}{y}\PYG{p}{,} \PYG{n}{label} \PYG{o}{=} \PYG{l+s+s2}{\PYGZdq{}}\PYG{l+s+s2}{track}\PYG{l+s+s2}{\PYGZdq{}}\PYG{p}{)}
\PYG{n}{plt}\PYG{o}{.}\PYG{n}{scatter}\PYG{p}{(}\PYG{n}{x1\PYGZus{}0}\PYG{p}{,}\PYG{n}{track}\PYG{p}{(}\PYG{n}{x1\PYGZus{}0}\PYG{p}{)}\PYG{p}{,}\PYG{n}{zorder} \PYG{o}{=} \PYG{l+m+mi}{2}\PYG{p}{,}\PYG{n}{label} \PYG{o}{=} \PYG{l+s+sa}{r}\PYG{l+s+s2}{\PYGZdq{}}\PYG{l+s+s2}{\PYGZdl{}x\PYGZus{}1\PYGZdl{}}\PYG{l+s+s2}{\PYGZdq{}}\PYG{p}{)}
\PYG{n}{plt}\PYG{o}{.}\PYG{n}{scatter}\PYG{p}{(}\PYG{n}{x2\PYGZus{}0}\PYG{p}{,}\PYG{n}{track}\PYG{p}{(}\PYG{n}{x2\PYGZus{}0}\PYG{p}{)}\PYG{p}{,}\PYG{n}{zorder} \PYG{o}{=} \PYG{l+m+mi}{2}\PYG{p}{,}\PYG{n}{label} \PYG{o}{=} \PYG{l+s+sa}{r}\PYG{l+s+s2}{\PYGZdq{}}\PYG{l+s+s2}{\PYGZdl{}x\PYGZus{}2\PYGZdl{}}\PYG{l+s+s2}{\PYGZdq{}}\PYG{p}{)}
\PYG{n}{plt}\PYG{o}{.}\PYG{n}{scatter}\PYG{p}{(}\PYG{n}{x3\PYGZus{}0}\PYG{p}{,}\PYG{n}{track}\PYG{p}{(}\PYG{n}{x3\PYGZus{}0}\PYG{p}{)}\PYG{p}{,}\PYG{n}{zorder} \PYG{o}{=} \PYG{l+m+mi}{2}\PYG{p}{,}\PYG{n}{label} \PYG{o}{=} \PYG{l+s+sa}{r}\PYG{l+s+s2}{\PYGZdq{}}\PYG{l+s+s2}{\PYGZdl{}x\PYGZus{}3\PYGZdl{}}\PYG{l+s+s2}{\PYGZdq{}}\PYG{p}{)}
\PYG{n}{plt}\PYG{o}{.}\PYG{n}{legend}\PYG{p}{(}\PYG{p}{)}
\PYG{n}{plt}\PYG{o}{.}\PYG{n}{grid}\PYG{p}{(}\PYG{p}{)}
\PYG{n}{plt}\PYG{o}{.}\PYG{n}{show}\PYG{p}{(}\PYG{p}{)}
\end{sphinxVerbatim}

\end{sphinxuseclass}\end{sphinxVerbatimInput}
\begin{sphinxVerbatimOutput}

\begin{sphinxuseclass}{cell_output}
\noindent\sphinxincludegraphics{{lagrange_1_7_0}.png}

\end{sphinxuseclass}\end{sphinxVerbatimOutput}

\end{sphinxuseclass}
\sphinxAtStartPar
\sphinxstylestrong{✅ Do this}
\begin{enumerate}
\sphinxsetlistlabels{\arabic}{enumi}{enumii}{}{.}%
\item {} 
\sphinxAtStartPar
Write down the equation(s) of constraint. How many coordinates do you actually need?

\item {} 
\sphinxAtStartPar
Write the energies of the system using your generalized coordinates.

\item {} 
\sphinxAtStartPar
Form the Lagrangian and find the equation(s?) of motion (how many are there?)

\item {} 
\sphinxAtStartPar
Are the dynamics of this system different that the dynamics of a system of just one roller coaster car?

\end{enumerate}

\sphinxstepscope


\chapter{7 Sep 23 \sphinxhyphen{} Numerical Integration and More Lagrangians}
\label{\detokenize{content/1_mechanics/lagrange_2:sep-23-numerical-integration-and-more-lagrangians}}\label{\detokenize{content/1_mechanics/lagrange_2::doc}}
\begin{sphinxuseclass}{cell}\begin{sphinxVerbatimInput}

\begin{sphinxuseclass}{cell_input}
\begin{sphinxVerbatim}[commandchars=\\\{\}]
\PYG{k+kn}{import} \PYG{n+nn}{matplotlib}\PYG{n+nn}{.}\PYG{n+nn}{pyplot} \PYG{k}{as} \PYG{n+nn}{plt}
\PYG{k+kn}{import} \PYG{n+nn}{numpy} \PYG{k}{as} \PYG{n+nn}{np}
\PYG{k+kn}{import} \PYG{n+nn}{matplotlib}\PYG{n+nn}{.}\PYG{n+nn}{pyplot} \PYG{k}{as} \PYG{n+nn}{plt}
\PYG{k+kn}{from} \PYG{n+nn}{scipy}\PYG{n+nn}{.}\PYG{n+nn}{integrate} \PYG{k+kn}{import} \PYG{n}{solve\PYGZus{}ivp}
\end{sphinxVerbatim}

\end{sphinxuseclass}\end{sphinxVerbatimInput}

\end{sphinxuseclass}
\sphinxAtStartPar
Now that we have an idea how to find a wealth of interesting ordinary differential equations using lagrangians, we’ll work on building up ways to understand these equations, their solutions and behavior. The issue with this is that \sphinxstylestrong{most ODEs do not have analytical solutions}. That means we can’t write down nice closed\sphinxhyphen{}form solutions for them using trancendental functions. However, don’t despair, because that does not mean there is no solution. In fact, the vast majority of non\sphinxhyphen{}pathological ODEs one might come across in physics are \sphinxstylestrong{garunteed} to have unique solutions (at least for finite time). We can easily calculate these solutions using \sphinxstylestrong{numerical integration}. Next week we’ll also see how we can characterize the behavior of ODEs even without acess to numerical integration.

\sphinxAtStartPar
\sphinxhref{https://en.wikipedia.org/wiki/Numerical\_integration}{Numerical Integration} is a vast and wide topic with lots of different approaches, important nuances, and difficult problems. Some of the most high profile numerical integration was done by NASA’s \sphinxhref{https://education.nationalgeographic.org/resource/women-nasa}{human computers} – a now well\sphinxhyphen{}known story thanks to the film \sphinxhref{https://en.wikipedia.org/wiki/Hidden\_Figures}{Hidden Figures}. Black women formed a core group of these especially talented scientists (including \sphinxhref{https://en.wikipedia.org/wiki/Mary\_Jackson\_(engineer)}{Mary Jackson}, \sphinxhref{https://en.wikipedia.org/wiki/Katherine\_Johnson}{Katherine Johnson}, and \sphinxhref{https://en.wikipedia.org/wiki/Dorothy\_Vaughan}{Dorothy Vaughn}), without whom, John Glenn would not have orbited the Earth in 1962. This is also a very interesting story about the importance of \sphinxhref{https://en.wikipedia.org/wiki/Historically\_black\_colleges\_and\_universities}{Historically Black Colleges and Universities} to American science.


\section{Harmonic Oscillator}
\label{\detokenize{content/1_mechanics/lagrange_2:harmonic-oscillator}}
\sphinxAtStartPar
Let’s start simple with everyone’s favorite differential equation, the simple harmonic oscillator. Recall that we can write the SHO as:
\begin{equation*}
\begin{split}
\ddot{x} =  -\omega_0^2
\end{split}
\end{equation*}
\sphinxAtStartPar
where \(\omega_0^2 = \frac{k}{m}\). This equation is 2nd order, but numerical integration techniques only work on 1st order equations. Thankfully they work on any number of potentially coupled 1st order equations. This means that with a quick change of variables, we can write the SHO as a system of 2 first order equations by introducing a new variable \(v\) equal to the velocity of the oscillator.
\begin{equation*}
\begin{split}
v = \dot{x}
\end{split}
\end{equation*}
\sphinxAtStartPar
Then the accelleration of the oscillator can be written as:
\begin{equation*}
\begin{split}
\dot{v}  = -\omega_0^2
\end{split}
\end{equation*}
\sphinxAtStartPar
This trick for writing higher order differential equations as first order equations is incredibly common.


\section{Setting up to numerically integrate}
\label{\detokenize{content/1_mechanics/lagrange_2:setting-up-to-numerically-integrate}}
\sphinxAtStartPar
We need a few things to numerically integrate using \sphinxcode{\sphinxupquote{solve\_ivp}} in python.


\subsection{1. Derivatives Function}
\label{\detokenize{content/1_mechanics/lagrange_2:derivatives-function}}
\sphinxAtStartPar
First, we need to set up a derivatives function that calculates and returns a list of the values of the first order derivatives given an imput list of current values. These current values represent a location in \sphinxstylestrong{phase space}. Phase Space is a space that contains all the information about the state of an ODE. The simple harmonic oscillator has a 2D phase space since its state is totally defined by its position and velocity.

\sphinxAtStartPar
Here’s what our derivatives function looks like for a SHO:

\begin{sphinxVerbatim}[commandchars=\\\{\}]
\PYG{k}{def} \PYG{n+nf}{diffyqs}\PYG{p}{(}\PYG{n}{t}\PYG{p}{,}\PYG{n}{curr\PYGZus{}vals}\PYG{p}{,} \PYG{n}{omega2}\PYG{p}{)}\PYG{p}{:}
    \PYG{c+c1}{\PYGZsh{} 2 first\PYGZhy{}order differential equations for a SHO}
    \PYG{c+c1}{\PYGZsh{} first 2 arguments are always t and curr\PYGZus{}vals, which are followed by any parameters of your ODEs}
    \PYG{n}{x}\PYG{p}{,} \PYG{n}{v} \PYG{o}{=} \PYG{n}{curr\PYGZus{}vals}   \PYG{c+c1}{\PYGZsh{} unpack current values}
    
    \PYG{n}{vdot} \PYG{o}{=} \PYG{o}{\PYGZhy{}}\PYG{n}{omega2} \PYG{o}{*} \PYG{n}{x} \PYG{c+c1}{\PYGZsh{} calculate derivative}

    \PYG{k}{return} \PYG{n}{v}\PYG{p}{,}\PYG{n}{vdot} \PYG{c+c1}{\PYGZsh{} return derivatives}
\end{sphinxVerbatim}

\sphinxAtStartPar
We will pass this function to our solver, which will give us back integrated solutions of our list of derivatives. So since \(v = \dot{x}\), our solution will return \(x\) first, and \(v\).


\subsection{2. Time Setup}
\label{\detokenize{content/1_mechanics/lagrange_2:time-setup}}
\sphinxAtStartPar
We need to define the time span to solve the ODE for AND the specific times we’d like solution points for. Here it is also convienient to choose a time step \(dt\). Here’s one way we could do this in python:

\begin{sphinxVerbatim}[commandchars=\\\{\}]
\PYG{n}{tmax} \PYG{o}{=} \PYG{l+m+mi}{15}
\PYG{n}{dt} \PYG{o}{=} \PYG{l+m+mf}{0.1}
\PYG{n}{tspan} \PYG{o}{=} \PYG{p}{(}\PYG{l+m+mi}{0}\PYG{p}{,}\PYG{n}{tmax}\PYG{p}{)}         \PYG{c+c1}{\PYGZsh{} time span}
\PYG{n}{t} \PYG{o}{=} \PYG{n}{np}\PYG{o}{.}\PYG{n}{arange}\PYG{p}{(}\PYG{l+m+mi}{0}\PYG{p}{,}\PYG{n}{tmax}\PYG{p}{,}\PYG{n}{dt}\PYG{p}{)} \PYG{c+c1}{\PYGZsh{} specific times to return solutions for}
\end{sphinxVerbatim}


\subsection{3. Parameters and Initial Conditions}
\label{\detokenize{content/1_mechanics/lagrange_2:parameters-and-initial-conditions}}
\sphinxAtStartPar
Since we’re dealing with ODEs, we need to supply an initial condition to be able to solve. The SHO has 2D phase space so we need 2 values for our initial condition. We’ll also define parameter value(s) in this step.

\begin{sphinxVerbatim}[commandchars=\\\{\}]
\PYG{n}{omega2} \PYG{o}{=} \PYG{l+m+mi}{2}
\PYG{n}{initial\PYGZus{}condition} \PYG{o}{=} \PYG{p}{[}\PYG{l+m+mi}{1}\PYG{p}{,} \PYG{l+m+mi}{0}\PYG{p}{]} \PYG{c+c1}{\PYGZsh{} pull back 1m, no initial velocity}
\end{sphinxVerbatim}


\subsection{4. Call Integrator}
\label{\detokenize{content/1_mechanics/lagrange_2:call-integrator}}
\sphinxAtStartPar
Now all we have left to do is to actually use \sphinxcode{\sphinxupquote{solve\_ivp}} to do the integration. The syntax for how to do this is shown below. We also get the oppourtunity to tell \sphinxcode{\sphinxupquote{solve\_ivp}} exactly what numerical integration method we’d like it to use. For now we can think of the integrator as a magic box and choose \sphinxcode{\sphinxupquote{RK45}}, or a Runge\sphinxhyphen{}Kutta 4th order method.

\begin{sphinxVerbatim}[commandchars=\\\{\}]
\PYG{n}{solved} \PYG{o}{=} \PYG{n}{solve\PYGZus{}ivp}\PYG{p}{(}\PYG{n}{diffyqs}\PYG{p}{,}\PYG{n}{tspan}\PYG{p}{,}\PYG{n}{initial\PYGZus{}condition}\PYG{p}{,}\PYG{n}{t\PYGZus{}eval} \PYG{o}{=} \PYG{n}{t}\PYG{p}{,} \PYG{n}{args} \PYG{o}{=} \PYG{p}{(}\PYG{n}{omega2}\PYG{p}{,}\PYG{p}{)}\PYG{p}{,}\PYG{n}{method}\PYG{o}{=}\PYG{l+s+s2}{\PYGZdq{}}\PYG{l+s+s2}{RK45}\PYG{l+s+s2}{\PYGZdq{}}\PYG{p}{)}
\end{sphinxVerbatim}

\sphinxAtStartPar
To access the solution directly, use \sphinxcode{\sphinxupquote{solved.y}}. \sphinxcode{\sphinxupquote{solved.y{[}0{]}}} is the solved for position array and \sphinxcode{\sphinxupquote{solved.y{[}1{]}}} is the velocity array in this case. Now let’s see a full implementation of this below, including some visualization that compares our numerical solution to the analytical solution of the SHO.

\begin{sphinxuseclass}{cell}\begin{sphinxVerbatimInput}

\begin{sphinxuseclass}{cell_input}
\begin{sphinxVerbatim}[commandchars=\\\{\}]
\PYG{c+c1}{\PYGZsh{} 1. Derivatives Function}
\PYG{k}{def} \PYG{n+nf}{diffyqs}\PYG{p}{(}\PYG{n}{t}\PYG{p}{,}\PYG{n}{curr\PYGZus{}vals}\PYG{p}{,} \PYG{n}{omega2}\PYG{p}{)}\PYG{p}{:}
    \PYG{n}{x}\PYG{p}{,} \PYG{n}{v} \PYG{o}{=} \PYG{n}{curr\PYGZus{}vals} 
    \PYG{n}{vdot} \PYG{o}{=} \PYG{o}{\PYGZhy{}}\PYG{n}{omega2} \PYG{o}{*} \PYG{n}{x}
    \PYG{k}{return} \PYG{n}{v}\PYG{p}{,}\PYG{n}{vdot}

\PYG{c+c1}{\PYGZsh{} 2. Time Setup}
\PYG{n}{tmax} \PYG{o}{=} \PYG{l+m+mi}{15}
\PYG{n}{dt} \PYG{o}{=} \PYG{l+m+mf}{0.1}
\PYG{n}{tspan} \PYG{o}{=} \PYG{p}{(}\PYG{l+m+mi}{0}\PYG{p}{,}\PYG{n}{tmax}\PYG{p}{)}
\PYG{n}{t} \PYG{o}{=} \PYG{n}{np}\PYG{o}{.}\PYG{n}{arange}\PYG{p}{(}\PYG{l+m+mi}{0}\PYG{p}{,}\PYG{n}{tmax}\PYG{p}{,}\PYG{n}{dt}\PYG{p}{)}

\PYG{c+c1}{\PYGZsh{} 3. Parameters and Initial Conditions}
\PYG{n}{omega2} \PYG{o}{=} \PYG{l+m+mi}{2}
\PYG{n}{initial\PYGZus{}condition} \PYG{o}{=} \PYG{p}{[}\PYG{l+m+mi}{1}\PYG{p}{,} \PYG{l+m+mi}{0}\PYG{p}{]} 

\PYG{c+c1}{\PYGZsh{} 4. Call Integrator}
\PYG{n}{solved} \PYG{o}{=} \PYG{n}{solve\PYGZus{}ivp}\PYG{p}{(}\PYG{n}{diffyqs}\PYG{p}{,}\PYG{n}{tspan}\PYG{p}{,}\PYG{n}{initial\PYGZus{}condition}\PYG{p}{,}\PYG{n}{t\PYGZus{}eval} \PYG{o}{=} \PYG{n}{t}\PYG{p}{,} \PYG{n}{args} \PYG{o}{=} \PYG{p}{(}\PYG{n}{omega2}\PYG{p}{,}\PYG{p}{)}\PYG{p}{,}\PYG{n}{method}\PYG{o}{=}\PYG{l+s+s2}{\PYGZdq{}}\PYG{l+s+s2}{RK45}\PYG{l+s+s2}{\PYGZdq{}}\PYG{p}{)}

\PYG{c+c1}{\PYGZsh{} 5. Visualization and Comparision to analytical solution}
\PYG{k}{def} \PYG{n+nf}{analytic\PYGZus{}sol}\PYG{p}{(}\PYG{n}{t}\PYG{p}{,}\PYG{n}{omega0}\PYG{p}{,}\PYG{n}{initial\PYGZus{}condition}\PYG{p}{)}\PYG{p}{:}
    \PYG{n}{x0}\PYG{p}{,}\PYG{n}{v0} \PYG{o}{=} \PYG{n}{initial\PYGZus{}condition}
    \PYG{k}{return} \PYG{p}{(}\PYG{n}{v0}\PYG{o}{/}\PYG{n}{omega0}\PYG{p}{)}\PYG{o}{*}\PYG{n}{np}\PYG{o}{.}\PYG{n}{sin}\PYG{p}{(}\PYG{n}{omega0}\PYG{o}{*}\PYG{n}{t}\PYG{p}{)} \PYG{o}{+} \PYG{n}{x0} \PYG{o}{*} \PYG{n}{np}\PYG{o}{.}\PYG{n}{cos}\PYG{p}{(}\PYG{n}{omega0}\PYG{o}{*}\PYG{n}{t}\PYG{p}{)}

\PYG{n}{plt}\PYG{o}{.}\PYG{n}{plot}\PYG{p}{(}\PYG{n}{t}\PYG{p}{,}\PYG{n}{analytic\PYGZus{}sol}\PYG{p}{(}\PYG{n}{t}\PYG{p}{,}\PYG{n}{omega2}\PYG{o}{*}\PYG{o}{*}\PYG{l+m+mf}{0.5}\PYG{p}{,}\PYG{n}{initial\PYGZus{}condition}\PYG{p}{)}\PYG{p}{,}\PYG{n}{label} \PYG{o}{=} \PYG{l+s+s2}{\PYGZdq{}}\PYG{l+s+s2}{Analytic Solution}\PYG{l+s+s2}{\PYGZdq{}}\PYG{p}{,}\PYG{n}{linewidth} \PYG{o}{=} \PYG{l+m+mi}{3}\PYG{p}{)}
\PYG{n}{plt}\PYG{o}{.}\PYG{n}{plot}\PYG{p}{(}\PYG{n}{t}\PYG{p}{,}\PYG{n}{solved}\PYG{o}{.}\PYG{n}{y}\PYG{p}{[}\PYG{l+m+mi}{0}\PYG{p}{]}\PYG{p}{,}\PYG{n}{label} \PYG{o}{=} \PYG{l+s+s2}{\PYGZdq{}}\PYG{l+s+s2}{Numerical Solution}\PYG{l+s+s2}{\PYGZdq{}}\PYG{p}{)}
\PYG{n}{plt}\PYG{o}{.}\PYG{n}{title}\PYG{p}{(}\PYG{l+s+s2}{\PYGZdq{}}\PYG{l+s+s2}{SHO}\PYG{l+s+s2}{\PYGZdq{}}\PYG{p}{)}
\PYG{n}{plt}\PYG{o}{.}\PYG{n}{xlabel}\PYG{p}{(}\PYG{l+s+s2}{\PYGZdq{}}\PYG{l+s+s2}{t}\PYG{l+s+s2}{\PYGZdq{}}\PYG{p}{)}
\PYG{n}{plt}\PYG{o}{.}\PYG{n}{ylabel}\PYG{p}{(}\PYG{l+s+s2}{\PYGZdq{}}\PYG{l+s+s2}{x}\PYG{l+s+s2}{\PYGZdq{}}\PYG{p}{)}
\PYG{n}{plt}\PYG{o}{.}\PYG{n}{legend}\PYG{p}{(}\PYG{p}{)}
\PYG{n}{plt}\PYG{o}{.}\PYG{n}{grid}\PYG{p}{(}\PYG{p}{)}
\PYG{n}{plt}\PYG{o}{.}\PYG{n}{show}\PYG{p}{(}\PYG{p}{)}
\end{sphinxVerbatim}

\end{sphinxuseclass}\end{sphinxVerbatimInput}
\begin{sphinxVerbatimOutput}

\begin{sphinxuseclass}{cell_output}
\noindent\sphinxincludegraphics{{lagrange_2_3_0}.png}

\end{sphinxuseclass}\end{sphinxVerbatimOutput}

\end{sphinxuseclass}
\sphinxAtStartPar
\sphinxstylestrong{✅ Do this}
\begin{enumerate}
\sphinxsetlistlabels{\arabic}{enumi}{enumii}{}{.}%
\item {} 
\sphinxAtStartPar
Plot the numerically calculated trajectory in phase space. Use velocity as the y axis and position as the y axis. What shape does it make?

\item {} 
\sphinxAtStartPar
Add a drag term equal to \(-\beta v\) with \(\beta \in [0,1)\) and numerically integrate again. What does this trajectory look like on x vs t plots and phase space plots?

\end{enumerate}


\section{Back to Paraboloid Paradise}
\label{\detokenize{content/1_mechanics/lagrange_2:back-to-paraboloid-paradise}}
\sphinxAtStartPar
Let’s consider the problem we we’re working on on Tuesday, Where a particle was constrained to move on the surface \(z = r^2\). The EOM we arrived at are complex (\(\ddot{r} = \frac{1}{1 + 4r^2}(- 4rv^2 + r\omega^2 -2gr)\) and \(\ddot{\theta} = -2\frac{v\omega}{r} \))and it was unclear if those ODEs had solutions at all. Now That we’re armed with numerical integration, we can tackle the problem.

\sphinxAtStartPar
\sphinxstylestrong{✅ Do this}

\sphinxAtStartPar
Introduce variables \(v\) and \(\omega\) to use our trick for reducing \(>1\) order differential equations to first order equations to write the equations of motion for this problem as a system of four first order differential equations (shown below).
\begin{equation*}
\begin{split}\dot{r} = ?? \end{split}
\end{equation*}\begin{equation*}
\begin{split}\dot{v} = ??\end{split}
\end{equation*}



\begin{equation*}
\begin{split}\dot{\theta} = ?? \end{split}
\end{equation*}\begin{equation*}
\begin{split}\dot{\omega} = ??\end{split}
\end{equation*}




\sphinxAtStartPar
\sphinxstylestrong{✅ Do this}

\sphinxAtStartPar
Use these equations to correct the \sphinxcode{\sphinxupquote{diffyqs}} function in the cell below.

\begin{sphinxuseclass}{cell}\begin{sphinxVerbatimInput}

\begin{sphinxuseclass}{cell_input}
\begin{sphinxVerbatim}[commandchars=\\\{\}]
\PYG{c+c1}{\PYGZsh{} 1. Derivatives Function}
\PYG{k}{def} \PYG{n+nf}{diffyqs}\PYG{p}{(}\PYG{n}{t}\PYG{p}{,}\PYG{n}{curr\PYGZus{}vals}\PYG{p}{,} \PYG{n}{g}\PYG{p}{)}\PYG{p}{:}

    \PYG{n}{r}\PYG{p}{,} \PYG{n}{v}\PYG{p}{,} \PYG{n}{theta}\PYG{p}{,} \PYG{n}{omega} \PYG{o}{=} \PYG{n}{curr\PYGZus{}vals}
    
    \PYG{n}{vdot} \PYG{o}{=} \PYG{l+m+mi}{0}

    \PYG{n}{omegadot} \PYG{o}{=} \PYG{l+m+mi}{0}

    \PYG{k}{return} \PYG{n}{v}\PYG{p}{,} \PYG{n}{vdot}\PYG{p}{,} \PYG{n}{omega}\PYG{p}{,} \PYG{n}{omegadot} \PYG{c+c1}{\PYGZsh{} solution will return in this order, but integrated (r,v,theta,ω)}

\PYG{c+c1}{\PYGZsh{} 2. Time Setup}
\PYG{n}{tmax} \PYG{o}{=} \PYG{l+m+mi}{40}
\PYG{n}{dt} \PYG{o}{=} \PYG{l+m+mf}{0.01} \PYG{c+c1}{\PYGZsh{} unneccecarily small dt to make plot super smooth}
\PYG{n}{t} \PYG{o}{=} \PYG{n}{np}\PYG{o}{.}\PYG{n}{arange}\PYG{p}{(}\PYG{l+m+mi}{0}\PYG{p}{,}\PYG{n}{tmax}\PYG{p}{,}\PYG{n}{dt}\PYG{p}{)}

\PYG{c+c1}{\PYGZsh{} 3. Parameters and Initial Conditions}
\PYG{n}{g} \PYG{o}{=} \PYG{l+m+mf}{9.81}
\PYG{n}{x0} \PYG{o}{=} \PYG{p}{[}\PYG{l+m+mf}{2.6}\PYG{p}{,}\PYG{l+m+mi}{0}\PYG{p}{,}\PYG{l+m+mi}{0}\PYG{p}{,}\PYG{l+m+mi}{2}\PYG{p}{]} 

\PYG{c+c1}{\PYGZsh{} 4. Call Integrator}
\PYG{n}{solved} \PYG{o}{=} \PYG{n}{solve\PYGZus{}ivp}\PYG{p}{(}\PYG{n}{diffyqs}\PYG{p}{,}\PYG{p}{(}\PYG{l+m+mi}{0}\PYG{p}{,}\PYG{n}{tmax}\PYG{p}{)}\PYG{p}{,}\PYG{n}{x0}\PYG{p}{,}\PYG{n}{t\PYGZus{}eval} \PYG{o}{=} \PYG{n}{t}\PYG{p}{,} \PYG{n}{args} \PYG{o}{=} \PYG{p}{(}\PYG{n}{g}\PYG{p}{,}\PYG{p}{)}\PYG{p}{,}\PYG{n}{method}\PYG{o}{=}\PYG{l+s+s2}{\PYGZdq{}}\PYG{l+s+s2}{RK45}\PYG{l+s+s2}{\PYGZdq{}}\PYG{p}{)}
\end{sphinxVerbatim}

\end{sphinxuseclass}\end{sphinxVerbatimInput}

\end{sphinxuseclass}
\sphinxAtStartPar
\sphinxstylestrong{✅ Do this}
\begin{enumerate}
\sphinxsetlistlabels{\arabic}{enumi}{enumii}{}{.}%
\item {} 
\sphinxAtStartPar
Make r vs t and theta vs t plots of this trajectory. Can you think of what that trajectory would look like in cartesian coordinates? 2. Run the cell below to see what the trajectory looks like in 3D. How does the true trajectory compare to your prediction?

\item {} 
\sphinxAtStartPar
Change the initial condtion to examine the following cases and plot the trajectories in 3d:

\sphinxAtStartPar
a. Particle starts from rest and is let go

\sphinxAtStartPar
b. Particle starts at a given height and is given a low speed (less than needed to orbit)

\sphinxAtStartPar
c. Particle starts at a given height and is given a low speed (more than needed to orbit)

\sphinxAtStartPar
d. Can you find a flat horizontal circular orbit?

\end{enumerate}

\begin{sphinxuseclass}{cell}\begin{sphinxVerbatimInput}

\begin{sphinxuseclass}{cell_input}
\begin{sphinxVerbatim}[commandchars=\\\{\}]
\PYG{k}{def} \PYG{n+nf}{parabaloid}\PYG{p}{(}\PYG{n}{x}\PYG{p}{,}\PYG{n}{y}\PYG{p}{,}\PYG{n}{alpha}\PYG{o}{=}\PYG{l+m+mf}{1.}\PYG{p}{)}\PYG{p}{:}
    \PYG{c+c1}{\PYGZsh{} function of a paraboloid in Cartesian coordinates}
    \PYG{k}{return} \PYG{n}{alpha} \PYG{o}{*} \PYG{p}{(}\PYG{n}{x}\PYG{o}{*}\PYG{o}{*}\PYG{l+m+mi}{2} \PYG{o}{+} \PYG{n}{y}\PYG{o}{*}\PYG{o}{*}\PYG{l+m+mi}{2}\PYG{p}{)}

\PYG{k}{def} \PYG{n+nf}{cylindrical\PYGZus{}to\PYGZus{}cartesian}\PYG{p}{(}\PYG{n}{r}\PYG{p}{,} \PYG{n}{th}\PYG{p}{)}\PYG{p}{:}
    \PYG{c+c1}{\PYGZsh{} convert back to cartesian coordinates for ease of plotting}
    \PYG{n}{r} \PYG{o}{=} \PYG{n}{np}\PYG{o}{.}\PYG{n}{array}\PYG{p}{(}\PYG{n}{r}\PYG{p}{)}
    \PYG{n}{th} \PYG{o}{=} \PYG{n}{np}\PYG{o}{.}\PYG{n}{array}\PYG{p}{(}\PYG{n}{th}\PYG{p}{)}
    \PYG{n}{x} \PYG{o}{=} \PYG{n}{r}\PYG{o}{*}\PYG{n}{np}\PYG{o}{.}\PYG{n}{cos}\PYG{p}{(}\PYG{n}{th}\PYG{p}{)}
    \PYG{n}{y} \PYG{o}{=} \PYG{n}{r}\PYG{o}{*}\PYG{n}{np}\PYG{o}{.}\PYG{n}{sin}\PYG{p}{(}\PYG{n}{th}\PYG{p}{)}
    \PYG{k}{return} \PYG{n}{x}\PYG{p}{,}\PYG{n}{y}\PYG{p}{,}\PYG{n}{parabaloid}\PYG{p}{(}\PYG{n}{x}\PYG{p}{,} \PYG{n}{y}\PYG{p}{)}

\PYG{k}{def} \PYG{n+nf}{plot\PYGZus{}solution}\PYG{p}{(}\PYG{n}{solved}\PYG{p}{)}\PYG{p}{:}
    \PYG{c+c1}{\PYGZsh{} Function to plot the trajectory }

    \PYG{c+c1}{\PYGZsh{} points of the surface to plot}
    \PYG{n}{x} \PYG{o}{=} \PYG{n}{np}\PYG{o}{.}\PYG{n}{linspace}\PYG{p}{(}\PYG{o}{\PYGZhy{}}\PYG{l+m+mf}{2.8}\PYG{p}{,} \PYG{l+m+mf}{2.8}\PYG{p}{,} \PYG{l+m+mi}{50}\PYG{p}{)}
    \PYG{n}{y} \PYG{o}{=} \PYG{n}{np}\PYG{o}{.}\PYG{n}{linspace}\PYG{p}{(}\PYG{o}{\PYGZhy{}}\PYG{l+m+mf}{2.8}\PYG{p}{,} \PYG{l+m+mf}{2.8}\PYG{p}{,} \PYG{l+m+mi}{50}\PYG{p}{)}
    \PYG{n}{alpha} \PYG{o}{=} \PYG{l+m+mi}{1}
    \PYG{c+c1}{\PYGZsh{} construct meshgrid for plotting}
    \PYG{n}{X}\PYG{p}{,} \PYG{n}{Y} \PYG{o}{=} \PYG{n}{np}\PYG{o}{.}\PYG{n}{meshgrid}\PYG{p}{(}\PYG{n}{x}\PYG{p}{,} \PYG{n}{y}\PYG{p}{)}
    \PYG{n}{Z} \PYG{o}{=} \PYG{n}{parabaloid}\PYG{p}{(}\PYG{n}{X}\PYG{p}{,} \PYG{n}{Y}\PYG{p}{,}\PYG{n}{alpha}\PYG{p}{)}

    \PYG{c+c1}{\PYGZsh{} get trajectory in cartesian coords}
    \PYG{n}{xtraj}\PYG{p}{,} \PYG{n}{ytraj}\PYG{p}{,} \PYG{n}{ztraj} \PYG{o}{=} \PYG{n}{cylindrical\PYGZus{}to\PYGZus{}cartesian}\PYG{p}{(}\PYG{n}{solved}\PYG{o}{.}\PYG{n}{y}\PYG{p}{[}\PYG{l+m+mi}{0}\PYG{p}{]}\PYG{p}{,} \PYG{n}{solved}\PYG{o}{.}\PYG{n}{y}\PYG{p}{[}\PYG{l+m+mi}{2}\PYG{p}{]}\PYG{p}{)}

    \PYG{c+c1}{\PYGZsh{} plot plot plot}
    \PYG{n}{fig} \PYG{o}{=} \PYG{n}{plt}\PYG{o}{.}\PYG{n}{figure}\PYG{p}{(}\PYG{n}{figsize} \PYG{o}{=} \PYG{p}{(}\PYG{l+m+mi}{10}\PYG{p}{,}\PYG{l+m+mi}{10}\PYG{p}{)}\PYG{p}{)}
    \PYG{n}{ax} \PYG{o}{=} \PYG{n}{plt}\PYG{o}{.}\PYG{n}{axes}\PYG{p}{(}\PYG{n}{projection}\PYG{o}{=}\PYG{l+s+s1}{\PYGZsq{}}\PYG{l+s+s1}{3d}\PYG{l+s+s1}{\PYGZsq{}}\PYG{p}{)}
    \PYG{n}{plt}\PYG{o}{.}\PYG{n}{title}\PYG{p}{(}\PYG{l+s+s2}{\PYGZdq{}}\PYG{l+s+s2}{Particle}\PYG{l+s+s2}{\PYGZsq{}}\PYG{l+s+s2}{s Path in 3d}\PYG{l+s+s2}{\PYGZdq{}}\PYG{p}{)}
    \PYG{n}{ax}\PYG{o}{.}\PYG{n}{plot\PYGZus{}surface}\PYG{p}{(}\PYG{n}{X}\PYG{p}{,} \PYG{n}{Y}\PYG{p}{,} \PYG{n}{Z}\PYG{p}{,} \PYG{n}{cmap}\PYG{o}{=}\PYG{l+s+s1}{\PYGZsq{}}\PYG{l+s+s1}{binary}\PYG{l+s+s1}{\PYGZsq{}}\PYG{p}{,} \PYG{n}{alpha}\PYG{o}{=}\PYG{l+m+mf}{0.5}\PYG{p}{)} 
    \PYG{n}{ax}\PYG{o}{.}\PYG{n}{plot3D}\PYG{p}{(}\PYG{n}{xtraj}\PYG{p}{,} \PYG{n}{ytraj}\PYG{p}{,} \PYG{n}{ztraj}\PYG{p}{,} \PYG{n}{c} \PYG{o}{=} \PYG{l+s+s2}{\PYGZdq{}}\PYG{l+s+s2}{\PYGZsh{}18453B}\PYG{l+s+s2}{\PYGZdq{}}\PYG{p}{)}
    \PYG{n}{ax}\PYG{o}{.}\PYG{n}{set\PYGZus{}xlim}\PYG{p}{(}\PYG{o}{\PYGZhy{}}\PYG{l+m+mi}{3}\PYG{p}{,} \PYG{l+m+mi}{3}\PYG{p}{)}\PYG{p}{;} \PYG{n}{ax}\PYG{o}{.}\PYG{n}{set\PYGZus{}ylim}\PYG{p}{(}\PYG{o}{\PYGZhy{}}\PYG{l+m+mi}{3}\PYG{p}{,} \PYG{l+m+mi}{3}\PYG{p}{)}\PYG{p}{;} \PYG{n}{ax}\PYG{o}{.}\PYG{n}{set\PYGZus{}zlim}\PYG{p}{(}\PYG{o}{\PYGZhy{}}\PYG{l+m+mi}{1} \PYG{p}{,}\PYG{l+m+mi}{15}\PYG{p}{)}
    \PYG{n}{ax}\PYG{o}{.}\PYG{n}{set\PYGZus{}xlabel}\PYG{p}{(}\PYG{l+s+s1}{\PYGZsq{}}\PYG{l+s+s1}{x}\PYG{l+s+s1}{\PYGZsq{}}\PYG{p}{)}
    \PYG{n}{ax}\PYG{o}{.}\PYG{n}{set\PYGZus{}ylabel}\PYG{p}{(}\PYG{l+s+s1}{\PYGZsq{}}\PYG{l+s+s1}{y}\PYG{l+s+s1}{\PYGZsq{}}\PYG{p}{)}
    \PYG{n}{ax}\PYG{o}{.}\PYG{n}{set\PYGZus{}zlabel}\PYG{p}{(}\PYG{l+s+s1}{\PYGZsq{}}\PYG{l+s+s1}{z}\PYG{l+s+s1}{\PYGZsq{}}\PYG{p}{)}
    \PYG{n}{plt}\PYG{o}{.}\PYG{n}{show}\PYG{p}{(}\PYG{p}{)}

\PYG{n}{plot\PYGZus{}solution}\PYG{p}{(}\PYG{n}{solved}\PYG{p}{)}
\end{sphinxVerbatim}

\end{sphinxuseclass}\end{sphinxVerbatimInput}
\begin{sphinxVerbatimOutput}

\begin{sphinxuseclass}{cell_output}
\noindent\sphinxincludegraphics{{lagrange_2_8_0}.png}

\end{sphinxuseclass}\end{sphinxVerbatimOutput}

\end{sphinxuseclass}

\section{How Does Numerical Integration actually work?}
\label{\detokenize{content/1_mechanics/lagrange_2:how-does-numerical-integration-actually-work}}
\sphinxAtStartPar
A computer understands things like updating individual variables with a change. It turns out this process of updating things in steps is the basis for numerical integration. We need a set of update equations. Making those update equations is effectively choosing our integrator.


\subsection{Update equations}
\label{\detokenize{content/1_mechanics/lagrange_2:update-equations}}
\sphinxAtStartPar
The critical part of numerical integration is approximating the change to variables you are investigating. Going back to our differential equations, we can rewrite them as approximate equation, which a computer understands because it involves discrete steps. How we choose to approximate this update indicates which integration routine we’ve chosen and sets the irreducible error we are stuck with (i.e., \(O((\Delta t)^2)\), \(O((\Delta t)^3)\), etc.)

\sphinxAtStartPar
We will illustrate three approximations to the slope of these functions:
\begin{itemize}
\item {} 
\sphinxAtStartPar
\sphinxstylestrong{Euler\sphinxhyphen{}Cromer (EC)} \sphinxhyphen{} definitely the most intuitive of the approaches, where we approximate the slope with two points separated by \(\Delta t\) in time. It is quick to write, slow to solve, and requires small steps for accurate results. Even so, it fails to integrate periodic motion well because it doesn’t always conserve energy in periodic motion. Turns out it’s the best tool to use when you have random noise added to the model though (e.g., \(\eta_n(\sigma(t))\)). For a first order eqn, \(\dot{x}=f(x,t)\),

\end{itemize}
\begin{equation*}
\begin{split}x(t+\Delta t) = x(t) + \textrm{change} = x(t) + \Delta t \left(f(x(t+\dfrac{1}{2}\Delta t), t+\dfrac{1}{2}\Delta t\right)\end{split}
\end{equation*}\begin{itemize}
\item {} 
\sphinxAtStartPar
\sphinxstylestrong{Runge\sphinxhyphen{}Kutta 2nd order (RK2)} \sphinxhyphen{} just a step above Euler\sphinxhyphen{}Cromer; it uses three points to approximate the slope giving two measures of the slope (hence, 2nd order). It’s not much more complex than Euler\sphinxhyphen{}Cromer, but gives an order of magnitude lower error. It’s a good starting point for simple systems. For a first order eqn, \(\dot{x}=f(x,t)\),

\end{itemize}
\begin{equation*}
\begin{split}k_1 = \Delta t\left(f(x,t)\right),\end{split}
\end{equation*}\begin{equation*}
\begin{split}k_2 =  \Delta t\left(x+\dfrac{1}{2}k_1, t+\dfrac{1}{2}\Delta t\right),\end{split}
\end{equation*}\begin{equation*}
\begin{split}x(t+\Delta t) = x(t) + \textrm{change} = x(t) + k_2\end{split}
\end{equation*}\begin{itemize}
\item {} 
\sphinxAtStartPar
\sphinxstylestrong{Runge Kutta 4th order (RK4)} \sphinxhyphen{} this is the gold standard. Most researchers start with RK4 on most problems. It uses 5 points to build 4 slope profiles and integrates the system in 4 steps. It is highly adaptable and supported – it can be modified to take smaller or longer steps depending on the specific nature of the problem at the time. I mean that it can change step size in the middle of its work; including within the step it is taking presently. For a first order eqn, \(\dot{x}=f(x,t)\),

\end{itemize}
\begin{equation*}
\begin{split}k_1 = \Delta t\left(f(x,t)\right),\end{split}
\end{equation*}\begin{equation*}
\begin{split}k_2 =  \Delta t\left(x+\dfrac{1}{2}k_1, t+\dfrac{1}{2}\Delta t\right),\end{split}
\end{equation*}\begin{equation*}
\begin{split}k_3 =  \Delta t\left(x+\dfrac{1}{2}k_2, t+\dfrac{1}{2}\Delta t\right),\end{split}
\end{equation*}\begin{equation*}
\begin{split}k_4 =  \Delta t\left(x+k_3, t+\Delta t\right),\end{split}
\end{equation*}\begin{equation*}
\begin{split}x(t+\Delta t) = x(t) + \textrm{change} = x(t) + \dfrac{1}{6}\left(k_1 + 2k_2 +2k_3 +k_4\right)\end{split}
\end{equation*}
\sphinxAtStartPar
We don’t expect you memorize these approaches or to derive them, but to understand how they work and what their limitations are.


\section{Numerically analyzing this system}
\label{\detokenize{content/1_mechanics/lagrange_2:numerically-analyzing-this-system}}

\subsection{Analysis of the Energy}
\label{\detokenize{content/1_mechanics/lagrange_2:analysis-of-the-energy}}
\sphinxAtStartPar
We know that this is a system that should conserve energy. There’s no dissipation and there’s only conservatives forces acting to change the speed of the object. The contact forces only change direction, so the system is “conservative”.

\sphinxAtStartPar
Let’s compute the energy:
\begin{equation*}
\begin{split}T = \dfrac{m}{2}\left(\dot{r}^2 + r^2\dot{\theta}^2 + \dot{z}^2\right) = (\dot{r}^2 + r^2\dot{\theta}^2 + 4r^2\dot{r}^2)\end{split}
\end{equation*}\begin{equation*}
\begin{split}U = mgz = mgr^2\end{split}
\end{equation*}
\sphinxAtStartPar
We have all these quantities except \(m\). Let’s divide it out for just set it to 1. The code below computes and plots the energies.

\begin{sphinxuseclass}{cell}\begin{sphinxVerbatimInput}

\begin{sphinxuseclass}{cell_input}
\begin{sphinxVerbatim}[commandchars=\\\{\}]
\PYG{c+c1}{\PYGZsh{} Unpack solution in a convienient way}
\PYG{n}{r}\PYG{p}{,}\PYG{n}{v}\PYG{p}{,}\PYG{n}{theta}\PYG{p}{,}\PYG{n}{omega} \PYG{o}{=} \PYG{n}{solved}\PYG{o}{.}\PYG{n}{y}

\PYG{c+c1}{\PYGZsh{}\PYGZsh{} Kinetic}
\PYG{n}{T} \PYG{o}{=} \PYG{l+m+mf}{0.5}\PYG{o}{*}\PYG{p}{(}\PYG{n}{v}\PYG{o}{*}\PYG{o}{*}\PYG{l+m+mi}{2} \PYG{o}{+} \PYG{n}{r}\PYG{o}{*}\PYG{o}{*}\PYG{l+m+mi}{2} \PYG{o}{*} \PYG{n}{omega}\PYG{o}{*}\PYG{o}{*}\PYG{l+m+mi}{2} \PYG{o}{+} \PYG{l+m+mi}{4} \PYG{o}{*} \PYG{n}{r}\PYG{o}{*}\PYG{o}{*}\PYG{l+m+mi}{2} \PYG{o}{*} \PYG{n}{v}\PYG{o}{*}\PYG{o}{*}\PYG{l+m+mi}{2}\PYG{p}{)}

\PYG{c+c1}{\PYGZsh{}\PYGZsh{} Potetial}
\PYG{n}{U} \PYG{o}{=} \PYG{n}{g}\PYG{o}{*}\PYG{n}{r}\PYG{o}{*}\PYG{o}{*}\PYG{l+m+mi}{2}

\PYG{c+c1}{\PYGZsh{}\PYGZsh{} Total}
\PYG{n}{E}\PYG{o}{=}\PYG{n}{T}\PYG{o}{+}\PYG{n}{U}

\PYG{n}{ax} \PYG{o}{=} \PYG{n}{plt}\PYG{o}{.}\PYG{n}{figure}\PYG{p}{(}\PYG{n}{figsize}\PYG{o}{=}\PYG{p}{(}\PYG{l+m+mi}{10}\PYG{p}{,}\PYG{l+m+mi}{5}\PYG{p}{)}\PYG{p}{)}
\PYG{n}{plt}\PYG{o}{.}\PYG{n}{plot}\PYG{p}{(}\PYG{n}{t}\PYG{p}{,} \PYG{n}{T}\PYG{p}{,} \PYG{n}{label}\PYG{o}{=}\PYG{l+s+s1}{\PYGZsq{}}\PYG{l+s+s1}{Kinetic}\PYG{l+s+s1}{\PYGZsq{}}\PYG{p}{)}
\PYG{n}{plt}\PYG{o}{.}\PYG{n}{plot}\PYG{p}{(}\PYG{n}{t}\PYG{p}{,} \PYG{n}{U}\PYG{p}{,} \PYG{n}{label}\PYG{o}{=}\PYG{l+s+s1}{\PYGZsq{}}\PYG{l+s+s1}{Potential}\PYG{l+s+s1}{\PYGZsq{}}\PYG{p}{)}
\PYG{n}{plt}\PYG{o}{.}\PYG{n}{plot}\PYG{p}{(}\PYG{n}{t}\PYG{p}{,} \PYG{n}{E}\PYG{p}{,} \PYG{n}{label}\PYG{o}{=}\PYG{l+s+s1}{\PYGZsq{}}\PYG{l+s+s1}{Total}\PYG{l+s+s1}{\PYGZsq{}}\PYG{p}{)}
\PYG{n}{plt}\PYG{o}{.}\PYG{n}{legend}\PYG{p}{(}\PYG{p}{)}
\PYG{n}{plt}\PYG{o}{.}\PYG{n}{xlabel}\PYG{p}{(}\PYG{l+s+s1}{\PYGZsq{}}\PYG{l+s+s1}{time}\PYG{l+s+s1}{\PYGZsq{}}\PYG{p}{)}
\PYG{n}{plt}\PYG{o}{.}\PYG{n}{ylabel}\PYG{p}{(}\PYG{l+s+s1}{\PYGZsq{}}\PYG{l+s+s1}{Energy/mass}\PYG{l+s+s1}{\PYGZsq{}}\PYG{p}{)}
\PYG{n}{plt}\PYG{o}{.}\PYG{n}{show}\PYG{p}{(}\PYG{p}{)}
\end{sphinxVerbatim}

\end{sphinxuseclass}\end{sphinxVerbatimInput}
\begin{sphinxVerbatimOutput}

\begin{sphinxuseclass}{cell_output}
\noindent\sphinxincludegraphics{{lagrange_2_11_0}.png}

\end{sphinxuseclass}\end{sphinxVerbatimOutput}

\end{sphinxuseclass}

\section{Angular Momentum Analysis}
\label{\detokenize{content/1_mechanics/lagrange_2:angular-momentum-analysis}}
\sphinxAtStartPar
We argued that the equation
\$\(\dfrac{d}{dt}\left(mr^2\dot{\theta}\right) = 0\)\$
was a statement of conservation of the z\sphinxhyphen{}component of angular momentum.

\sphinxAtStartPar
Recall that angular momentum is a vector quantity and can be conserved in total, but also a given component might be conserved while others are not. Let’s compute the angular momentum and see what the deal is. This will involve taken cross products in cylindrical coordinates (which also obey the right hand rule!).

\sphinxAtStartPar
Starting with the classical relationship:
\begin{equation*}
\begin{split}\dfrac{\mathbf{L}}{m} = \mathbf{r} \times \mathbf{v}\end{split}
\end{equation*}
\sphinxAtStartPar
We can write down position and velocity vectors in general:
\begin{equation*}
\begin{split}\mathbf{r} = r\hat{r} + z\hat{z}\end{split}
\end{equation*}\begin{equation*}
\begin{split}\mathbf{v} = v_r\hat{r} + v_{\theta}\hat{\theta} + v_z \hat{z}\end{split}
\end{equation*}
\sphinxAtStartPar
Let’s take the cross product:
\begin{equation*}
\begin{split}\mathbf{r} \times \mathbf{v} = \left(r\hat{r} + z\hat{z}\right) \times
\left( v_r\hat{r} + v_{\theta}\hat{\theta} + v_z \hat{z} \right)\end{split}
\end{equation*}
\sphinxAtStartPar
Which is
\begin{equation*}
\begin{split}\dfrac{\mathbf{L}}{m} = \mathbf{r} \times \mathbf{v} = -(z v_{\theta})\hat{r} + (z v_r -r v_z)\hat{\theta} + rv_{\theta}\hat{z}\end{split}
\end{equation*}
\sphinxAtStartPar
Or:
\begin{equation*}
\begin{split}\dfrac{L_r}{m} = -(z v_{\theta})\end{split}
\end{equation*}\begin{equation*}
\begin{split}\dfrac{L_{\theta}}{m} = (z v_r -r v_z)\end{split}
\end{equation*}\begin{equation*}
\begin{split}\dfrac{L_z}{m} = rv_{\theta}\end{split}
\end{equation*}
\sphinxAtStartPar
Yep, \(L_z\) just pops out:
\begin{equation*}
\begin{split}\dfrac{L_z}{m} = rv_{\theta} = r^2\dot{\theta}\end{split}
\end{equation*}\begin{equation*}
\begin{split}L_z = m r^2\dot{\theta}\end{split}
\end{equation*}
\sphinxAtStartPar
A good physics question is ‘why?’

\sphinxAtStartPar
Let’s plot it

\sphinxAtStartPar
\sphinxstylestrong{✅ Do this}

\sphinxAtStartPar
Calculate and plot \(L_z /m\) vs \(t\).

\begin{sphinxuseclass}{cell}\begin{sphinxVerbatimInput}

\begin{sphinxuseclass}{cell_input}
\begin{sphinxVerbatim}[commandchars=\\\{\}]
\PYG{c+c1}{\PYGZsh{}\PYGZsh{} your code here}
\end{sphinxVerbatim}

\end{sphinxuseclass}\end{sphinxVerbatimInput}

\end{sphinxuseclass}
\sphinxAtStartPar
\sphinxstylestrong{✅ Do this}

\sphinxAtStartPar
Calculate \(L_r /m\), \(L_\theta / m\), and \(L_{tot} /m\). Plot these against time alongside \(L_z/m\), all on the same plot. What do you observe?

\begin{sphinxuseclass}{cell}\begin{sphinxVerbatimInput}

\begin{sphinxuseclass}{cell_input}
\begin{sphinxVerbatim}[commandchars=\\\{\}]
\PYG{c+c1}{\PYGZsh{} your code here}
\end{sphinxVerbatim}

\end{sphinxuseclass}\end{sphinxVerbatimInput}

\end{sphinxuseclass}

\section{Constraints Revisited \sphinxhyphen{} Lagrange Multipliers}
\label{\detokenize{content/1_mechanics/lagrange_2:constraints-revisited-lagrange-multipliers}}
\sphinxAtStartPar
Thus far we’ve been treating constrained motion problems by including the constraint information in the generalized coordinates themselves, but there is a more general approach for these kinds of problems \sphinxhyphen{} Lagrange Multipliers. Toward finding a modified Euler\sphinxhyphen{}Lagrange equation that deals with these, we can start by looking at a way that one might arrive at the original Euler\sphinxhyphen{}Lagrange equation.

\sphinxAtStartPar
We start By defining \(L = T-V\) and we define the action of our system by:
\begin{equation*}
\begin{split}
S = \int_{t_1}^{t_2}(\mathbf{q},\dot{\mathbf{q}},t) dt
\end{split}
\end{equation*}
\sphinxAtStartPar
Via the principle of least action we know that \(\delta S = 0\) for our ideal path since the \(S\) integral is stationary along that path. If we nudge our generalized coordinates a bit:
\begin{equation*}
\begin{split}
\mathbf{q} \rightarrow \mathbf{q} + \delta \mathbf{q} 
\end{split}
\end{equation*}
\sphinxAtStartPar
This lets us define a small change in the lagrangian:
\begin{equation*}
\begin{split}
\delta L = \frac{\partial L}{\partial \mathbf{q}} \cdot \delta \mathbf{q} + \frac{\partial L}{\partial \dot{\mathbf{q}}}\cdot \delta \dot{\mathbf{q}}
\end{split}
\end{equation*}
\sphinxAtStartPar
Which in\sphinxhyphen{}turn gives us a small change in the action:
\begin{equation*}
\begin{split}
\delta S = \int_{t_1}^{t_2} \left(  \frac{\partial L}{\partial \mathbf{q}} \cdot \delta \mathbf{q} + \frac{\partial L}{\partial \dot{\mathbf{q}}}\cdot \delta \dot{\mathbf{q}} \right) dt 
\end{split}
\end{equation*}
\sphinxAtStartPar
Integrating by parts gives:
\$\(
= \int_{t_1}^{t_2} \left(\frac{\partial L}{\partial \mathbf{q}}\cdot  \delta \mathbf{q} - \frac{d}{dt}\left(\frac{\partial L}{\partial \dot{\mathbf{q}}} \right)\cdot\delta\mathbf{q}  \right)dt + \left[\frac{\partial L}{\partial \dot{\mathbf{q}}}\cdot \delta \mathbf{q}\right]_{t_1}^{t_2}
= \int_{t_1}^{t_2} \left(\frac{\partial L}{\partial \mathbf{q}} - \frac{d}{dt}\left(\frac{\partial L}{\partial \mathbf{q}}\right) \right)\cdot \delta\mathbf{q} dt
\)\$

\sphinxAtStartPar
Which finally lets us argue:
\$\(
\sum_{i=0}^n \left(\frac{\partial L}{\partial q_i} - \frac{d}{dt}\left(\frac{\partial L}{\partial \dot{q}_i}\right)  \right) \delta q_i = 0
\)\$

\sphinxAtStartPar
Since our generalized coordinates are independent, this gives us the original Euler\sphinxhyphen{}Lagrange equation:
\begin{equation*}
\begin{split}
\frac{\partial L}{\partial q_i} - \frac{d}{dt}\left(\frac{\partial L}{\partial \dot{q}_i}\right) = 0
\end{split}
\end{equation*}
\sphinxAtStartPar
But let’s say our coordinates are constrained by some \(f(\mathbf{q},t) = 0\). Noting that \(\delta f = \frac{\partial f}{\partial \mathbf{q}}\cdot \delta \mathbf{q} = 0\), we can modify the above equation to read:
\begin{equation*}
\begin{split}
\sum_{i=0}^n \left(\frac{\partial L}{\partial q_i} - \frac{d}{dt}\left(\frac{\partial L}{\partial \dot{q}_i}\right)  + \lambda \frac{\partial f}{\partial q_i}\right) \delta q_i = 0
\end{split}
\end{equation*}
\sphinxAtStartPar
Where we’ve introduced \(\lambda\), a \sphinxstylestrong{lagrange multiplier}. This gives us a new modified version of the Euler\sphinxhyphen{}Lagrange Equation:
\begin{equation*}
\begin{split}
\frac{\partial L}{\partial q_i} - \frac{d}{dt}\left(\frac{\partial L}{\partial \dot{q}_i}\right)  + \lambda \frac{\partial f}{\partial q_i} = 0
\end{split}
\end{equation*}
\sphinxAtStartPar
The beautiful thing about this equation is that \(\lambda\) itself often will take the form of the constraint force itself (though this is nuanced and depends on the system at hand). So if you do want to know what your constraint forces are \sphinxhyphen{} for example say you’re designing a roller coaster and you need to know the forces that passengers experience for safety reasons \sphinxhyphen{} you can still obtain them via the lagrangian approach, which still lets you bypass the Newtonian framework.

\sphinxAtStartPar
\sphinxstylestrong{✅ Do this}

\sphinxAtStartPar
Use the modified Euler\sphinxhyphen{}Lagrange equation with generalized coordinates \(\mathbf{q} = (r,\theta, z)\) and constraint equation \(f = r^2 - z = 0\) to find the equations of motion for the paraboloid problem. Also solve for \(\lambda\). What is the physical meaning of this lagrange multiplier?

\sphinxAtStartPar
Note: The procedure here is a bit different now, since you impose the constraint \sphinxstylestrong{after} iterating the modified Euler\sphinxhyphen{}Lagrange equation 3 times. Then you’ll have a system of equations to solve for \(\ddot{r}\) and \(\lambda\).

\sphinxstepscope


\chapter{12 Sep 23 \sphinxhyphen{} The Dynamical Systems Approach and Phase Portraits}
\label{\detokenize{content/1_mechanics/dynamical_1:sep-23-the-dynamical-systems-approach-and-phase-portraits}}\label{\detokenize{content/1_mechanics/dynamical_1::doc}}
\sphinxAtStartPar
Up to now, most of your work with models in physics are those you can solve analytically in terms of known functions. Think about solving differential equations that produce polynomials or sines and cosines.

\sphinxAtStartPar
But what happens when the solution to the problem is not obviously tractable in an analytical form. Rather, how can we investigate systems that are new to us?

\sphinxAtStartPar
In today’s activity you will:
\begin{itemize}
\item {} 
\sphinxAtStartPar
Remind yourself how to interpret a 1d phase portrait using the differential equation \(\dot{x} = x^2 + 1\)

\item {} 
\sphinxAtStartPar
Remind yourself how to interpret a 2d phase portrait (phase space plot) using the SHO model

\item {} 
\sphinxAtStartPar
Explain what you see in the phase space figure for the SHO

\item {} 
\sphinxAtStartPar
Develop the ODE for the large angle pendulum

\item {} 
\sphinxAtStartPar
Show how we can recover the SHO using mathematics and graphs

\item {} 
\sphinxAtStartPar
Use an existing program to work with a new system

\item {} 
\sphinxAtStartPar
Explain the insights developed from a phase space plot of the Large Angle Pendulum

\item {} 
\sphinxAtStartPar
(if time) Plot a trajectories in the phase space

\item {} 
\sphinxAtStartPar
(if time) Add damping to the model

\item {} 
\sphinxAtStartPar
(if time) Analyze fixed points for 2d systems

\end{itemize}


\section{1D System}
\label{\detokenize{content/1_mechanics/dynamical_1:d-system}}
\sphinxAtStartPar
Let’s look at a simple 1d system first, given by the differential equation:
\begin{equation*}
\begin{split}
\dot{x} = x^2 - 1
\end{split}
\end{equation*}
\sphinxAtStartPar
Let’s start by looking at what the plot of this differential equation looks like:

\begin{sphinxuseclass}{cell}\begin{sphinxVerbatimInput}

\begin{sphinxuseclass}{cell_input}
\begin{sphinxVerbatim}[commandchars=\\\{\}]
\PYG{k+kn}{import} \PYG{n+nn}{numpy} \PYG{k}{as} \PYG{n+nn}{np}
\PYG{k+kn}{import} \PYG{n+nn}{matplotlib}\PYG{n+nn}{.}\PYG{n+nn}{pyplot} \PYG{k}{as} \PYG{n+nn}{plt}
\PYG{n}{x} \PYG{o}{=} \PYG{n}{np}\PYG{o}{.}\PYG{n}{arange}\PYG{p}{(}\PYG{o}{\PYGZhy{}}\PYG{l+m+mi}{2}\PYG{p}{,}\PYG{l+m+mi}{2}\PYG{p}{,}\PYG{l+m+mf}{0.01}\PYG{p}{)}
\PYG{n}{xdot} \PYG{o}{=} \PYG{n}{x}\PYG{o}{*}\PYG{o}{*}\PYG{l+m+mi}{2} \PYG{o}{\PYGZhy{}} \PYG{l+m+mi}{1}
\PYG{n}{plt}\PYG{o}{.}\PYG{n}{title}\PYG{p}{(}\PYG{l+s+sa}{r}\PYG{l+s+s2}{\PYGZdq{}}\PYG{l+s+s2}{\PYGZdl{}}\PYG{l+s+s2}{\PYGZbs{}}\PYG{l+s+s2}{dot}\PYG{l+s+si}{\PYGZob{}x\PYGZcb{}}\PYG{l+s+s2}{ = x\PYGZca{}2 \PYGZhy{} 1\PYGZdl{}}\PYG{l+s+s2}{\PYGZdq{}}\PYG{p}{)}
\PYG{n}{plt}\PYG{o}{.}\PYG{n}{axvline}\PYG{p}{(}\PYG{n}{x}\PYG{o}{=}\PYG{l+m+mi}{0}\PYG{p}{,} \PYG{n}{c}\PYG{o}{=}\PYG{l+s+s2}{\PYGZdq{}}\PYG{l+s+s2}{black}\PYG{l+s+s2}{\PYGZdq{}}\PYG{p}{)}
\PYG{n}{plt}\PYG{o}{.}\PYG{n}{axhline}\PYG{p}{(}\PYG{n}{y}\PYG{o}{=}\PYG{l+m+mi}{0}\PYG{p}{,} \PYG{n}{c}\PYG{o}{=}\PYG{l+s+s2}{\PYGZdq{}}\PYG{l+s+s2}{black}\PYG{l+s+s2}{\PYGZdq{}}\PYG{p}{)}
\PYG{n}{plt}\PYG{o}{.}\PYG{n}{plot}\PYG{p}{(}\PYG{n}{x}\PYG{p}{,}\PYG{n}{xdot}\PYG{p}{)}
\PYG{n}{plt}\PYG{o}{.}\PYG{n}{grid}\PYG{p}{(}\PYG{p}{)}
\PYG{n}{plt}\PYG{o}{.}\PYG{n}{show}\PYG{p}{(}\PYG{p}{)}
\end{sphinxVerbatim}

\end{sphinxuseclass}\end{sphinxVerbatimInput}
\begin{sphinxVerbatimOutput}

\begin{sphinxuseclass}{cell_output}
\noindent\sphinxincludegraphics{{dynamical_1_2_0}.png}

\end{sphinxuseclass}\end{sphinxVerbatimOutput}

\end{sphinxuseclass}
\sphinxAtStartPar
Since this is a first order equation we have a nice physical way of thinking about it: at position \(x\), a particles velocity is \(x^2 -1\). So when the \(x^2-1\) is positive, the particle’s velocity is positive (or to the right) and vice versa. We can use this knowledge to work out qualitative behavor of our system.

\sphinxAtStartPar
\sphinxstylestrong{✅ Do this}
\begin{enumerate}
\sphinxsetlistlabels{\arabic}{enumi}{enumii}{}{.}%
\item {} 
\sphinxAtStartPar
Consider a trajectory that starts at \(x = -1.5\). Does this trajectory move to the right or left?

\item {} 
\sphinxAtStartPar
What direction does a trajectory that starts at \(x = 0\) go?

\item {} 
\sphinxAtStartPar
What direction does a trajectory that starts at \( x = 1.5\) go?

\item {} 
\sphinxAtStartPar
What trends do you notice? Do trajectories settle down at certain locations or blast off to infinity?

\end{enumerate}


\section{Fixed Points}
\label{\detokenize{content/1_mechanics/dynamical_1:fixed-points}}
\sphinxAtStartPar
The \sphinxstylestrong{fixed points} of a system are where its derivative vanishes, that is \(\dot{\mathbf{x}} = 0\). At any fixed point, a system is constant (why?). In the dynamical systems approach, we often care about charactarizing the behavor of systems near fixed points. In 1d, most fixed points are either stable (they attract trajectories) or unstable (they repel trajectories).

\sphinxAtStartPar
\sphinxstylestrong{✅ Do this}

\sphinxAtStartPar
Find and characterize the fixed points of \(\dot{x} = x^2-1\) as stable or unstable.

\sphinxAtStartPar
Another way we can visualize these is with \sphinxstylestrong{slope fields}. Here we essentially plot the slopes \(\dot{x}(x)\) over and over again for many values of \(t\). The actual solution of the differential equation will be a curve that is always tangent to the local slope. Below we’ve plotted this using both \sphinxcode{\sphinxupquote{plt.quiver}} and \sphinxcode{\sphinxupquote{plt.streamplot}}

\begin{sphinxuseclass}{cell}\begin{sphinxVerbatimInput}

\begin{sphinxuseclass}{cell_input}
\begin{sphinxVerbatim}[commandchars=\\\{\}]
\PYG{k+kn}{import} \PYG{n+nn}{numpy} \PYG{k}{as} \PYG{n+nn}{np}
\PYG{k+kn}{import} \PYG{n+nn}{matplotlib}\PYG{n+nn}{.}\PYG{n+nn}{pyplot} \PYG{k}{as} \PYG{n+nn}{plt}

\PYG{n}{t} \PYG{o}{=} \PYG{n}{np}\PYG{o}{.}\PYG{n}{arange}\PYG{p}{(}\PYG{l+m+mi}{0}\PYG{p}{,} \PYG{l+m+mi}{11}\PYG{p}{,} \PYG{l+m+mf}{0.8}\PYG{p}{)}
\PYG{n}{x} \PYG{o}{=} \PYG{n}{np}\PYG{o}{.}\PYG{n}{arange}\PYG{p}{(}\PYG{o}{\PYGZhy{}}\PYG{l+m+mi}{2}\PYG{p}{,} \PYG{l+m+mi}{2}\PYG{p}{,} \PYG{l+m+mf}{0.4}\PYG{p}{)}

\PYG{c+c1}{\PYGZsh{} Make grid}
\PYG{n}{T}\PYG{p}{,} \PYG{n}{X} \PYG{o}{=} \PYG{n}{np}\PYG{o}{.}\PYG{n}{meshgrid}\PYG{p}{(}\PYG{n}{t}\PYG{p}{,} \PYG{n}{x}\PYG{p}{)}

\PYG{c+c1}{\PYGZsh{} calculate derivative (dt is const so just use ones)}
\PYG{n}{dx} \PYG{o}{=} \PYG{n}{X}\PYG{o}{*}\PYG{o}{*}\PYG{l+m+mi}{2} \PYG{o}{\PYGZhy{}} \PYG{l+m+mi}{1}
\PYG{n}{dt} \PYG{o}{=} \PYG{n}{np}\PYG{o}{.}\PYG{n}{ones}\PYG{p}{(}\PYG{n}{dx}\PYG{o}{.}\PYG{n}{shape}\PYG{p}{)}

\PYG{c+c1}{\PYGZsh{} plot}
\PYG{n}{fig} \PYG{o}{=} \PYG{n}{plt}\PYG{o}{.}\PYG{n}{figure}\PYG{p}{(}\PYG{n}{figsize} \PYG{o}{=} \PYG{p}{(}\PYG{l+m+mi}{10}\PYG{p}{,}\PYG{l+m+mi}{5}\PYG{p}{)}\PYG{p}{)}
\PYG{n}{plt}\PYG{o}{.}\PYG{n}{subplot}\PYG{p}{(}\PYG{l+m+mi}{1}\PYG{p}{,}\PYG{l+m+mi}{2}\PYG{p}{,}\PYG{l+m+mi}{1}\PYG{p}{)}
\PYG{n}{plt}\PYG{o}{.}\PYG{n}{quiver}\PYG{p}{(}\PYG{n}{T}\PYG{p}{,}\PYG{n}{X}\PYG{p}{,}\PYG{n}{dt}\PYG{p}{,}\PYG{n}{dx}\PYG{p}{)}
\PYG{n}{plt}\PYG{o}{.}\PYG{n}{xlabel}\PYG{p}{(}\PYG{l+s+s2}{\PYGZdq{}}\PYG{l+s+s2}{t}\PYG{l+s+s2}{\PYGZdq{}}\PYG{p}{)}
\PYG{n}{plt}\PYG{o}{.}\PYG{n}{ylabel}\PYG{p}{(}\PYG{l+s+s2}{\PYGZdq{}}\PYG{l+s+s2}{x}\PYG{l+s+s2}{\PYGZdq{}}\PYG{p}{)}
\PYG{n}{plt}\PYG{o}{.}\PYG{n}{subplot}\PYG{p}{(}\PYG{l+m+mi}{1}\PYG{p}{,}\PYG{l+m+mi}{2}\PYG{p}{,}\PYG{l+m+mi}{2}\PYG{p}{)}
\PYG{n}{plt}\PYG{o}{.}\PYG{n}{streamplot}\PYG{p}{(}\PYG{n}{T}\PYG{p}{,}\PYG{n}{X}\PYG{p}{,}\PYG{n}{dt}\PYG{p}{,}\PYG{n}{dx}\PYG{p}{)}
\PYG{n}{plt}\PYG{o}{.}\PYG{n}{xlabel}\PYG{p}{(}\PYG{l+s+s2}{\PYGZdq{}}\PYG{l+s+s2}{t}\PYG{l+s+s2}{\PYGZdq{}}\PYG{p}{)}
\PYG{n}{plt}\PYG{o}{.}\PYG{n}{ylabel}\PYG{p}{(}\PYG{l+s+s2}{\PYGZdq{}}\PYG{l+s+s2}{x}\PYG{l+s+s2}{\PYGZdq{}}\PYG{p}{)}
\PYG{n}{plt}\PYG{o}{.}\PYG{n}{suptitle}\PYG{p}{(}\PYG{l+s+s2}{\PYGZdq{}}\PYG{l+s+s2}{Slope Field}\PYG{l+s+s2}{\PYGZdq{}}\PYG{p}{)}
\PYG{n}{plt}\PYG{o}{.}\PYG{n}{show}\PYG{p}{(}\PYG{p}{)}
\end{sphinxVerbatim}

\end{sphinxuseclass}\end{sphinxVerbatimInput}
\begin{sphinxVerbatimOutput}

\begin{sphinxuseclass}{cell_output}
\noindent\sphinxincludegraphics{{dynamical_1_5_0}.png}

\end{sphinxuseclass}\end{sphinxVerbatimOutput}

\end{sphinxuseclass}

\section{The Phase Portrait of the SHO}
\label{\detokenize{content/1_mechanics/dynamical_1:the-phase-portrait-of-the-sho}}
\sphinxAtStartPar
To get this started, let’s remind ourselves of the phase portrait of the SHO. Recall that we separated the second order ODE into two first order ODEs, one for \(x\) and one for \(v_x\),
\begin{equation*}
\begin{split}\dot{x} = v_x\end{split}
\end{equation*}\begin{equation*}
\begin{split}\dot{v}_x=-\omega^2x\end{split}
\end{equation*}
\sphinxAtStartPar
We then map out the phase space with the following conceptual interpretation:
\begin{itemize}
\item {} 
\sphinxAtStartPar
Phase space is a space in which all possible states of the system can be shown
\begin{itemize}
\item {} 
\sphinxAtStartPar
a state is a collection of conditions of the state (it’s known position and velocity in our case)

\end{itemize}

\item {} 
\sphinxAtStartPar
Each state is a unique point in phase space
\begin{itemize}
\item {} 
\sphinxAtStartPar
Think about ordered Cartesian pairs, there’s a pair of numbers for every point in a 2D space

\end{itemize}

\item {} 
\sphinxAtStartPar
Remember that knowing \(x_0\) and \(v_{x,0}\) means we can know \(x(t)\) for all time (for that one trajectory/particular solution) given a linear ODE

\end{itemize}

\sphinxAtStartPar
We map the differential equation to the following conceptual interpretation: \sphinxstylestrong{How the state changes depends on location in phase space.} We can understand this as the time derivative for \(x\) and \(v_x\) change throughout the space.

\sphinxAtStartPar
For our 2D SHO case we are saying that how \(x\) and \(v_x\) change is proportional to the position in space:
\begin{equation*}
\begin{split}\langle \dot{x}, \dot{v}_x \rangle = \langle v_x, -\omega^2 x\rangle\end{split}
\end{equation*}
\sphinxAtStartPar
The process is:
\begin{enumerate}
\sphinxsetlistlabels{\arabic}{enumi}{enumii}{}{.}%
\item {} 
\sphinxAtStartPar
Determine the location(s) of interest (i.e., \(x\), \(v_x\))

\item {} 
\sphinxAtStartPar
Compute the change in those quantities at the location (i.e., calculate \(\dot{x}\) and \(\dot{v}_x\) using our prescribed 1st order ODEs above)

\item {} 
\sphinxAtStartPar
At a given point (\(x_0\), \(v_{x,0}\)), create an arrow the indicates the direction and magnitude of the changes to \(x\) and \(v_x\) at that location.
\begin{itemize}
\item {} 
\sphinxAtStartPar
That arrow represents the local flow of the system at that point

\end{itemize}

\item {} 
\sphinxAtStartPar
Repeat for all points of interest

\item {} 
\sphinxAtStartPar
Plot arrows to demonstrate flow of the solutions in phase space

\end{enumerate}


\subsection{Let’s focus on axes first}
\label{\detokenize{content/1_mechanics/dynamical_1:let-s-focus-on-axes-first}}
\sphinxAtStartPar
We talked about how we can look at the axes (\(x=0\) and \(v_x =0\)) to help get a sense of the flow in phase space. Below, we have some code that does this in two parts:
\begin{enumerate}
\sphinxsetlistlabels{\arabic}{enumi}{enumii}{}{.}%
\item {} 
\sphinxAtStartPar
We created a function to produce arrows of the right length given a line of points

\item {} 
\sphinxAtStartPar
We call that function for each axis and for a line at a diagonal

\end{enumerate}


\subsection{Discussion Question}
\label{\detokenize{content/1_mechanics/dynamical_1:discussion-question}}
\sphinxAtStartPar
\sphinxstylestrong{✅ Do this}
\begin{enumerate}
\sphinxsetlistlabels{\arabic}{enumi}{enumii}{}{.}%
\item {} 
\sphinxAtStartPar
Review the phase portraits below. Talk with your neighbors about how they are constructed.
\begin{itemize}
\item {} 
\sphinxAtStartPar
Work to identify which components of the code look familiar and which you have more questions about

\end{itemize}

\item {} 
\sphinxAtStartPar
Make a fourth plot that looks at the other diagonal line that runs at a 45 degree angle to each axes

\end{enumerate}

\sphinxAtStartPar
\sphinxstylestrong{You should be able to explain what the code is doing.} We avoided using meshgrid here to make this a smaller bit of code.


\subsection{PlotPhaseSpaceAxesSHO}
\label{\detokenize{content/1_mechanics/dynamical_1:plotphasespaceaxessho}}
\sphinxAtStartPar
This function is computing the arrows for a given line of points in phase space. Send it a line of points in two arrays (one for \(x\) and one for \(v_x\)) and it plots the resulting arrows. The code is documented below with comments and then used several times.

\begin{sphinxuseclass}{cell}\begin{sphinxVerbatimInput}

\begin{sphinxuseclass}{cell_input}
\begin{sphinxVerbatim}[commandchars=\\\{\}]
\PYG{k}{def} \PYG{n+nf}{PlotPhaseSpaceAxesSHO}\PYG{p}{(}\PYG{n}{x}\PYG{p}{,} \PYG{n}{vx}\PYG{p}{,} \PYG{n}{N}\PYG{o}{=}\PYG{l+m+mi}{20}\PYG{p}{)}\PYG{p}{:}
    \PYG{l+s+sd}{\PYGZdq{}\PYGZdq{}\PYGZdq{}Takes two one\PYGZhy{}dimensional arrays}
\PYG{l+s+sd}{    and computes the resulting arrow to}
\PYG{l+s+sd}{    represent the flow of the system in }
\PYG{l+s+sd}{    phase space. This code is specifically}
\PYG{l+s+sd}{    designed for the SHO with omega=1\PYGZdq{}\PYGZdq{}\PYGZdq{}}

    \PYG{c+c1}{\PYGZsh{}\PYGZsh{} Map the points to the arrows using the }
    \PYG{c+c1}{\PYGZsh{}\PYGZsh{} 1st order ODEs for the SHO}
    \PYG{c+c1}{\PYGZsh{}\PYGZsh{} Returns two arrays of the same length}
    \PYG{c+c1}{\PYGZsh{}\PYGZsh{} as the inputs}
    \PYG{n}{xdot}\PYG{p}{,} \PYG{n}{vxdot} \PYG{o}{=} \PYG{n}{vx}\PYG{p}{,} \PYG{o}{\PYGZhy{}}\PYG{l+m+mi}{1}\PYG{o}{*}\PYG{n}{x}

    \PYG{c+c1}{\PYGZsh{}\PYGZsh{} Create a figure with a known size}
    \PYG{n}{plt}\PYG{o}{.}\PYG{n}{figure}\PYG{p}{(}\PYG{n}{figsize}\PYG{o}{=}\PYG{p}{(}\PYG{l+m+mi}{10}\PYG{p}{,}\PYG{l+m+mi}{8}\PYG{p}{)}\PYG{p}{)}

    \PYG{c+c1}{\PYGZsh{}\PYGZsh{} Go through all the arrays we created to plot the arrows}
    \PYG{c+c1}{\PYGZsh{}\PYGZsh{} Syntax for arrow is:}
    \PYG{c+c1}{\PYGZsh{}\PYGZsh{} arrow(xpos, ypos, xchange, ychange, other\PYGZus{}parameters)}
    \PYG{k}{for} \PYG{n}{i} \PYG{o+ow}{in} \PYG{n}{np}\PYG{o}{.}\PYG{n}{arange}\PYG{p}{(}\PYG{n}{N}\PYG{p}{)}\PYG{p}{:}
    
        \PYG{n}{plt}\PYG{o}{.}\PYG{n}{arrow}\PYG{p}{(}\PYG{n}{x}\PYG{p}{[}\PYG{n}{i}\PYG{p}{]}\PYG{p}{,} \PYG{n}{vx}\PYG{p}{[}\PYG{n}{i}\PYG{p}{]}\PYG{p}{,} \PYG{n}{xdot}\PYG{p}{[}\PYG{n}{i}\PYG{p}{]}\PYG{p}{,} \PYG{n}{vxdot}\PYG{p}{[}\PYG{n}{i}\PYG{p}{]}\PYG{p}{,} 
                  \PYG{n}{head\PYGZus{}width}\PYG{o}{=}\PYG{l+m+mf}{0.2}\PYG{p}{,} 
                  \PYG{n}{head\PYGZus{}length}\PYG{o}{=}\PYG{l+m+mf}{0.2}\PYG{p}{)}
        \PYG{n}{plt}\PYG{o}{.}\PYG{n}{xlabel}\PYG{p}{(}\PYG{l+s+s1}{\PYGZsq{}}\PYG{l+s+s1}{\PYGZdl{}x\PYGZdl{}}\PYG{l+s+s1}{\PYGZsq{}}\PYG{p}{)}
        \PYG{n}{plt}\PYG{o}{.}\PYG{n}{ylabel}\PYG{p}{(}\PYG{l+s+s1}{\PYGZsq{}}\PYG{l+s+s1}{\PYGZdl{}v\PYGZus{}x\PYGZdl{}}\PYG{l+s+s1}{\PYGZsq{}}\PYG{p}{)}
        
    \PYG{n}{plt}\PYG{o}{.}\PYG{n}{grid}\PYG{p}{(}\PYG{p}{)}
\end{sphinxVerbatim}

\end{sphinxuseclass}\end{sphinxVerbatimInput}

\end{sphinxuseclass}
\begin{sphinxuseclass}{cell}\begin{sphinxVerbatimInput}

\begin{sphinxuseclass}{cell_input}
\begin{sphinxVerbatim}[commandchars=\\\{\}]
\PYG{c+c1}{\PYGZsh{}\PYGZsh{} Plotting along the vx axis}
\PYG{n}{N} \PYG{o}{=} \PYG{l+m+mi}{20}

\PYG{n}{x} \PYG{o}{=} \PYG{n}{np}\PYG{o}{.}\PYG{n}{zeros}\PYG{p}{(}\PYG{n}{N}\PYG{p}{)}
\PYG{n}{vx} \PYG{o}{=} \PYG{n}{np}\PYG{o}{.}\PYG{n}{linspace}\PYG{p}{(}\PYG{o}{\PYGZhy{}}\PYG{l+m+mi}{5}\PYG{p}{,}\PYG{l+m+mi}{6}\PYG{p}{,}\PYG{n}{N}\PYG{p}{)}

\PYG{n}{PlotPhaseSpaceAxesSHO}\PYG{p}{(}\PYG{n}{x}\PYG{p}{,} \PYG{n}{vx}\PYG{p}{,} \PYG{n}{N}\PYG{p}{)}
\end{sphinxVerbatim}

\end{sphinxuseclass}\end{sphinxVerbatimInput}
\begin{sphinxVerbatimOutput}

\begin{sphinxuseclass}{cell_output}
\noindent\sphinxincludegraphics{{dynamical_1_9_0}.png}

\end{sphinxuseclass}\end{sphinxVerbatimOutput}

\end{sphinxuseclass}

\subsection{Plotting along the x axis}
\label{\detokenize{content/1_mechanics/dynamical_1:plotting-along-the-x-axis}}
\begin{sphinxuseclass}{cell}\begin{sphinxVerbatimInput}

\begin{sphinxuseclass}{cell_input}
\begin{sphinxVerbatim}[commandchars=\\\{\}]
\PYG{c+c1}{\PYGZsh{}\PYGZsh{} Plotting along the x axis}
\PYG{n}{N} \PYG{o}{=} \PYG{l+m+mi}{20}

\PYG{n}{x} \PYG{o}{=} \PYG{n}{np}\PYG{o}{.}\PYG{n}{linspace}\PYG{p}{(}\PYG{o}{\PYGZhy{}}\PYG{l+m+mi}{5}\PYG{p}{,}\PYG{l+m+mi}{6}\PYG{p}{,}\PYG{n}{N}\PYG{p}{)}
\PYG{n}{vx} \PYG{o}{=} \PYG{n}{np}\PYG{o}{.}\PYG{n}{zeros}\PYG{p}{(}\PYG{n}{N}\PYG{p}{)}

\PYG{n}{PlotPhaseSpaceAxesSHO}\PYG{p}{(}\PYG{n}{x}\PYG{p}{,} \PYG{n}{vx}\PYG{p}{,} \PYG{n}{N}\PYG{p}{)}
\end{sphinxVerbatim}

\end{sphinxuseclass}\end{sphinxVerbatimInput}
\begin{sphinxVerbatimOutput}

\begin{sphinxuseclass}{cell_output}
\noindent\sphinxincludegraphics{{dynamical_1_11_0}.png}

\end{sphinxuseclass}\end{sphinxVerbatimOutput}

\end{sphinxuseclass}

\subsection{Plotting along the 45 degree line between the x and vx axes}
\label{\detokenize{content/1_mechanics/dynamical_1:plotting-along-the-45-degree-line-between-the-x-and-vx-axes}}
\begin{sphinxuseclass}{cell}\begin{sphinxVerbatimInput}

\begin{sphinxuseclass}{cell_input}
\begin{sphinxVerbatim}[commandchars=\\\{\}]
\PYG{c+c1}{\PYGZsh{}\PYGZsh{} Plotting along the 45 degree line between the x and vx axes}
\PYG{n}{N} \PYG{o}{=} \PYG{l+m+mi}{20}

\PYG{n}{x} \PYG{o}{=} \PYG{n}{np}\PYG{o}{.}\PYG{n}{linspace}\PYG{p}{(}\PYG{o}{\PYGZhy{}}\PYG{l+m+mi}{5}\PYG{p}{,}\PYG{l+m+mi}{6}\PYG{p}{,}\PYG{n}{N}\PYG{p}{)}
\PYG{n}{vx} \PYG{o}{=} \PYG{n}{np}\PYG{o}{.}\PYG{n}{linspace}\PYG{p}{(}\PYG{o}{\PYGZhy{}}\PYG{l+m+mi}{5}\PYG{p}{,}\PYG{l+m+mi}{6}\PYG{p}{,}\PYG{n}{N}\PYG{p}{)}

\PYG{n}{PlotPhaseSpaceAxesSHO}\PYG{p}{(}\PYG{n}{x}\PYG{p}{,} \PYG{n}{vx}\PYG{p}{,} \PYG{n}{N}\PYG{p}{)}
\end{sphinxVerbatim}

\end{sphinxuseclass}\end{sphinxVerbatimInput}
\begin{sphinxVerbatimOutput}

\begin{sphinxuseclass}{cell_output}
\noindent\sphinxincludegraphics{{dynamical_1_13_0}.png}

\end{sphinxuseclass}\end{sphinxVerbatimOutput}

\end{sphinxuseclass}

\subsection{Make a Graph}
\label{\detokenize{content/1_mechanics/dynamical_1:make-a-graph}}
\sphinxAtStartPar
\sphinxstylestrong{✅ Do this}

\sphinxAtStartPar
Make a fourth plot that looks at the other diagonal line that runs at a 45 degree angle to each axes


\section{Phase Portrait of the Simple Harmonic Oscillator}
\label{\detokenize{content/1_mechanics/dynamical_1:phase-portrait-of-the-simple-harmonic-oscillator}}
\sphinxAtStartPar
Below, we have written code that makes a phase portrait from the simple harmonic oscillator. It’s written in terms of three functions that serve three purposes that you might want to modify in your own work:
\begin{itemize}
\item {} 
\sphinxAtStartPar
\sphinxcode{\sphinxupquote{SHOPhasePortrait}} is a function that simply returns the relationship between the locations in phase space and how the phase variables change at that location.

\item {} 
\sphinxAtStartPar
\sphinxcode{\sphinxupquote{ComputeSHOPhase}} is a function that uses that relationship and computes the values of the changes at every location. It returns two arrays, which contain all those values.

\item {} 
\sphinxAtStartPar
\sphinxcode{\sphinxupquote{SHOTrajectory}} is a function that takes a pair of points in space and computes the trajectory in phase space

\end{itemize}

\sphinxAtStartPar
By separating these ideas, we are illustrating the process for computing these phase portraits:
\begin{itemize}
\item {} 
\sphinxAtStartPar
Translate one  \(Nth\) order differential equation to \(N\) 1st order (Done earlier in this case)

\item {} 
\sphinxAtStartPar
Put that into a code so you can compute the value of the changes at a location (\sphinxcode{\sphinxupquote{SHOPhasePotrait}})

\item {} 
\sphinxAtStartPar
Call that computation a bunch to compute it at every location you want (\sphinxcode{\sphinxupquote{ComputeSHOPhase}})

\item {} 
\sphinxAtStartPar
investigate specific trajectories in the space (\sphinxcode{\sphinxupquote{SHOTrajectory}})

\end{itemize}

\sphinxAtStartPar
We can then call these functions can plots the results.

\begin{sphinxuseclass}{cell}\begin{sphinxVerbatimInput}

\begin{sphinxuseclass}{cell_input}
\begin{sphinxVerbatim}[commandchars=\\\{\}]
\PYG{k}{def} \PYG{n+nf}{SHOPhasePortrait}\PYG{p}{(}\PYG{n}{x}\PYG{p}{,} \PYG{n}{vx}\PYG{p}{,} \PYG{n}{omega}\PYG{p}{)}\PYG{p}{:}
    \PYG{l+s+sd}{\PYGZsq{}\PYGZsq{}\PYGZsq{}SHOPhasePortrait returns the value of}
\PYG{l+s+sd}{    the change in the phase variables at a given location}
\PYG{l+s+sd}{    in phase space for the SHO model\PYGZsq{}\PYGZsq{}\PYGZsq{}}
    
    \PYG{n}{xdot}\PYG{p}{,} \PYG{n}{vxdot} \PYG{o}{=} \PYG{p}{[}\PYG{n}{vx}\PYG{p}{,} \PYG{o}{\PYGZhy{}}\PYG{l+m+mi}{1}\PYG{o}{*}\PYG{n}{omega}\PYG{o}{*}\PYG{o}{*}\PYG{l+m+mi}{2}\PYG{o}{*}\PYG{n}{x}\PYG{p}{]} \PYG{c+c1}{\PYGZsh{}\PYGZsh{} Specific to this problem}
    
    \PYG{k}{return} \PYG{n}{xdot}\PYG{p}{,} \PYG{n}{vxdot}

\PYG{k}{def} \PYG{n+nf}{ComputeSHOPhase}\PYG{p}{(}\PYG{n}{X}\PYG{p}{,} \PYG{n}{VX}\PYG{p}{,} \PYG{n}{omega}\PYG{p}{)}\PYG{p}{:}
    \PYG{l+s+sd}{\PYGZsq{}\PYGZsq{}\PYGZsq{}ComputeSHOPhase returns the changes in }
\PYG{l+s+sd}{    the phase variables across a grid of locations}
\PYG{l+s+sd}{    that are specified\PYGZsq{}\PYGZsq{}\PYGZsq{}}
    
    \PYG{c+c1}{\PYGZsh{}\PYGZsh{} Prep the arrays with zeros at the right size}
    \PYG{n}{xdot}\PYG{p}{,} \PYG{n}{vxdot} \PYG{o}{=} \PYG{n}{np}\PYG{o}{.}\PYG{n}{zeros}\PYG{p}{(}\PYG{n}{X}\PYG{o}{.}\PYG{n}{shape}\PYG{p}{)}\PYG{p}{,} \PYG{n}{np}\PYG{o}{.}\PYG{n}{zeros}\PYG{p}{(}\PYG{n}{VX}\PYG{o}{.}\PYG{n}{shape}\PYG{p}{)}

    \PYG{c+c1}{\PYGZsh{}\PYGZsh{} Set the limits of the loop based on how }
    \PYG{c+c1}{\PYGZsh{}\PYGZsh{} many points in the arrays we have}
    \PYG{n}{Xlim}\PYG{p}{,} \PYG{n}{Ylim} \PYG{o}{=} \PYG{n}{X}\PYG{o}{.}\PYG{n}{shape}
    
    \PYG{c+c1}{\PYGZsh{}\PYGZsh{} Calculate the changes at each location and add them to the arrays}
    \PYG{k}{for} \PYG{n}{i} \PYG{o+ow}{in} \PYG{n+nb}{range}\PYG{p}{(}\PYG{n}{Xlim}\PYG{p}{)}\PYG{p}{:}
        \PYG{k}{for} \PYG{n}{j} \PYG{o+ow}{in} \PYG{n+nb}{range}\PYG{p}{(}\PYG{n}{Ylim}\PYG{p}{)}\PYG{p}{:}
            \PYG{n}{xloc} \PYG{o}{=} \PYG{n}{X}\PYG{p}{[}\PYG{n}{i}\PYG{p}{,} \PYG{n}{j}\PYG{p}{]}
            \PYG{n}{yloc} \PYG{o}{=} \PYG{n}{VX}\PYG{p}{[}\PYG{n}{i}\PYG{p}{,} \PYG{n}{j}\PYG{p}{]}
            \PYG{n}{xdot}\PYG{p}{[}\PYG{n}{i}\PYG{p}{,}\PYG{n}{j}\PYG{p}{]}\PYG{p}{,} \PYG{n}{vxdot}\PYG{p}{[}\PYG{n}{i}\PYG{p}{,}\PYG{n}{j}\PYG{p}{]} \PYG{o}{=} \PYG{n}{SHOPhasePortrait}\PYG{p}{(}\PYG{n}{xloc}\PYG{p}{,} \PYG{n}{yloc}\PYG{p}{,} \PYG{n}{omega}\PYG{p}{)}
            
    \PYG{k}{return} \PYG{n}{xdot}\PYG{p}{,} \PYG{n}{vxdot}

\PYG{k}{def} \PYG{n+nf}{SHOTrajectory}\PYG{p}{(}\PYG{n}{x0}\PYG{p}{,} \PYG{n}{vx0}\PYG{p}{,} \PYG{n}{omega}\PYG{p}{,} \PYG{n}{N}\PYG{o}{=}\PYG{l+m+mi}{100}\PYG{p}{)}\PYG{p}{:}
    \PYG{l+s+sd}{\PYGZsq{}\PYGZsq{}\PYGZsq{}SHOTrajectory computes the phase space}
\PYG{l+s+sd}{    trjectory using the analytical forms of the}
\PYG{l+s+sd}{    solution. Note this sloppy analytical approach}
\PYG{l+s+sd}{    only works because the SHO is perfectly sinusoidal.\PYGZsq{}\PYGZsq{}\PYGZsq{}}
    
    \PYG{c+c1}{\PYGZsh{}\PYGZsh{} Only work with one period}
    \PYG{n}{T} \PYG{o}{=} \PYG{l+m+mi}{2}\PYG{o}{*}\PYG{n}{np}\PYG{o}{.}\PYG{n}{pi}\PYG{o}{/}\PYG{n}{omega}
    \PYG{n}{t} \PYG{o}{=} \PYG{n}{np}\PYG{o}{.}\PYG{n}{arange}\PYG{p}{(}\PYG{l+m+mi}{0}\PYG{p}{,}\PYG{n}{T}\PYG{p}{,}\PYG{n}{T}\PYG{o}{/}\PYG{n}{N}\PYG{p}{)}
    
    \PYG{c+c1}{\PYGZsh{}\PYGZsh{} I derived this in general with Acos(wt+phi)}
    \PYG{c+c1}{\PYGZsh{}\PYGZsh{} It\PYGZsq{}s not in general a good approach}
    \PYG{c+c1}{\PYGZsh{}\PYGZsh{} because you are not guaranteed analytical }
    \PYG{c+c1}{\PYGZsh{}\PYGZsh{} closed form trajectories in phase space}
    
    \PYG{n}{phi} \PYG{o}{=} \PYG{n}{np}\PYG{o}{.}\PYG{n}{arctan2}\PYG{p}{(}\PYG{o}{\PYGZhy{}}\PYG{l+m+mi}{1}\PYG{o}{*}\PYG{n}{vx0}\PYG{p}{,} \PYG{n}{omega}\PYG{o}{*}\PYG{n}{x0}\PYG{p}{)} \PYG{c+c1}{\PYGZsh{}\PYGZsh{} arctan(\PYGZhy{}vxo/(omega*x0)) taken correctly for the quadrant}
    \PYG{n}{A} \PYG{o}{=} \PYG{n}{x0}\PYG{o}{/}\PYG{n}{np}\PYG{o}{.}\PYG{n}{cos}\PYG{p}{(}\PYG{n}{phi}\PYG{p}{)}
    \PYG{n}{x\PYGZus{}traj} \PYG{o}{=} \PYG{n}{A}\PYG{o}{*}\PYG{n}{np}\PYG{o}{.}\PYG{n}{cos}\PYG{p}{(}\PYG{n}{omega}\PYG{o}{*}\PYG{n}{t}\PYG{o}{+}\PYG{n}{phi}\PYG{p}{)}
    \PYG{n}{v\PYGZus{}traj} \PYG{o}{=} \PYG{o}{\PYGZhy{}}\PYG{n}{omega}\PYG{o}{*}\PYG{n}{A}\PYG{o}{*}\PYG{n}{np}\PYG{o}{.}\PYG{n}{sin}\PYG{p}{(}\PYG{n}{omega}\PYG{o}{*}\PYG{n}{t}\PYG{o}{+}\PYG{n}{phi}\PYG{p}{)}
    
    \PYG{k}{return} \PYG{n}{x\PYGZus{}traj}\PYG{p}{,} \PYG{n}{v\PYGZus{}traj}
\end{sphinxVerbatim}

\end{sphinxuseclass}\end{sphinxVerbatimInput}

\end{sphinxuseclass}

\subsection{Putting the functions to use}
\label{\detokenize{content/1_mechanics/dynamical_1:putting-the-functions-to-use}}
\sphinxAtStartPar
With these two functions, all we are left to do is specify the size of the space and the grid points (that is where exactly we are computing the changes). We use \sphinxhref{https://numpy.org/doc/stable/reference/generated/numpy.meshgrid.html}{meshgrid} to make those arrays a set of Cartesian coordinates and then send that to our functions.

\sphinxAtStartPar
We then plot the results.

\begin{sphinxuseclass}{cell}\begin{sphinxVerbatimInput}

\begin{sphinxuseclass}{cell_input}
\begin{sphinxVerbatim}[commandchars=\\\{\}]
\PYG{c+c1}{\PYGZsh{}\PYGZsh{} Setting parameters and the phase space variables}

\PYG{n}{omega} \PYG{o}{=} \PYG{l+m+mi}{1}
\PYG{n}{x} \PYG{o}{=} \PYG{n}{np}\PYG{o}{.}\PYG{n}{linspace}\PYG{p}{(}\PYG{o}{\PYGZhy{}}\PYG{l+m+mf}{10.0}\PYG{p}{,} \PYG{l+m+mf}{10.0}\PYG{p}{,} \PYG{l+m+mi}{20}\PYG{p}{)}
\PYG{n}{vx} \PYG{o}{=} \PYG{n}{np}\PYG{o}{.}\PYG{n}{linspace}\PYG{p}{(}\PYG{o}{\PYGZhy{}}\PYG{l+m+mf}{10.0}\PYG{p}{,} \PYG{l+m+mf}{10.0}\PYG{p}{,} \PYG{l+m+mi}{20}\PYG{p}{)}

\PYG{c+c1}{\PYGZsh{}\PYGZsh{} Get back pairs of coordinates for every point in the space}
\PYG{n}{X}\PYG{p}{,} \PYG{n}{VX} \PYG{o}{=} \PYG{n}{np}\PYG{o}{.}\PYG{n}{meshgrid}\PYG{p}{(}\PYG{n}{x}\PYG{p}{,} \PYG{n}{vx}\PYG{p}{)}

\PYG{c+c1}{\PYGZsh{}\PYGZsh{} Run our calculations}
\PYG{n}{xdot}\PYG{p}{,} \PYG{n}{vxdot} \PYG{o}{=} \PYG{n}{ComputeSHOPhase}\PYG{p}{(}\PYG{n}{X}\PYG{p}{,} \PYG{n}{VX}\PYG{p}{,} \PYG{n}{omega}\PYG{p}{)}

\PYG{n}{x0} \PYG{o}{=} \PYG{l+m+mi}{5}
\PYG{n}{vx0} \PYG{o}{=} \PYG{l+m+mi}{3}
\PYG{n}{x\PYGZus{}traj}\PYG{p}{,} \PYG{n}{v\PYGZus{}traj} \PYG{o}{=} \PYG{n}{SHOTrajectory}\PYG{p}{(}\PYG{n}{x0}\PYG{p}{,} \PYG{n}{vx0}\PYG{p}{,} \PYG{n}{omega}\PYG{p}{)}

\PYG{c+c1}{\PYGZsh{}\PYGZsh{} Plot. plot. plot.}
\PYG{n}{ax} \PYG{o}{=} \PYG{n}{plt}\PYG{o}{.}\PYG{n}{figure}\PYG{p}{(}\PYG{n}{figsize}\PYG{o}{=}\PYG{p}{(}\PYG{l+m+mi}{15}\PYG{p}{,}\PYG{l+m+mi}{7}\PYG{p}{)}\PYG{p}{)}
\PYG{n}{plt}\PYG{o}{.}\PYG{n}{subplot}\PYG{p}{(}\PYG{l+m+mi}{1}\PYG{p}{,}\PYG{l+m+mi}{2}\PYG{p}{,}\PYG{l+m+mi}{1}\PYG{p}{)}

\PYG{c+c1}{\PYGZsh{}\PYGZsh{} Plot with Quiver}
\PYG{n}{Q} \PYG{o}{=} \PYG{n}{plt}\PYG{o}{.}\PYG{n}{quiver}\PYG{p}{(}\PYG{n}{X}\PYG{p}{,} \PYG{n}{VX}\PYG{p}{,} \PYG{n}{xdot}\PYG{p}{,} \PYG{n}{vxdot}\PYG{p}{,} \PYG{n}{color}\PYG{o}{=}\PYG{l+s+s1}{\PYGZsq{}}\PYG{l+s+s1}{k}\PYG{l+s+s1}{\PYGZsq{}}\PYG{p}{)}

\PYG{c+c1}{\PYGZsh{}\PYGZsh{} Plot trajectory and the starting location}
\PYG{n}{plt}\PYG{o}{.}\PYG{n}{plot}\PYG{p}{(}\PYG{n}{x\PYGZus{}traj}\PYG{p}{,}\PYG{n}{v\PYGZus{}traj}\PYG{p}{)}
\PYG{n}{plt}\PYG{o}{.}\PYG{n}{plot}\PYG{p}{(}\PYG{n}{x0}\PYG{p}{,} \PYG{n}{vx0}\PYG{p}{,} \PYG{l+s+s1}{\PYGZsq{}}\PYG{l+s+s1}{r*}\PYG{l+s+s1}{\PYGZsq{}}\PYG{p}{,} \PYG{n}{markersize}\PYG{o}{=}\PYG{l+m+mi}{10}\PYG{p}{)}

\PYG{n}{plt}\PYG{o}{.}\PYG{n}{xlabel}\PYG{p}{(}\PYG{l+s+s1}{\PYGZsq{}}\PYG{l+s+s1}{\PYGZdl{}x\PYGZdl{}}\PYG{l+s+s1}{\PYGZsq{}}\PYG{p}{)}
\PYG{n}{plt}\PYG{o}{.}\PYG{n}{ylabel}\PYG{p}{(}\PYG{l+s+s1}{\PYGZsq{}}\PYG{l+s+s1}{\PYGZdl{}v\PYGZus{}x\PYGZdl{}}\PYG{l+s+s1}{\PYGZsq{}}\PYG{p}{)}
\PYG{n}{plt}\PYG{o}{.}\PYG{n}{grid}\PYG{p}{(}\PYG{p}{)}

\PYG{n}{plt}\PYG{o}{.}\PYG{n}{subplot}\PYG{p}{(}\PYG{l+m+mi}{1}\PYG{p}{,}\PYG{l+m+mi}{2}\PYG{p}{,}\PYG{l+m+mi}{2}\PYG{p}{)}
\PYG{c+c1}{\PYGZsh{}\PYGZsh{} Plot with streamplot for subplot}
\PYG{n}{Q} \PYG{o}{=} \PYG{n}{plt}\PYG{o}{.}\PYG{n}{streamplot}\PYG{p}{(}\PYG{n}{X}\PYG{p}{,} \PYG{n}{VX}\PYG{p}{,} \PYG{n}{xdot}\PYG{p}{,} \PYG{n}{vxdot}\PYG{p}{,} \PYG{n}{color}\PYG{o}{=}\PYG{l+s+s1}{\PYGZsq{}}\PYG{l+s+s1}{k}\PYG{l+s+s1}{\PYGZsq{}}\PYG{p}{)}
\PYG{n}{plt}\PYG{o}{.}\PYG{n}{plot}\PYG{p}{(}\PYG{n}{x\PYGZus{}traj}\PYG{p}{,}\PYG{n}{v\PYGZus{}traj}\PYG{p}{)}
\PYG{n}{plt}\PYG{o}{.}\PYG{n}{plot}\PYG{p}{(}\PYG{n}{x0}\PYG{p}{,} \PYG{n}{vx0}\PYG{p}{,} \PYG{l+s+s1}{\PYGZsq{}}\PYG{l+s+s1}{r*}\PYG{l+s+s1}{\PYGZsq{}}\PYG{p}{,} \PYG{n}{markersize}\PYG{o}{=}\PYG{l+m+mi}{10}\PYG{p}{)}

\PYG{n}{plt}\PYG{o}{.}\PYG{n}{xlabel}\PYG{p}{(}\PYG{l+s+s1}{\PYGZsq{}}\PYG{l+s+s1}{\PYGZdl{}x\PYGZdl{}}\PYG{l+s+s1}{\PYGZsq{}}\PYG{p}{)}
\PYG{n}{plt}\PYG{o}{.}\PYG{n}{ylabel}\PYG{p}{(}\PYG{l+s+s1}{\PYGZsq{}}\PYG{l+s+s1}{\PYGZdl{}v\PYGZus{}x\PYGZdl{}}\PYG{l+s+s1}{\PYGZsq{}}\PYG{p}{)}
\PYG{n}{plt}\PYG{o}{.}\PYG{n}{grid}\PYG{p}{(}\PYG{p}{)}
\end{sphinxVerbatim}

\end{sphinxuseclass}\end{sphinxVerbatimInput}
\begin{sphinxVerbatimOutput}

\begin{sphinxuseclass}{cell_output}
\noindent\sphinxincludegraphics{{dynamical_1_18_0}.png}

\end{sphinxuseclass}\end{sphinxVerbatimOutput}

\end{sphinxuseclass}

\subsection{What can phase space help us do?}
\label{\detokenize{content/1_mechanics/dynamical_1:what-can-phase-space-help-us-do}}
\sphinxAtStartPar
\sphinxstylestrong{✅ Do this}

\sphinxAtStartPar
Let’s remember a few things about the SHO.
\begin{enumerate}
\sphinxsetlistlabels{\arabic}{enumi}{enumii}{}{.}%
\item {} 
\sphinxAtStartPar
With your neighbors, list all the things you know about the SHO. Include anything we haven’t discussed (e.g., the energetics of the problem).

\item {} 
\sphinxAtStartPar
Name which of those things you can see in the phase diagram. Which things are you sure you can see? What things don’t seem to be able to be seen from the phase diagram?

\item {} 
\sphinxAtStartPar
What do you remember about the energy of an SHO? Consider a harmonic oscillator in a known trajectory (\(x(t) = A\cos(\omega t)\)). Compute the total (conserved) energy of the oscillator as a function of time.
\begin{itemize}
\item {} 
\sphinxAtStartPar
Explain how your expression for energy conservation can be seen in your phase diagram.

\item {} 
\sphinxAtStartPar
You might try to show analytically that the ellipse above is related to your energy conservation expression

\end{itemize}

\item {} 
\sphinxAtStartPar
What are the fixed points for this system? Can it be classified as stable/unstable or do we need to think of something new?

\end{enumerate}

\sphinxAtStartPar
What do these plots tell you about all potential solutions?


\section{The Large Angle Pendulum}
\label{\detokenize{content/1_mechanics/dynamical_1:the-large-angle-pendulum}}
\sphinxAtStartPar
The Large Angle Pendulum is the first of a number of nonlinear differential equations out there. This one is quite special in that the integral that solves for the period of this pendulum has a name! It’s called and \sphinxhref{https://en.wikipedia.org/wiki/Elliptic\_integral}{Elliptical Integral of the First Kind}. Elliptical because of the nature of the kernel of the integral, which has an elliptic form (in our case, one over the square root of a quantity squared subtracted from one, yes, seriously, we have a name for that).

\sphinxAtStartPar
Here’s the pendulum in all it’s glory.



\sphinxAtStartPar
The analytical solution for the period is given by:
\begin{equation*}
\begin{split}T = 4\sqrt{\dfrac{L}{g}}\int_0^{\pi/2}\dfrac{d\theta}{\sqrt{1-k^2\sin^2(\theta)}}\end{split}
\end{equation*}
\sphinxAtStartPar
To find the period, we have to use some form of computation, even it’s the “well\sphinxhyphen{}known” \sphinxhref{https://en.wikipedia.org/wiki/Elliptic\_integral\#Complete\_elliptic\_integral\_of\_the\_first\_kind}{recurrence relationship} that was used for centuries to compute this integral \sphinxstylestrong{by hand}.

\sphinxAtStartPar
But let’s try to gain insight from the phase space instead. We can {[}derive{]} the differential equation that describes the motion of the pendulum through an angle \(\theta\) thusly:
\begin{equation*}
\begin{split}\ddot{\theta} = -\dfrac{g}{L}\sin(\theta)\end{split}
\end{equation*}
\sphinxAtStartPar
You have a second order differential equation for \(\theta\).


\subsection{Make a new phase portrait}
\label{\detokenize{content/1_mechanics/dynamical_1:make-a-new-phase-portrait}}
\sphinxAtStartPar
\sphinxstylestrong{✅ Do this}

\sphinxAtStartPar
With your partners,
\begin{enumerate}
\sphinxsetlistlabels{\arabic}{enumi}{enumii}{}{.}%
\item {} 
\sphinxAtStartPar
Take the 2nd order ODE and make it two 1st order ODEs (one for \(\theta\) and one for \(\omega=\dot{\theta}\)). Make sure you agree on the analytics.

\item {} 
\sphinxAtStartPar
Add those expressions to the function \sphinxcode{\sphinxupquote{LAPPhasePortrait}}

\item {} 
\sphinxAtStartPar
The rest of the code runs the same as before (we’ve engaged in reproducible and adaptable work!), so make some phase portraits.
\begin{itemize}
\item {} 
\sphinxAtStartPar
What do you notice?

\item {} 
\sphinxAtStartPar
What physics is new?

\item {} 
\sphinxAtStartPar
What physics is old?

\end{itemize}

\item {} 
\sphinxAtStartPar
Play with parameters and build a story for what is going on with the motion.

\end{enumerate}

\begin{sphinxuseclass}{cell}\begin{sphinxVerbatimInput}

\begin{sphinxuseclass}{cell_input}
\begin{sphinxVerbatim}[commandchars=\\\{\}]
\PYG{k}{def} \PYG{n+nf}{LAPPhasePortrait}\PYG{p}{(}\PYG{n}{x}\PYG{p}{,} \PYG{n}{vx}\PYG{p}{,} \PYG{n}{omega0} \PYG{o}{=} \PYG{l+m+mi}{10}\PYG{p}{)}\PYG{p}{:}
    
    \PYG{c+c1}{\PYGZsh{}\PYGZsh{}\PYGZsh{}\PYGZsh{}\PYGZsh{}\PYGZsh{}\PYGZsh{}\PYGZsh{}\PYGZsh{}\PYGZsh{}\PYGZsh{}\PYGZsh{}\PYGZsh{}\PYGZsh{}\PYGZsh{}\PYGZsh{}\PYGZsh{}}
    \PYG{c+c1}{\PYGZsh{}\PYGZsh{} CHANGE THIS \PYGZsh{}\PYGZsh{}}
    \PYG{c+c1}{\PYGZsh{}\PYGZsh{}\PYGZsh{}\PYGZsh{}\PYGZsh{}\PYGZsh{}\PYGZsh{}\PYGZsh{}\PYGZsh{}\PYGZsh{}\PYGZsh{}\PYGZsh{}\PYGZsh{}\PYGZsh{}\PYGZsh{}\PYGZsh{}\PYGZsh{}}
    \PYG{n}{xdot}\PYG{p}{,} \PYG{n}{vxdot} \PYG{o}{=} \PYG{p}{[}\PYG{l+m+mi}{1}\PYG{p}{,} \PYG{l+m+mi}{1}\PYG{p}{]} \PYG{c+c1}{\PYGZsh{}\PYGZsh{} Specific to the problem}
    
    \PYG{k}{return} \PYG{n}{xdot}\PYG{p}{,} \PYG{n}{vxdot}

\PYG{k}{def} \PYG{n+nf}{ComputeLAPPhase}\PYG{p}{(}\PYG{n}{X}\PYG{p}{,} \PYG{n}{VX}\PYG{p}{,} \PYG{n}{omega0}\PYG{p}{)}\PYG{p}{:}
    
    \PYG{n}{xdot}\PYG{p}{,} \PYG{n}{vxdot} \PYG{o}{=} \PYG{n}{np}\PYG{o}{.}\PYG{n}{zeros}\PYG{p}{(}\PYG{n}{X}\PYG{o}{.}\PYG{n}{shape}\PYG{p}{)}\PYG{p}{,} \PYG{n}{np}\PYG{o}{.}\PYG{n}{zeros}\PYG{p}{(}\PYG{n}{VX}\PYG{o}{.}\PYG{n}{shape}\PYG{p}{)}

    \PYG{n}{Xlim}\PYG{p}{,} \PYG{n}{Ylim} \PYG{o}{=} \PYG{n}{X}\PYG{o}{.}\PYG{n}{shape}
    
    \PYG{k}{for} \PYG{n}{i} \PYG{o+ow}{in} \PYG{n+nb}{range}\PYG{p}{(}\PYG{n}{Xlim}\PYG{p}{)}\PYG{p}{:}
        \PYG{k}{for} \PYG{n}{j} \PYG{o+ow}{in} \PYG{n+nb}{range}\PYG{p}{(}\PYG{n}{Ylim}\PYG{p}{)}\PYG{p}{:}
            \PYG{n}{xloc} \PYG{o}{=} \PYG{n}{X}\PYG{p}{[}\PYG{n}{i}\PYG{p}{,} \PYG{n}{j}\PYG{p}{]}
            \PYG{n}{yloc} \PYG{o}{=} \PYG{n}{VX}\PYG{p}{[}\PYG{n}{i}\PYG{p}{,} \PYG{n}{j}\PYG{p}{]}
            \PYG{n}{xdot}\PYG{p}{[}\PYG{n}{i}\PYG{p}{,}\PYG{n}{j}\PYG{p}{]}\PYG{p}{,} \PYG{n}{vxdot}\PYG{p}{[}\PYG{n}{i}\PYG{p}{,}\PYG{n}{j}\PYG{p}{]} \PYG{o}{=} \PYG{n}{LAPPhasePortrait}\PYG{p}{(}\PYG{n}{xloc}\PYG{p}{,} \PYG{n}{yloc}\PYG{p}{,} \PYG{n}{omega0}\PYG{p}{)}
            
    \PYG{k}{return} \PYG{n}{xdot}\PYG{p}{,} \PYG{n}{vxdot}

\PYG{n}{omega0} \PYG{o}{=} \PYG{l+m+mi}{2}
\PYG{n}{N} \PYG{o}{=} \PYG{l+m+mi}{40}

\PYG{n}{x} \PYG{o}{=} \PYG{n}{np}\PYG{o}{.}\PYG{n}{linspace}\PYG{p}{(}\PYG{o}{\PYGZhy{}}\PYG{l+m+mf}{6.0}\PYG{p}{,} \PYG{l+m+mf}{6.0}\PYG{p}{,} \PYG{n}{N}\PYG{p}{)}
\PYG{n}{vx} \PYG{o}{=} \PYG{n}{np}\PYG{o}{.}\PYG{n}{linspace}\PYG{p}{(}\PYG{o}{\PYGZhy{}}\PYG{l+m+mf}{8.0}\PYG{p}{,} \PYG{l+m+mf}{8.0}\PYG{p}{,} \PYG{n}{N}\PYG{p}{)}

\PYG{n}{X}\PYG{p}{,} \PYG{n}{VX} \PYG{o}{=} \PYG{n}{np}\PYG{o}{.}\PYG{n}{meshgrid}\PYG{p}{(}\PYG{n}{x}\PYG{p}{,} \PYG{n}{vx}\PYG{p}{)}

\PYG{n}{xdot}\PYG{p}{,} \PYG{n}{vxdot} \PYG{o}{=} \PYG{n}{ComputeLAPPhase}\PYG{p}{(}\PYG{n}{X}\PYG{p}{,} \PYG{n}{VX}\PYG{p}{,} \PYG{n}{omega0}\PYG{p}{)}

\PYG{n}{ax} \PYG{o}{=} \PYG{n}{plt}\PYG{o}{.}\PYG{n}{figure}\PYG{p}{(}\PYG{n}{figsize}\PYG{o}{=}\PYG{p}{(}\PYG{l+m+mi}{10}\PYG{p}{,}\PYG{l+m+mi}{6}\PYG{p}{)}\PYG{p}{)}
\PYG{n}{Q} \PYG{o}{=} \PYG{n}{plt}\PYG{o}{.}\PYG{n}{quiver}\PYG{p}{(}\PYG{n}{X}\PYG{p}{,} \PYG{n}{VX}\PYG{p}{,} \PYG{n}{xdot}\PYG{p}{,} \PYG{n}{vxdot}\PYG{p}{,} \PYG{n}{color}\PYG{o}{=}\PYG{l+s+s1}{\PYGZsq{}}\PYG{l+s+s1}{k}\PYG{l+s+s1}{\PYGZsq{}}\PYG{p}{)}
\PYG{n}{plt}\PYG{o}{.}\PYG{n}{grid}\PYG{p}{(}\PYG{p}{)}

\PYG{n}{plt}\PYG{o}{.}\PYG{n}{xlabel}\PYG{p}{(}\PYG{l+s+s1}{\PYGZsq{}}\PYG{l+s+s1}{\PYGZdl{}x\PYGZdl{}}\PYG{l+s+s1}{\PYGZsq{}}\PYG{p}{)}
\PYG{n}{plt}\PYG{o}{.}\PYG{n}{ylabel}\PYG{p}{(}\PYG{l+s+s1}{\PYGZsq{}}\PYG{l+s+s1}{\PYGZdl{}v\PYGZus{}x\PYGZdl{}}\PYG{l+s+s1}{\PYGZsq{}}\PYG{p}{)}
\PYG{n}{plt}\PYG{o}{.}\PYG{n}{show}\PYG{p}{(}\PYG{p}{)}
\end{sphinxVerbatim}

\end{sphinxuseclass}\end{sphinxVerbatimInput}
\begin{sphinxVerbatimOutput}

\begin{sphinxuseclass}{cell_output}
\noindent\sphinxincludegraphics{{dynamical_1_22_0}.png}

\end{sphinxuseclass}\end{sphinxVerbatimOutput}

\end{sphinxuseclass}
\sphinxAtStartPar
\sphinxstylestrong{✅ Do this}

\sphinxAtStartPar
Find the fixed points of this system. Are they the same as in the SHO case? If they’re different, what about this system makes it different?


\section{(Time Permitting) Fixed Points in 2 Dimensions}
\label{\detokenize{content/1_mechanics/dynamical_1:time-permitting-fixed-points-in-2-dimensions}}
\sphinxAtStartPar
If we don’t get to this in class today don’t worry, we’ll start out with this on thursday!

\sphinxAtStartPar
Now that we’ve stepped into two dimensional phase space, the amount of interesting geometry that our solutions can have is much greater. In particular, local behavior near fixed points can now exhibit much more complex behavior than just being stable or not, as we can see in this figure.



\sphinxAtStartPar
So how do we definitively say what the behavior of our system is like near a fixed point? The anwer lies in \sphinxstylestrong{linearization}. A linear system in 2d is a system of the form:
\begin{equation*}
\begin{split}
\dot{x} = ax + by \hspace{1in} \dot{y} = cx + dy
\end{split}
\end{equation*}
\sphinxAtStartPar
Note we can convieniently write this in matrix notation:
\begin{equation*}
\begin{split}
\dot{\mathbf{x}} = A\mathbf{x}
\end{split}
\end{equation*}
\sphinxAtStartPar
Where
\begin{equation*}
\begin{split}
A = 
\begin{bmatrix}
a & b \\
c & d
\end{bmatrix}
\hspace{0.5in}\text{and} \hspace{0.5in} \mathbf{x} = \begin{bmatrix} x \\ y \end{bmatrix}
\end{split}
\end{equation*}
\sphinxAtStartPar
These systems are understood quite well and play very nice mathematically (see strogatz chapter 5). Miraculously, the mathematical tools for classifying fixed points of linear systems carry over into nonlinear systems with little tweaking. This is because most nonlinear systems behave in linear ways near fixed points. This means we can \sphinxstylestrong{linearize} nonlinear systems. Here’s how you do it:

\sphinxAtStartPar
Suppose you have a system given by:
\begin{equation*}
\begin{split}
\dot{x} = f(x,y)
\end{split}
\end{equation*}\begin{equation*}
\begin{split}
\dot{y} = g(x,y)
\end{split}
\end{equation*}
\sphinxAtStartPar
With fixed point \((x^*,y^*)\). Near this fixed point, distrubances to the system will evolve approximatley according to:
\begin{equation*}
\begin{split}
\begin{bmatrix} \dot{x} \\ \dot{y} \end{bmatrix} = \begin{bmatrix} \frac{\partial f}{\partial x} & \frac{\partial f}{\partial y} \\ \frac{\partial g}{\partial x} & \frac{\partial g}{\partial y}\end{bmatrix} \begin{bmatrix}x - x^* \\ y - y^*\end{bmatrix}
\end{split}
\end{equation*}
\sphinxAtStartPar
This is called the \sphinxstylestrong{linearized system}. The matrix
\begin{equation*}
\begin{split}
A = \begin{bmatrix} \frac{\partial f}{\partial x} & \frac{\partial f}{\partial y} \\ \frac{\partial g}{\partial x} & \frac{\partial g}{\partial y}\end{bmatrix}_{(x^*,y^*)}
\end{split}
\end{equation*}
\sphinxAtStartPar
is the \sphinxstylestrong{Jacobian matrix} at this fixed point, which is the multivariable version of the derivative. To categorize the fixed points of a given system at a fixed point, calculate \(A\) for said fixed point, then find its Eigenvalues. Recall Eigenvalues are the \(\lambda\) in \(A\mathbf{v} = \lambda \mathbf{v}\). \(2\times2\) matricies have (up to) 2 eigenvalues. These eigenvalues tell you about the stability of the system:
\begin{itemize}
\item {} 
\sphinxAtStartPar
\(\mathrm{Re}(\lambda) > 0 \) for both eigenvalues: Repeller/Source (unstable)

\item {} 
\sphinxAtStartPar
\(\mathrm{Re}(\lambda) < 0 \) for both eigenvalues: Attractor/Sink  (stable)

\item {} 
\sphinxAtStartPar
One eigenvalue positive, one negative: Saddle

\item {} 
\sphinxAtStartPar
Both eigenvalues pure imaginary: Center

\end{itemize}

\sphinxAtStartPar
In fact one can learn quite a bit more from a these eigenvalues (see strogatz chapter 6 or section 5.4 \sphinxhref{https://users.math.msu.edu/users/gnagy/teaching/ode.pdf}{here}), but these charactarizations are a great starting point.

\sphinxAtStartPar
\sphinxstylestrong{✅ Do this}

\sphinxAtStartPar
Calculate the Jacobian matrix \(A\) for a fixed point of the large angle pendulum.



\sphinxstepscope


\chapter{14 Sep 23 \sphinxhyphen{} Dynamical Systems Continued}
\label{\detokenize{content/1_mechanics/dynamical_2:sep-23-dynamical-systems-continued}}\label{\detokenize{content/1_mechanics/dynamical_2::doc}}
\sphinxAtStartPar
Last time we investigated the phase portrait of the large angle pendulum, we we could arrive at by re\sphinxhyphen{}writing the differential equation
\begin{equation*}
\begin{split}
\ddot{\theta} = -\dfrac{g}{L}\sin(\theta)
\end{split}
\end{equation*}
\sphinxAtStartPar
as 2 first\sphinxhyphen{}order differential equations:
\begin{equation*}
\begin{split}
\dot{\theta} = \omega \hspace{0.5in}\text{and}\hspace{0.5in} \dot{\omega} = -\frac{g}{L}\sin(\theta)
\end{split}
\end{equation*}
\sphinxAtStartPar
By setting both of these equations equal to zero simultaneously, we also argued that this system has (\sphinxhref{https://faculty.math.illinois.edu/~kapovich/417-16/card.pdf}{countably}) infinite fixed points at \((n\pi, 0)\) for  \(n\in \mathbb{Z}\) in \((\theta,\omega)\) phase space.

\sphinxAtStartPar
Now we turn to the challenge of characterizing these fixed points with the linearization of the system (see the end of tuesday’s activiy for some more notes on this). Recall that we can do this by finding the eigenvalues of the Jacobian Matrix of the system at its fixed point. For the system \(\dot{x} = f(x,y)\), \(\dot{y} = g(x,y)\) the jacobian matrix looks like this:
\begin{equation*}
\begin{split}
A = \begin{bmatrix} \frac{\partial f}{\partial x} & \frac{\partial f}{\partial y} \\ \frac{\partial g}{\partial x} & \frac{\partial g}{\partial y}\end{bmatrix}_{(x^*,y^*)}
\end{split}
\end{equation*}
\sphinxAtStartPar
\sphinxstylestrong{✅ Do this}
(this is the same problem as the last problem from tuesday)

\sphinxAtStartPar
Calculate the general Jacobian matrix \(A\) for this system, then calculate what it is at the fixed point \((0,0)\).

\sphinxAtStartPar
We have the Jacobian at \((0,0)\) now but we still need to find its eigenvalues. Let’s take a quick detour to remember how to do that.


\section{Eigenvalues}
\label{\detokenize{content/1_mechanics/dynamical_2:eigenvalues}}
\sphinxAtStartPar
Eigenvalues and the closely related Eigenvectors are indispensible in physics, math, and computational science. These ideas for the basis (pun somewhat intened) for countless problems, from the \sphinxhref{https://phys.libretexts.org/Bookshelves/Nuclear\_and\_Particle\_Physics/Introduction\_to\_Applied\_Nuclear\_Physics\_(Cappellaro)/02\%3A\_Introduction\_to\_Quantum\_Mechanics/2.04\%3A\_Energy\_Eigenvalue\_Problem}{energy eigenvalue equation} that is the founcation of quantum mechanics, to the stability of complex nonlinear systems, to Normal Modes of oscillators, which we’ll study later in this course, eigenproblems show up all over in physics. I can’t resist a brief tangent: Once some scientists were using an eigenvalue driven algorithm called principal component analysis to study the genes of people that live in Europe. They found that these egenvalues/vectors reproduced a map of Europe with surprising accuracy (\sphinxhref{https://www.ncbi.nlm.nih.gov/pmc/articles/PMC2735096/}{link}). So these tools are extremely, and often unreasonably powerful.

\sphinxAtStartPar
Eigenvalues are the \(\lambda\) in the equation:
\begin{equation*}
\begin{split}
A\mathbf{v} = \lambda \mathbf{v}
\end{split}
\end{equation*}
\sphinxAtStartPar
Where \(A\) is a linear operator of the vector space that \(\mathbf{v}\) lives in. In finite\sphinxhyphen{}dimensional vector spaces, like what we’re considering today, these linear operators are always matricies. There is a bit of physical intuition behind this equation: An eigenvector of \(A\) is a vector that only gets stretched or squished by \(\lambda\) when \(A\) acts on \(\mathbf{v}\). Here’s a gif from Grant Sanderson’s \sphinxhref{https://www.youtube.com/watch?v=PFDu9oVAE-g}{fantastic video} on eigenvalues and eigenvectors that shows this:




\subsection{Finding Eigenvalues}
\label{\detokenize{content/1_mechanics/dynamical_2:finding-eigenvalues}}
\sphinxAtStartPar
To actually find the eigenvalues of a matrix, you solve the \sphinxstylestrong{characteristic polynomial} of the matrix, which you obtain by solving the equation:
\begin{equation*}
\begin{split}
|A - \lambda I | = 0 
\end{split}
\end{equation*}
\sphinxAtStartPar
Where the vertical bars means determinant.

\sphinxAtStartPar
To find Eigenvectors, simply plug in the values you found for \(\lambda\) into the original eigenvalue equation \(A\mathbf{v} = \lambda \mathbf{v}\), using \(\mathbf{v} = \begin{bmatrix}x \\ y\end{bmatrix}\). You’ll find some simple relationship between \(x\) and \(y\). Any scalar multiple of an eigenvector is also an eigenvector so we usually just choose the simplest one. Say if you found that \(x = -y\). Then for a nice clean looking eigenvector you could choose \(\begin{bmatrix} -1 \\ 1\end{bmatrix}\).

\sphinxAtStartPar
\sphinxstylestrong{✅ Do this}

\sphinxAtStartPar
Analytically, find the eigenvalues of the Jacobian matrix you calculated earlier. Use the below bullets to identify these eigenvalues with the type of the fixed point.
\begin{itemize}
\item {} 
\sphinxAtStartPar
\(\mathrm{Re}(\lambda) > 0 \) for both eigenvalues: Repeller/Source (unstable)

\item {} 
\sphinxAtStartPar
\(\mathrm{Re}(\lambda) < 0 \) for both eigenvalues: Attractor/Sink  (stable)

\item {} 
\sphinxAtStartPar
One eigenvalue positive, one negative: Saddle

\item {} 
\sphinxAtStartPar
Both eigenvalues pure imaginary: Center

\end{itemize}

\sphinxAtStartPar
Note: You can actually learn quite a bit more from this analysis, see Strogatz chaper 6.


\subsection{Eigenvalues, Computationally}
\label{\detokenize{content/1_mechanics/dynamical_2:eigenvalues-computationally}}
\sphinxAtStartPar
We can use \sphinxcode{\sphinxupquote{np.linalg.eig()}} to find the eigenvalues (and normalized eigenvectors) of a matrix which we represent as numpy array. Below is some doe that does this (note the imaginary unit is represented as \(j\) in python):

\begin{sphinxuseclass}{cell}\begin{sphinxVerbatimInput}

\begin{sphinxuseclass}{cell_input}
\begin{sphinxVerbatim}[commandchars=\\\{\}]
\PYG{k+kn}{import} \PYG{n+nn}{numpy} \PYG{k}{as} \PYG{n+nn}{np}
\PYG{n}{A} \PYG{o}{=} \PYG{n}{np}\PYG{o}{.}\PYG{n}{array}\PYG{p}{(}\PYG{p}{[}\PYG{p}{[}\PYG{l+m+mi}{0}\PYG{p}{,}\PYG{l+m+mi}{1}\PYG{p}{]}\PYG{p}{,}\PYG{p}{[}\PYG{o}{\PYGZhy{}}\PYG{l+m+mi}{1}\PYG{p}{,}\PYG{l+m+mi}{0}\PYG{p}{]}\PYG{p}{]}\PYG{p}{)}
\PYG{n}{eigvals} \PYG{o}{=} \PYG{n}{np}\PYG{o}{.}\PYG{n}{linalg}\PYG{o}{.}\PYG{n}{eig}\PYG{p}{(}\PYG{n}{A}\PYG{p}{)}\PYG{p}{[}\PYG{l+m+mi}{0}\PYG{p}{]}
\PYG{n}{eigvecs} \PYG{o}{=} \PYG{n}{np}\PYG{o}{.}\PYG{n}{linalg}\PYG{o}{.}\PYG{n}{eig}\PYG{p}{(}\PYG{n}{A}\PYG{p}{)}\PYG{p}{[}\PYG{l+m+mi}{1}\PYG{p}{]}

\PYG{n+nb}{print}\PYG{p}{(}\PYG{l+s+s2}{\PYGZdq{}}\PYG{l+s+s2}{eigenvalues:}\PYG{l+s+s2}{\PYGZdq{}}\PYG{p}{,} \PYG{n}{eigvals}\PYG{p}{)}
\end{sphinxVerbatim}

\end{sphinxuseclass}\end{sphinxVerbatimInput}
\begin{sphinxVerbatimOutput}

\begin{sphinxuseclass}{cell_output}
\begin{sphinxVerbatim}[commandchars=\\\{\}]
eigenvalues: [0.+1.j 0.\PYGZhy{}1.j]
\end{sphinxVerbatim}

\end{sphinxuseclass}\end{sphinxVerbatimOutput}

\end{sphinxuseclass}
\sphinxAtStartPar
This can be super handy when you just need to do some quick caracterization from the eigenvalues of a matrix. However, be warned \sphinxhyphen{} since you only get numerical answers you can lose quite a bit of the nuance that comes from if you had calculated these. We’ll see how that can be an issue later in the semester when we tackle normal modes.


\section{Activity: Lotka \sphinxhyphen{} Volterra Equations}
\label{\detokenize{content/1_mechanics/dynamical_2:activity-lotka-volterra-equations}}
\sphinxAtStartPar
The Lotka \sphinxhyphen{}Volterra Equations are a pair of coupled ODEs
\begin{equation*}
\begin{split}\dot{x}= x(A − Bx - Cy)\end{split}
\end{equation*}\begin{equation*}
\begin{split}\dot{y}= y(D - Fx - Gy)\end{split}
\end{equation*}
\sphinxAtStartPar
with \(A,B,C,D,F,G > 0\)

\sphinxAtStartPar
That model the time evolution of the competition between two species, say rabbits and sheep. We’ll say \(x\) is the number of rabbits while \(y\) is the number of sheep. This model reduces to the logistic growth model if we were to ignore the competition, say if \(\dot{x}= x(A − Bx)\)

\sphinxAtStartPar
\sphinxstylestrong{✅ Do this}
\begin{enumerate}
\sphinxsetlistlabels{\arabic}{enumi}{enumii}{}{.}%
\item {} 
\sphinxAtStartPar
What do each of the parameters \(A,B,C,D,F,G\) represent? Why do you say so?

\item {} 
\sphinxAtStartPar
Identify the fixed points of this system (there might be more than 2!)

\item {} 
\sphinxAtStartPar
Find the Jacobian for these equations

\item {} 
\sphinxAtStartPar
Modify the starter code below to so it gives you the eigenvalues of the jacobian for a given \(A,B,C,D,x^*,y^*\).

\item {} 
\sphinxAtStartPar
For the set of values of \(A,B,C,D,F,G\) given in the code below, sketch what you expect the phase portrait of this system to look like. Then run the the code 2 cells below to see how well you did.

\item {} 
\sphinxAtStartPar
Experiment with choosing different values for A,B,C,D,F,G. Does the behavior of the system change for with different choices? (the initial values given below should be a good starting point).

\end{enumerate}

\begin{sphinxuseclass}{cell}\begin{sphinxVerbatimInput}

\begin{sphinxuseclass}{cell_input}
\begin{sphinxVerbatim}[commandchars=\\\{\}]
\PYG{k}{def} \PYG{n+nf}{jacobian}\PYG{p}{(}\PYG{n}{A}\PYG{p}{,}\PYG{n}{B}\PYG{p}{,}\PYG{n}{C}\PYG{p}{,}\PYG{n}{D}\PYG{p}{,}\PYG{n}{x}\PYG{p}{,}\PYG{n}{y}\PYG{p}{)}\PYG{p}{:}
    \PYG{k}{return} \PYG{n}{np}\PYG{o}{.}\PYG{n}{array}\PYG{p}{(}\PYG{p}{[}\PYG{p}{[}\PYG{l+m+mi}{0}\PYG{p}{,}\PYG{l+m+mi}{0}\PYG{p}{]}\PYG{p}{,}\PYG{p}{[}\PYG{l+m+mi}{0}\PYG{p}{,}\PYG{l+m+mi}{0}\PYG{p}{]}\PYG{p}{]}\PYG{p}{)} \PYG{c+c1}{\PYGZsh{} CHANGE}

\PYG{n}{A}\PYG{p}{,}\PYG{n}{B}\PYG{p}{,}\PYG{n}{C}\PYG{p}{,}\PYG{n}{D}\PYG{p}{,}\PYG{n}{F}\PYG{p}{,}\PYG{n}{G} \PYG{o}{=} \PYG{l+m+mi}{3}\PYG{p}{,}\PYG{l+m+mi}{1}\PYG{p}{,}\PYG{l+m+mi}{2}\PYG{p}{,}\PYG{l+m+mi}{2}\PYG{p}{,}\PYG{l+m+mi}{1}\PYG{p}{,}\PYG{l+m+mi}{1}
\PYG{n}{x1}\PYG{p}{,}\PYG{n}{y1} \PYG{o}{=} \PYG{l+m+mi}{0}\PYG{p}{,}\PYG{l+m+mi}{0} \PYG{c+c1}{\PYGZsh{} 1st fixed point}
\PYG{n}{x2}\PYG{p}{,}\PYG{n}{y2} \PYG{o}{=} \PYG{l+m+mi}{0}\PYG{p}{,}\PYG{l+m+mi}{0} \PYG{c+c1}{\PYGZsh{} 2nd fixed point CHANGE}
\PYG{c+c1}{\PYGZsh{} more fixed points here...}

\PYG{n+nb}{print}\PYG{p}{(}\PYG{l+s+s2}{\PYGZdq{}}\PYG{l+s+s2}{eigenvalues, 1st fixed point:}\PYG{l+s+s2}{\PYGZdq{}}\PYG{p}{,}\PYG{n}{np}\PYG{o}{.}\PYG{n}{linalg}\PYG{o}{.}\PYG{n}{eig}\PYG{p}{(}\PYG{n}{jacobian}\PYG{p}{(}\PYG{n}{A}\PYG{p}{,}\PYG{n}{B}\PYG{p}{,}\PYG{n}{C}\PYG{p}{,}\PYG{n}{D}\PYG{p}{,}\PYG{n}{x1}\PYG{p}{,}\PYG{n}{y1}\PYG{p}{)}\PYG{p}{)}\PYG{p}{[}\PYG{l+m+mi}{0}\PYG{p}{]}\PYG{p}{)}
\PYG{n+nb}{print}\PYG{p}{(}\PYG{l+s+s2}{\PYGZdq{}}\PYG{l+s+s2}{eigenvalues, 2nd fixed point:}\PYG{l+s+s2}{\PYGZdq{}}\PYG{p}{,}\PYG{n}{np}\PYG{o}{.}\PYG{n}{linalg}\PYG{o}{.}\PYG{n}{eig}\PYG{p}{(}\PYG{n}{jacobian}\PYG{p}{(}\PYG{n}{A}\PYG{p}{,}\PYG{n}{B}\PYG{p}{,}\PYG{n}{C}\PYG{p}{,}\PYG{n}{D}\PYG{p}{,}\PYG{n}{x2}\PYG{p}{,}\PYG{n}{y2}\PYG{p}{)}\PYG{p}{)}\PYG{p}{[}\PYG{l+m+mi}{0}\PYG{p}{]}\PYG{p}{)}
\end{sphinxVerbatim}

\end{sphinxuseclass}\end{sphinxVerbatimInput}
\begin{sphinxVerbatimOutput}

\begin{sphinxuseclass}{cell_output}
\begin{sphinxVerbatim}[commandchars=\\\{\}]
eigenvalues, 1st fixed point: [0. 0.]
eigenvalues, 2nd fixed point: [0. 0.]
\end{sphinxVerbatim}

\end{sphinxuseclass}\end{sphinxVerbatimOutput}

\end{sphinxuseclass}
\begin{sphinxuseclass}{cell}\begin{sphinxVerbatimInput}

\begin{sphinxuseclass}{cell_input}
\begin{sphinxVerbatim}[commandchars=\\\{\}]
\PYG{k+kn}{import} \PYG{n+nn}{matplotlib}\PYG{n+nn}{.}\PYG{n+nn}{pyplot} \PYG{k}{as} \PYG{n+nn}{plt}
\PYG{k}{def} \PYG{n+nf}{LV\PYGZus{}eqns}\PYG{p}{(}\PYG{n}{x}\PYG{p}{,} \PYG{n}{y}\PYG{p}{)}\PYG{p}{:}
    \PYG{n}{xdot}\PYG{p}{,} \PYG{n}{ydot} \PYG{o}{=} \PYG{p}{[}\PYG{n}{x}\PYG{o}{*}\PYG{p}{(}\PYG{n}{A} \PYG{o}{\PYGZhy{}} \PYG{n}{B}\PYG{o}{*}\PYG{n}{x} \PYG{o}{\PYGZhy{}} \PYG{n}{C}\PYG{o}{*}\PYG{n}{y}\PYG{p}{)}\PYG{p}{,} \PYG{n}{y}\PYG{o}{*}\PYG{p}{(}\PYG{n}{D} \PYG{o}{\PYGZhy{}} \PYG{n}{F}\PYG{o}{*}\PYG{n}{x} \PYG{o}{\PYGZhy{}} \PYG{n}{G}\PYG{o}{*}\PYG{n}{y}\PYG{p}{)}\PYG{p}{]}
    \PYG{k}{return} \PYG{n}{xdot}\PYG{p}{,} \PYG{n}{ydot}

\PYG{k}{def} \PYG{n+nf}{LV\PYGZus{}phase}\PYG{p}{(}\PYG{n}{X}\PYG{p}{,} \PYG{n}{VX}\PYG{p}{)}\PYG{p}{:}
    \PYG{n}{xdot}\PYG{p}{,} \PYG{n}{ydot} \PYG{o}{=} \PYG{n}{np}\PYG{o}{.}\PYG{n}{zeros}\PYG{p}{(}\PYG{n}{X}\PYG{o}{.}\PYG{n}{shape}\PYG{p}{)}\PYG{p}{,} \PYG{n}{np}\PYG{o}{.}\PYG{n}{zeros}\PYG{p}{(}\PYG{n}{VX}\PYG{o}{.}\PYG{n}{shape}\PYG{p}{)}
    \PYG{n}{Xlim}\PYG{p}{,} \PYG{n}{Ylim} \PYG{o}{=} \PYG{n}{X}\PYG{o}{.}\PYG{n}{shape}
    \PYG{k}{for} \PYG{n}{i} \PYG{o+ow}{in} \PYG{n+nb}{range}\PYG{p}{(}\PYG{n}{Xlim}\PYG{p}{)}\PYG{p}{:}
        \PYG{k}{for} \PYG{n}{j} \PYG{o+ow}{in} \PYG{n+nb}{range}\PYG{p}{(}\PYG{n}{Ylim}\PYG{p}{)}\PYG{p}{:}
            \PYG{n}{xloc} \PYG{o}{=} \PYG{n}{X}\PYG{p}{[}\PYG{n}{i}\PYG{p}{,} \PYG{n}{j}\PYG{p}{]}
            \PYG{n}{yloc} \PYG{o}{=} \PYG{n}{VX}\PYG{p}{[}\PYG{n}{i}\PYG{p}{,} \PYG{n}{j}\PYG{p}{]}
            \PYG{n}{xdot}\PYG{p}{[}\PYG{n}{i}\PYG{p}{,}\PYG{n}{j}\PYG{p}{]}\PYG{p}{,} \PYG{n}{ydot}\PYG{p}{[}\PYG{n}{i}\PYG{p}{,}\PYG{n}{j}\PYG{p}{]} \PYG{o}{=} \PYG{n}{LV\PYGZus{}eqns}\PYG{p}{(}\PYG{n}{xloc}\PYG{p}{,} \PYG{n}{yloc}\PYG{p}{)}
    \PYG{k}{return} \PYG{n}{xdot}\PYG{p}{,} \PYG{n}{ydot}

\PYG{n}{N} \PYG{o}{=} \PYG{l+m+mi}{40}
\PYG{n}{x} \PYG{o}{=} \PYG{n}{np}\PYG{o}{.}\PYG{n}{linspace}\PYG{p}{(}\PYG{l+m+mf}{0.}\PYG{p}{,} \PYG{l+m+mf}{3.5}\PYG{p}{,} \PYG{n}{N}\PYG{p}{)}
\PYG{n}{y} \PYG{o}{=} \PYG{n}{np}\PYG{o}{.}\PYG{n}{linspace}\PYG{p}{(}\PYG{l+m+mf}{0.}\PYG{p}{,} \PYG{l+m+mf}{3.5}\PYG{p}{,} \PYG{n}{N}\PYG{p}{)}
\PYG{n}{X}\PYG{p}{,} \PYG{n}{Y} \PYG{o}{=} \PYG{n}{np}\PYG{o}{.}\PYG{n}{meshgrid}\PYG{p}{(}\PYG{n}{x}\PYG{p}{,} \PYG{n}{y}\PYG{p}{)}
\PYG{n}{xdot}\PYG{p}{,} \PYG{n}{ydot} \PYG{o}{=} \PYG{n}{LV\PYGZus{}phase}\PYG{p}{(}\PYG{n}{X}\PYG{p}{,} \PYG{n}{Y}\PYG{p}{)}
\PYG{n}{ax} \PYG{o}{=} \PYG{n}{plt}\PYG{o}{.}\PYG{n}{figure}\PYG{p}{(}\PYG{n}{figsize}\PYG{o}{=}\PYG{p}{(}\PYG{l+m+mi}{10}\PYG{p}{,}\PYG{l+m+mi}{10}\PYG{p}{)}\PYG{p}{)}
\PYG{n}{Q} \PYG{o}{=} \PYG{n}{plt}\PYG{o}{.}\PYG{n}{streamplot}\PYG{p}{(}\PYG{n}{X}\PYG{p}{,} \PYG{n}{Y}\PYG{p}{,} \PYG{n}{xdot}\PYG{p}{,} \PYG{n}{ydot}\PYG{p}{,} \PYG{n}{color}\PYG{o}{=}\PYG{l+s+s1}{\PYGZsq{}}\PYG{l+s+s1}{k}\PYG{l+s+s1}{\PYGZsq{}}\PYG{p}{,}\PYG{n}{broken\PYGZus{}streamlines} \PYG{o}{=} \PYG{k+kc}{False}\PYG{p}{)}
\PYG{n}{plt}\PYG{o}{.}\PYG{n}{scatter}\PYG{p}{(}\PYG{n}{x1}\PYG{p}{,}\PYG{n}{y1}\PYG{p}{,} \PYG{n}{label} \PYG{o}{=} \PYG{l+s+s1}{\PYGZsq{}}\PYG{l+s+s1}{fixed point 1}\PYG{l+s+s1}{\PYGZsq{}}\PYG{p}{)}
\PYG{n}{plt}\PYG{o}{.}\PYG{n}{scatter}\PYG{p}{(}\PYG{n}{x2}\PYG{p}{,}\PYG{n}{y2}\PYG{p}{,} \PYG{n}{label} \PYG{o}{=} \PYG{l+s+s2}{\PYGZdq{}}\PYG{l+s+s2}{fixed point 2}\PYG{l+s+s2}{\PYGZdq{}}\PYG{p}{)}
\PYG{n}{plt}\PYG{o}{.}\PYG{n}{legend}\PYG{p}{(}\PYG{p}{)}
\PYG{n}{plt}\PYG{o}{.}\PYG{n}{grid}\PYG{p}{(}\PYG{p}{)}
\PYG{n}{plt}\PYG{o}{.}\PYG{n}{xlabel}\PYG{p}{(}\PYG{l+s+s1}{\PYGZsq{}}\PYG{l+s+s1}{\PYGZdl{}rabbits\PYGZdl{}}\PYG{l+s+s1}{\PYGZsq{}}\PYG{p}{)}
\PYG{n}{plt}\PYG{o}{.}\PYG{n}{ylabel}\PYG{p}{(}\PYG{l+s+s1}{\PYGZsq{}}\PYG{l+s+s1}{\PYGZdl{}sheep\PYGZdl{}}\PYG{l+s+s1}{\PYGZsq{}}\PYG{p}{)}
\PYG{n}{plt}\PYG{o}{.}\PYG{n}{show}\PYG{p}{(}\PYG{p}{)}
\end{sphinxVerbatim}

\end{sphinxuseclass}\end{sphinxVerbatimInput}
\begin{sphinxVerbatimOutput}

\begin{sphinxuseclass}{cell_output}
\begin{sphinxVerbatim}[commandchars=\\\{\}]
\PYG{g+gt}{\PYGZhy{}\PYGZhy{}\PYGZhy{}\PYGZhy{}\PYGZhy{}\PYGZhy{}\PYGZhy{}\PYGZhy{}\PYGZhy{}\PYGZhy{}\PYGZhy{}\PYGZhy{}\PYGZhy{}\PYGZhy{}\PYGZhy{}\PYGZhy{}\PYGZhy{}\PYGZhy{}\PYGZhy{}\PYGZhy{}\PYGZhy{}\PYGZhy{}\PYGZhy{}\PYGZhy{}\PYGZhy{}\PYGZhy{}\PYGZhy{}\PYGZhy{}\PYGZhy{}\PYGZhy{}\PYGZhy{}\PYGZhy{}\PYGZhy{}\PYGZhy{}\PYGZhy{}\PYGZhy{}\PYGZhy{}\PYGZhy{}\PYGZhy{}\PYGZhy{}\PYGZhy{}\PYGZhy{}\PYGZhy{}\PYGZhy{}\PYGZhy{}\PYGZhy{}\PYGZhy{}\PYGZhy{}\PYGZhy{}\PYGZhy{}\PYGZhy{}\PYGZhy{}\PYGZhy{}\PYGZhy{}\PYGZhy{}\PYGZhy{}\PYGZhy{}\PYGZhy{}\PYGZhy{}\PYGZhy{}\PYGZhy{}\PYGZhy{}\PYGZhy{}\PYGZhy{}\PYGZhy{}\PYGZhy{}\PYGZhy{}\PYGZhy{}\PYGZhy{}\PYGZhy{}\PYGZhy{}\PYGZhy{}\PYGZhy{}\PYGZhy{}\PYGZhy{}}
\PYG{n+ne}{TypeError}\PYG{g+gWhitespace}{                                 }Traceback (most recent call last)
\PYG{n+nn}{Input In [3],} in \PYG{n+ni}{\PYGZlt{}cell line: 22\PYGZgt{}}\PYG{n+nt}{()}
\PYG{g+gWhitespace}{     }\PYG{l+m+mi}{20} \PYG{n}{xdot}\PYG{p}{,} \PYG{n}{ydot} \PYG{o}{=} \PYG{n}{LV\PYGZus{}phase}\PYG{p}{(}\PYG{n}{X}\PYG{p}{,} \PYG{n}{Y}\PYG{p}{)}
\PYG{g+gWhitespace}{     }\PYG{l+m+mi}{21} \PYG{n}{ax} \PYG{o}{=} \PYG{n}{plt}\PYG{o}{.}\PYG{n}{figure}\PYG{p}{(}\PYG{n}{figsize}\PYG{o}{=}\PYG{p}{(}\PYG{l+m+mi}{10}\PYG{p}{,}\PYG{l+m+mi}{10}\PYG{p}{)}\PYG{p}{)}
\PYG{n+ne}{\PYGZhy{}\PYGZhy{}\PYGZhy{}\PYGZgt{} }\PYG{l+m+mi}{22} \PYG{n}{Q} \PYG{o}{=} \PYG{n}{plt}\PYG{o}{.}\PYG{n}{streamplot}\PYG{p}{(}\PYG{n}{X}\PYG{p}{,} \PYG{n}{Y}\PYG{p}{,} \PYG{n}{xdot}\PYG{p}{,} \PYG{n}{ydot}\PYG{p}{,} \PYG{n}{color}\PYG{o}{=}\PYG{l+s+s1}{\PYGZsq{}}\PYG{l+s+s1}{k}\PYG{l+s+s1}{\PYGZsq{}}\PYG{p}{,}\PYG{n}{broken\PYGZus{}streamlines} \PYG{o}{=} \PYG{k+kc}{False}\PYG{p}{)}
\PYG{g+gWhitespace}{     }\PYG{l+m+mi}{23} \PYG{n}{plt}\PYG{o}{.}\PYG{n}{scatter}\PYG{p}{(}\PYG{n}{x1}\PYG{p}{,}\PYG{n}{y1}\PYG{p}{,} \PYG{n}{label} \PYG{o}{=} \PYG{l+s+s1}{\PYGZsq{}}\PYG{l+s+s1}{fixed point 1}\PYG{l+s+s1}{\PYGZsq{}}\PYG{p}{)}
\PYG{g+gWhitespace}{     }\PYG{l+m+mi}{24} \PYG{n}{plt}\PYG{o}{.}\PYG{n}{scatter}\PYG{p}{(}\PYG{n}{x2}\PYG{p}{,}\PYG{n}{y2}\PYG{p}{,} \PYG{n}{label} \PYG{o}{=} \PYG{l+s+s2}{\PYGZdq{}}\PYG{l+s+s2}{fixed point 2}\PYG{l+s+s2}{\PYGZdq{}}\PYG{p}{)}

\PYG{n+ne}{TypeError}: streamplot() got an unexpected keyword argument \PYGZsq{}broken\PYGZus{}streamlines\PYGZsq{}
\end{sphinxVerbatim}

\begin{sphinxVerbatim}[commandchars=\\\{\}]
\PYGZlt{}Figure size 720x720 with 0 Axes\PYGZgt{}
\end{sphinxVerbatim}

\end{sphinxuseclass}\end{sphinxVerbatimOutput}

\end{sphinxuseclass}

\section{Investigating the Van der Pol Oscillator}
\label{\detokenize{content/1_mechanics/dynamical_2:investigating-the-van-der-pol-oscillator}}
\sphinxAtStartPar
It turns out there is some more interesting behavior other than just the behavior around fixed points. Toward seeing that, let’s look at the Van der Pol Oscillator. This equation originates from lonlinear circuits in early radios, but has now also been used in neuroscience and geology. It is given by the differential equation:
\begin{equation*}
\begin{split}
\ddot{x} = -\mu (x^2 - 1)\dot{x} - x
\end{split}
\end{equation*}
\sphinxAtStartPar
or, written as two first order equations:
\begin{equation*}
\begin{split}
\dot{x} = v \hspace{1in} \dot{v} = -\mu (x^2 - 1)v - x
\end{split}
\end{equation*}
\sphinxAtStartPar
With \(\mu > 0\). Note that this equation is simply the harmonic oscillator when \(\mu = 0\). The strange \(-\mu (x^2 - 1)\dot{x}\) represents damping, but this damping behaves strangely, because when \(|x|<1\) it is negative damping, that is it boosts oscillations smaller than \(1\), while still slowing down oscillations larger than \(1\).

\sphinxAtStartPar
Now we play the usual game of trying to figure out how this system behaves:

\sphinxAtStartPar
\sphinxstylestrong{✅ Do this}
\begin{enumerate}
\sphinxsetlistlabels{\arabic}{enumi}{enumii}{}{.}%
\item {} 
\sphinxAtStartPar
Identify the fixed point of this system. Follow the linearization procedure to characterize it.

\item {} 
\sphinxAtStartPar
Edit the code below to produce a phase plot for the Van der Pol oscillator. This code also numerically integrates a trajectory and plots it. Add a second trajectory and plot that as well.

\item {} 
\sphinxAtStartPar
What happens to phase space when you change the value of \(\mu\)? What if you make it negative?

\item {} 
\sphinxAtStartPar
What behavior do you notice here that’s different than you’ve seen before? What is attracting the trajectories?

\end{enumerate}

\begin{sphinxuseclass}{cell}\begin{sphinxVerbatimInput}

\begin{sphinxuseclass}{cell_input}
\begin{sphinxVerbatim}[commandchars=\\\{\}]
\PYG{k+kn}{import} \PYG{n+nn}{matplotlib}\PYG{n+nn}{.}\PYG{n+nn}{pyplot} \PYG{k}{as} \PYG{n+nn}{plt}
\PYG{k+kn}{from} \PYG{n+nn}{scipy}\PYG{n+nn}{.}\PYG{n+nn}{integrate} \PYG{k+kn}{import} \PYG{n}{solve\PYGZus{}ivp}

\PYG{k}{def} \PYG{n+nf}{VP\PYGZus{}eqn}\PYG{p}{(}\PYG{n}{x}\PYG{p}{,} \PYG{n}{v}\PYG{p}{,} \PYG{n}{mu} \PYG{o}{=} \PYG{l+m+mf}{1.}\PYG{p}{)}\PYG{p}{:}
    \PYG{n}{xdot}\PYG{p}{,} \PYG{n}{vdot} \PYG{o}{=} \PYG{p}{[}\PYG{l+m+mi}{0}\PYG{p}{,}\PYG{l+m+mi}{0}\PYG{p}{]} \PYG{c+c1}{\PYGZsh{}\PYGZsh{} CHANGE}
    \PYG{k}{return} \PYG{n}{xdot}\PYG{p}{,} \PYG{n}{vdot}

\PYG{k}{def} \PYG{n+nf}{VP\PYGZus{}phase}\PYG{p}{(}\PYG{n}{X}\PYG{p}{,} \PYG{n}{VX}\PYG{p}{,} \PYG{n}{mu}\PYG{p}{)}\PYG{p}{:}
    \PYG{n}{xdot}\PYG{p}{,} \PYG{n}{vdot} \PYG{o}{=} \PYG{n}{np}\PYG{o}{.}\PYG{n}{zeros}\PYG{p}{(}\PYG{n}{X}\PYG{o}{.}\PYG{n}{shape}\PYG{p}{)}\PYG{p}{,} \PYG{n}{np}\PYG{o}{.}\PYG{n}{zeros}\PYG{p}{(}\PYG{n}{VX}\PYG{o}{.}\PYG{n}{shape}\PYG{p}{)}
    \PYG{n}{Xlim}\PYG{p}{,} \PYG{n}{Ylim} \PYG{o}{=} \PYG{n}{X}\PYG{o}{.}\PYG{n}{shape}
    \PYG{k}{for} \PYG{n}{i} \PYG{o+ow}{in} \PYG{n+nb}{range}\PYG{p}{(}\PYG{n}{Xlim}\PYG{p}{)}\PYG{p}{:}
        \PYG{k}{for} \PYG{n}{j} \PYG{o+ow}{in} \PYG{n+nb}{range}\PYG{p}{(}\PYG{n}{Ylim}\PYG{p}{)}\PYG{p}{:}
            \PYG{n}{xloc} \PYG{o}{=} \PYG{n}{X}\PYG{p}{[}\PYG{n}{i}\PYG{p}{,} \PYG{n}{j}\PYG{p}{]}
            \PYG{n}{yloc} \PYG{o}{=} \PYG{n}{VX}\PYG{p}{[}\PYG{n}{i}\PYG{p}{,} \PYG{n}{j}\PYG{p}{]}
            \PYG{n}{xdot}\PYG{p}{[}\PYG{n}{i}\PYG{p}{,}\PYG{n}{j}\PYG{p}{]}\PYG{p}{,} \PYG{n}{vdot}\PYG{p}{[}\PYG{n}{i}\PYG{p}{,}\PYG{n}{j}\PYG{p}{]} \PYG{o}{=} \PYG{n}{VP\PYGZus{}eqn}\PYG{p}{(}\PYG{n}{xloc}\PYG{p}{,} \PYG{n}{yloc}\PYG{p}{,}\PYG{n}{mu}\PYG{p}{)}
    \PYG{k}{return} \PYG{n}{xdot}\PYG{p}{,} \PYG{n}{vdot}

\PYG{k}{def} \PYG{n+nf}{VP\PYGZus{}eqn\PYGZus{}for\PYGZus{}solve\PYGZus{}ivp}\PYG{p}{(}\PYG{n}{t}\PYG{p}{,}\PYG{n}{curr\PYGZus{}vals}\PYG{p}{,} \PYG{n}{mu}\PYG{o}{=}\PYG{l+m+mi}{1}\PYG{p}{)}\PYG{p}{:} \PYG{c+c1}{\PYGZsh{} need to rephrase this to work with what solve\PYGZus{}ivp expects}
    \PYG{n}{x}\PYG{p}{,} \PYG{n}{v} \PYG{o}{=} \PYG{n}{curr\PYGZus{}vals} 
    \PYG{n}{xdot}\PYG{p}{,} \PYG{n}{vdot} \PYG{o}{=} \PYG{n}{VP\PYGZus{}eqn}\PYG{p}{(}\PYG{n}{x}\PYG{p}{,}\PYG{n}{v}\PYG{p}{,}\PYG{n}{mu}\PYG{p}{)}
    \PYG{k}{return} \PYG{n}{xdot}\PYG{p}{,}\PYG{n}{vdot}

\PYG{c+c1}{\PYGZsh{} Numerical Integration}
\PYG{n}{tmax} \PYG{o}{=} \PYG{l+m+mi}{20}
\PYG{n}{dt} \PYG{o}{=} \PYG{l+m+mf}{0.05}
\PYG{n}{tspan} \PYG{o}{=} \PYG{p}{(}\PYG{l+m+mi}{0}\PYG{p}{,}\PYG{n}{tmax}\PYG{p}{)}
\PYG{n}{t} \PYG{o}{=} \PYG{n}{np}\PYG{o}{.}\PYG{n}{arange}\PYG{p}{(}\PYG{l+m+mi}{0}\PYG{p}{,}\PYG{n}{tmax}\PYG{p}{,}\PYG{n}{dt}\PYG{p}{)}
\PYG{n}{mu} \PYG{o}{=} \PYG{l+m+mf}{1.}
\PYG{n}{initial\PYGZus{}condition} \PYG{o}{=} \PYG{p}{[}\PYG{l+m+mi}{1}\PYG{p}{,} \PYG{l+m+mi}{1}\PYG{p}{]} 
\PYG{n}{solved} \PYG{o}{=} \PYG{n}{solve\PYGZus{}ivp}\PYG{p}{(}\PYG{n}{VP\PYGZus{}eqn\PYGZus{}for\PYGZus{}solve\PYGZus{}ivp}\PYG{p}{,}\PYG{n}{tspan}\PYG{p}{,}\PYG{n}{initial\PYGZus{}condition}\PYG{p}{,}\PYG{n}{t\PYGZus{}eval} \PYG{o}{=} \PYG{n}{t}\PYG{p}{,} \PYG{n}{args} \PYG{o}{=} \PYG{p}{(}\PYG{n}{mu}\PYG{p}{,}\PYG{p}{)}\PYG{p}{,}\PYG{n}{method}\PYG{o}{=}\PYG{l+s+s2}{\PYGZdq{}}\PYG{l+s+s2}{RK45}\PYG{l+s+s2}{\PYGZdq{}}\PYG{p}{)}


\PYG{c+c1}{\PYGZsh{} Plotting stuff}
\PYG{n}{N} \PYG{o}{=} \PYG{l+m+mi}{40}
\PYG{n}{x} \PYG{o}{=} \PYG{n}{np}\PYG{o}{.}\PYG{n}{linspace}\PYG{p}{(}\PYG{o}{\PYGZhy{}}\PYG{l+m+mf}{3.}\PYG{p}{,} \PYG{l+m+mf}{3.}\PYG{p}{,} \PYG{n}{N}\PYG{p}{)}
\PYG{n}{v} \PYG{o}{=} \PYG{n}{np}\PYG{o}{.}\PYG{n}{linspace}\PYG{p}{(}\PYG{o}{\PYGZhy{}}\PYG{l+m+mf}{3.}\PYG{p}{,} \PYG{l+m+mf}{3.}\PYG{p}{,} \PYG{n}{N}\PYG{p}{)}
\PYG{n}{X}\PYG{p}{,} \PYG{n}{V} \PYG{o}{=} \PYG{n}{np}\PYG{o}{.}\PYG{n}{meshgrid}\PYG{p}{(}\PYG{n}{x}\PYG{p}{,} \PYG{n}{v}\PYG{p}{)}
\PYG{n}{xdot}\PYG{p}{,} \PYG{n}{vdot} \PYG{o}{=} \PYG{n}{VP\PYGZus{}phase}\PYG{p}{(}\PYG{n}{X}\PYG{p}{,} \PYG{n}{V}\PYG{p}{,}\PYG{n}{mu}\PYG{p}{)}
\PYG{n}{ax} \PYG{o}{=} \PYG{n}{plt}\PYG{o}{.}\PYG{n}{figure}\PYG{p}{(}\PYG{n}{figsize}\PYG{o}{=}\PYG{p}{(}\PYG{l+m+mi}{10}\PYG{p}{,}\PYG{l+m+mi}{10}\PYG{p}{)}\PYG{p}{)}
\PYG{n}{Q} \PYG{o}{=} \PYG{n}{plt}\PYG{o}{.}\PYG{n}{streamplot}\PYG{p}{(}\PYG{n}{X}\PYG{p}{,} \PYG{n}{V}\PYG{p}{,} \PYG{n}{xdot}\PYG{p}{,} \PYG{n}{vdot}\PYG{p}{,} \PYG{n}{color}\PYG{o}{=}\PYG{l+s+s1}{\PYGZsq{}}\PYG{l+s+s1}{k}\PYG{l+s+s1}{\PYGZsq{}}\PYG{p}{,}\PYG{n}{broken\PYGZus{}streamlines} \PYG{o}{=} \PYG{k+kc}{False}\PYG{p}{)}
\PYG{n}{plt}\PYG{o}{.}\PYG{n}{plot}\PYG{p}{(}\PYG{n}{solved}\PYG{o}{.}\PYG{n}{y}\PYG{p}{[}\PYG{l+m+mi}{0}\PYG{p}{]}\PYG{p}{,}\PYG{n}{solved}\PYG{o}{.}\PYG{n}{y}\PYG{p}{[}\PYG{l+m+mi}{1}\PYG{p}{]}\PYG{p}{,}\PYG{n}{lw} \PYG{o}{=} \PYG{l+m+mi}{3}\PYG{p}{,}\PYG{n}{c} \PYG{o}{=} \PYG{l+s+s1}{\PYGZsq{}}\PYG{l+s+s1}{red}\PYG{l+s+s1}{\PYGZsq{}}\PYG{p}{)} \PYG{c+c1}{\PYGZsh{} plot trajectory from solve\PYGZus{}ivp}
\PYG{n}{plt}\PYG{o}{.}\PYG{n}{grid}\PYG{p}{(}\PYG{p}{)}
\PYG{n}{plt}\PYG{o}{.}\PYG{n}{xlabel}\PYG{p}{(}\PYG{l+s+s1}{\PYGZsq{}}\PYG{l+s+s1}{\PYGZdl{}x\PYGZdl{}}\PYG{l+s+s1}{\PYGZsq{}}\PYG{p}{)}
\PYG{n}{plt}\PYG{o}{.}\PYG{n}{ylabel}\PYG{p}{(}\PYG{l+s+s1}{\PYGZsq{}}\PYG{l+s+s1}{\PYGZdl{}v\PYGZdl{}}\PYG{l+s+s1}{\PYGZsq{}}\PYG{p}{)}
\PYG{n}{plt}\PYG{o}{.}\PYG{n}{show}\PYG{p}{(}\PYG{p}{)}
\end{sphinxVerbatim}

\end{sphinxuseclass}\end{sphinxVerbatimInput}
\begin{sphinxVerbatimOutput}

\begin{sphinxuseclass}{cell_output}
\noindent\sphinxincludegraphics{{dynamical_2_10_0}.png}

\end{sphinxuseclass}\end{sphinxVerbatimOutput}

\end{sphinxuseclass}
\sphinxAtStartPar
\sphinxstylestrong{✅ Do this}

\sphinxAtStartPar
Based on the phase space diagram, what do you expect actual trajectories to look like in \(x\) vs \(t\) space? Use the numerically integrated trajectories to plot that.

\begin{sphinxuseclass}{cell}\begin{sphinxVerbatimInput}

\begin{sphinxuseclass}{cell_input}
\begin{sphinxVerbatim}[commandchars=\\\{\}]
\PYG{c+c1}{\PYGZsh{}\PYGZsh{} your code here}
\end{sphinxVerbatim}

\end{sphinxuseclass}\end{sphinxVerbatimInput}

\end{sphinxuseclass}

\subsection{Limit Cycles}
\label{\detokenize{content/1_mechanics/dynamical_2:limit-cycles}}
\sphinxAtStartPar
The new behavior we’ve seen from this equation is what’s called a \sphinxstylestrong{limit cycle}, where the system is attracted/reppeled from a closed curve instead of a fixed point(s). There’s a lot of really great math here that’s a bit beyond what we can cover in class, but it would be a great thing to look into for a project!

\sphinxAtStartPar
\sphinxstylestrong{✅ Do this}

\sphinxAtStartPar
Spend the rest of class investigating the Van der Pol oscillator. Here are a few investigations you could do:
\begin{itemize}
\item {} 
\sphinxAtStartPar
When \(\mu\) changes from negative to positive, this system undergoes what is known as a \sphinxstylestrong{Hopf Bifurcation} Look up what bifurcations are to understand what this means and show that it is true using numerical integration.

\item {} 
\sphinxAtStartPar
Add an \(A\sin(t)\) driving force term to the differential equation and numerically integrate. What do these trajectories look like in \(x\) vs \(t\) and in phase space?

\item {} 
\sphinxAtStartPar
Examine the energetics of this system. Is energy conserved or does it have some interesting behavior? Why?

\end{itemize}

\begin{sphinxuseclass}{cell}\begin{sphinxVerbatimInput}

\begin{sphinxuseclass}{cell_input}
\begin{sphinxVerbatim}[commandchars=\\\{\}]
\PYG{c+c1}{\PYGZsh{} code here}
\end{sphinxVerbatim}

\end{sphinxuseclass}\end{sphinxVerbatimInput}

\end{sphinxuseclass}
\sphinxstepscope


\chapter{19 Sep 23 \sphinxhyphen{} CHAOS}
\label{\detokenize{content/1_mechanics/CHAOS:sep-23-chaos}}\label{\detokenize{content/1_mechanics/CHAOS::doc}}
\sphinxAtStartPar
Over the last couple weeks we’ve been looking into a lot of 2 dimensional autonomous (the differential equations don’t depend on time) systems of differential equations.
Another thing we’ve noticed is that initial conditions in these systems tend to either blast off to infinity, form a closed loop, or are drawn to attracting fixed point(s) or limit cycle(s). In the case of attracting fixed points or limit cycles, many different initial conditions can be drawn to the same fixed point or limit cycle. In fact, we saw that it is possible for \sphinxstylestrong{every} initial condition to be drawn to the same limit cycle for the Van der Pol oscillator. This means that trajectories that are initially far apart eventually become close together.

\sphinxAtStartPar
This seems all well and good, but when we step up to 3 dimensional autonomous systems (or as we’ll see later, non\sphinxhyphen{}autonomous 2D ones), some more interesting behavior starts to emerge. Trajectories that start out very close to each other can diverge from each other, while still not diverging to infinity. And it turns out that these sorts of \sphinxstylestrong{chaotic} systems arise surprisingly often in systems we’re interested in in physics. The classic example of this is the double pendulum. Its also been shown that the solar system is chaotic for large time scales, and countless other systems exhibit this property.


\section{Definition of Chaos}
\label{\detokenize{content/1_mechanics/CHAOS:definition-of-chaos}}
\sphinxAtStartPar
Here we’ll follow Strogats’s definition of chaos, which is:

\sphinxAtStartPar
\sphinxstyleemphasis{\sphinxstylestrong{Chaos}} \sphinxstyleemphasis{is aperiodic long\sphinxhyphen{}term behavior in a deterministic system that exhibits sensitive dependence on initial conditions.}
\begin{enumerate}
\sphinxsetlistlabels{\arabic}{enumi}{enumii}{}{.}%
\item {} 
\sphinxAtStartPar
\sphinxstylestrong{Aperiodic long\sphinxhyphen{}term behavior} means that some trajectories don’t settle into fixed points, periodic orbits, or quasi\sphinxhyphen{}periodic orbits, while still not diverging to infinity.

\item {} 
\sphinxAtStartPar
\sphinxstylestrong{Deterministic} means that the equations evolve in totally predictable ways, without any randomness. There are no noisy parameters or imputs.

\item {} 
\sphinxAtStartPar
\sphinxstylestrong{Sensitive dependence on initial conditions} means that nearby trajectories separate exponentially fast.

\end{enumerate}

\sphinxAtStartPar
Let’s investigate this phenomenon for a particular system.


\section{Halvorsen Attractor}
\label{\detokenize{content/1_mechanics/CHAOS:halvorsen-attractor}}
\sphinxAtStartPar
One system we can start looking at is given by the \sphinxstylestrong{Halvorsen Equations:}
\begin{equation*}
\begin{split}
\dot{x} = -ax - 4y - 4z - y^2
\end{split}
\end{equation*}\begin{equation*}
\begin{split}
\dot{y} = -ay - 4z - 4x - z^2
\end{split}
\end{equation*}\begin{equation*}
\begin{split}
\dot{z} = -az - 4x - 4y - x^2
\end{split}
\end{equation*}
\sphinxAtStartPar
\sphinxstylestrong{✅ Do this}
\begin{enumerate}
\sphinxsetlistlabels{\arabic}{enumi}{enumii}{}{.}%
\item {} 
\sphinxAtStartPar
Review the code below. Talk to your neighbors about what it is doing. Feel free to experiment with more or less initial conditions. \sphinxstylestrong{You should be able to explain what this code is doing}

\item {} 
\sphinxAtStartPar
Based on our definition above, is this system chaotic? Why or why not? Discuss with your neighbors.
\begin{itemize}
\item {} 
\sphinxAtStartPar
If it is chaotic and there is no limit cycle, what is drawing the trajectories?

\end{itemize}

\end{enumerate}

\begin{sphinxuseclass}{cell}\begin{sphinxVerbatimInput}

\begin{sphinxuseclass}{cell_input}
\begin{sphinxVerbatim}[commandchars=\\\{\}]
\PYG{k+kn}{import} \PYG{n+nn}{numpy} \PYG{k}{as} \PYG{n+nn}{np}
\PYG{k+kn}{import} \PYG{n+nn}{matplotlib}\PYG{n+nn}{.}\PYG{n+nn}{pyplot} \PYG{k}{as} \PYG{n+nn}{plt}
\PYG{k+kn}{from} \PYG{n+nn}{scipy}\PYG{n+nn}{.}\PYG{n+nn}{integrate} \PYG{k+kn}{import} \PYG{n}{solve\PYGZus{}ivp}

\PYG{k}{def} \PYG{n+nf}{Halvorsen}\PYG{p}{(}\PYG{n}{t}\PYG{p}{,}\PYG{n}{curr\PYGZus{}vals}\PYG{p}{,} \PYG{n}{a}\PYG{p}{)}\PYG{p}{:} 
    \PYG{c+c1}{\PYGZsh{} Derivatives function for solve\PYGZus{}ivp}
    \PYG{n}{x}\PYG{p}{,}\PYG{n}{y}\PYG{p}{,}\PYG{n}{z} \PYG{o}{=} \PYG{n}{curr\PYGZus{}vals} 
    \PYG{n}{xdot} \PYG{o}{=} \PYG{o}{\PYGZhy{}}\PYG{n}{a}\PYG{o}{*}\PYG{n}{x} \PYG{o}{\PYGZhy{}} \PYG{l+m+mi}{4}\PYG{o}{*}\PYG{n}{y} \PYG{o}{\PYGZhy{}} \PYG{l+m+mi}{4}\PYG{o}{*}\PYG{n}{z} \PYG{o}{\PYGZhy{}} \PYG{n}{y}\PYG{o}{*}\PYG{o}{*}\PYG{l+m+mi}{2} 
    \PYG{n}{ydot} \PYG{o}{=} \PYG{o}{\PYGZhy{}}\PYG{n}{a}\PYG{o}{*}\PYG{n}{y} \PYG{o}{\PYGZhy{}} \PYG{l+m+mi}{4}\PYG{o}{*}\PYG{n}{z} \PYG{o}{\PYGZhy{}} \PYG{l+m+mi}{4}\PYG{o}{*}\PYG{n}{x} \PYG{o}{\PYGZhy{}} \PYG{n}{z}\PYG{o}{*}\PYG{o}{*}\PYG{l+m+mi}{2}
    \PYG{n}{zdot} \PYG{o}{=} \PYG{o}{\PYGZhy{}}\PYG{n}{a}\PYG{o}{*}\PYG{n}{z} \PYG{o}{\PYGZhy{}} \PYG{l+m+mi}{4}\PYG{o}{*}\PYG{n}{x} \PYG{o}{\PYGZhy{}} \PYG{l+m+mi}{4}\PYG{o}{*}\PYG{n}{y} \PYG{o}{\PYGZhy{}} \PYG{n}{x}\PYG{o}{*}\PYG{o}{*}\PYG{l+m+mi}{2}
    \PYG{k}{return} \PYG{n}{xdot}\PYG{p}{,}\PYG{n}{ydot}\PYG{p}{,}\PYG{n}{zdot}

\PYG{c+c1}{\PYGZsh{} Time Setup}
\PYG{n}{tmax} \PYG{o}{=} \PYG{l+m+mi}{10}
\PYG{n}{dt} \PYG{o}{=} \PYG{l+m+mf}{0.01}
\PYG{n}{tspan} \PYG{o}{=} \PYG{p}{(}\PYG{l+m+mi}{0}\PYG{p}{,}\PYG{n}{tmax}\PYG{p}{)}
\PYG{n}{t} \PYG{o}{=} \PYG{n}{np}\PYG{o}{.}\PYG{n}{arange}\PYG{p}{(}\PYG{l+m+mi}{0}\PYG{p}{,}\PYG{n}{tmax}\PYG{p}{,}\PYG{n}{dt}\PYG{p}{)}

\PYG{c+c1}{\PYGZsh{} Parameters and initial conditions}
\PYG{n}{a} \PYG{o}{=} \PYG{l+m+mf}{1.4}
\PYG{n}{n\PYGZus{}ics} \PYG{o}{=} \PYG{l+m+mi}{4} \PYG{c+c1}{\PYGZsh{} number of initial conditions}
\PYG{n}{n\PYGZus{}dim} \PYG{o}{=} \PYG{l+m+mi}{3} \PYG{c+c1}{\PYGZsh{} 3 dimensional problem}
\PYG{n}{np}\PYG{o}{.}\PYG{n}{random}\PYG{o}{.}\PYG{n}{seed}\PYG{p}{(}\PYG{l+m+mi}{1}\PYG{p}{)} \PYG{c+c1}{\PYGZsh{} control randomness}
\PYG{n}{initial\PYGZus{}conditions} \PYG{o}{=} \PYG{n}{np}\PYG{o}{.}\PYG{n}{random}\PYG{o}{.}\PYG{n}{uniform}\PYG{p}{(}\PYG{o}{\PYGZhy{}}\PYG{l+m+mf}{0.05}\PYG{p}{,}\PYG{l+m+mf}{0.05}\PYG{p}{,}\PYG{p}{(}\PYG{n}{n\PYGZus{}ics}\PYG{p}{,}\PYG{n}{n\PYGZus{}dim}\PYG{p}{)}\PYG{p}{)} \PYG{c+c1}{\PYGZsh{} get n\PYGZus{}ics initial conditions randomly from small box by the origin}

\PYG{c+c1}{\PYGZsh{} Call integrator for each initial condition}
\PYG{n}{solutions} \PYG{o}{=} \PYG{p}{[}\PYG{p}{]}
\PYG{k}{for} \PYG{n}{initial\PYGZus{}condition} \PYG{o+ow}{in} \PYG{n}{initial\PYGZus{}conditions}\PYG{p}{:}
    \PYG{n}{solved} \PYG{o}{=} \PYG{n}{solve\PYGZus{}ivp}\PYG{p}{(}\PYG{n}{Halvorsen}\PYG{p}{,}\PYG{n}{tspan}\PYG{p}{,}\PYG{n}{initial\PYGZus{}condition}\PYG{p}{,}\PYG{n}{t\PYGZus{}eval} \PYG{o}{=} \PYG{n}{t}\PYG{p}{,} \PYG{n}{args} \PYG{o}{=} \PYG{p}{(}\PYG{n}{a}\PYG{p}{,}\PYG{p}{)}\PYG{p}{)}
    \PYG{n}{solutions}\PYG{o}{.}\PYG{n}{append}\PYG{p}{(}\PYG{n}{solved}\PYG{o}{.}\PYG{n}{y}\PYG{p}{)}

\PYG{c+c1}{\PYGZsh{} Plotting}
\PYG{c+c1}{\PYGZsh{}\PYGZpc{}matplotlib widget \PYGZsh{}\PYGZsh{} UNCOMMENT TO BE ABLE TO PAN AROUND}
\PYG{n}{fig} \PYG{o}{=} \PYG{n}{plt}\PYG{o}{.}\PYG{n}{figure}\PYG{p}{(}\PYG{n}{figsize} \PYG{o}{=} \PYG{p}{(}\PYG{l+m+mi}{10}\PYG{p}{,}\PYG{l+m+mi}{10}\PYG{p}{)}\PYG{p}{)}
\PYG{n}{ax} \PYG{o}{=} \PYG{n}{plt}\PYG{o}{.}\PYG{n}{axes}\PYG{p}{(}\PYG{n}{projection}\PYG{o}{=}\PYG{l+s+s1}{\PYGZsq{}}\PYG{l+s+s1}{3d}\PYG{l+s+s1}{\PYGZsq{}}\PYG{p}{)}
\PYG{n}{ax}\PYG{o}{.}\PYG{n}{view\PYGZus{}init}\PYG{p}{(}\PYG{l+m+mi}{30}\PYG{p}{,} \PYG{l+m+mi}{45}\PYG{p}{)} \PYG{c+c1}{\PYGZsh{} Pick a nice initial viewing angle}
\PYG{k}{for} \PYG{n}{i}\PYG{p}{,}\PYG{n}{initial\PYGZus{}condition} \PYG{o+ow}{in} \PYG{n+nb}{enumerate}\PYG{p}{(}\PYG{n}{initial\PYGZus{}conditions}\PYG{p}{)}\PYG{p}{:}
    \PYG{n}{x}\PYG{p}{,}\PYG{n}{y}\PYG{p}{,}\PYG{n}{z} \PYG{o}{=} \PYG{n}{solutions}\PYG{p}{[}\PYG{n}{i}\PYG{p}{]}
    \PYG{n}{ax}\PYG{o}{.}\PYG{n}{plot3D}\PYG{p}{(}\PYG{n}{x}\PYG{p}{,}\PYG{n}{y}\PYG{p}{,}\PYG{n}{z}\PYG{p}{,} \PYG{n}{label} \PYG{o}{=} \PYG{l+s+s2}{\PYGZdq{}}\PYG{l+s+s2}{IC: }\PYG{l+s+s2}{\PYGZdq{}} \PYG{o}{+}\PYG{n+nb}{str}\PYG{p}{(}\PYG{n}{initial\PYGZus{}condition}\PYG{p}{)}\PYG{p}{,}\PYG{n}{lw} \PYG{o}{=} \PYG{l+m+mf}{0.5}\PYG{p}{)}
\PYG{n}{ax}\PYG{o}{.}\PYG{n}{set\PYGZus{}xlabel}\PYG{p}{(}\PYG{l+s+s1}{\PYGZsq{}}\PYG{l+s+s1}{x}\PYG{l+s+s1}{\PYGZsq{}}\PYG{p}{)}
\PYG{n}{ax}\PYG{o}{.}\PYG{n}{set\PYGZus{}ylabel}\PYG{p}{(}\PYG{l+s+s1}{\PYGZsq{}}\PYG{l+s+s1}{y}\PYG{l+s+s1}{\PYGZsq{}}\PYG{p}{)}
\PYG{n}{ax}\PYG{o}{.}\PYG{n}{set\PYGZus{}zlabel}\PYG{p}{(}\PYG{l+s+s1}{\PYGZsq{}}\PYG{l+s+s1}{z}\PYG{l+s+s1}{\PYGZsq{}}\PYG{p}{)}
\PYG{n}{plt}\PYG{o}{.}\PYG{n}{legend}\PYG{p}{(}\PYG{p}{)}
\PYG{n}{plt}\PYG{o}{.}\PYG{n}{show}\PYG{p}{(}\PYG{p}{)}
\end{sphinxVerbatim}

\end{sphinxuseclass}\end{sphinxVerbatimInput}
\begin{sphinxVerbatimOutput}

\begin{sphinxuseclass}{cell_output}
\noindent\sphinxincludegraphics{{CHAOS_2_0}.png}

\end{sphinxuseclass}\end{sphinxVerbatimOutput}

\end{sphinxuseclass}

\section{Lyapunov Exponents}
\label{\detokenize{content/1_mechanics/CHAOS:lyapunov-exponents}}
\sphinxAtStartPar
One of our criteria for a system to be chaotic was that nearby trajectories must separate exponentially fast. Let’s see if this holds true for this system:

\sphinxAtStartPar
\sphinxstylestrong{✅ Do this}
\begin{enumerate}
\sphinxsetlistlabels{\arabic}{enumi}{enumii}{}{.}%
\item {} 
\sphinxAtStartPar
Numerically calculate and plot the Euclidean distance (\(||\delta(t)||\)) between the first two trajectories  as a function of time. What do you notice?

\item {} 
\sphinxAtStartPar
Also calculate  \(\log ||\delta(t)||\) and plot it. What do you notice now?

\end{enumerate}

\begin{sphinxuseclass}{cell}\begin{sphinxVerbatimInput}

\begin{sphinxuseclass}{cell_input}
\begin{sphinxVerbatim}[commandchars=\\\{\}]
\PYG{c+c1}{\PYGZsh{}\PYGZsh{} your code here}
\end{sphinxVerbatim}

\end{sphinxuseclass}\end{sphinxVerbatimInput}

\end{sphinxuseclass}
\sphinxAtStartPar
The way that we characterize this divergence mathematically is with what are known as \sphinxstylestrong{Lyapunov Exponents}. Suppose we have some trajectory of a dynamical system \(\mathbf{x}_1(t)\). Then we could write a nearby trajectory, lets say \(\mathbf{x}_2(t)\) as \(\mathbf{x}_2(t) = \mathbf{x}_1(t) + \delta(t)\) where \(\delta(t) = \mathbf{x}_2(t) - \mathbf{x}_1(t)\). If trajectories initially separated by \(\delta_0\) separate exponentially fast, then we would expect to see:
\begin{equation*}
\begin{split}
||\delta(t)|| \sim ||\delta_0||e^{\lambda t}
\end{split}
\end{equation*}
\sphinxAtStartPar
Where we call \(\lambda\) the \sphinxstylestrong{Lyapunov Exponent} (technically it is the largest one of multiple). This also lets us write a very useful equation for how long a prediction of a chaotic system is within tolerance \(a\):
\begin{equation*}
\begin{split}
t_{\text{horizon}} \sim O\left(\frac{1}{\lambda}\log{\frac{a}{||\delta_0||}}\right)
\end{split}
\end{equation*}
\sphinxAtStartPar
\sphinxstylestrong{✅ Do this}
\begin{enumerate}
\sphinxsetlistlabels{\arabic}{enumi}{enumii}{}{.}%
\item {} 
\sphinxAtStartPar
Use \(\log ||\delta(t)||\) that you calculated above and \sphinxcode{\sphinxupquote{np.polyfit}} to estimate the value of \(\lambda\) for the Halvorsen system.
\begin{itemize}
\item {} 
\sphinxAtStartPar
You’ll need to eyeball where \(\log ||\delta(t)||\) stops being linear.

\end{itemize}

\item {} 
\sphinxAtStartPar
Repeat this calculation for another set of 2 trajectories. Do you get a similar value for \(\lambda\)?

\item {} 
\sphinxAtStartPar
Many systems have negative values for \(\lambda\). What does a negative value for \(\lambda\) mean?

\end{enumerate}

\begin{sphinxuseclass}{cell}\begin{sphinxVerbatimInput}

\begin{sphinxuseclass}{cell_input}
\begin{sphinxVerbatim}[commandchars=\\\{\}]
\PYG{c+c1}{\PYGZsh{}\PYGZsh{} your code here}
\end{sphinxVerbatim}

\end{sphinxuseclass}\end{sphinxVerbatimInput}

\end{sphinxuseclass}

\section{Driven Damped Pendulum}
\label{\detokenize{content/1_mechanics/CHAOS:driven-damped-pendulum}}
\sphinxAtStartPar
Now let’s turn our attention to a more physics\sphinxhyphen{}y chaotic system, the Driven Damped Pendulum (DDP). This system is similar to the large angle pendulum that we’ve studied before, but it has a damping term \(-2\beta\dot{\theta}\), as well as a driving time\sphinxhyphen{}dependent torque \(\gamma \omega_0^2\cos(\omega t)\) . The full equation after sufficient non\sphinxhyphen{}dimensionalization looks like this (note: \(\omega_0 \neq \omega\)):
\begin{equation*}
\begin{split}
\ddot{\theta} = - 2\beta\dot{\theta} - \omega_0^2\sin(\theta) + \gamma \omega_0^2 \cos(\omega t)
\end{split}
\end{equation*}
\sphinxAtStartPar
It may be tempting to try to construct a phase portrait of this thing since it looks so similar to the large\sphinxhyphen{}angle pendulum, but since this equation is now time dependent, the phase portrait itself is time dependent, which makes it tricky to visualize without a third dimension or animations (making that could be part of a project though!).

\sphinxAtStartPar
The code below numerically integrates and plots the two trajectories of similar initial condition of the DPP for a set of parameters that are known to be chaotic.

\begin{sphinxuseclass}{cell}\begin{sphinxVerbatimInput}

\begin{sphinxuseclass}{cell_input}
\begin{sphinxVerbatim}[commandchars=\\\{\}]
\PYG{k}{def} \PYG{n+nf}{DDP}\PYG{p}{(}\PYG{n}{t}\PYG{p}{,}\PYG{n}{curr\PYGZus{}vals}\PYG{p}{,} \PYG{n}{beta}\PYG{p}{,}\PYG{n}{omega\PYGZus{}natural}\PYG{p}{,}\PYG{n}{gamma}\PYG{p}{,}\PYG{n}{omega\PYGZus{}drive}\PYG{p}{)}\PYG{p}{:} 
    \PYG{c+c1}{\PYGZsh{} Derivatives function for solve\PYGZus{}ivp}
    \PYG{n}{theta}\PYG{p}{,}\PYG{n}{v} \PYG{o}{=} \PYG{n}{curr\PYGZus{}vals} 
    \PYG{n}{thetadot} \PYG{o}{=} \PYG{n}{v}
    \PYG{n}{vdot} \PYG{o}{=} \PYG{o}{\PYGZhy{}}\PYG{l+m+mi}{2}\PYG{o}{*}\PYG{n}{beta}\PYG{o}{*}\PYG{n}{v} \PYG{o}{\PYGZhy{}} \PYG{n}{omega\PYGZus{}natural} \PYG{o}{*} \PYG{n}{np}\PYG{o}{.}\PYG{n}{sin}\PYG{p}{(}\PYG{n}{theta}\PYG{p}{)} \PYG{o}{+} \PYG{n}{gamma}\PYG{o}{*}\PYG{n}{omega\PYGZus{}natural}\PYG{o}{*}\PYG{n}{np}\PYG{o}{.}\PYG{n}{cos}\PYG{p}{(}\PYG{n}{omega\PYGZus{}drive}\PYG{o}{*}\PYG{n}{t}\PYG{p}{)}
    \PYG{k}{return} \PYG{n}{thetadot}\PYG{p}{,}\PYG{n}{vdot}

\PYG{c+c1}{\PYGZsh{} Parameters and initial conditions (chosen so that they give chaos)}
\PYG{n}{beta} \PYG{o}{=} \PYG{l+m+mf}{0.375}\PYG{o}{/}\PYG{l+m+mi}{2}
\PYG{n}{omega\PYGZus{}natural} \PYG{o}{=} \PYG{l+m+mf}{1.5}\PYG{o}{*}\PYG{o}{*}\PYG{l+m+mi}{2}
\PYG{n}{omega\PYGZus{}drive} \PYG{o}{=} \PYG{l+m+mi}{1}
\PYG{n}{gamma} \PYG{o}{=} \PYG{l+m+mf}{1.5}
\PYG{n}{initial\PYGZus{}condition} \PYG{o}{=} \PYG{p}{[}\PYG{l+m+mf}{0.}\PYG{p}{,}\PYG{l+m+mf}{0.}\PYG{p}{]}
\PYG{n}{initial\PYGZus{}condition2} \PYG{o}{=} \PYG{p}{[}\PYG{l+m+mf}{0.}\PYG{p}{,}\PYG{l+m+mf}{0.01}\PYG{p}{]}

\PYG{c+c1}{\PYGZsh{} Time Setup}
\PYG{n}{tmax} \PYG{o}{=} \PYG{l+m+mf}{100.}
\PYG{n}{dt} \PYG{o}{=} \PYG{l+m+mf}{0.01}
\PYG{n}{t} \PYG{o}{=} \PYG{n}{np}\PYG{o}{.}\PYG{n}{arange}\PYG{p}{(}\PYG{l+m+mi}{0}\PYG{p}{,}\PYG{n}{tmax}\PYG{p}{,} \PYG{n}{dt}\PYG{p}{)} 
\PYG{n}{tspan} \PYG{o}{=} \PYG{p}{(}\PYG{n}{t}\PYG{p}{[}\PYG{l+m+mi}{0}\PYG{p}{]}\PYG{p}{,}\PYG{n}{t}\PYG{p}{[}\PYG{o}{\PYGZhy{}}\PYG{l+m+mi}{1}\PYG{p}{]}\PYG{p}{)}

\PYG{c+c1}{\PYGZsh{} Call integrator for each initial condition}
\PYG{n}{solved} \PYG{o}{=} \PYG{n}{solve\PYGZus{}ivp}\PYG{p}{(}\PYG{n}{DDP}\PYG{p}{,}\PYG{n}{tspan}\PYG{p}{,}\PYG{n}{initial\PYGZus{}condition}\PYG{p}{,}\PYG{n}{t\PYGZus{}eval} \PYG{o}{=} \PYG{n}{t}\PYG{p}{,} \PYG{n}{args} \PYG{o}{=} \PYG{p}{(}\PYG{n}{beta}\PYG{p}{,}\PYG{n}{omega\PYGZus{}natural}\PYG{p}{,}\PYG{n}{gamma}\PYG{p}{,}\PYG{n}{omega\PYGZus{}drive}\PYG{p}{)}\PYG{p}{)}
\PYG{n}{solved2} \PYG{o}{=} \PYG{n}{solve\PYGZus{}ivp}\PYG{p}{(}\PYG{n}{DDP}\PYG{p}{,}\PYG{n}{tspan}\PYG{p}{,}\PYG{n}{initial\PYGZus{}condition2}\PYG{p}{,}\PYG{n}{t\PYGZus{}eval} \PYG{o}{=} \PYG{n}{t}\PYG{p}{,} \PYG{n}{args} \PYG{o}{=} \PYG{p}{(}\PYG{n}{beta}\PYG{p}{,}\PYG{n}{omega\PYGZus{}natural}\PYG{p}{,}\PYG{n}{gamma}\PYG{p}{,}\PYG{n}{omega\PYGZus{}drive}\PYG{p}{)}\PYG{p}{)}

\PYG{c+c1}{\PYGZsh{} Plotting}
\PYG{n}{fig} \PYG{o}{=} \PYG{n}{plt}\PYG{o}{.}\PYG{n}{figure}\PYG{p}{(}\PYG{n}{figsize} \PYG{o}{=} \PYG{p}{(}\PYG{l+m+mi}{15}\PYG{p}{,}\PYG{l+m+mi}{6}\PYG{p}{)}\PYG{p}{)}
\PYG{n}{plt}\PYG{o}{.}\PYG{n}{subplot}\PYG{p}{(}\PYG{l+m+mi}{1}\PYG{p}{,}\PYG{l+m+mi}{2}\PYG{p}{,}\PYG{l+m+mi}{1}\PYG{p}{)}
\PYG{n}{plt}\PYG{o}{.}\PYG{n}{plot}\PYG{p}{(}\PYG{n}{t}\PYG{p}{,}\PYG{n}{solved}\PYG{o}{.}\PYG{n}{y}\PYG{p}{[}\PYG{l+m+mi}{0}\PYG{p}{]}\PYG{p}{,}\PYG{n}{label} \PYG{o}{=}  \PYG{l+s+s2}{\PYGZdq{}}\PYG{l+s+s2}{IC = }\PYG{l+s+s2}{\PYGZdq{}} \PYG{o}{+} \PYG{n+nb}{str}\PYG{p}{(}\PYG{n}{initial\PYGZus{}condition}\PYG{p}{)}\PYG{p}{)}
\PYG{n}{plt}\PYG{o}{.}\PYG{n}{plot}\PYG{p}{(}\PYG{n}{t}\PYG{p}{,}\PYG{n}{solved2}\PYG{o}{.}\PYG{n}{y}\PYG{p}{[}\PYG{l+m+mi}{0}\PYG{p}{]}\PYG{p}{,} \PYG{n}{label} \PYG{o}{=} \PYG{l+s+s2}{\PYGZdq{}}\PYG{l+s+s2}{IC = }\PYG{l+s+s2}{\PYGZdq{}} \PYG{o}{+} \PYG{n+nb}{str}\PYG{p}{(}\PYG{n}{initial\PYGZus{}condition2}\PYG{p}{)} \PYG{p}{)}
\PYG{n}{plt}\PYG{o}{.}\PYG{n}{xlabel}\PYG{p}{(}\PYG{l+s+s2}{\PYGZdq{}}\PYG{l+s+s2}{t}\PYG{l+s+s2}{\PYGZdq{}}\PYG{p}{)}
\PYG{n}{plt}\PYG{o}{.}\PYG{n}{ylabel}\PYG{p}{(}\PYG{l+s+sa}{r}\PYG{l+s+s2}{\PYGZdq{}}\PYG{l+s+s2}{\PYGZdl{}}\PYG{l+s+s2}{\PYGZbs{}}\PYG{l+s+s2}{theta\PYGZdl{}}\PYG{l+s+s2}{\PYGZdq{}}\PYG{p}{)}
\PYG{n}{plt}\PYG{o}{.}\PYG{n}{legend}\PYG{p}{(}\PYG{p}{)}
\PYG{n}{plt}\PYG{o}{.}\PYG{n}{grid}\PYG{p}{(}\PYG{p}{)}
\PYG{n}{plt}\PYG{o}{.}\PYG{n}{subplot}\PYG{p}{(}\PYG{l+m+mi}{1}\PYG{p}{,}\PYG{l+m+mi}{2}\PYG{p}{,}\PYG{l+m+mi}{2}\PYG{p}{)}
\PYG{n}{plt}\PYG{o}{.}\PYG{n}{plot}\PYG{p}{(}\PYG{n}{solved}\PYG{o}{.}\PYG{n}{y}\PYG{p}{[}\PYG{l+m+mi}{0}\PYG{p}{]}\PYG{p}{,}\PYG{n}{solved}\PYG{o}{.}\PYG{n}{y}\PYG{p}{[}\PYG{l+m+mi}{1}\PYG{p}{]}\PYG{p}{,} \PYG{n}{label} \PYG{o}{=}  \PYG{l+s+s2}{\PYGZdq{}}\PYG{l+s+s2}{IC = }\PYG{l+s+s2}{\PYGZdq{}} \PYG{o}{+} \PYG{n+nb}{str}\PYG{p}{(}\PYG{n}{initial\PYGZus{}condition}\PYG{p}{)}\PYG{p}{)}
\PYG{n}{plt}\PYG{o}{.}\PYG{n}{plot}\PYG{p}{(}\PYG{n}{solved2}\PYG{o}{.}\PYG{n}{y}\PYG{p}{[}\PYG{l+m+mi}{0}\PYG{p}{]}\PYG{p}{,}\PYG{n}{solved2}\PYG{o}{.}\PYG{n}{y}\PYG{p}{[}\PYG{l+m+mi}{1}\PYG{p}{]}\PYG{p}{,} \PYG{n}{label} \PYG{o}{=} \PYG{l+s+s2}{\PYGZdq{}}\PYG{l+s+s2}{IC = }\PYG{l+s+s2}{\PYGZdq{}} \PYG{o}{+} \PYG{n+nb}{str}\PYG{p}{(}\PYG{n}{initial\PYGZus{}condition2}\PYG{p}{)}\PYG{p}{)}
\PYG{n}{plt}\PYG{o}{.}\PYG{n}{xlabel}\PYG{p}{(}\PYG{l+s+sa}{r}\PYG{l+s+s2}{\PYGZdq{}}\PYG{l+s+s2}{\PYGZdl{}}\PYG{l+s+s2}{\PYGZbs{}}\PYG{l+s+s2}{theta\PYGZdl{}}\PYG{l+s+s2}{\PYGZdq{}}\PYG{p}{)}
\PYG{n}{plt}\PYG{o}{.}\PYG{n}{ylabel}\PYG{p}{(}\PYG{l+s+sa}{r}\PYG{l+s+s2}{\PYGZdq{}}\PYG{l+s+s2}{\PYGZdl{}v\PYGZdl{}}\PYG{l+s+s2}{\PYGZdq{}}\PYG{p}{)}
\PYG{n}{plt}\PYG{o}{.}\PYG{n}{legend}\PYG{p}{(}\PYG{p}{)}
\PYG{n}{plt}\PYG{o}{.}\PYG{n}{grid}\PYG{p}{(}\PYG{p}{)}
\PYG{n}{plt}\PYG{o}{.}\PYG{n}{show}\PYG{p}{(}\PYG{p}{)}
\end{sphinxVerbatim}

\end{sphinxuseclass}\end{sphinxVerbatimInput}
\begin{sphinxVerbatimOutput}

\begin{sphinxuseclass}{cell_output}
\noindent\sphinxincludegraphics{{CHAOS_8_0}.png}

\end{sphinxuseclass}\end{sphinxVerbatimOutput}

\end{sphinxuseclass}
\sphinxAtStartPar
\sphinxstylestrong{✅ Do this}
\begin{enumerate}
\sphinxsetlistlabels{\arabic}{enumi}{enumii}{}{.}%
\item {} 
\sphinxAtStartPar
Discuss the above plots with your neighbors.
\begin{itemize}
\item {} 
\sphinxAtStartPar
Do you notice any patterns or is the motion totally unpredictable?

\item {} 
\sphinxAtStartPar
Try increasing the integration time. Does that reveal any structure?

\item {} 
\sphinxAtStartPar
Can you think of looking at this system another way that would reveal more?

\end{itemize}

\end{enumerate}


\section{Poincaré Section}
\label{\detokenize{content/1_mechanics/CHAOS:poincare-section}}
\sphinxAtStartPar
To get a better sense of what’s actually going on here, and to maybe have a hope of actually seeing if this thing has an attractor, we can use what is called a \sphinxstylestrong{Poincaré section}. The idea behind a Poincaré section is as follows: since the force term of this system is periodic, we should only look at points in phase space where the force term is at the same point in its cycle. Doing this reveals the breautiful fractal cross section of an attractor.

\sphinxAtStartPar
\sphinxstylestrong{✅ Do this}

\sphinxAtStartPar
Modify the code below to create a Poincaré section of the DPP.
\begin{itemize}
\item {} 
\sphinxAtStartPar
Hint for line 5: How long is a drive period?

\item {} 
\sphinxAtStartPar
Hint for line 17: The range of values that \(\theta\) takes on is too large. What can you restrict them to?

\end{itemize}

\begin{sphinxuseclass}{cell}\begin{sphinxVerbatimInput}

\begin{sphinxuseclass}{cell_input}
\begin{sphinxVerbatim}[commandchars=\\\{\}]
\PYG{c+c1}{\PYGZsh{} Time Setup}
\PYG{n}{N} \PYG{o}{=} \PYG{l+m+mi}{5000} \PYG{c+c1}{\PYGZsh{} Number of Drive Periods to integrate for}

\PYG{c+c1}{\PYGZsh{}\PYGZsh{}\PYGZsh{}\PYGZsh{}\PYGZsh{}\PYGZsh{}\PYGZsh{}\PYGZsh{}\PYGZsh{}\PYGZsh{}}
\PYG{n}{t\PYGZus{}period} \PYG{o}{=} \PYG{l+m+mf}{2.} \PYG{c+c1}{\PYGZsh{}\PYGZsh{} CHANGE }
\PYG{c+c1}{\PYGZsh{}\PYGZsh{}\PYGZsh{}\PYGZsh{}\PYGZsh{}\PYGZsh{}\PYGZsh{}\PYGZsh{}\PYGZsh{}\PYGZsh{}}

\PYG{n}{t} \PYG{o}{=} \PYG{n}{np}\PYG{o}{.}\PYG{n}{linspace}\PYG{p}{(}\PYG{l+m+mi}{0}\PYG{p}{,}\PYG{n}{N}\PYG{o}{*}\PYG{n}{t\PYGZus{}period}\PYG{p}{,} \PYG{n}{N}\PYG{o}{+}\PYG{l+m+mi}{1}\PYG{p}{)} \PYG{c+c1}{\PYGZsh{} Note: this is not the same as the points used for integration, so we need not worry about this ruining our numerical accuracy}
\PYG{n}{tspan} \PYG{o}{=} \PYG{p}{(}\PYG{n}{t}\PYG{p}{[}\PYG{l+m+mi}{0}\PYG{p}{]}\PYG{p}{,}\PYG{n}{t}\PYG{p}{[}\PYG{o}{\PYGZhy{}}\PYG{l+m+mi}{1}\PYG{p}{]}\PYG{p}{)}

\PYG{c+c1}{\PYGZsh{} Call integrator for each initial condition}
\PYG{n}{solved} \PYG{o}{=} \PYG{n}{solve\PYGZus{}ivp}\PYG{p}{(}\PYG{n}{DDP}\PYG{p}{,}\PYG{n}{tspan}\PYG{p}{,}\PYG{n}{initial\PYGZus{}condition}\PYG{p}{,}\PYG{n}{t\PYGZus{}eval} \PYG{o}{=} \PYG{n}{t}\PYG{p}{,} \PYG{n}{args} \PYG{o}{=} \PYG{p}{(}\PYG{n}{beta}\PYG{p}{,}\PYG{n}{omega\PYGZus{}natural}\PYG{p}{,}\PYG{n}{gamma}\PYG{p}{,}\PYG{n}{omega\PYGZus{}drive}\PYG{p}{)}\PYG{p}{)}

\PYG{n}{poincare\PYGZus{}theta}\PYG{p}{,} \PYG{n}{v} \PYG{o}{=} \PYG{n}{solved}\PYG{o}{.}\PYG{n}{y}\PYG{p}{[}\PYG{l+m+mi}{0}\PYG{p}{]}\PYG{p}{,}\PYG{n}{solved}\PYG{o}{.}\PYG{n}{y}\PYG{p}{[}\PYG{l+m+mi}{1}\PYG{p}{]}

\PYG{c+c1}{\PYGZsh{}\PYGZsh{}\PYGZsh{}\PYGZsh{}\PYGZsh{}\PYGZsh{}\PYGZsh{}}
\PYG{n}{poincare\PYGZus{}theta} \PYG{o}{=} \PYG{n}{poincare\PYGZus{}theta} \PYG{c+c1}{\PYGZsh{}\PYGZsh{} CHANGE}
\PYG{c+c1}{\PYGZsh{}\PYGZsh{}\PYGZsh{}\PYGZsh{}\PYGZsh{}\PYGZsh{}\PYGZsh{}}

\PYG{c+c1}{\PYGZsh{} Plotting}
\PYG{n}{fig} \PYG{o}{=} \PYG{n}{plt}\PYG{o}{.}\PYG{n}{figure}\PYG{p}{(}\PYG{n}{figsize} \PYG{o}{=} \PYG{p}{(}\PYG{l+m+mi}{10}\PYG{p}{,}\PYG{l+m+mi}{10}\PYG{p}{)}\PYG{p}{,}\PYG{n}{dpi} \PYG{o}{=} \PYG{l+m+mi}{300}\PYG{p}{)}
\PYG{n}{plt}\PYG{o}{.}\PYG{n}{scatter}\PYG{p}{(}\PYG{n}{poincare\PYGZus{}theta}\PYG{p}{,}\PYG{n}{v}\PYG{p}{,}\PYG{n}{s} \PYG{o}{=} \PYG{l+m+mf}{0.1}\PYG{p}{)}
\PYG{n}{plt}\PYG{o}{.}\PYG{n}{title}\PYG{p}{(}\PYG{l+s+s2}{\PYGZdq{}}\PYG{l+s+s2}{Poincaré Section}\PYG{l+s+s2}{\PYGZdq{}}\PYG{p}{)}
\PYG{n}{plt}\PYG{o}{.}\PYG{n}{xlabel}\PYG{p}{(}\PYG{l+s+sa}{r}\PYG{l+s+s2}{\PYGZdq{}}\PYG{l+s+s2}{\PYGZdl{}}\PYG{l+s+s2}{\PYGZbs{}}\PYG{l+s+s2}{theta\PYGZdl{}}\PYG{l+s+s2}{\PYGZdq{}}\PYG{p}{)}
\PYG{n}{plt}\PYG{o}{.}\PYG{n}{ylabel}\PYG{p}{(}\PYG{l+s+sa}{r}\PYG{l+s+s2}{\PYGZdq{}}\PYG{l+s+s2}{\PYGZdl{}v\PYGZdl{}}\PYG{l+s+s2}{\PYGZdq{}}\PYG{p}{)}
\PYG{n}{plt}\PYG{o}{.}\PYG{n}{grid}\PYG{p}{(}\PYG{p}{)}
\PYG{n}{plt}\PYG{o}{.}\PYG{n}{show}\PYG{p}{(}\PYG{p}{)}
\end{sphinxVerbatim}

\end{sphinxuseclass}\end{sphinxVerbatimInput}
\begin{sphinxVerbatimOutput}

\begin{sphinxuseclass}{cell_output}
\noindent\sphinxincludegraphics{{CHAOS_11_0}.png}

\end{sphinxuseclass}\end{sphinxVerbatimOutput}

\end{sphinxuseclass}
\sphinxAtStartPar
\sphinxstylestrong{✅ Do this}

\sphinxAtStartPar
Spend any time you have left in class investigating the DDP. Some things to try:
\begin{itemize}
\item {} 
\sphinxAtStartPar
Try decreasing the strength of the driving force by decreasing \(\gamma\).
\begin{itemize}
\item {} 
\sphinxAtStartPar
Is the system always chaotic for any \(\gamma >0\)?

\item {} 
\sphinxAtStartPar
How does the system’s behavior change as you vary \(\gamma\)?

\item {} 
\sphinxAtStartPar
Here you might want to look up \sphinxstylestrong{transient chaos} and \sphinxstylestrong{fractal basin boundaries}. Strogatz chapter 12 explains this well (and has some pretty pictures!).

\end{itemize}

\item {} 
\sphinxAtStartPar
What does the Poincare section look like when you have more predictable motion?
\begin{itemize}
\item {} 
\sphinxAtStartPar
How many points does it have?

\item {} 
\sphinxAtStartPar
Can you vary ICs or parameters to change how many points it has without giving in to total chaos?

\item {} 
\sphinxAtStartPar
Here you might want to look up \sphinxstylestrong{period doubling} and \sphinxstylestrong{fiegenbaum number} Taylor classical mechanics chapter 12 has some great stuff on this.

\end{itemize}

\end{itemize}

\sphinxstepscope


\chapter{21 Sept 23 \sphinxhyphen{} Activity: ODE Games}
\label{\detokenize{content/1_mechanics/ODE_games:sept-23-activity-ode-games}}\label{\detokenize{content/1_mechanics/ODE_games::doc}}
\sphinxAtStartPar
Y’all have worked with several different ODEs. We’ve learned that we can use phase space to investigate different potential families of solutions. We’ve learned how to read information from phase space for systems we are familiar with like the SHO and the large angle Pendulum. We’ve also learned how to use numerical integration to find trajectories of the system in time and in phase space, and how chaotic systems behave. Today, you will investigate a new model. There’s quite a few well\sphinxhyphen{}known models.

\sphinxAtStartPar
Some of these models are listed below with links to articles describing them or the ideas related to them. \sphinxstyleemphasis{This is for reference, you don’t need to read or understand anything deeply from these articles.}

\sphinxAtStartPar
A few 2nd order models:
\begin{enumerate}
\sphinxsetlistlabels{\arabic}{enumi}{enumii}{}{.}%
\item {} 
\sphinxAtStartPar
\sphinxhref{https://en.wikipedia.org/wiki/Double-well\_potential}{Double Well Potential:}

\end{enumerate}
\begin{equation*}
\begin{split}\dot{x}=y; \dot{y} = -x+y(1-x^2)\end{split}
\end{equation*}\begin{enumerate}
\sphinxsetlistlabels{\arabic}{enumi}{enumii}{}{.}%
\item {} 
\sphinxAtStartPar
\sphinxhref{https://en.wikipedia.org/wiki/Stability\_theory}{Dipole Fixed Points:}

\end{enumerate}
\begin{equation*}
\begin{split}\dot{x} = 2xy; \dot{y} = y^2 - x^2\end{split}
\end{equation*}\begin{enumerate}
\sphinxsetlistlabels{\arabic}{enumi}{enumii}{}{.}%
\item {} 
\sphinxAtStartPar
\sphinxhref{https://en.wikipedia.org/wiki/Anharmonicity}{Anharmonic Oscillator (Symmetric)}

\end{enumerate}
\begin{equation*}
\begin{split}m\ddot{x} + b\dot{x} + k_1 x + k_2 x^3 = 0\end{split}
\end{equation*}\begin{enumerate}
\sphinxsetlistlabels{\arabic}{enumi}{enumii}{}{.}%
\item {} 
\sphinxAtStartPar
\sphinxhref{https://en.wikipedia.org/wiki/Duffing\_equation}{Duffing Oscillator}

\end{enumerate}
\begin{equation*}
\begin{split}\ddot{x} + x + \epsilon x^3=0\end{split}
\end{equation*}\begin{enumerate}
\sphinxsetlistlabels{\arabic}{enumi}{enumii}{}{.}%
\item {} 
\sphinxAtStartPar
\sphinxhref{https://en.wikipedia.org/wiki/Glycolytic\_oscillation}{Glycolysis model}

\end{enumerate}
\begin{equation*}
\begin{split}\dot{x} = -x + ay + x^2 y ; \dot{y} = b - ay - x^2y\end{split}
\end{equation*}\begin{enumerate}
\sphinxsetlistlabels{\arabic}{enumi}{enumii}{}{.}%
\item {} 
\sphinxAtStartPar
\sphinxhref{https://en.wikipedia.org/wiki/Kuramoto\_model}{Coupled synchronizing oscillators}:

\end{enumerate}
\begin{equation*}
\begin{split}\dot{\theta}_1 = \omega_1 + K_1 \sin(\theta_2 - \theta_1) ; \dot{\theta}_2 = \omega_2 + K_2\sin(\theta_1 - \theta_2) \end{split}
\end{equation*}\begin{enumerate}
\sphinxsetlistlabels{\arabic}{enumi}{enumii}{}{.}%
\item {} 
\sphinxAtStartPar
\sphinxhref{https://www.princeton.edu/~stonelab/Publications/pdfs/From\%20Howard/JFM/StoneNadimStrogatzChaoticStreamlinesJFM1990.pdf}{Drop in stokes flow}

\end{enumerate}
\begin{equation*}
\begin{split}\dot{x} = \frac{\sqrt{2}}{4}x (x-1)\sin(\phi); \dot{\phi} = \frac{1}{2}\left[\beta - \frac{1}{\sqrt 2}\cos(\phi) - \frac{1}{8\sqrt{2}}x\cos(\phi)\right]\end{split}
\end{equation*}
\sphinxAtStartPar
We want to let your group decide what you want to explore. But a little guidance when making those choices:
\begin{enumerate}
\sphinxsetlistlabels{\arabic}{enumi}{enumii}{}{.}%
\item {} 
\sphinxAtStartPar
The 2nd order oscillators are relatively straightforward to implement in the code below if you break them into 1st order ODEs. Remember that you have to do that!

\item {} 
\sphinxAtStartPar
Several of these models have parameters, which you can choose, but maybe look into reasonable values. These models are likely the more challenging for this assignment because you have to keep track of and pass parameters.

\item {} 
\sphinxAtStartPar
Some of these models have a versions with time depdendent components , i.e., some \(F(t)\). Adding these components once you’ve explored the non\sphinxhyphen{}time\sphinxhyphen{}dependent version gives a ton more to explore!

\end{enumerate}


\section{Activity}
\label{\detokenize{content/1_mechanics/ODE_games:activity}}
\sphinxAtStartPar
Pick an ODE as a group and explore it as we have done in class. The critical element here is not only working on making the models but finding where the key components of our model evaluation framework appear in your work. It is ok if it doesn’t all show up for you. We will dsicuss as a class how we might see these components in our modeling work. As a reminder, here’s the framework:


\subsection{Goal: Investigate physical systems (0.30)}
\label{\detokenize{content/1_mechanics/ODE_games:goal-investigate-physical-systems-0-30}}\begin{itemize}
\item {} 
\sphinxAtStartPar
How well does your computational essay predict or explain the system of interest?

\item {} 
\sphinxAtStartPar
How well does your computational essay allow the user to explore and investigate the system?

\end{itemize}


\subsection{Goal: Construct and document a reproducible process (0.10)}
\label{\detokenize{content/1_mechanics/ODE_games:goal-construct-and-document-a-reproducible-process-0-10}}\begin{itemize}
\item {} 
\sphinxAtStartPar
How well does your computational essay reproduce your results and claims?

\item {} 
\sphinxAtStartPar
How  well documented is your computational essay?

\end{itemize}


\subsection{Goal: Use analytical, computational, and graphical approaches (0.30)}
\label{\detokenize{content/1_mechanics/ODE_games:goal-use-analytical-computational-and-graphical-approaches-0-30}}\begin{itemize}
\item {} 
\sphinxAtStartPar
How well does your computational essay document your assumptions?

\item {} 
\sphinxAtStartPar
How well does your computational essay produce an understandable and parsimonious model?

\item {} 
\sphinxAtStartPar
How well does your computational essay explain the limitations of your analysis?

\end{itemize}


\subsection{Goal: Provide evidence of the quality of their work}
\label{\detokenize{content/1_mechanics/ODE_games:goal-provide-evidence-of-the-quality-of-their-work}}\begin{itemize}
\item {} 
\sphinxAtStartPar
How well does your computational essay present  the case for its claims?

\item {} 
\sphinxAtStartPar
How well validated  is your model?

\end{itemize}


\subsection{Goal: Collaborate effectively}
\label{\detokenize{content/1_mechanics/ODE_games:goal-collaborate-effectively}}\begin{itemize}
\item {} 
\sphinxAtStartPar
How well did you share  in the class’s knowledge?
\begin{itemize}
\item {} 
\sphinxAtStartPar
How well is that documented in your computational essay?

\end{itemize}

\item {} 
\sphinxAtStartPar
How well did you work with your partner ? \sphinxstyleemphasis{For those choosing to do so}

\end{itemize}

\sphinxAtStartPar
Below is some starter code to help you get started.


\subsection{Starter Code}
\label{\detokenize{content/1_mechanics/ODE_games:starter-code}}
\sphinxAtStartPar
Here’s some working code from last week when we looked at the Van der Pol oscillator to help you get started on your invesigation.

\sphinxAtStartPar
For reference, the Van der Pol oscillator is given by:
\begin{equation*}
\begin{split}
\dot{x} = v \hspace{1in} \dot{v} = -\mu (x^2 - 1)v - x
\end{split}
\end{equation*}
\begin{sphinxuseclass}{cell}\begin{sphinxVerbatimInput}

\begin{sphinxuseclass}{cell_input}
\begin{sphinxVerbatim}[commandchars=\\\{\}]
\PYG{k+kn}{import} \PYG{n+nn}{numpy} \PYG{k}{as} \PYG{n+nn}{np}
\PYG{k+kn}{import} \PYG{n+nn}{matplotlib}\PYG{n+nn}{.}\PYG{n+nn}{pyplot} \PYG{k}{as} \PYG{n+nn}{plt}
\PYG{k+kn}{from} \PYG{n+nn}{scipy}\PYG{n+nn}{.}\PYG{n+nn}{integrate} \PYG{k+kn}{import} \PYG{n}{solve\PYGZus{}ivp}

\PYG{k}{def} \PYG{n+nf}{VP\PYGZus{}eqn}\PYG{p}{(}\PYG{n}{x}\PYG{p}{,} \PYG{n}{v}\PYG{p}{,} \PYG{n}{mu} \PYG{o}{=} \PYG{l+m+mf}{1.}\PYG{p}{)}\PYG{p}{:}
    \PYG{n}{xdot}\PYG{p}{,} \PYG{n}{vdot} \PYG{o}{=} \PYG{p}{[}\PYG{n}{v}\PYG{p}{,}\PYG{o}{\PYGZhy{}}\PYG{n}{mu} \PYG{o}{*} \PYG{p}{(}\PYG{n}{x}\PYG{o}{*}\PYG{o}{*}\PYG{l+m+mi}{2} \PYG{o}{\PYGZhy{}} \PYG{l+m+mi}{1}\PYG{p}{)}\PYG{o}{*}\PYG{n}{v} \PYG{o}{\PYGZhy{}} \PYG{n}{x}\PYG{p}{]}
    \PYG{k}{return} \PYG{n}{xdot}\PYG{p}{,} \PYG{n}{vdot}

\PYG{k}{def} \PYG{n+nf}{VP\PYGZus{}phase}\PYG{p}{(}\PYG{n}{X}\PYG{p}{,} \PYG{n}{VX}\PYG{p}{,} \PYG{n}{mu}\PYG{p}{)}\PYG{p}{:}
    \PYG{n}{xdot}\PYG{p}{,} \PYG{n}{vdot} \PYG{o}{=} \PYG{n}{np}\PYG{o}{.}\PYG{n}{zeros}\PYG{p}{(}\PYG{n}{X}\PYG{o}{.}\PYG{n}{shape}\PYG{p}{)}\PYG{p}{,} \PYG{n}{np}\PYG{o}{.}\PYG{n}{zeros}\PYG{p}{(}\PYG{n}{VX}\PYG{o}{.}\PYG{n}{shape}\PYG{p}{)}
    \PYG{n}{Xlim}\PYG{p}{,} \PYG{n}{Ylim} \PYG{o}{=} \PYG{n}{X}\PYG{o}{.}\PYG{n}{shape}
    \PYG{k}{for} \PYG{n}{i} \PYG{o+ow}{in} \PYG{n+nb}{range}\PYG{p}{(}\PYG{n}{Xlim}\PYG{p}{)}\PYG{p}{:}
        \PYG{k}{for} \PYG{n}{j} \PYG{o+ow}{in} \PYG{n+nb}{range}\PYG{p}{(}\PYG{n}{Ylim}\PYG{p}{)}\PYG{p}{:}
            \PYG{n}{xloc} \PYG{o}{=} \PYG{n}{X}\PYG{p}{[}\PYG{n}{i}\PYG{p}{,} \PYG{n}{j}\PYG{p}{]}
            \PYG{n}{yloc} \PYG{o}{=} \PYG{n}{VX}\PYG{p}{[}\PYG{n}{i}\PYG{p}{,} \PYG{n}{j}\PYG{p}{]}
            \PYG{n}{xdot}\PYG{p}{[}\PYG{n}{i}\PYG{p}{,}\PYG{n}{j}\PYG{p}{]}\PYG{p}{,} \PYG{n}{vdot}\PYG{p}{[}\PYG{n}{i}\PYG{p}{,}\PYG{n}{j}\PYG{p}{]} \PYG{o}{=} \PYG{n}{VP\PYGZus{}eqn}\PYG{p}{(}\PYG{n}{xloc}\PYG{p}{,} \PYG{n}{yloc}\PYG{p}{,}\PYG{n}{mu}\PYG{p}{)}
    \PYG{k}{return} \PYG{n}{xdot}\PYG{p}{,} \PYG{n}{vdot}

\PYG{k}{def} \PYG{n+nf}{VP\PYGZus{}eqn\PYGZus{}for\PYGZus{}solve\PYGZus{}ivp}\PYG{p}{(}\PYG{n}{t}\PYG{p}{,}\PYG{n}{curr\PYGZus{}vals}\PYG{p}{,} \PYG{n}{mu}\PYG{o}{=}\PYG{l+m+mi}{1}\PYG{p}{)}\PYG{p}{:} \PYG{c+c1}{\PYGZsh{} need to rephrase this to work with what solve\PYGZus{}ivp expects}
    \PYG{n}{x}\PYG{p}{,} \PYG{n}{v} \PYG{o}{=} \PYG{n}{curr\PYGZus{}vals} 
    \PYG{n}{xdot}\PYG{p}{,} \PYG{n}{vdot} \PYG{o}{=} \PYG{n}{VP\PYGZus{}eqn}\PYG{p}{(}\PYG{n}{x}\PYG{p}{,}\PYG{n}{v}\PYG{p}{,}\PYG{n}{mu}\PYG{p}{)}
    \PYG{k}{return} \PYG{n}{xdot}\PYG{p}{,}\PYG{n}{vdot}

\PYG{c+c1}{\PYGZsh{} Numerical Integration}
\PYG{n}{tmax} \PYG{o}{=} \PYG{l+m+mi}{20}
\PYG{n}{dt} \PYG{o}{=} \PYG{l+m+mf}{0.05}
\PYG{n}{tspan} \PYG{o}{=} \PYG{p}{(}\PYG{l+m+mi}{0}\PYG{p}{,}\PYG{n}{tmax}\PYG{p}{)}
\PYG{n}{t} \PYG{o}{=} \PYG{n}{np}\PYG{o}{.}\PYG{n}{arange}\PYG{p}{(}\PYG{l+m+mi}{0}\PYG{p}{,}\PYG{n}{tmax}\PYG{p}{,}\PYG{n}{dt}\PYG{p}{)}
\PYG{n}{mu} \PYG{o}{=} \PYG{l+m+mf}{1.}
\PYG{n}{initial\PYGZus{}condition} \PYG{o}{=} \PYG{p}{[}\PYG{l+m+mi}{1}\PYG{p}{,} \PYG{l+m+mi}{1}\PYG{p}{]} 
\PYG{n}{solved} \PYG{o}{=} \PYG{n}{solve\PYGZus{}ivp}\PYG{p}{(}\PYG{n}{VP\PYGZus{}eqn\PYGZus{}for\PYGZus{}solve\PYGZus{}ivp}\PYG{p}{,}\PYG{n}{tspan}\PYG{p}{,}\PYG{n}{initial\PYGZus{}condition}\PYG{p}{,}\PYG{n}{t\PYGZus{}eval} \PYG{o}{=} \PYG{n}{t}\PYG{p}{,} \PYG{n}{args} \PYG{o}{=} \PYG{p}{(}\PYG{n}{mu}\PYG{p}{,}\PYG{p}{)}\PYG{p}{,}\PYG{n}{method}\PYG{o}{=}\PYG{l+s+s2}{\PYGZdq{}}\PYG{l+s+s2}{RK45}\PYG{l+s+s2}{\PYGZdq{}}\PYG{p}{)}

\PYG{c+c1}{\PYGZsh{} Plotting stuff}
\PYG{n}{N} \PYG{o}{=} \PYG{l+m+mi}{40}
\PYG{n}{x} \PYG{o}{=} \PYG{n}{np}\PYG{o}{.}\PYG{n}{linspace}\PYG{p}{(}\PYG{o}{\PYGZhy{}}\PYG{l+m+mf}{3.}\PYG{p}{,} \PYG{l+m+mf}{3.}\PYG{p}{,} \PYG{n}{N}\PYG{p}{)}
\PYG{n}{v} \PYG{o}{=} \PYG{n}{np}\PYG{o}{.}\PYG{n}{linspace}\PYG{p}{(}\PYG{o}{\PYGZhy{}}\PYG{l+m+mf}{3.}\PYG{p}{,} \PYG{l+m+mf}{3.}\PYG{p}{,} \PYG{n}{N}\PYG{p}{)}
\PYG{n}{X}\PYG{p}{,} \PYG{n}{V} \PYG{o}{=} \PYG{n}{np}\PYG{o}{.}\PYG{n}{meshgrid}\PYG{p}{(}\PYG{n}{x}\PYG{p}{,} \PYG{n}{v}\PYG{p}{)}
\PYG{n}{xdot}\PYG{p}{,} \PYG{n}{vdot} \PYG{o}{=} \PYG{n}{VP\PYGZus{}phase}\PYG{p}{(}\PYG{n}{X}\PYG{p}{,} \PYG{n}{V}\PYG{p}{,}\PYG{n}{mu}\PYG{p}{)}
\PYG{n}{ax} \PYG{o}{=} \PYG{n}{plt}\PYG{o}{.}\PYG{n}{figure}\PYG{p}{(}\PYG{n}{figsize}\PYG{o}{=}\PYG{p}{(}\PYG{l+m+mi}{10}\PYG{p}{,}\PYG{l+m+mi}{10}\PYG{p}{)}\PYG{p}{)}
\PYG{n}{Q} \PYG{o}{=} \PYG{n}{plt}\PYG{o}{.}\PYG{n}{streamplot}\PYG{p}{(}\PYG{n}{X}\PYG{p}{,} \PYG{n}{V}\PYG{p}{,} \PYG{n}{xdot}\PYG{p}{,} \PYG{n}{vdot}\PYG{p}{,} \PYG{n}{color}\PYG{o}{=}\PYG{l+s+s1}{\PYGZsq{}}\PYG{l+s+s1}{k}\PYG{l+s+s1}{\PYGZsq{}}\PYG{p}{,}\PYG{n}{broken\PYGZus{}streamlines} \PYG{o}{=} \PYG{k+kc}{False}\PYG{p}{)}
\PYG{n}{plt}\PYG{o}{.}\PYG{n}{plot}\PYG{p}{(}\PYG{n}{solved}\PYG{o}{.}\PYG{n}{y}\PYG{p}{[}\PYG{l+m+mi}{0}\PYG{p}{]}\PYG{p}{,}\PYG{n}{solved}\PYG{o}{.}\PYG{n}{y}\PYG{p}{[}\PYG{l+m+mi}{1}\PYG{p}{]}\PYG{p}{,}\PYG{n}{lw} \PYG{o}{=} \PYG{l+m+mi}{3}\PYG{p}{,}\PYG{n}{c} \PYG{o}{=} \PYG{l+s+s1}{\PYGZsq{}}\PYG{l+s+s1}{red}\PYG{l+s+s1}{\PYGZsq{}}\PYG{p}{)} \PYG{c+c1}{\PYGZsh{} plot trajectory from solve\PYGZus{}ivp}
\PYG{n}{plt}\PYG{o}{.}\PYG{n}{grid}\PYG{p}{(}\PYG{p}{)}
\PYG{n}{plt}\PYG{o}{.}\PYG{n}{xlabel}\PYG{p}{(}\PYG{l+s+s1}{\PYGZsq{}}\PYG{l+s+s1}{\PYGZdl{}x\PYGZdl{}}\PYG{l+s+s1}{\PYGZsq{}}\PYG{p}{)}
\PYG{n}{plt}\PYG{o}{.}\PYG{n}{ylabel}\PYG{p}{(}\PYG{l+s+s1}{\PYGZsq{}}\PYG{l+s+s1}{\PYGZdl{}v\PYGZdl{}}\PYG{l+s+s1}{\PYGZsq{}}\PYG{p}{)}
\PYG{n}{plt}\PYG{o}{.}\PYG{n}{show}\PYG{p}{(}\PYG{p}{)}

\PYG{n}{plt}\PYG{o}{.}\PYG{n}{figure}\PYG{p}{(}\PYG{p}{)}
\PYG{n}{plt}\PYG{o}{.}\PYG{n}{plot}\PYG{p}{(}\PYG{n}{t}\PYG{p}{,}\PYG{n}{solved}\PYG{o}{.}\PYG{n}{y}\PYG{p}{[}\PYG{l+m+mi}{0}\PYG{p}{]}\PYG{p}{)}
\PYG{n}{plt}\PYG{o}{.}\PYG{n}{plot}\PYG{p}{(}\PYG{n}{t}\PYG{p}{,}\PYG{n}{solved}\PYG{o}{.}\PYG{n}{y}\PYG{p}{[}\PYG{l+m+mi}{1}\PYG{p}{]}\PYG{p}{)}
\PYG{n}{plt}\PYG{o}{.}\PYG{n}{show}\PYG{p}{(}\PYG{p}{)}
\end{sphinxVerbatim}

\end{sphinxuseclass}\end{sphinxVerbatimInput}
\begin{sphinxVerbatimOutput}

\begin{sphinxuseclass}{cell_output}
\noindent\sphinxincludegraphics{{ODE_games_3_0}.png}

\noindent\sphinxincludegraphics{{ODE_games_3_1}.png}

\end{sphinxuseclass}\end{sphinxVerbatimOutput}

\end{sphinxuseclass}
\sphinxstepscope


\part{2 \sphinxhyphen{} E\&M and PDEs}

\sphinxstepscope


\chapter{Intro to Electricity and Magnetism}
\label{\detokenize{content/2_EM/EM_intro:intro-to-electricity-and-magnetism}}\label{\detokenize{content/2_EM/EM_intro::doc}}
\sphinxstepscope


\chapter{26 Sep 23 \sphinxhyphen{} Graphing Electric Fields}
\label{\detokenize{content/2_EM/E_fields_graphing_184:sep-23-graphing-electric-fields}}\label{\detokenize{content/2_EM/E_fields_graphing_184::doc}}
\begin{sphinxuseclass}{cell}\begin{sphinxVerbatimInput}

\begin{sphinxuseclass}{cell_input}
\begin{sphinxVerbatim}[commandchars=\\\{\}]
\PYG{k+kn}{import} \PYG{n+nn}{numpy} \PYG{k}{as} \PYG{n+nn}{np}
\PYG{k+kn}{import} \PYG{n+nn}{matplotlib}\PYG{n+nn}{.}\PYG{n+nn}{pyplot} \PYG{k}{as} \PYG{n+nn}{plt}

\PYG{n}{x} \PYG{o}{=} \PYG{n}{np}\PYG{o}{.}\PYG{n}{linspace}\PYG{p}{(}\PYG{o}{\PYGZhy{}}\PYG{l+m+mi}{5}\PYG{p}{,}\PYG{l+m+mi}{5}\PYG{p}{,}\PYG{l+m+mi}{5}\PYG{p}{)}
\PYG{n}{y} \PYG{o}{=} \PYG{n}{np}\PYG{o}{.}\PYG{n}{linspace}\PYG{p}{(}\PYG{o}{\PYGZhy{}}\PYG{l+m+mi}{5}\PYG{p}{,}\PYG{l+m+mi}{5}\PYG{p}{,}\PYG{l+m+mi}{5}\PYG{p}{)}
\PYG{n}{X}\PYG{p}{,}\PYG{n}{Y} \PYG{o}{=} \PYG{n}{np}\PYG{o}{.}\PYG{n}{meshgrid}\PYG{p}{(}\PYG{n}{x}\PYG{p}{,}\PYG{n}{y}\PYG{p}{)}
\PYG{n}{u} \PYG{o}{=} \PYG{n}{X}\PYG{o}{*}\PYG{o}{*}\PYG{l+m+mi}{2}
\PYG{n}{v} \PYG{o}{=} \PYG{n}{Y}\PYG{o}{*}\PYG{o}{*}\PYG{l+m+mi}{2}

\PYG{n}{q} \PYG{o}{=} \PYG{l+m+mf}{1e\PYGZhy{}4}
\PYG{n}{r\PYGZus{}source} \PYG{o}{=} \PYG{n}{np}\PYG{o}{.}\PYG{n}{array}\PYG{p}{(}\PYG{p}{[}\PYG{l+m+mi}{0}\PYG{p}{,}\PYG{l+m+mi}{0}\PYG{p}{]}\PYG{p}{)}
\PYG{n}{k} \PYG{o}{=} \PYG{l+m+mf}{9e9}

\PYG{k}{def} \PYG{n+nf}{VP\PYGZus{}eqn}\PYG{p}{(}\PYG{n}{x}\PYG{p}{,} \PYG{n}{v}\PYG{p}{,} \PYG{n}{mu} \PYG{o}{=} \PYG{l+m+mf}{1.}\PYG{p}{)}\PYG{p}{:}
    \PYG{n}{xdot}\PYG{p}{,} \PYG{n}{vdot} \PYG{o}{=} \PYG{p}{[}\PYG{n}{v}\PYG{p}{,}\PYG{o}{\PYGZhy{}}\PYG{n}{mu} \PYG{o}{*} \PYG{p}{(}\PYG{n}{x}\PYG{o}{*}\PYG{o}{*}\PYG{l+m+mi}{2} \PYG{o}{\PYGZhy{}} \PYG{l+m+mi}{1}\PYG{p}{)}\PYG{o}{*}\PYG{n}{v} \PYG{o}{\PYGZhy{}} \PYG{n}{x}\PYG{p}{]}
    \PYG{k}{return} \PYG{n}{xdot}\PYG{p}{,} \PYG{n}{vdot}

\PYG{k}{def} \PYG{n+nf}{point\PYGZus{}charge\PYGZus{}E}\PYG{p}{(}\PYG{n}{X}\PYG{p}{,} \PYG{n}{Y}\PYG{p}{,} \PYG{n}{q}\PYG{p}{,}\PYG{n}{r\PYGZus{}source}\PYG{p}{)}\PYG{p}{:}
    \PYG{n}{xdot}\PYG{p}{,} \PYG{n}{vdot} \PYG{o}{=} \PYG{n}{np}\PYG{o}{.}\PYG{n}{zeros}\PYG{p}{(}\PYG{n}{X}\PYG{o}{.}\PYG{n}{shape}\PYG{p}{)}\PYG{p}{,} \PYG{n}{np}\PYG{o}{.}\PYG{n}{zeros}\PYG{p}{(}\PYG{n}{Y}\PYG{o}{.}\PYG{n}{shape}\PYG{p}{)}
    \PYG{n}{Xlim}\PYG{p}{,} \PYG{n}{Ylim} \PYG{o}{=} \PYG{n}{X}\PYG{o}{.}\PYG{n}{shape}
    \PYG{k}{for} \PYG{n}{i} \PYG{o+ow}{in} \PYG{n+nb}{range}\PYG{p}{(}\PYG{n}{Xlim}\PYG{p}{)}\PYG{p}{:}
        \PYG{k}{for} \PYG{n}{j} \PYG{o+ow}{in} \PYG{n+nb}{range}\PYG{p}{(}\PYG{n}{Ylim}\PYG{p}{)}\PYG{p}{:}
            \PYG{n}{xloc} \PYG{o}{=} \PYG{n}{X}\PYG{p}{[}\PYG{n}{i}\PYG{p}{,} \PYG{n}{j}\PYG{p}{]}
            \PYG{n}{yloc} \PYG{o}{=} \PYG{n}{Y}\PYG{p}{[}\PYG{n}{i}\PYG{p}{,} \PYG{n}{j}\PYG{p}{]}
            \PYG{n}{xdot}\PYG{p}{[}\PYG{n}{i}\PYG{p}{,}\PYG{n}{j}\PYG{p}{]}\PYG{p}{,} \PYG{n}{vdot}\PYG{p}{[}\PYG{n}{i}\PYG{p}{,}\PYG{n}{j}\PYG{p}{]} \PYG{o}{=} \PYG{n}{VP\PYGZus{}eqn}\PYG{p}{(}\PYG{n}{xloc}\PYG{p}{,} \PYG{n}{yloc}\PYG{p}{,}\PYG{n}{mu}\PYG{p}{)}
    \PYG{k}{return} \PYG{n}{xdot}\PYG{p}{,} \PYG{n}{vdot}

\PYG{n}{N} \PYG{o}{=} \PYG{l+m+mi}{40}
\PYG{n}{x} \PYG{o}{=} \PYG{n}{np}\PYG{o}{.}\PYG{n}{linspace}\PYG{p}{(}\PYG{o}{\PYGZhy{}}\PYG{l+m+mf}{3.}\PYG{p}{,} \PYG{l+m+mf}{3.}\PYG{p}{,} \PYG{n}{N}\PYG{p}{)}
\PYG{n}{v} \PYG{o}{=} \PYG{n}{np}\PYG{o}{.}\PYG{n}{linspace}\PYG{p}{(}\PYG{o}{\PYGZhy{}}\PYG{l+m+mf}{3.}\PYG{p}{,} \PYG{l+m+mf}{3.}\PYG{p}{,} \PYG{n}{N}\PYG{p}{)}
\PYG{n}{X}\PYG{p}{,} \PYG{n}{V} \PYG{o}{=} \PYG{n}{np}\PYG{o}{.}\PYG{n}{meshgrid}\PYG{p}{(}\PYG{n}{x}\PYG{p}{,} \PYG{n}{v}\PYG{p}{)}
\PYG{n}{xdot}\PYG{p}{,} \PYG{n}{vdot} \PYG{o}{=} \PYG{n}{point\PYGZus{}charge\PYGZus{}E}\PYG{p}{(}\PYG{n}{X}\PYG{p}{,} \PYG{n}{V}\PYG{p}{,}\PYG{n}{mu}\PYG{p}{)}
\PYG{n}{ax} \PYG{o}{=} \PYG{n}{plt}\PYG{o}{.}\PYG{n}{figure}\PYG{p}{(}\PYG{n}{figsize}\PYG{o}{=}\PYG{p}{(}\PYG{l+m+mi}{5}\PYG{p}{,}\PYG{l+m+mi}{5}\PYG{p}{)}\PYG{p}{)}
\PYG{n}{Q} \PYG{o}{=} \PYG{n}{plt}\PYG{o}{.}\PYG{n}{quiver}\PYG{p}{(}\PYG{n}{X}\PYG{p}{,} \PYG{n}{V}\PYG{p}{,} \PYG{n}{xdot}\PYG{p}{,} \PYG{n}{vdot}\PYG{p}{,} \PYG{n}{color}\PYG{o}{=}\PYG{l+s+s1}{\PYGZsq{}}\PYG{l+s+s1}{k}\PYG{l+s+s1}{\PYGZsq{}}\PYG{p}{)}
\PYG{n}{plt}\PYG{o}{.}\PYG{n}{grid}\PYG{p}{(}\PYG{p}{)}
\PYG{n}{plt}\PYG{o}{.}\PYG{n}{xlabel}\PYG{p}{(}\PYG{l+s+s1}{\PYGZsq{}}\PYG{l+s+s1}{\PYGZdl{}x\PYGZdl{}}\PYG{l+s+s1}{\PYGZsq{}}\PYG{p}{)}
\PYG{n}{plt}\PYG{o}{.}\PYG{n}{ylabel}\PYG{p}{(}\PYG{l+s+s1}{\PYGZsq{}}\PYG{l+s+s1}{\PYGZdl{}v\PYGZdl{}}\PYG{l+s+s1}{\PYGZsq{}}\PYG{p}{)}
\PYG{n}{plt}\PYG{o}{.}\PYG{n}{show}\PYG{p}{(}\PYG{p}{)}
\end{sphinxVerbatim}

\end{sphinxuseclass}\end{sphinxVerbatimInput}
\begin{sphinxVerbatimOutput}

\begin{sphinxuseclass}{cell_output}
\begin{sphinxVerbatim}[commandchars=\\\{\}]
\PYG{g+gt}{\PYGZhy{}\PYGZhy{}\PYGZhy{}\PYGZhy{}\PYGZhy{}\PYGZhy{}\PYGZhy{}\PYGZhy{}\PYGZhy{}\PYGZhy{}\PYGZhy{}\PYGZhy{}\PYGZhy{}\PYGZhy{}\PYGZhy{}\PYGZhy{}\PYGZhy{}\PYGZhy{}\PYGZhy{}\PYGZhy{}\PYGZhy{}\PYGZhy{}\PYGZhy{}\PYGZhy{}\PYGZhy{}\PYGZhy{}\PYGZhy{}\PYGZhy{}\PYGZhy{}\PYGZhy{}\PYGZhy{}\PYGZhy{}\PYGZhy{}\PYGZhy{}\PYGZhy{}\PYGZhy{}\PYGZhy{}\PYGZhy{}\PYGZhy{}\PYGZhy{}\PYGZhy{}\PYGZhy{}\PYGZhy{}\PYGZhy{}\PYGZhy{}\PYGZhy{}\PYGZhy{}\PYGZhy{}\PYGZhy{}\PYGZhy{}\PYGZhy{}\PYGZhy{}\PYGZhy{}\PYGZhy{}\PYGZhy{}\PYGZhy{}\PYGZhy{}\PYGZhy{}\PYGZhy{}\PYGZhy{}\PYGZhy{}\PYGZhy{}\PYGZhy{}\PYGZhy{}\PYGZhy{}\PYGZhy{}\PYGZhy{}\PYGZhy{}\PYGZhy{}\PYGZhy{}\PYGZhy{}\PYGZhy{}\PYGZhy{}\PYGZhy{}\PYGZhy{}}
\PYG{n+ne}{NameError}\PYG{g+gWhitespace}{                                 }Traceback (most recent call last)
\PYG{n+nn}{Input In [1],} in \PYG{n+ni}{\PYGZlt{}cell line: 32\PYGZgt{}}\PYG{n+nt}{()}
\PYG{g+gWhitespace}{     }\PYG{l+m+mi}{30} \PYG{n}{v} \PYG{o}{=} \PYG{n}{np}\PYG{o}{.}\PYG{n}{linspace}\PYG{p}{(}\PYG{o}{\PYGZhy{}}\PYG{l+m+mf}{3.}\PYG{p}{,} \PYG{l+m+mf}{3.}\PYG{p}{,} \PYG{n}{N}\PYG{p}{)}
\PYG{g+gWhitespace}{     }\PYG{l+m+mi}{31} \PYG{n}{X}\PYG{p}{,} \PYG{n}{V} \PYG{o}{=} \PYG{n}{np}\PYG{o}{.}\PYG{n}{meshgrid}\PYG{p}{(}\PYG{n}{x}\PYG{p}{,} \PYG{n}{v}\PYG{p}{)}
\PYG{n+ne}{\PYGZhy{}\PYGZhy{}\PYGZhy{}\PYGZgt{} }\PYG{l+m+mi}{32} \PYG{n}{xdot}\PYG{p}{,} \PYG{n}{vdot} \PYG{o}{=} \PYG{n}{point\PYGZus{}charge\PYGZus{}E}\PYG{p}{(}\PYG{n}{X}\PYG{p}{,} \PYG{n}{V}\PYG{p}{,}\PYG{n}{mu}\PYG{p}{)}
\PYG{g+gWhitespace}{     }\PYG{l+m+mi}{33} \PYG{n}{ax} \PYG{o}{=} \PYG{n}{plt}\PYG{o}{.}\PYG{n}{figure}\PYG{p}{(}\PYG{n}{figsize}\PYG{o}{=}\PYG{p}{(}\PYG{l+m+mi}{5}\PYG{p}{,}\PYG{l+m+mi}{5}\PYG{p}{)}\PYG{p}{)}
\PYG{g+gWhitespace}{     }\PYG{l+m+mi}{34} \PYG{n}{Q} \PYG{o}{=} \PYG{n}{plt}\PYG{o}{.}\PYG{n}{quiver}\PYG{p}{(}\PYG{n}{X}\PYG{p}{,} \PYG{n}{V}\PYG{p}{,} \PYG{n}{xdot}\PYG{p}{,} \PYG{n}{vdot}\PYG{p}{,} \PYG{n}{color}\PYG{o}{=}\PYG{l+s+s1}{\PYGZsq{}}\PYG{l+s+s1}{k}\PYG{l+s+s1}{\PYGZsq{}}\PYG{p}{)}

\PYG{n+ne}{NameError}: name \PYGZsq{}mu\PYGZsq{} is not defined
\end{sphinxVerbatim}

\end{sphinxuseclass}\end{sphinxVerbatimOutput}

\end{sphinxuseclass}
\begin{sphinxuseclass}{cell}\begin{sphinxVerbatimInput}

\begin{sphinxuseclass}{cell_input}
\begin{sphinxVerbatim}[commandchars=\\\{\}]
\PYG{k+kn}{import} \PYG{n+nn}{numpy} \PYG{k}{as} \PYG{n+nn}{np}
\PYG{k+kn}{import} \PYG{n+nn}{matplotlib}\PYG{n+nn}{.}\PYG{n+nn}{pyplot} \PYG{k}{as} \PYG{n+nn}{plt}
\PYG{k+kn}{from} \PYG{n+nn}{scipy}\PYG{n+nn}{.}\PYG{n+nn}{integrate} \PYG{k+kn}{import} \PYG{n}{solve\PYGZus{}ivp}

\PYG{k}{def} \PYG{n+nf}{VP\PYGZus{}eqn}\PYG{p}{(}\PYG{n}{x}\PYG{p}{,} \PYG{n}{v}\PYG{p}{,} \PYG{n}{mu} \PYG{o}{=} \PYG{l+m+mf}{1.}\PYG{p}{)}\PYG{p}{:}
    \PYG{n}{xdot}\PYG{p}{,} \PYG{n}{vdot} \PYG{o}{=} \PYG{p}{[}\PYG{n}{v}\PYG{p}{,}\PYG{o}{\PYGZhy{}}\PYG{n}{mu} \PYG{o}{*} \PYG{p}{(}\PYG{n}{x}\PYG{o}{*}\PYG{o}{*}\PYG{l+m+mi}{2} \PYG{o}{\PYGZhy{}} \PYG{l+m+mi}{1}\PYG{p}{)}\PYG{o}{*}\PYG{n}{v} \PYG{o}{\PYGZhy{}} \PYG{n}{x}\PYG{p}{]}
    \PYG{k}{return} \PYG{n}{xdot}\PYG{p}{,} \PYG{n}{vdot}

\PYG{k}{def} \PYG{n+nf}{VP\PYGZus{}phase}\PYG{p}{(}\PYG{n}{X}\PYG{p}{,} \PYG{n}{Y}\PYG{p}{,} \PYG{n}{mu}\PYG{p}{)}\PYG{p}{:}
    \PYG{n}{xdot}\PYG{p}{,} \PYG{n}{vdot} \PYG{o}{=} \PYG{n}{np}\PYG{o}{.}\PYG{n}{zeros}\PYG{p}{(}\PYG{n}{X}\PYG{o}{.}\PYG{n}{shape}\PYG{p}{)}\PYG{p}{,} \PYG{n}{np}\PYG{o}{.}\PYG{n}{zeros}\PYG{p}{(}\PYG{n}{Y}\PYG{o}{.}\PYG{n}{shape}\PYG{p}{)}
    \PYG{n}{Xlim}\PYG{p}{,} \PYG{n}{Ylim} \PYG{o}{=} \PYG{n}{X}\PYG{o}{.}\PYG{n}{shape}
    \PYG{k}{for} \PYG{n}{i} \PYG{o+ow}{in} \PYG{n+nb}{range}\PYG{p}{(}\PYG{n}{Xlim}\PYG{p}{)}\PYG{p}{:}
        \PYG{k}{for} \PYG{n}{j} \PYG{o+ow}{in} \PYG{n+nb}{range}\PYG{p}{(}\PYG{n}{Ylim}\PYG{p}{)}\PYG{p}{:}
            \PYG{n}{xloc} \PYG{o}{=} \PYG{n}{X}\PYG{p}{[}\PYG{n}{i}\PYG{p}{,} \PYG{n}{j}\PYG{p}{]}
            \PYG{n}{yloc} \PYG{o}{=} \PYG{n}{Y}\PYG{p}{[}\PYG{n}{i}\PYG{p}{,} \PYG{n}{j}\PYG{p}{]}
            \PYG{n}{xdot}\PYG{p}{[}\PYG{n}{i}\PYG{p}{,}\PYG{n}{j}\PYG{p}{]}\PYG{p}{,} \PYG{n}{vdot}\PYG{p}{[}\PYG{n}{i}\PYG{p}{,}\PYG{n}{j}\PYG{p}{]} \PYG{o}{=} \PYG{n}{VP\PYGZus{}eqn}\PYG{p}{(}\PYG{n}{xloc}\PYG{p}{,} \PYG{n}{yloc}\PYG{p}{,}\PYG{n}{mu}\PYG{p}{)}
    \PYG{k}{return} \PYG{n}{xdot}\PYG{p}{,} \PYG{n}{vdot}

\PYG{k}{def} \PYG{n+nf}{VP\PYGZus{}eqn\PYGZus{}for\PYGZus{}solve\PYGZus{}ivp}\PYG{p}{(}\PYG{n}{t}\PYG{p}{,}\PYG{n}{curr\PYGZus{}vals}\PYG{p}{,} \PYG{n}{mu}\PYG{o}{=}\PYG{l+m+mi}{1}\PYG{p}{)}\PYG{p}{:} \PYG{c+c1}{\PYGZsh{} need to rephrase this to work with what solve\PYGZus{}ivp expects}
    \PYG{n}{x}\PYG{p}{,} \PYG{n}{v} \PYG{o}{=} \PYG{n}{curr\PYGZus{}vals} 
    \PYG{n}{xdot}\PYG{p}{,} \PYG{n}{vdot} \PYG{o}{=} \PYG{n}{VP\PYGZus{}eqn}\PYG{p}{(}\PYG{n}{x}\PYG{p}{,}\PYG{n}{v}\PYG{p}{,}\PYG{n}{mu}\PYG{p}{)}
    \PYG{k}{return} \PYG{n}{xdot}\PYG{p}{,}\PYG{n}{vdot}

\PYG{c+c1}{\PYGZsh{} Numerical Integration}
\PYG{n}{tmax} \PYG{o}{=} \PYG{l+m+mi}{20}
\PYG{n}{dt} \PYG{o}{=} \PYG{l+m+mf}{0.05}
\PYG{n}{tspan} \PYG{o}{=} \PYG{p}{(}\PYG{l+m+mi}{0}\PYG{p}{,}\PYG{n}{tmax}\PYG{p}{)}
\PYG{n}{t} \PYG{o}{=} \PYG{n}{np}\PYG{o}{.}\PYG{n}{arange}\PYG{p}{(}\PYG{l+m+mi}{0}\PYG{p}{,}\PYG{n}{tmax}\PYG{p}{,}\PYG{n}{dt}\PYG{p}{)}
\PYG{n}{mu} \PYG{o}{=} \PYG{l+m+mf}{1.}
\PYG{n}{initial\PYGZus{}condition} \PYG{o}{=} \PYG{p}{[}\PYG{l+m+mi}{1}\PYG{p}{,} \PYG{l+m+mi}{1}\PYG{p}{]} 
\PYG{n}{solved} \PYG{o}{=} \PYG{n}{solve\PYGZus{}ivp}\PYG{p}{(}\PYG{n}{VP\PYGZus{}eqn\PYGZus{}for\PYGZus{}solve\PYGZus{}ivp}\PYG{p}{,}\PYG{n}{tspan}\PYG{p}{,}\PYG{n}{initial\PYGZus{}condition}\PYG{p}{,}\PYG{n}{t\PYGZus{}eval} \PYG{o}{=} \PYG{n}{t}\PYG{p}{,} \PYG{n}{args} \PYG{o}{=} \PYG{p}{(}\PYG{n}{mu}\PYG{p}{,}\PYG{p}{)}\PYG{p}{,}\PYG{n}{method}\PYG{o}{=}\PYG{l+s+s2}{\PYGZdq{}}\PYG{l+s+s2}{RK45}\PYG{l+s+s2}{\PYGZdq{}}\PYG{p}{)}

\PYG{c+c1}{\PYGZsh{} Plotting stuff}
\PYG{n}{N} \PYG{o}{=} \PYG{l+m+mi}{40}
\PYG{n}{x} \PYG{o}{=} \PYG{n}{np}\PYG{o}{.}\PYG{n}{linspace}\PYG{p}{(}\PYG{o}{\PYGZhy{}}\PYG{l+m+mf}{3.}\PYG{p}{,} \PYG{l+m+mf}{3.}\PYG{p}{,} \PYG{n}{N}\PYG{p}{)}
\PYG{n}{v} \PYG{o}{=} \PYG{n}{np}\PYG{o}{.}\PYG{n}{linspace}\PYG{p}{(}\PYG{o}{\PYGZhy{}}\PYG{l+m+mf}{3.}\PYG{p}{,} \PYG{l+m+mf}{3.}\PYG{p}{,} \PYG{n}{N}\PYG{p}{)}
\PYG{n}{X}\PYG{p}{,} \PYG{n}{V} \PYG{o}{=} \PYG{n}{np}\PYG{o}{.}\PYG{n}{meshgrid}\PYG{p}{(}\PYG{n}{x}\PYG{p}{,} \PYG{n}{v}\PYG{p}{)}
\PYG{n}{xdot}\PYG{p}{,} \PYG{n}{vdot} \PYG{o}{=} \PYG{n}{VP\PYGZus{}phase}\PYG{p}{(}\PYG{n}{X}\PYG{p}{,} \PYG{n}{V}\PYG{p}{,}\PYG{n}{mu}\PYG{p}{)}
\PYG{n}{ax} \PYG{o}{=} \PYG{n}{plt}\PYG{o}{.}\PYG{n}{figure}\PYG{p}{(}\PYG{n}{figsize}\PYG{o}{=}\PYG{p}{(}\PYG{l+m+mi}{10}\PYG{p}{,}\PYG{l+m+mi}{10}\PYG{p}{)}\PYG{p}{)}
\PYG{n}{Q} \PYG{o}{=} \PYG{n}{plt}\PYG{o}{.}\PYG{n}{quiver}\PYG{p}{(}\PYG{n}{X}\PYG{p}{,} \PYG{n}{V}\PYG{p}{,} \PYG{n}{xdot}\PYG{p}{,} \PYG{n}{vdot}\PYG{p}{,} \PYG{n}{color}\PYG{o}{=}\PYG{l+s+s1}{\PYGZsq{}}\PYG{l+s+s1}{k}\PYG{l+s+s1}{\PYGZsq{}}\PYG{p}{)}
\PYG{n}{plt}\PYG{o}{.}\PYG{n}{grid}\PYG{p}{(}\PYG{p}{)}
\PYG{n}{plt}\PYG{o}{.}\PYG{n}{xlabel}\PYG{p}{(}\PYG{l+s+s1}{\PYGZsq{}}\PYG{l+s+s1}{\PYGZdl{}x\PYGZdl{}}\PYG{l+s+s1}{\PYGZsq{}}\PYG{p}{)}
\PYG{n}{plt}\PYG{o}{.}\PYG{n}{ylabel}\PYG{p}{(}\PYG{l+s+s1}{\PYGZsq{}}\PYG{l+s+s1}{\PYGZdl{}v\PYGZdl{}}\PYG{l+s+s1}{\PYGZsq{}}\PYG{p}{)}
\PYG{n}{plt}\PYG{o}{.}\PYG{n}{show}\PYG{p}{(}\PYG{p}{)}
\end{sphinxVerbatim}

\end{sphinxuseclass}\end{sphinxVerbatimInput}
\begin{sphinxVerbatimOutput}

\begin{sphinxuseclass}{cell_output}
\noindent\sphinxincludegraphics{{E_fields_graphing_184_3_0}.png}

\end{sphinxuseclass}\end{sphinxVerbatimOutput}

\end{sphinxuseclass}
\sphinxstepscope


\chapter{28 Sep 23 \sphinxhyphen{} Laplace’s Equation}
\label{\detokenize{content/2_EM/laplace_eq:sep-23-laplace-s-equation}}\label{\detokenize{content/2_EM/laplace_eq::doc}}
\sphinxstepscope


\chapter{3 Oct 23 \sphinxhyphen{} More PDEs}
\label{\detokenize{content/2_EM/more_PDEs:oct-23-more-pdes}}\label{\detokenize{content/2_EM/more_PDEs::doc}}
\sphinxstepscope


\chapter{5 sep 23 \sphinxhyphen{} Method of Relaxation}
\label{\detokenize{content/2_EM/relaxation:sep-23-method-of-relaxation}}\label{\detokenize{content/2_EM/relaxation::doc}}
\sphinxstepscope


\chapter{10 Oct 23 \sphinxhyphen{} Magnetic Fields}
\label{\detokenize{content/2_EM/B_fields:oct-23-magnetic-fields}}\label{\detokenize{content/2_EM/B_fields::doc}}
\sphinxstepscope


\chapter{Electromagnetic Waves \& the Wave Equation}
\label{\detokenize{content/2_EM/EM_waves:electromagnetic-waves-the-wave-equation}}\label{\detokenize{content/2_EM/EM_waves::doc}}
\sphinxstepscope


\part{3 \sphinxhyphen{} Waves and Complex Analysis}

\sphinxstepscope


\chapter{Intro to Waves}
\label{\detokenize{content/3_waves/waves_intro:intro-to-waves}}\label{\detokenize{content/3_waves/waves_intro::doc}}
\sphinxstepscope


\chapter{17 Oct 23 \sphinxhyphen{} Normal Modes}
\label{\detokenize{content/3_waves/normal_modes:oct-23-normal-modes}}\label{\detokenize{content/3_waves/normal_modes::doc}}

\section{Three Coupled Oscillators}
\label{\detokenize{content/3_waves/normal_modes:three-coupled-oscillators}}
\sphinxAtStartPar
Consider the setup below consisting of three masses connected by springs to each other. We intend to find the normal modes of the system by denoting each mass’s displacement (\(x_1\), \(x_2\), and \(x_3\)).




\subsection{Finding the Normal Mode Frequencies}
\label{\detokenize{content/3_waves/normal_modes:finding-the-normal-mode-frequencies}}
\sphinxAtStartPar
\sphinxstylestrong{✅ Do this}

\sphinxAtStartPar
This is not magic as we will see, it follows from our choices of solution. Here’s the steps and what you might notice about them:
\begin{enumerate}
\sphinxsetlistlabels{\arabic}{enumi}{enumii}{}{.}%
\item {} 
\sphinxAtStartPar
Guess what the normal modes might look like? Write your guesses down; how should the masses move? (It’s ok if you are not sure about all of them, try to determine one of them)

\item {} 
\sphinxAtStartPar
Write down the energy for the whole system, \(T\) and \(U\) (We have done this before, but not for this many particles)

\item {} 
\sphinxAtStartPar
Use the Euler\sphinxhyphen{}Lagrange Equation to find the equations of motion for \(x_1\), \(x_2\), and \(x_3\). (We have done this lots, so make sure it feels solid)

\item {} 
\sphinxAtStartPar
Reformulate the equations of motion as a matrix equation \(\ddot{\mathbf{x}} = \mathbf{A} \mathbf{x}\). What is \(\mathbf{A}\)? (We have done this, but only quickly, so take your time)

\item {} 
\sphinxAtStartPar
Consider solutions of the form \(Ce^{i{\omega}t}\), plug that into \(x_1\), \(x_2\), and \(x_3\) to show you get \(\mathbf{A}\mathbf{x} = -\omega^2 \mathbf{x}\). (We have not done this, we just assumed it works! It’s ok if this is annoying, we only have to show it once.)

\item {} 
\sphinxAtStartPar
Find the normal mode frequencies by taking the determinant of \(\mathbf{A} - \mathbf{I}\lambda\). Note that this produces the following definition: \(\lambda = -\omega^2\)

\end{enumerate}


\subsection{Finding the Normal Modes Amplitudes}
\label{\detokenize{content/3_waves/normal_modes:finding-the-normal-modes-amplitudes}}
\sphinxAtStartPar
Ok, now we need to find the normal mode amplitudes. That is we assumed sinusoidal oscillations, but at what amplitudes? We will show how to do this with one frequency (\(\omega_1\)), and then break up the work of the the other two. These frequencies are:
\begin{equation*}
\begin{split}\omega_A = 2\dfrac{k}{m}; \qquad \omega_B = \left(2-\sqrt{2}\right)\dfrac{k}{m}; \qquad \omega_C = \left(2+\sqrt{2}\right)\dfrac{k}{m}\qquad\end{split}
\end{equation*}
\sphinxAtStartPar
\sphinxstylestrong{✅ Do this}

\sphinxAtStartPar
After we do the first one, pick another frequencies and repeat. Answer the follow questions:
\begin{enumerate}
\sphinxsetlistlabels{\arabic}{enumi}{enumii}{}{.}%
\item {} 
\sphinxAtStartPar
What does this motion physically look like? What are the masses doing?

\item {} 
\sphinxAtStartPar
How does the frequency of oscillation make sense? Why is it higher or lower than \(\omega_A\)?

\end{enumerate}

\sphinxAtStartPar
The two cells below have some code that shows how you could’ve used python to help you when solving this problem:

\begin{sphinxuseclass}{cell}\begin{sphinxVerbatimInput}

\begin{sphinxuseclass}{cell_input}
\begin{sphinxVerbatim}[commandchars=\\\{\}]
\PYG{k+kn}{import} \PYG{n+nn}{numpy} \PYG{k}{as} \PYG{n+nn}{np}
\PYG{k+kn}{import} \PYG{n+nn}{matplotlib}\PYG{n+nn}{.}\PYG{n+nn}{pyplot} \PYG{k}{as} \PYG{n+nn}{plt}
\PYG{k+kn}{from} \PYG{n+nn}{sympy} \PYG{k+kn}{import} \PYG{o}{*}
\PYG{k+kn}{from} \PYG{n+nn}{sympy} \PYG{k+kn}{import} \PYG{n}{Matrix} \PYG{c+c1}{\PYGZsh{} get symbolic matrix methods}
\PYG{n}{init\PYGZus{}printing}\PYG{p}{(}\PYG{n}{use\PYGZus{}unicode}\PYG{o}{=}\PYG{k+kc}{True}\PYG{p}{)} \PYG{c+c1}{\PYGZsh{} make math display good}

\PYG{n}{A} \PYG{o}{=} \PYG{n}{np}\PYG{o}{.}\PYG{n}{array}\PYG{p}{(}\PYG{p}{[}\PYG{p}{[}\PYG{o}{\PYGZhy{}}\PYG{l+m+mi}{2}\PYG{p}{,} \PYG{l+m+mi}{1}\PYG{p}{,} \PYG{l+m+mi}{0}\PYG{p}{]}\PYG{p}{,} \PYG{p}{[}\PYG{l+m+mi}{1}\PYG{p}{,} \PYG{o}{\PYGZhy{}}\PYG{l+m+mi}{2}\PYG{p}{,} \PYG{l+m+mi}{1}\PYG{p}{]}\PYG{p}{,} \PYG{p}{[}\PYG{l+m+mi}{0}\PYG{p}{,} \PYG{l+m+mi}{1}\PYG{p}{,} \PYG{o}{\PYGZhy{}}\PYG{l+m+mi}{2}\PYG{p}{]}\PYG{p}{]}\PYG{p}{)} \PYG{c+c1}{\PYGZsh{}\PYGZsh{} numpy matrix}
\PYG{n}{A\PYGZus{}sympy} \PYG{o}{=} \PYG{n}{Matrix}\PYG{p}{(}\PYG{n}{M}\PYG{p}{)} \PYG{c+c1}{\PYGZsh{}\PYGZsh{} Take numpy matrix and make it a sympy one}

\PYG{n}{eigenvals}\PYG{p}{,} \PYG{n}{eigenvects} \PYG{o}{=} \PYG{n}{np}\PYG{o}{.}\PYG{n}{linalg}\PYG{o}{.}\PYG{n}{eig}\PYG{p}{(}\PYG{n}{A}\PYG{p}{)} \PYG{c+c1}{\PYGZsh{} numpy numerical methods}
\PYG{n+nb}{print}\PYG{p}{(}\PYG{l+s+s2}{\PYGZdq{}}\PYG{l+s+s2}{numpy eigenvals:}\PYG{l+s+s2}{\PYGZdq{}}\PYG{p}{,}\PYG{n}{eigenvals}\PYG{p}{)}
\PYG{n+nb}{print}\PYG{p}{(}\PYG{l+s+s2}{\PYGZdq{}}\PYG{l+s+s2}{numpy eigenvects:}\PYG{l+s+s2}{\PYGZdq{}}\PYG{p}{,}\PYG{n}{eigenvects}\PYG{p}{)}
\PYG{n+nb}{print}\PYG{p}{(}\PYG{l+s+s2}{\PYGZdq{}}\PYG{l+s+s2}{sympy eigenvals:}\PYG{l+s+s2}{\PYGZdq{}}\PYG{p}{)}
\PYG{n}{A\PYGZus{}sympy}\PYG{o}{.}\PYG{n}{eigenvals}\PYG{p}{(}\PYG{p}{)} \PYG{c+c1}{\PYGZsh{} sympy symbolic methods WARNING: slow for big matrices}
\end{sphinxVerbatim}

\end{sphinxuseclass}\end{sphinxVerbatimInput}
\begin{sphinxVerbatimOutput}

\begin{sphinxuseclass}{cell_output}
\begin{sphinxVerbatim}[commandchars=\\\{\}]
\PYG{g+gt}{\PYGZhy{}\PYGZhy{}\PYGZhy{}\PYGZhy{}\PYGZhy{}\PYGZhy{}\PYGZhy{}\PYGZhy{}\PYGZhy{}\PYGZhy{}\PYGZhy{}\PYGZhy{}\PYGZhy{}\PYGZhy{}\PYGZhy{}\PYGZhy{}\PYGZhy{}\PYGZhy{}\PYGZhy{}\PYGZhy{}\PYGZhy{}\PYGZhy{}\PYGZhy{}\PYGZhy{}\PYGZhy{}\PYGZhy{}\PYGZhy{}\PYGZhy{}\PYGZhy{}\PYGZhy{}\PYGZhy{}\PYGZhy{}\PYGZhy{}\PYGZhy{}\PYGZhy{}\PYGZhy{}\PYGZhy{}\PYGZhy{}\PYGZhy{}\PYGZhy{}\PYGZhy{}\PYGZhy{}\PYGZhy{}\PYGZhy{}\PYGZhy{}\PYGZhy{}\PYGZhy{}\PYGZhy{}\PYGZhy{}\PYGZhy{}\PYGZhy{}\PYGZhy{}\PYGZhy{}\PYGZhy{}\PYGZhy{}\PYGZhy{}\PYGZhy{}\PYGZhy{}\PYGZhy{}\PYGZhy{}\PYGZhy{}\PYGZhy{}\PYGZhy{}\PYGZhy{}\PYGZhy{}\PYGZhy{}\PYGZhy{}\PYGZhy{}\PYGZhy{}\PYGZhy{}\PYGZhy{}\PYGZhy{}\PYGZhy{}\PYGZhy{}\PYGZhy{}}
\PYG{n+ne}{NameError}\PYG{g+gWhitespace}{                                 }Traceback (most recent call last)
\PYG{n+nn}{Input In [1],} in \PYG{n+ni}{\PYGZlt{}cell line: 8\PYGZgt{}}\PYG{n+nt}{()}
\PYG{g+gWhitespace}{      }\PYG{l+m+mi}{5} \PYG{n}{init\PYGZus{}printing}\PYG{p}{(}\PYG{n}{use\PYGZus{}unicode}\PYG{o}{=}\PYG{k+kc}{True}\PYG{p}{)} \PYG{c+c1}{\PYGZsh{} make math display good}
\PYG{g+gWhitespace}{      }\PYG{l+m+mi}{7} \PYG{n}{A} \PYG{o}{=} \PYG{n}{np}\PYG{o}{.}\PYG{n}{array}\PYG{p}{(}\PYG{p}{[}\PYG{p}{[}\PYG{o}{\PYGZhy{}}\PYG{l+m+mi}{2}\PYG{p}{,} \PYG{l+m+mi}{1}\PYG{p}{,} \PYG{l+m+mi}{0}\PYG{p}{]}\PYG{p}{,} \PYG{p}{[}\PYG{l+m+mi}{1}\PYG{p}{,} \PYG{o}{\PYGZhy{}}\PYG{l+m+mi}{2}\PYG{p}{,} \PYG{l+m+mi}{1}\PYG{p}{]}\PYG{p}{,} \PYG{p}{[}\PYG{l+m+mi}{0}\PYG{p}{,} \PYG{l+m+mi}{1}\PYG{p}{,} \PYG{o}{\PYGZhy{}}\PYG{l+m+mi}{2}\PYG{p}{]}\PYG{p}{]}\PYG{p}{)} \PYG{c+c1}{\PYGZsh{}\PYGZsh{} numpy matrix}
\PYG{n+ne}{\PYGZhy{}\PYGZhy{}\PYGZhy{}\PYGZhy{}\PYGZgt{} }\PYG{l+m+mi}{8} \PYG{n}{A\PYGZus{}sympy} \PYG{o}{=} \PYG{n}{Matrix}\PYG{p}{(}\PYG{n}{M}\PYG{p}{)} \PYG{c+c1}{\PYGZsh{}\PYGZsh{} Take numpy matrix and make it a sympy one}
\PYG{g+gWhitespace}{     }\PYG{l+m+mi}{10} \PYG{n}{eigenvals}\PYG{p}{,} \PYG{n}{eigenvects} \PYG{o}{=} \PYG{n}{np}\PYG{o}{.}\PYG{n}{linalg}\PYG{o}{.}\PYG{n}{eig}\PYG{p}{(}\PYG{n}{A}\PYG{p}{)} \PYG{c+c1}{\PYGZsh{} numpy numerical methods}
\PYG{g+gWhitespace}{     }\PYG{l+m+mi}{11} \PYG{n+nb}{print}\PYG{p}{(}\PYG{l+s+s2}{\PYGZdq{}}\PYG{l+s+s2}{numpy eigenvals:}\PYG{l+s+s2}{\PYGZdq{}}\PYG{p}{,}\PYG{n}{eigenvals}\PYG{p}{)}

\PYG{n+ne}{NameError}: name \PYGZsq{}M\PYGZsq{} is not defined
\end{sphinxVerbatim}

\end{sphinxuseclass}\end{sphinxVerbatimOutput}

\end{sphinxuseclass}
\begin{sphinxuseclass}{cell}\begin{sphinxVerbatimInput}

\begin{sphinxuseclass}{cell_input}
\begin{sphinxVerbatim}[commandchars=\\\{\}]
\PYG{n+nb}{print}\PYG{p}{(}\PYG{l+s+s2}{\PYGZdq{}}\PYG{l+s+s2}{sympy eigenvects:}\PYG{l+s+s2}{\PYGZdq{}}\PYG{p}{)}
\PYG{n}{A\PYGZus{}sympy}\PYG{o}{.}\PYG{n}{eigenvects}\PYG{p}{(}\PYG{p}{)}
\end{sphinxVerbatim}

\end{sphinxuseclass}\end{sphinxVerbatimInput}
\begin{sphinxVerbatimOutput}

\begin{sphinxuseclass}{cell_output}
\begin{sphinxVerbatim}[commandchars=\\\{\}]
sympy eigenvects:
\end{sphinxVerbatim}

\noindent\sphinxincludegraphics{{normal_modes_4_1}.png}

\end{sphinxuseclass}\end{sphinxVerbatimOutput}

\end{sphinxuseclass}

\section{Extending your work}
\label{\detokenize{content/3_waves/normal_modes:extending-your-work}}
\sphinxAtStartPar
Given what we have done thus far, you can see that we could easily construct the matrix for a \(N\) dimensional chain of 1D oscillators. So let’s do that.

\sphinxAtStartPar
\sphinxstylestrong{✅ Do this}

\sphinxAtStartPar
Repeat this analysis for a set of \(N\) oscillators. Your code should be able to:
\begin{enumerate}
\sphinxsetlistlabels{\arabic}{enumi}{enumii}{}{.}%
\item {} 
\sphinxAtStartPar
Take a value of \(N\) and construct the right matrix representation

\item {} 
\sphinxAtStartPar
Find the eigenvalues and eigenvectors for this matrix.

\item {} 
\sphinxAtStartPar
(BONUS) plots the modes automatically

\item {} 
\sphinxAtStartPar
(CHALLENGE) time the execution of the analysis

\end{enumerate}

\sphinxAtStartPar
Be careful not to pick too large of an \(N\) value to work with because you could melt your CPU easily. Make sure your code can do something like \(N=10\). If you get the timing working, plot time vs number of objects to see how the problem scales with more oscillators.

\begin{sphinxuseclass}{cell}\begin{sphinxVerbatimInput}

\begin{sphinxuseclass}{cell_input}
\begin{sphinxVerbatim}[commandchars=\\\{\}]
\PYG{c+c1}{\PYGZsh{}\PYGZsh{} Your code here}
\end{sphinxVerbatim}

\end{sphinxuseclass}\end{sphinxVerbatimInput}

\end{sphinxuseclass}

\section{Even further}
\label{\detokenize{content/3_waves/normal_modes:even-further}}
\sphinxAtStartPar
These models can be used with lattices (solid objects). Draw a sketch of 4 oscillators in a plane connected together in a square shape. Write down the energy equations for this system (assume the springs do not move laterally much). What do the EOMs look like?


\subsection{Notes}
\label{\detokenize{content/3_waves/normal_modes:notes}}\begin{itemize}
\item {} 
\sphinxAtStartPar
\sphinxhref{https://github.com/dannycab/phy415msu/blob/main/MMIPbook/assets/pdfs/notes/Notes\_2\_Three\_Coupled\_Oscillators.pdf}{Partial Solution to Activity}

\end{itemize}

\sphinxstepscope


\chapter{19 Oct 23 \sphinxhyphen{} Mechanical Waves}
\label{\detokenize{content/3_waves/mech_waves:oct-23-mechanical-waves}}\label{\detokenize{content/3_waves/mech_waves::doc}}
\sphinxstepscope


\chapter{26 Oct 23 \sphinxhyphen{} Examples of Waves}
\label{\detokenize{content/3_waves/wave_examples:oct-23-examples-of-waves}}\label{\detokenize{content/3_waves/wave_examples::doc}}
\sphinxstepscope


\chapter{31 Oct 23 \sphinxhyphen{} Complex Analysis \& the Fourier Transform}
\label{\detokenize{content/3_waves/complex:oct-23-complex-analysis-the-fourier-transform}}\label{\detokenize{content/3_waves/complex::doc}}
\sphinxstepscope


\chapter{2 Nov 23 \sphinxhyphen{} Discrete and Fast Fourier Transforms}
\label{\detokenize{content/3_waves/DFT_FFT:nov-23-discrete-and-fast-fourier-transforms}}\label{\detokenize{content/3_waves/DFT_FFT::doc}}
\sphinxstepscope


\part{4 \sphinxhyphen{} Random Processes and Distributions}

\sphinxstepscope


\chapter{Intro to Distributions}
\label{\detokenize{content/4_distributions/distributions_intro:intro-to-distributions}}\label{\detokenize{content/4_distributions/distributions_intro::doc}}






\renewcommand{\indexname}{Index}
\printindex
\end{document}